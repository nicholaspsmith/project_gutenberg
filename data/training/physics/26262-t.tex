% %%%%%%%%%%%%%%%%%%%%%%%%%%%%%%%%%%%%%%%%%%%%%%%%%%%%%%%%%%%%%%%%%%%%%%% %
%                                                                         %
% Project Gutenberg's Utility of Quaternions in Physics, by Alexander McAulay
%                                                                         %
% This eBook is for the use of anyone anywhere at no cost and with        %
% almost no restrictions whatsoever.  You may copy it, give it away or    %
% re-use it under the terms of the Project Gutenberg License included     %
% with this eBook or online at www.gutenberg.org                          %
%                                                                         %
%                                                                         %
% Title: Utility of Quaternions in Physics                                %
%                                                                         %
% Author: Alexander McAulay                                               %
%                                                                         %
% Release Date: August 11, 2008 [EBook #26262]                            %
%                                                                         %
% Language: English                                                       %
%                                                                         %
% Character set encoding: ISO-8859-1                                      %
%                                                                         %
% *** START OF THIS PROJECT GUTENBERG EBOOK UTILITY OF QUATERNIONS IN PHYSICS ***
%                                                                         %
% %%%%%%%%%%%%%%%%%%%%%%%%%%%%%%%%%%%%%%%%%%%%%%%%%%%%%%%%%%%%%%%%%%%%%%% %

\def\ebook{26262}
%%%%%%%%%%%%%%%%%%%%%%%%%%%%%%%%%%%%%%%%%%%%%%%%%%%%%%%%%%%%%%%%%%%%%%
%%                                                                  %%
%% Packages and substitutions:                                      %%
%%                                                                  %%
%% book:     Required.                                              %%
%% geometry: Enhanced page layout package. Required.                %%
%%                                                                  %%
%% inputenc: Standard DP encoding. Required.                        %%
%% amsmath:  AMS mathematics enhancements. Required.                %%
%% amssymb:  Additional mathematical symbols. Required.             %%
%% yfonts:   Fraktur fonts, fall back to bold. Optional.            %%
%% alltt:    Fixed-width font environment. Required.                %%
%% footmisc: Enhanced footnote capabilities. Required.              %%
%% perpage:  Restart footnote numbering on each page. Required.     %%
%% txfonts:  PostScript Times fonts. Required.                      %%
%% indentfirst: Indent after headings. Recommended.                 %%
%%                                                                  %%
%% ifthen:   Logical conditionals. Required.                        %%
%%                                                                  %%
%% fancyhdr: Enhanced running headers and footers. Required.        %%
%% extramarks: Part of fancyhdr                                     %%
%% graphicx: Standard interface for graphics inclusion. Required.   %%
%%                                                                  %%
%% hyperref: Hypertext embellishments for pdf output. Required.     %%
%%                                                                  %%
%% Producer's Comments: The book contains a lot of in-line math and %%
%%                      uses idiosyncratic notation (particularly   %%
%%                      an upside-down D symbol), unusual titling,  %%
%%                      numbering, and ToC; the pagination should   %%
%%                      be straightforward, however.                %%
%%                      Custom \Chapter, \Section, and \Paragraph   %%
%%                      commands implement the book's divisions,    %%
%%                      see "formatting details" below.             %%
%%                                                                  %%
%%                                                                  %%
%% Things to Check:                                                 %%
%%                                                                  %%
%% Spellcheck: OK                                                   %%
%% Smoothreading pool: Yes                                          %%
%% lacheck: OK, false positives:                                    %%
%%     unmatched braces, et. al.  (ten instances)                   %%
%%     bad character in label     (three instances, in macro defs)  %%
%%     missing `\@' before `.'    (in commented region)             %%
%%     deliberate space after {   (seven instances)                 %%
%%     multiple complaints from log file                            %%
%%                                                                  %%
%% Lprep/gutcheck: OK                                               %%
%% PDF pages: 134                                                   %%
%% PDF page size: US Letter (8.5" x 11")                            %%
%% PDF bookmarks: created, point to ToC entries                     %%
%% PDF document info: filled in                                     %%
%% ToC page numbers: OK                                             %%
%% Images: 2 PDF                                                    %%
%%                                                                  %%
%% Summary of log file:                                             %%
%% * Six underfull vboxes.                                          %%
%% * One overfull hbox (smaller than 2.24pt).                       %%
%%                                                                  %%
%% Formatting details:                                              %%
%%                                                                  %%
%% The book's divisions don't map well to plain LaTeX, so the       %%
%% preamble defines the following commands:                         %%
%%                                                                  %%
%% 1. \Chapter: Primary chapter-like headings, which the book calls %%
%%   ``Sections''.                                                  %%
%%                                                                  %%
%% 1a. \FNChapter (footnote chapter): The title of ``Section III''  %%
%%   contains a footnote in the title, which should not go into the %%
%%   toc or running head.                                           %%
%%                                                                  %%
%% 1b. \HeadChapter: The title of ``Section I'' is preceded by a    %%
%%   half title page.                                               %%
%%                                                                  %%
%% 1c. \Preface: Formatted like a \Chapter, but no toc entry.       %%
%%                                                                  %%
%%                                                                  %%
%% 2. \Paragraph: Numbered divisions formatted like paragraphs,     %%
%%   but referenced internally with the ``section'' symbol \S. By   %%
%%   default these are not listed in the toc, but their numbering   %%
%%   affects the toc.                                               %%
%%                                                                  %%
%%   The number of \Paragraph{94} is deliberately not indented, to  %%
%%   leave space on the line for an in-line math expression.        %%
%%                                                                  %%
%% 2a. \Section: \Paragraph having a toc entry, and labeled in the  %%
%%   text with a centered italic title.                             %%
%%                                                                  %%
%% 2b. \SectionTitle: No toc entry, but having a centered italic    %%
%%   title.                                                         %%
%%                                                                  %%
%% 2c. \TCSection (``ToC section''): A \Section whose title and     %%
%%   toc entry differ (e.g., because of a footnote).                %%
%%                                                                  %%
%%                                                                  %%
%% 3. Equations are tagged and referenced with \EqnTag and \Eref.   %%
%%   \EqnTag{section}{equation} prints a tag and issues a label     %%
%%   command, \Eref{section}{equation} generates the corresponding  %%
%%   reference. For instance, \EqnTag{IV}{16} makes its equation    %%
%%   referrable with \Eref{IV}{16}.                                 %%
%%                                                                  %%
%%                                                                  %%
%% Compile History:                                                 %%
%%                                                                  %%
%% August, 2008: adhere (Andrew D. Hwang)                           %%
%%               texlive2007, GNU/Linux                             %%
%%                                                                  %%
%% August, 2008: dcwilson                                           %%
%%         MiKTeX 2.7, Windows XP Pro                               %%
%%                                                                  %%
%%     pdflatex x3 (Run pdflatex three times)                       %%
%%                                                                  %%
%%                                                                  %%
%% August 2008: pglatex.                                            %%
%%   Compile this project with:                                     %%
%%   pdflatex 26262-t.tex ..... THREE times                         %%
%%                                                                  %%
%%   pdfeTeX, Version 3.141592-1.30.5-2.2 (Web2C 7.5.5)             %%
%%                                                                  %%
%%%%%%%%%%%%%%%%%%%%%%%%%%%%%%%%%%%%%%%%%%%%%%%%%%%%%%%%%%%%%%%%%%%%%%

%%%%%%%%%%%%%%%%%%%%%%%%%%%%% PACKAGES %%%%%%%%%%%%%%%%%%%%%%%%%%%%%%%

\listfiles
\documentclass[12pt,letterpaper]{book}[2005/09/16]

\usepackage{amsmath}[2000/07/18]  % AMS math enhancements
\usepackage{amssymb}[2002/01/22]  % Extra math symbols

\usepackage{ifthen}[2001/05/26]   % Logical conditionals
\usepackage[latin1]{inputenc}[2006/05/05] % DP standard encoding

\usepackage{graphicx}[1999/02/16] % publisher's mark, upside-down D operator

% Define \textgoth; document subsequently defines \Mygoth
\IfFileExists{yfonts.sty}%
{\usepackage{yfonts}[2003/01/08]} % fraktur font (titlepage only)
{\providecommand{\textgoth}[1]{\textbf{##1}}} % fallback if no yfonts

\usepackage{alltt}[1997/06/16]    % for boilerplate, credits, license

\usepackage{indentfirst}[1995/11/23]

\usepackage[symbol]{footmisc}[2005/03/17]
\usepackage{perpage}[2006/07/15]  % footnote numbering restarts on each page

\usepackage{txfonts}[2005/01/03]  % PostScript Times font
\usepackage{fancyhdr,extramarks}  % running heads


%%%%%%%%%%%%%%%%%%%%%%%%%%%%%%%%%%%%%%%%%%%%%%%%%%%%%%%%%%%%%%%%%
%%%% Interlude:  Set up PRINTING (default) or SCREEN VIEWING %%%%
%%%%%%%%%%%%%%%%%%%%%%%%%%%%%%%%%%%%%%%%%%%%%%%%%%%%%%%%%%%%%%%%%

% ForPrinting=true (default)           false
% Asymmetric margins                   Symmetric margins
% Black hyperlinks                     Blue hyperlinks
% Even-odd page headings               Uniform page headings
% Start Preface, ToC, etc. recto       No blank verso pages
%
% Chapter-like ``Sections'' start both recto and verso in the scanned
% book. This behavior has been retained.
\newboolean{ForPrinting}

%% *** COMMENT the next line for a SCREEN-OPTIMIZED VERSION *** %%
\setboolean{ForPrinting}{true}

%% Initialize values to ForPrinting=false
\newcommand{\Margins}{hmarginratio=1:1}     % Symmetric margins
\newcommand{\HLinkColor}{blue}              % Hyperlink color
\newcommand{\PDFPageLayout}{SinglePage}
\newcommand{\TransNote}{Transcriber's Note}
\newcommand{\TransNoteText}{%
  This PDF file is formatted for screen viewing, but may be easily
  recompiled for printing. Please see the preamble of the \LaTeX\
  source file for instructions.
}

%% Re-set if ForPrinting=true
\ifthenelse{\boolean{ForPrinting}}{%
  \renewcommand{\Margins}{hmarginratio=2:3} % Asymmetric margins
  \renewcommand{\HLinkColor}{black}         % Hyperlink color
  \renewcommand{\PDFPageLayout}{TwoPageRight}
  \renewcommand{\TransNote}{Transcriber's Notes}
  \renewcommand{\TransNoteText}{%
    Table of contents entries, section references, and equation
    numbers are internal hyperlinks.
    \bigskip

    This PDF file is formatted for printing, but may be easily
    recompiled for screen viewing. Please see the preamble of the
    \LaTeX\ source file for instructions.
  }
}{% If ForPrinting=false, don't skip to recto
  \renewcommand{\cleardoublepage}{\clearpage}
}
%%%% End of PRINTING/SCREEN VIEWING code; back to packages %%%%


%%%% Text block size %%%%
% The printed book's text block is well-approximated by hscale=0.5, but
% the painstakingly-chosen value hscale=0.63 minimizes overfull hboxes
% and is not uncomfortable to read, ~65 chars per line.
% The vscale value is the package default, specified explicitly to guard
% against changes to geometry.sty.
% N.B. global letterpaper option above
\usepackage[hscale=0.63,vscale=0.7,\Margins]{geometry}[2002/07/08]

%% Alternative text block settings for Computer Modern Roman %%
%\usepackage[hscale=0.67,vscale=0.7,\Margins]{geometry}[2002/07/08]

%%%% PDF and hyperref options %%%%
\providecommand{\ebook}{00000}    % Overridden during white-washing
\usepackage[pdftex,
  hyperref,
  hyperfootnotes=false,
  pdftitle={The Project Gutenberg eBook \#\ebook: Utility of Quaternions in Physics},
  pdfauthor={Alexander McAulay},
  pdfkeywords={Joshua Hutchinson, Andrew D. Hwang, Carolyn Bottomley,
               Project Gutenberg Online Distributed Proofreading Team,
               Cornell University},
  pdfstartview=Fit,    % default value
  pdfstartpage=1,      % default value
  pdfpagemode=UseNone, % default value
  bookmarks=true,      % default value
  linktocpage=true,
  pdfdisplaydoctitle,
  pdfpagelayout=\PDFPageLayout,
  pdfpagelabels=true,
  bookmarksopen=true,
  bookmarksopenlevel=1,
  colorlinks=true,
  linkcolor=\HLinkColor]{hyperref}[2007/02/07]

% Re-crop screen-formatted version
\ifthenelse{\boolean{ForPrinting}}
  {}
  {\hypersetup{pdfpagescrop= 65 90 547 752}}


%%%%%%%%%%%%%%%%%%%%%%%%%%%%% PREAMBLE %%%%%%%%%%%%%%%%%%%%%%%%%%%%%%%

% No number on first page of Table of Contents
\AtBeginDocument{\addtocontents{toc}{\protect\thispagestyle{empty}}}

% To prevent underfull hboxes
\newcommand{\stretchyspace}{\spaceskip0.5em plus 0.5em minus 0.25em}
% To improve spacing on titlepage
\newcommand{\Stretchout}{\spaceskip0.85em}

%%%% Fixed-width environment to format PG boilerplate %%%%
% 9.2pt leaves no overfull hbox at 80 char line width
\newenvironment{PGtext}{%
\begin{alltt}
\fontsize{9.2}{10.5}\ttfamily\selectfont}%
{\end{alltt}}

%%%% Reset Table of Contents heading %%%%
\renewcommand{\contentsname}
  {\protect\centering\protect\normalfont\protect\scshape{%
      \protect\large CONTENTS.}}


%%%% Footnote numbers restart on each page %%%%
\MakePerPage{footnote}


%%%% \Chapter and related macros %%%%

% Set up page for start of \Chapter
\newcommand{\ChapterPrep}[1]{%
  \clearpage % [sic]; ``Sections'' start both recto and verso
  \phantomsection
  \vspace*{0.75in}
  \pagestyle{fancy}
  \thispagestyle{empty}
  \fancyhead{}
  \fancyfoot{}
  \renewcommand{\headrulewidth}{0pt}
  \fancyhead[C]{{\scshape\small\MakeLowercase{#1.}}}
  \ifthenelse{\boolean{ForPrinting}}
    {\fancyhead[RO,LE]{\thepage}}
    {\fancyhead[R]{\thepage}}
  \setlength{\headheight}{14.5pt}
}

% increment counter, issue label
\newcommand{\ChapterRef}[2]{%
  \refstepcounter{chapter}
  \label{chapter:\thechapter}
  \pdfbookmark[0]{#2}{#2}
  \addtocontents{toc}
    {\protect\bigskip\normalfont\scshape\protect\centering %
       #1.\protect\quad #2\protect\medskip}
}

%% Macros used in file body %%
% Special format, no toc for Preface
\newcommand{\Preface}{%
  \ChapterPrep{Preface}

  \begin{center}
    \textsc{\Large Preface}.
  \end{center}
}


% For Chapter I, which has a pre-title
\newcommand{\HeadChapter}[3]{%
  \ChapterPrep{#2}
  \ChapterRef{#1}{#2}

  \begin{center}
    \textsc{\Large #3}.
    \bigskip

    \textsc{\Large #1}.
    \bigskip

    \textsc{\large #2}.
  \end{center}
}

% For Chapter III, whose title contains a footnote
\newcommand{\FNChapter}[3]{%
  \ChapterPrep{#2}
  \ChapterRef{#1}{#2}

  \begin{center}
    \textsc{\Large #1}.
    \bigskip

    \textsc{\large #3}.
  \end{center}
}

% For others. Usage: \Chapter{Section N}{Title}
\newcommand{\Chapter}[2]{\FNChapter{#1}{#2}{#2}}


%%%% \Paragraph, \Section, and related macros %%%%

% Space between Section number and title in ToC
\newlength{\Tocskip}
\setlength{\Tocskip}{2em}

% Add toc line: \Tocline{n}{title} or \Tocline{}{title}
\newcommand{\Tocline}[2]{%
  \phantomsection
  \ifthenelse{\equal{#1}{}}{% #1 is the null string?
    \addcontentsline{toc}{section}{%
      \texorpdfstring{\protect\makebox[\Tocskip]{}%
        \protect\normalfont\protect\footnotesize{\protect{#2}}
      }{\protect{#2}}
    }
  }{% Otherwise
    \addcontentsline{toc}{section}{%
      \texorpdfstring{\protect\makebox[\Tocskip][l]{\protect{#1}.}% add period
        \protect\normalfont\protect\footnotesize{\protect{#2}}
      }{\protect{#2}}
    }
  }
}

% For paragraphs 3a, 16a, and 45. Usage: \FNParagraph{label}{tag}
\newcommand{\FNParagraph}[2]{% [** PP: Dirty trick -- don't indent \S94]
  \ifthenelse{\equal{#2}{94}}{\paragraph{#2.}}{\paragraph{\indent#2.}}
  \phantomsection
  \label{section:#1}
  % Set up section numbers in running head
  \extramarks{\S~#1.}{\S~#1.}
  \ifthenelse{\boolean{ForPrinting}}
  {\fancyhead[RE]{[\,\firstleftxmark} % Printed: par. range verso-recto
   \fancyhead[LO]{\lastrightxmark\,]}}
%
  {\fancyhead[L]{[\,\firstleftxmark\,]}} % Screen: ``initial'' par.
%
  \setlength{\headheight}{14.5pt} % Either case
}

\newcommand{\Paragraph}[1]{\FNParagraph{#1}{#1}}


% Title only; used in \TCSection defn, and four times in the text
\newcommand{\SectionTitle}[1]{%
  \subsubsection*{\centering\normalfont\itshape #1.} % Add dot
}

% Most general Section-setting command; handles sections whose
% title contains a footnote. Usage: \TCSection{tag}{ToC title}{Title}
\newcommand{\TCSection}[3]{%
  \SectionTitle{#3}
  \Tocline{#1}{#2}
}

% Most common Section-setting command; title = ToC entry
\newcommand{\Section}[2]{\TCSection{#1}{#2}{#2}}


%%%% Equation and section cross-referencing %%%%
% ``equation tag'', labels equation with Roman section and equation number
\newcommand{\EqnTag}[2]{%
  \tag{#2}%
  \label{eqn:#1.#2}% e.g. \label{eqn:IV.45} = equation 45 in Section IV
}

% ``equation reference'', e.g., \Eref{V}{45} -> \EqnTag{V}{45}
\newcommand{\Eref}[2]{\hyperref[eqn:#1.#2]{(#2)}}
\newcommand{\equationref}[2]{\hyperref[eqn:#1.#2]{equa\-tion~(#2)}}
\newcommand{\Equationref}[2]{\hyperref[eqn:#1.#2]{Equa\-tion~(#2)}}

% ``section reference''
\newcommand{\Secref}[1]{\hyperref[section:#1]{\S\,#1}}

% Pointers to errata
\newcommand{\Erratum}[1]{\phantomsection\label{#1}}

%%%% Miscellaneous macros %%%%
\newcommand{\McAulay}{M\raisebox{0.45ex}{c}AULAY}

% Reverts to \textbf if yfonts not available
\newcommand{\Mygoth}[1]{\textgoth{#1}}

\newcommand{\ds}{d\mathfrak{s}}

\newcommand{\smalliint}{{\textstyle\iint}}
\newcommand{\smalliiint}{{\textstyle\iiint}}

% Upside-down D operator
\newcommand{\Dop}{\raisebox{-0.4ex}{\rotatebox[origin=c]{180}{$\mathrm{D}$}}}

% set first arg as affix on second arg
\newcommand{\Pre}[2]{{}_{#1}{#2}}
\newcommand{\Nabla}[1]{\Pre{#1}{\!\nabla}}

%%%%%%%%%%%%%%%%%%%%%%%% START OF DOCUMENT %%%%%%%%%%%%%%%%%%%%%%%%%%

\begin{document}

\pagestyle{empty}
\pagenumbering{Roman}

\pdfbookmark[0]{Project Gutenberg Boilerplate}{Project Gutenberg Boilerplate}

\begin{center}
\begin{minipage}{\textwidth}
\begin{PGtext}
Project Gutenberg's Utility of Quaternions in Physics, by Alexander McAulay

This eBook is for the use of anyone anywhere at no cost and with
almost no restrictions whatsoever.  You may copy it, give it away or
re-use it under the terms of the Project Gutenberg License included
with this eBook or online at www.gutenberg.org


Title: Utility of Quaternions in Physics

Author: Alexander McAulay

Release Date: August 11, 2008 [EBook #26262]

Language: English

Character set encoding: ISO-8859-1

*** START OF THIS PROJECT GUTENBERG EBOOK UTILITY OF QUATERNIONS IN PHYSICS ***
\end{PGtext}
\end{minipage}
\end{center}
\vfill

%% -----File: 001.png---

% Formatted, commented production note
\iffalse
\null\vfill
\begin{center}
\begin{minipage}{3in}
\renewcommand{\baselinestretch}{0.8}
\begin{verbatim}
       Production Note

Cornell   University    Library
produced this volume to replace
the  irreparably   deteriorated
original.  It was scanned using
Xerox software and equipment at
600  dots  per inch  resolution
and compressed prior to storage
using      CCITT     Group    4
compression.   The digital data
were  used  to create Cornell's
replacement   volume  on  paper
that  meets the  ANSI  Standard
Z39.48-1984.  The production of
this  volume  was  supported in
part   by  the  Commission   on
Preservation and Access and the
Xerox Corporation.  1991.
\end{verbatim}
\renewcommand{\baselinestretch}{1} % restore default value
\end{minipage}
\end{center}
\vfill
\fi

%% -----File: 002.png---
% [Blank Page]
%% -----File: 003.png---

\iffalse

Cornell University Library

BOUGHT WITH THE INCOME
FROM THE

SAGE ENDOWMENT FUND

THE GIFT OF

Henry W. Sage
1891

A.52629 MATHEMATICS 24/10/93
\fi
%% -----File: 004.png---
% [Blank Page]
%% -----File: 005.png---Folio i-------

%%%%%%%%%%%%%%%%%%%%%%%%%%% FRONT MATTER %%%%%%%%%%%%%%%%%%%%%%%%%%

\frontmatter

\pagestyle{empty}
\pagenumbering{Alph}

%% Half title page
\null
\vfil
\begin{center}
\textbf{\Large UTILITY}\\[36pt]

\textsc{of}\\[36pt]

\textbf{\Large\Stretchout QUATERNIONS IN PHYSICS.}
\end{center}
\vfil
\clearpage

%% -----File: 006.png---Folio ii-------

% [Illustration: *Emblem* M & M Co.]

\null
\vspace{0pt plus 2fil}
\begin{center}
  \includegraphics[width=1.5in]{./images/pubmark.pdf}
\end{center}
\vfil

\clearpage

%% -----File: 007.png---Folio iii-------

%% Title page

% [** PP: Pagesize-dependent dimension]
\enlargethispage{12pt}

\noindent\begin{minipage}{\textwidth}
\centering
\textbf{\huge UTILITY}\\[36pt]

\textsc{\small of}\\[36pt]

\textbf{\huge\Stretchout QUATERNIONS IN PHYSICS.}\\[144pt]

\textsc{\small by}\\[18pt]

\textbf{\Stretchout A.~\McAulay, M.A.,}

\textsc{\scriptsize{%
LECTURER IN MATHEMATICS AND PHYSICS IN THE UNIVERSITY OF TASMANIA}.%
}\\[144pt]

\large \Mygoth{London}:

\textbf{\Stretchout MACMILLAN AND CO.}

\textbf{\footnotesize\Stretchout AND NEW YORK.}

1893\\[4pt]

[\textit{\footnotesize All Rights reserved}.]
\end{minipage}
\clearpage

%% -----File: 008.png---Folio iv-------

\iffalse
\null
\vspace{0pt plus 2fil}

\begin{center}
\textbf{\Mygoth{Cambridge}}:
\medskip

\textsc{\scriptsize PRINTED BY C. J. CLAY, M.A., AND SONS,}
\medskip

\textsc{\scriptsize AT THE UNIVERSITY PRESS.}
\end{center}
\vfil
\clearpage
\fi


\begin{center}
\begin{minipage}{\textwidth}
\begin{PGtext}
Produced by Joshua Hutchinson, Andrew D. Hwang, Carolyn
Bottomley and the Online Distributed Proofreading Team at
http://www.pgdp.net (This ebook was produced from images
from the Cornell University Library: Historical Mathematics
Monographs collection.)
\end{PGtext}
\end{minipage}
\end{center}
\vfill

\begin{minipage}{0.8\textwidth}
\small
\SectionTitle{\TransNote}

\raggedright
% Behavior of \TransNoteText depends on value of ForPrinting.
% If true, explains that page numbers et. al. are hyperlinks.
% In either case, explains how to recompile to change the option.
\TransNoteText

\end{minipage}
\null
\vspace{0pt plus 2fill}

\begin{center}
\textbf{\Mygoth{Cambridge}}:
\medskip

\textsc{\scriptsize PRINTED BY C.~J.~CLAY, M.A., AND SONS,}
\medskip

\textsc{\scriptsize AT THE UNIVERSITY PRESS.}
\end{center}
\cleardoublepage


%% -----File: 009.png---Folio v-------

\pagenumbering{roman}

\Preface

The present publication is an essay that was sent in
(December,~1887) to compete for the Smith's Prizes at Cambridge.

To the onlooker it is always a mournful thing to see what
he considers splendid abilities or opportunities wasted for lack of
knowledge of some paltry common-place truth. Such is in the
main my feeling when considering the neglect of the study of
Quaternions by that wonderful corporation the University of Cambridge.
To the alumnus she is apt to appear as the leader in all
branches of Mathematics. To the outsider she appears rather as
the leader in \emph{Applied} Mathematics and as a ready welcomer of
other branches.

If Quaternions were simply a branch of Pure Mathematics we
could understand why the study of them was almost confined
to the University which gave birth to them, but as the truth is
quite otherwise it is hard to shew good reason why they have not
struck root also in Cambridge. The prophet on whom Hamilton's
mantle has fallen is more than a mathematician and more than a
natural philosopher---he is both, and it is to be noted also that he
is a Cambridge man. He has preached in season and out of
season (if that were possible) that Quaternions are especially useful
in Physical applications. Why then has his Alma Mater turned a
deaf ear? I cannot believe that she is in her dotage and has lost
her hearing. The problem is beyond me and I give it up.

But I wish to add my little efforts to Prof.~Tait's powerful
advocacy to bring about another state of affairs. Cambridge is the
prepared ground on which if anywhere the study of the Physical
applications of Quaternions ought to flourish.

%% -----File: 010.png---Folio vi-------

When I sent in the essay I had a faint misgiving that perchance
there was not a single man in Cambridge who could
understand it without much labour---and yet it is a straightforward
application of Hamilton's principles. I cannot say what
transformation scene has taken place in the five years that have
elapsed, but an encouraging fact is that one professor at any rate
has been imported from Dublin.

There is no lack in Cambridge of the cultivation of Quaternions
\emph{as an algebra}, but this cultivation is not Hamiltonian,
though an evidence of the great fecundity of Hamilton's work.
Hamilton looked upon Quaternions as a \emph{geometrical} method, and it
is in this respect that he has as yet failed to find worthy followers
resident in Cambridge. [The chapter contributed by Prof.~Cayley
to Prof.~Tait's 3rd~ed.\ of `Quaternions' deals with quite a different
subject from the rest of the treatise, a subject that deserves a
distinctive name, say, \emph{Cayleyan Quaternions}.]

I have delayed for a considerable time the present publication
in order at the last if possible to make it more effective. I have
waited till I could by a more striking example than any in the
essay shew the immense utility of Quaternions in the regions in
which I believe them to be especially powerful. This I believe
has been done in the `Phil.~Trans.'\ 1892, p.~685. Certainly on
two occasions copious extracts have been published, viz.\ in the
P. R. S. E., 1890--1, p.~98, and in the `Phil.~Mag.' June~1892,
p.~477, but the reasons are simple. The first was published after the
subject of the `Phil.~Trans.'\ paper had been considered sufficiently
to afford clear daylight ahead in that direction, and the second
after that paper had actually been despatched for publication.

At the time of writing the essay I possessed little more than
faith in the potentiality of Quaternions, and I felt that something
more than faith was needed to convince scientists. It was thought
that rather than publish in driblets it were better to wait for
a more copious shower on the principle that a well-directed heavy
blow is more effective than a long-continued series of little
pushes.

Perhaps more harm has been done than by any other cause to
the study of Quaternions in their Physical applications by a silly
superstition with which the nurses of Cambridge are wont to
%% -----File: 011.png---Folio vii-------
frighten their too timorous charges. This is the belief that the
subject of Quaternions is difficult. It is difficult in one sense and
in no other, in that sense in which the subject of analytical conics
is difficult to the schoolboy, in the sense in which every subject
is difficult whose fundamental ideas and methods are different
from any the student has hitherto been introduced to. The only
way to convince the nurses that Quaternions form a healthy diet
for the \emph{young} mathematician is to prove to them that they will
``pay'' in the first part of the Tripos. Of course this is an impossible
task while the only questions set in the Tripos on the subject
are in the second part and average one in two years. [This
solitary biennial question is rarely if ever anything but an exercise
in \emph{algebra}. The very form in which candidates are invited, or at
any rate were in my day, to study Quaternions is an insult to the
memory of Hamilton. The monstrosity ``Quaternions and other
% [** PP: Changing ``paralleled'' to ``parallelled'']
non-commutative algebras'' can only be parallelled by ``Cartesian
Geometry and other commutative algebras.'' When I was in Cambridge
it was currently reported that if an answer to a Mathematical
Tripos question were couched in Hebrew the candidate
would or would not get credit for the answer according as one
or more of the examiners did or did not understand Hebrew,
and that in this respect Hebrew or Quaternions were strictly
analogous.]

Is it hopeless to appeal to the charges? I will try. Let me
suppose that some budding Cambridge Mathematician has followed
me so far. I now address myself to him. Have you ever
felt a joy in Mathematics? Probably you have, but it was before
your schoolmasters had found you out and resolved to fashion you
into an examinee. Even now you occasionally have feelings like
the dimly remembered ones. Now and then you forget that you
are nerving yourself for that Juggernaut the Tripos. Let me
implore you as though your soul's salvation depended on it to let
these trances run their utmost course in spite of solemn warnings
from your nurse. You will in time be rewarded by a soul-thrilling
dream whose subject is the Universe and whose organ to look
upon the Universe withal is the sense called Quaternions. Steep
yourself in the delirious pleasures. When you wake you will have
forgotten the Tripos and in the fulness of time will develop into a
%% -----File: 012.png---Folio viii-------
financial wreck, but in possession of the memory of that heaven-sent
dream you will be a far happier and richer man than the
millionest millionaire.

To pass to earth---from the few papers I have published it
will be evident that the subject treated of here is one I have very
much at heart, and I think that the publication of the essay is
likely to conduce to an acceptance of the view that \emph{it is now the
duty of mathematical physicists to study Quaternions seriously}.
I have been told by more than one of the few who have read some
of my papers that they prove rather stiff reading. The reasons
for this are not in the papers I believe but in matters which have
already been indicated. Now the present essay reproduces the
order in which the subject was developed in my own mind. The
less complete treatment of a subject, especially if more diffuse, is
often easier to follow than the finished product. It is therefore
probable that the present essay is likely to prove more easy
reading than my other papers.

Moreover I wish it to be studied by a class of readers who are
not in the habit of consulting the proceedings,~\&c., of learned
societies. I want the slaves of examination to be arrested and to
read, for it is apparently to the rising generation that we must
look to wipe off the blot from the escutcheon of Cambridge.

And now as to the essay itself. But one real alteration has
been made. A passage has been suppressed in which were made
uncomplimentary remarks concerning a certain author for what
the writer regards as his abuse of Quaternion methods. The
author in question would no doubt have been perfectly well able
to take care of himself, so that perhaps there was no very good
reason for suppressing the passage as it still represents my convictions,
but I did not want a side issue to be raised that would
serve to distract attention from the main one. To bring the
notation into harmony with my later papers $d\nu$ and~$\nabla'$ which
occur in the manuscript have been changed throughout to $d\Sigma$
and~$\Delta$ respectively. To facilitate printing the solidus has been freely
introduced and the vinculum abjured. Mere slips of the pen
have been corrected. A formal prefatory note required by the
conditions of competition has been omitted. The Table of Contents
was not prefixed to the original essay. It consists of little
%% -----File: 013.png---Folio ix-------
more than a collection of the headings scattered through the
essay. Several notes have been added, all indicated by square
brackets and the date (1892 or~1893). Otherwise the essay
remains absolutely unaltered. The name originally given to the
essay is at the head of p.~1 below. The name on the title-page
is adopted to prevent confusion of the essay with the `Phil.~Mag.',
% [** PP: Changing ``caligraphy'' to ``calligraphy'']
paper referred to above. What in the peculiar calligraphy of the
manuscript was meant for the familiar $\smalliiint()\,d\varsigma$ has been consistently
rendered by the printer as $\smalliiint()\,\ds$. As the mental operation of
substituting the former for the latter is not laborious I have not
thought it necessary to make the requisite extensive alterations in
the proofs.
\bigskip

I wish here to express my great indebtedness to Prof.~Tait, not
only for having through his published works given me such
knowledge of Quaternions as I possess but for giving me private
encouragement at a time I sorely needed it. There was a time
when I felt tempted to throw my convictions to the winds and
follow the line of least resistance. To break down the solid and
well-nigh universal scepticism as to the utility of Quaternions in
Physics seemed too much like casting one's pearls---at least like
crying in the wilderness.

But though I recognise that I am fighting under Prof.~Tait's
banner, yet, as every subaltern could have conducted a campaign
better than his general, so in some details I feel compelled to
differ from Professor Tait. Some two or three years ago he was
good enough to read the present essay. He somewhat severely
criticised certain points but did not convince me on all.

Among other things he pointed out that I sprung on the
unsuspicious reader without due warning and explanation what
may be considered as a peculiarity in symbolisation. I take this
opportunity therefore of remedying the omission. In Quaternions
on account of the non-commutative nature of multiplication we
have not the same unlimited choice of order of the terms in a
product as we have in ordinary algebra, and the same is true of
certain quaternion operators. It is thus inconvenient in many
cases to use the familiar method of indicating the connection
between an operator and its operand by placing the former
%% -----File: 014.png---Folio x-------
immediately before the latter. Another method is adopted. With
this other method the operator may be separated from the operand,
but it seems that there has been a tacit convention among users
of this method that the separated operator is still to be restricted
to precedence of the operand. There is of course nothing in the
nature of things why this should be so, though its violation may
seem a trifle strange at first, just as the tyro in Latin is puzzled
by the unexpected corners of a sentence in which adjectives
(operators) and their nouns (operands) turn up. Indeed a
Roman may be said to have anticipated in every detail the
method of indicating the connection now under discussion, for he
did so by the similarity of the suffixes of his operators and
operands. In this essay his example is followed and therefore no
restrictions except such as result from the genius of the language
(the laws of Quaternions) are placed on the relative positions in a
product of operators and operands. With this warning the reader
ought to find no difficulty.

One of Prof.~Tait's criticisms already alluded to appears in the
third edition of his `Quaternions.' The process held up in \S~500
of this edition as an example of ``how not to do it'' is contained
in \Secref{6} below and was first given in the `Mess.\ of Math.,' 1884.
He implies that the process is a ``most intensely artificial application
of'' Quaternions. If this were true I should consider it a
perfectly legitimate criticism, but I hold that it is the exact
reverse of the truth. In the course of Physical investigations
certain volume integrals are found to be capable of, or by general
considerations are obviously capable of transformation into surface
integrals. We are led to seek for the correct expression in the
latter form. Starting from this we can by a long, and in my
opinion, tedious process arrive at the most general type of volume
integral which is capable of transformation into a surface integral.
[I may remark in passing that Prof.~Tait did not however arrive
at quite the most general type.] Does it follow that this is the
most natural course of procedure? Certainly not, as I think. It
would be the most natural course for the empiricist, but not for
the scientist. When he has been introduced to one or two volume
integrals capable of the transformation the \emph{natural} course of the
mathematician is to ask himself what is the most general volume
%% -----File: 015.png---Folio xi-------
integral of the kind. By quite elementary considerations he sees
that while only such volume integrals as satisfy certain conditions
are transformable into surface integrals, yet \emph{any} surface integral
which is continuous and applies to the complete boundary of any
finite volume can be expressed as a volume integral throughout
that volume. He is thus led to start from the surface integral
and deduces by the briefest of processes the most general volume
integral of the type required. Needless to say, when giving his
demonstration he does not bare his soul in this way. He thinks
rightly that any mathematician can at once divine the exact road
he has followed. Where is the artificiality?

Let me in conclusion say that even now I scarcely dare state
what I believe to be the proper place of Quaternions in a
Physical education, for fear my statements be regarded as the
uninspired babblings of a misdirected enthusiast, but I cannot
refrain from saying that I look forward to the time when
Quaternions will appear in every Physical text-book that assumes
the knowledge of (say) elementary plane trigonometry.

I am much indebted to Mr~G.~H.~A.~Wilson of Clare College,
Cambridge, for helping me in the revision of the proofs, and take
this opportunity of thanking him for the time and trouble he has
devoted to the work.

\bigskip

\null\hfill \textsc{ALEX. \McAulay.}\qquad
\bigskip

\textsc{\footnotesize University of Tasmania,}

\qquad\textsc{\footnotesize Hobart.}

\qquad\qquad{\footnotesize \emph{March}~26, 1893.}
\cleardoublepage

%% -----File: 016.png---Folio xii-------
%[Blank Page]
%% -----File: 017.png---Folio xiii-------

% [*** Page number on first ToC page]
\pagestyle{fancy}
\fancyhead{}
\fancyfoot{}
\renewcommand{\headrulewidth}{0pt}
\fancyhead[C]{{\scshape\small contents.}}
\ifthenelse{\boolean{ForPrinting}}
 {\fancyhead[RO,LE]{\thepage}}
 {\fancyhead[R]{\thepage}}

\pdfbookmark[0]{Table of Contents}{Table of Contents}
\thispagestyle{empty}

\tableofcontents

\cleardoublepage

\iffalse
%% Scanned text from book's ToC
CONTENTS.

/*
SECTION I. INTRODUCTION.

PAGE

General remarks on the place of Quaternions in Physics      1

Cartesian form of some of the results to follow      6


SECTION II. QUATERNION THEOREMS.

1. Definitions      12

2. Properties of $\zeta$      15

5. Fundamental property of~$\Dop$      18

6. Theorems in Integration      19

9. Potentials      22


SECTION III. ELASTIC SOLIDS.

11. Brief recapitulation of previous work in this branch      25

12. Strain, stress-force, stress-couple      26

14. Stress in terms of strain      27

16. The equations of equilibrium      32

16a. Variation of temperature      35

17. Small strains      37

20. Isotropic bodies      40

22. Particular integral of the equation of equilibrium      42

24. Orthogonal coordinates      44

27. Saint-Venant's torsion problem      46

29. Wires      49


SECTION IV. ELECTRICITY AND MAGNETISM.

34. ELECTROSTATICS---general problem      55

41. The force in particular cases      63

43. Nature of the stress      65

46. MAGNETISM---magnetic potential, force, induction      67

49. Magnetic solenoids and shells      69

54. ELECTRO-MAGNETISM---general theory      71

60. Electro-magnetic stress      74
*/

%% -----File: 018.png---Folio xiv-------
/*
SECTION V. HYDRODYNAMICS.

PAGE

61. Preliminary      76

62. Notation      76

63. Euler's equations      77

68. The Lagrangian equations      80

69. Cauchy's integrals of these equations      81

71. Flow, circulation, vortex-motion      82

74. Irrotational motion      84

76. Motion of a solid through a liquid      85

79. The velocity in terms of the convergences and spins      88

83. Viscosity      91


SECTION VI. THE VORTEX-ATOM THEORY.

85. Preliminary      94

86. Statement of Sir Wm. Thomson's and Prof. Hicks's theories      94

87. General considerations concerning these theories      95

88. Description of the method here adopted      96

89. Acceleration in terms of the convergences, their time-fluxes and the
spins      97

91. Sir Wm. Thomson's theory      98

93. Prof. Hicks's theory      100

94. Consideration of all the terms except $-\nabla(\sigma^2)/2$       101

96. Consideration of the term $-\nabla\centerdot(\sigma^2)/2$      103
*/
\fi
%% -----File: 019.png---Folio 1-------


\mainmatter
\pagenumbering{arabic}


\HeadChapter{Section I}{Introduction}
            {quaternions as a practical instrument \\ of physical research}


\Tocline{}{General remarks on the place of Quaternions in Physics}

It is a curious phenomenon in the History of Mathematics
that the greatest work of the greatest Mathematician of the
century which prides itself upon being the most enlightened the
world has yet seen, has suffered the most chilling neglect.

The cause of this is not at first sight obvious. We have here
little to do with the benefit provided by Quaternions to Pure
Mathematics. The reason for the neglect here \emph{may} be that
Hamilton himself has developed the Science to such an extent as
to make successors an impossibility. One cannot however resist
a strong suspicion that were the subject even studied we should
hear more from Pure Mathematicians, of Hamilton's valuable
results. This reason at any rate cannot be assigned for the
neglect of the Physical side of Quaternions. Hamilton has done
but little in this field, and yet when we ask what Mathematical
Physicists have been tempted by the bait to win easy laurels (to
put the incentive on no higher grounds), the answer must be
scarcely one. Prof.~Tait is the grand exception to this. But well-known
Physicist though he be, his fellow-workers for the most
part render themselves incapable of appreciating his valuable
services by studying the subject if at all only as dilettanti. The
number who read a small amount in Quaternions is by no means
small, but those who get further than what is recommended by
Maxwell as imperatively necessary are but a small percentage of
the whole.

I cannot help thinking that this state of affairs is owing chiefly
to a prejudice. This prejudice is well seen in Maxwell's well-known
statement---``I am convinced that the introduction of the
%% -----File: 020.png---Folio 2-------
ideas, \emph{as distinguished from the operations and methods} of Quaternions,
will be of great use to us in all parts of our
subject.''\footnote{\emph{Elect.\ and~Mag.} Vol.~I. \S~10.}
Now what I hold and what the main object of this essay is to
prove is that the ``operations and methods'' of Quaternions are as
much better qualified to deal with Physics than the ordinary ones
as are the ``ideas''.

But, what has produced this notion, that the subject of
Quaternions is only a pretty toy that has nothing to do with the
serious work of practical Physics? It must be the fact that it has
hitherto produced few results that appeal strongly to Physicists.
This I acknowledge, but that the deduction is correct I strongly
disbelieve. As well might an instrument of which nobody has
attempted to master the principles be blamed for not being of
much use. Workers naturally find themselves while still inexperienced
in the use of Quaternions incapable of clearly thinking
through them and of making them do the work of Cartesian Geometry,
and they conclude that Quaternions do not provide suitable
treatment for what they have in hand. The fact is that the subject
requires a slight development in order readily to apply to the
practical consideration of most physical subjects. The first steps
of this, which consist chiefly in the invention of new symbols of
operation and a slight examination of their chief properties, I
have endeavoured to give in the following pages.

I may now state what I hold to be the mission of Quaternions
to Physics. I believe that Physics would advance with both
more rapid and surer strides were Quaternions introduced to
serious study to the almost total exclusion of Cartesian Geometry,
except in an insignificant way as a particular case of
the former. All the geometrical processes occurring in Physical
\emph{theories} and \emph{general} Physical problems are much more graceful
in their Quaternion than in their Cartesian garb. To illustrate
what is here meant by ``theory'' and ``general problem''
let us take the case of Elasticity treated below. That by the
methods advocated not only are the already well-known results of
the general \emph{theory} of Elasticity better proved, but more general
results are obtained, will I think be acknowledged after a perusal
of \Secref{12} to~\Secref{21} below. That Quaternions are superior to Cartesian
Geometry in considering the \emph{general problems} of (1)~an infinite
isotropic solid, (2)~the torsion and bending of prisms and cylinders,
(3)~the general theory of wires, I have endeavoured to shew in
%% -----File: 021.png---Folio 3-------
\Secref{22}--\Secref{33}. But for \emph{particular} problems such as the torsion
problem for a cylinder of given shape, we require of course the
various theories specially constructed for the solution of particular
problems such as Fourier's theories, complex variables, spherical
harmonics,~\&c. It will thus be seen that I do not propose to
banish these theories but merely Cartesian Geometry.

So mistaken are the common notions concerning the pretensions
of advocates of Quaternions that I was asked by one well-known
Mathematician whether Quaternions furnished methods for
the solution of differential equations, as he asserted that this was
all that remained for Mathematics in the domain of Physics!
Quaternions can no more solve differential equations than Cartesian
Geometry, but the solution of such equations can be performed
as readily, in fact generally more so, in the Quaternion shape as in
the Cartesian. But that the sole work of Physical Mathematics
to-day is the solution of differential equations I beg to question.
There are many and important Physical questions and extensions
of Physical theories that have little or nothing to do with such
solutions. As witness I may call attention to the new Physical
work which occurs below.

If only on account of the extreme simplicity of Quaternion
notation, large advances in the parts of Physics now indicated, are
to be expected. Expressions which are far too cumbrous to be of
much use in the Cartesian shape become so simple when translated
into Quaternions, that they admit of easy interpretation and, what
is perhaps of more importance, of easy manipulation. Compare for
instance the particular case of \equationref{III}{15$m$} \Secref{16} below when
$\mathbf{F} = 0$ with the same thing as considered in Thomson and Tait's
\textit{Nat.~Phil.}, App.~C. The Quaternion equation is
\[
\rho'_1 S \nabla_{1\Psi}\Dop  w\Delta = 0.
\]

The Cartesian \emph{exact} equivalent consists of Thomson and Tait's
equations~(7), viz.
\begin{align*}
\frac{d}{dx}
&\left\{ 2\frac{dw}{dA} \left( \frac{d\alpha}{dx} + 1 \right)
        + \frac{dw}{db} \frac{d\alpha}{dz}
        + \frac{dw}{dc} \frac{d\alpha}{dy} \right\} \\
%
&{} + \frac{d}{dy}
 \left\{ 2\frac{dw}{dB} \frac{d\alpha}{dy}
        + \frac{dw}{da} \frac{d\alpha}{dz}
 + \frac{dw}{dc} \left( \frac{d\alpha}{dx} + 1 \right) \right\} \\
%
&{}+ \frac{d}{dz}
 \left\{ 2\frac{dw}{dC} \frac{d\alpha}{dz}
        + \frac{dw}{da} \frac{d\alpha}{dy}
 + \frac{dw}{db} \left( \frac{d\alpha}{dx} + 1 \right) \right\} = 0,
\end{align*}
and two similar equations.
%% -----File: 022.png---Folio 4-------

Many of the equations indeed in the part of the essay where
this occurs, although quite simple enough to be thoroughly useful
in their present form, lead to much more complicated equations
than those just given when translated into Cartesian notation.

It will thus be seen that there are two statements to make
good:---(1) that Quaternions are in such a stage of development
as already to justify the practically complete banishment of Cartesian
Geometry from Physical questions of a general nature, and
(2) that Quaternions will in Physics produce many \emph{new} results that
cannot be produced by the rival and older theory.

To establish completely the first of these propositions it would
be necessary to go over \emph{all} the ground covered by Mathematical
Physical Theories, by means of our present subject, and compare
the proofs with the ordinary ones. This of course is impossible in
an essay. It would require a treatise of no small dimensions.
But the principle can be followed to a small extent. I have therefore
taken three typical theories and applied Quaternions to most
of the general propositions in them. The three subjects are those
of Elastic Solids, with the thermodynamic considerations necessary,
Electricity and Magnetism, and Hydrodynamics. It is impossible
without greatly exceeding due limits of space to consider in addition,
Conduction of Heat, Acoustics, Physical Optics, and the
Kinetic Theory of Gases. With the exception of the first of these
subjects I do not profess even to have attempted hitherto the
desired applications, but one would seem almost justified in arguing
that, since Quaternions have been found so applicable to the
subjects considered, they are very likely to prove useful to about
the same extent in similar theories. Again, only in one of the
subjects chosen, viz., Hydrodynamics, have I given the \emph{whole} of
the general theory which usually appears in text-books. For
instance, in Electricity and Magnetism I have not considered Electric
Conduction in three dimensions which, as Maxwell remarks,
lends itself very readily to Quaternion treatment, nor Magnetic
Induction, nor the Electro-Magnetic Theory of Light. Again, I
have left out the consideration of Poynting's theories of Electricity
which are very beautifully treated by Quaternions, and I
felt much tempted to introduce some considerations in connection
with the Molecular Current theory of Magnetism. With similar
reluctance I have been compelled to omit many applications in
the Theory of Elastic Solids, but the already too large size of the
essay admitted of no other course. Notwithstanding these omissions,
%% -----File: 023.png---Folio 5-------
I think that what I have done in this part will go far to
bear out the truth of the first proposition I have stated above.

But it is the second that I would especially lay stress upon.
In the first it is merely stated that Cartesian Geometry is an antiquated
machine that ought to be thrown aside to make room for
modern improvements. But the second asserts that the improved
machinery will not only do the work of the old better, but will
also do much work that the old is quite incapable of doing at all.
Should this be satisfactorily established and should Physicists in
that case still refuse to have anything to do with Quaternions,
they would place themselves in the position of the traditional
workmen who so strongly objected to the introduction of machinery
to supplant manual labour.

But in a few months and synchronously with the work I have
already described, to arrive at a \emph{large} number of new results is too
much to expect even from such a subject as that now under discussion.
There are however some few such results to shew. I
have endeavoured to advance each of the theories chosen in at
least one direction. In the subject of Elastic Solids I have expressed
the stress in terms of the strain in the most general case,
i.e.\ where the strain is not small, where the ordinary assumption
of no stress-couple is not made and where no assumption is made
as to homogeneity, isotropy,~\&c. I have also obtained the equations
of motion when there is given an external force and couple per
unit volume of the unstrained solid. These two problems, as will
be seen, are by no means identical. In Electrostatics I have considered
the most general mechanical results flowing from Maxwell's
theory, and their explanation by stress in the dielectric. These
results are not known, as might be inferred from this mode of
statement, for to solve the problem we require to know forty-two
independent constants to express the properties of the dielectric at
a given state of strain at each point. These are the six coefficients
of specific inductive capacity and their thirty-six differential coefficients
with regard to the six coordinates of pure strain. But, as
far as I am aware, only such particular cases of this have already
been considered as make the forty-two constants reduce at most to
three. In Hydrodynamics I have endeavoured to deduce certain
general phenomena which would be exhibited by vortex-atoms
acting upon one another. This has been done by examination of
an equation which has not, I believe, been hitherto given. The
result of this part of the essay is to lead to a presumption against
%% -----File: 024.png---Folio 6-------
Sir William Thomson's Vortex-Atom Theory and in favour of
Hicks's.

As one of the objects of this introduction is to give a bird's-eye
view of the merits of Quaternions as opposed to Cartesian Geometry,
it will not be out of place to give side by side the Quaternion
and the Cartesian forms of most of the new results I have been
speaking about. It must be premised, as already hinted, that the
usefulness of these results must be judged not by the Cartesian
but by the Quaternion form.


\TCSection{}{Cartesian form of some of the results to follow}{Elasticity}

Let the point $(x, y, z)$ of an elastic solid be displaced to
$(x', y', z')$. The strain at any point that is caused may be supposed
due to a pure strain followed by a rotation. In Section III. below,
this pure strain is called $\psi$. Let its coordinates be $e$, $f$, $g$,
$a/2$, $b/2$, $c/2$; i.e.\ if the vector $(\xi, \eta, \zeta)$ becomes $(\xi', \eta', \zeta')$ by means
of the pure strain, then
\begin{gather*}
\xi' = e\xi + \tfrac{1}{2} c\eta + \tfrac{1}{2} b\zeta,  \\
\text{ \&c., \&c.}
\end{gather*}
Thus when the strain is small $e$, $f$, $g$ reduce to Thomson and Tait's
$1+e$, $1+f$, $1+g$ and $a$, $b$, $c$ are the same both in their case and
the present one. Now let the coordinates of $\Psi$, \Secref{16} below, be
$E$, $F$, $G$, $A/2$, $B/2$, $C/2$. \Equationref{III}{15}, \Secref{16} below, viz.
\[
\footnotemark\Psi\omega = \psi^2 \omega = \chi'\chi\omega
= \nabla_1 S \rho'_1 \rho'_2 S\omega\nabla_2,
\]
gives in our present notation
\Erratum{err6a}%[** PP: Break second equation at equals sign]
\begin{align*}
E &= e^2 + c^2/4 + b^2/4 = (dx'/dx)^2 + (dy'/dx)^2 + (dz'/dx)^2, \\
& \text{\qquad\qquad\qquad\qquad\qquad\&c., \&c.} \\
%
A &= a(f+g) + bc/2 \\ % [** PP: Re-breaking]
  &= 2\bigl\{(dx'/dy)(dx'/dz) + (dy'/dy)(dy'/dz) + (dz'/dy)(dz'/dz)\bigr\}, \\
& \text{\qquad\qquad\qquad\qquad\qquad\&c., \&c.}
\end{align*}
which shew that the present $E$, $F$, $G$, $A/2$, $B/2$, $C/2$ are the
$A$, $B$, $C$, $a$, $b$, $c$ of Thomson and Tait's \textit{Nat.\ Phil.}, App.~C.
\footnotetext{This result is one of Tait's (\textit{Quaternions} \S~365 where he has $\phi'\phi = \overline{\varpi}^2$). It is
given here for completeness.}
% end of footnote

%% -----File: 025.png---Folio 7-------

Let us put
\begin{align*}
J
\begin{pmatrix}
  x' y' z' \\
  x\, y\, z
\end{pmatrix} &= J \\
%
J
\begin{pmatrix}
  y' z' \\
  y\, z
\end{pmatrix}\phantom{z} &=J_{11},\quad \text{\&c., \&c.,} \\
%
J
\begin{pmatrix}
  z' x' \\
  y\, z
\end{pmatrix}\phantom{z} &= J_{12},
J
\begin{pmatrix}
  y' z' \\
  z\, x
\end{pmatrix} = J_{21},\quad \text{\&c., \&c., \&c., \&c.}
\end{align*}

I have shewn in \Secref{14} below that the \emph{stress-couple is quite
independent of the strain}. Thus we may consider the stress to
consist of two parts---an ordinary stress $PQRSTU$ as in Thomson
and Tait's \emph{Nat.~Phil.} and a stress which causes a couple per unit
volume $L' M' N'$. The former only of these will depend on strain.
The result of the two will be to cause a force (as indeed can be
seen from the expressions in \Secref{13} below) per unit area on the $x$-interface
$P$, $U+ N'/2$, $T-M'/2$, and so for the other interfaces. If
$L$, $M$, $N$ be the external couple per unit volume of the \emph{unstrained}
solid we shall have
\[
L' = -L/J, \quad
M' = -M/J, \quad
N' = -N/J,
\]
for the external couple and the stress-couple are \emph{always} equal and
opposite. Thus the force on the $x$-interface becomes
\[
P, \quad U-N/2J, \quad T+M/2J
\]
and similarly for the other interfaces.

To express the part of the stress ($P$~\&c.)\ which depends on
the strain in terms of that strain, consider $w$ the potential energy
per unit volume of the unstrained solid as a function of $E$~\&c.
In the general thermodynamic case $w$ may be defined by saying
that
\Erratum{err7a}% [** PP: Displaying this equation for readability]
\begin{align*}
w &\times (\text{\small the element of volume}) \\
&= (\text{the intrinsic energy of the element}) \\
&\quad-  (\text{\small the entropy of the element}
   \times \text{\small its absolute temperature}
   \times \text{\small Joule's coefficient}).
\end{align*}
Of course~$w$ may be, and indeed is in
\Secref{14}, \Secref{15} below, regarded as a function of~$e$~\&c.

The equation for stress is~\Eref{III}{15$b$} \Secref{16} below, viz.,
\[
J\overline{\phi}\omega
= 2\chi \Pre{\Psi}{\Dop}  w\chi'\omega
= 2\rho_1' S \rho_2'\omega S\nabla_{1\Psi}\Dop w\nabla_2.
\]

The second of the expressions is in terms of the strain and the
third in terms of the displacement and its derivatives. In our
present notation this last is
%% -----File: 026.png---Folio 8-------
\begin{align*}
\frac{JP}{2}
&=
   \left(\frac{dx'}{dx}\right)^2 \frac{dw}{dE}
 + \left(\frac{dx'}{dy}\right)^2 \frac{dw}{dF}
 + \left(\frac{dx'}{dz}\right)^2 \frac{dw}{dG}  \\[1ex]
& \quad
+ 2\frac{dx'}{dy} \frac{dx'}{dz} \frac{dw}{dA}
+ 2\frac{dx'}{dz} \frac{dx'}{dx} \frac{dw}{dB}
+ 2\frac{dx'}{dx} \frac{dx'}{dy} \frac{dw}{dC}, \\
&\qquad\qquad\qquad\qquad\qquad\text{\&c., \&c.} \\
\frac{JS}{2} &=
   \frac{dy'}{dx} \frac{dz'}{dx} \frac{dw}{dE}
 + \frac{dy'}{dy} \frac{dz'}{dy} \frac{dw}{dF}
 + \frac{dy'}{dz} \frac{dz'}{dz} \frac{dw}{dG}  \\[1ex]
& \quad
+ \left(\frac{dy'}{dy} \frac{dz'}{dz}
+ \frac{dy'}{dz} \frac{dz'}{dy}\right)\frac{dw}{dA}
+ \left(\frac{dy'}{dz} \frac{dz'}{dx}
+ \frac{dy'}{dx} \frac{dz'}{dz}\right)\frac{dw}{dB} \\[1ex]
& \quad
% skip right by width of first summand...
\phantom{+ \left(\frac{dy'}{dx} \frac{dz'}{dy}
         + \frac{dy'}{dy} \frac{dz'}{dx}\right)\frac{dw}{dC}}
% plus a bit
\qquad
+ \left(\frac{dy'}{dx} \frac{dz'}{dy}
+ \frac{dy'}{dy} \frac{dz'}{dx}\right)\frac{dw}{dC}, \\
&\qquad\qquad\qquad\qquad\qquad\text{\&c., \&c.}
\end{align*}

In \Secref{14} I also obtain this part of the stress explicitly in terms
of $e$, $f$, $g$, $a$, $b$, $c$, of $w$ as a function of these quantities and of the
axis and amount of rotation. But these results are so very complicated in their Cartesian shape that it is quite useless to give
them.

To put down the equations of motion let $X_x$, $Y_x$, $Z_x$ be the
force due to stress on what before strain was unit area perpendicular to the axis of $x$. Similarly for $X_y$, \&c. Next suppose
that $X$, $Y$, $Z$ is the external force per unit volume of the unstrained
solid and let $D$ be the original density of the solid. Then
the equation of motion~\Eref{III}{15$n$} \Secref{16$a$} below, viz.\
\[
D\ddot{\rho}' = \mathbf{F} + \tau\Delta,
\]
gives in our present notation
\[
X + dX_x/dx + dX_y/dy + dX_z/dz = \ddot{x}'D,\quad \text{\&c., \&c.}
\]

It remains to express $X_x$~\&c.\ in terms of the displacement
and~$LMN$. This is done in \equationref{III}{15$l$} \Secref{16} below, viz.\
\[
\tau\omega
= -2\rho_1' S \nabla_{1\Psi}\Dop w \omega
  + 3V\mathbf{M} V\rho_1' \rho_2'
    S\omega\nabla_1 \nabla_2 /2 S\nabla_1 \nabla_2 \nabla_3
    S\rho_1'\rho_2'\rho_3'.\footnotemark
\]
\footnotetext{The second term on the right contains \emph{in full} the nine terms corresponding to
$(J_{12}N-J_{13}M)/2J$. Quaternion notation is therefore here, as in nearly all cases
which occur in Physics, considerably more compact even than the notations of
determinants or Jacobians.}

In our present notation this consists of the following nine
equations:
%% -----File: 027.png---Folio 9-------
\begin{align*}
X_x &= 2\left(
   \frac{dw}{dE} \frac{dx'}{dx}
 + \frac{dw}{dC} \frac{dx'}{dy}
 + \frac{dw}{dB} \frac{dx'}{dz}\right) + \frac{J_{12}N - J_{13}M}{2J}, \\
%
Y_x &= 2\left(
   \frac{dw}{dE} \frac{dy'}{dx}
 + \frac{dw}{dC} \frac{dy'}{dy}
 + \frac{dw}{dB} \frac{dy'}{dz}\right) + \frac{J_{13}L - J_{11}N}{2J}, \\
%
Z_x &= 2\left(
   \frac{dw}{dE} \frac{dz'}{dx}
 + \frac{dw}{dC} \frac{dz'}{dy}
 + \frac{dw}{dB} \frac{dz'}{dz}\right) + \frac{J_{11}M - J_{12}L}{2J}, \\
\end{align*}
and six similar equations.

We thus see that in the case where $LMN$ are zero, our present
$X_x$, $X_y$, $X_z$ are the $PQR$ of Thomson and Tait's \emph{Nat.~Phil}.\ App.~C
(d), and therefore equations~(7) of that article agree with our
equations of motion when we put both the external force and the
acceleration zero.

These are some of the new results in Elasticity, but, as I have
hinted, there are others in \Secref{14}, \Secref{15} which it would be waste of
time to give in their Cartesian form.


\SectionTitle{Electricity}

In Section~IV. below I have considered, as already stated, the
most general mechanical results flowing from Maxwell's theory of
Electrostatics. I have shewn that here, as in the particular cases
considered by others, the forces, whether per unit volume or per
unit surface, can be explained by a stress in the dielectric. It is
easiest to describe these forces by means of the stress.

Let the coordinates of the stress be $PQRSTU$. Then $F_1 F_2 F_3$
the mechanical force, due to the field per unit volume, exerted
upon the dielectric where there is no discontinuity in the stress, is
given by
\[
F_1 = dP/dx + dU/dy + dT/dz,\quad \text{\&c., \&c.}
\]
and $(l, m, n)$ being the direction cosines of the normal to any
surface, pointing away from the region considered
\[
F_1' = - [lP + mU + nT]_a - [\ ]_b,\quad \text{\&c., \&c.,}
\]
where $a$, $b$ indicate the two sides of the surface and ${F_1}'$, ${F_2}'$, ${F_3}'$ is
the force due to the field per unit surface.

It remains to find $P$~\&c. Let $X$, $Y$, $Z$ be the electro-motive
force, $\alpha$, $\beta$, $\gamma$ the displacement, $w$ the potential energy per unit
volume and $K_{xx}$, $K_{yy}$, $K_{zz}$, $K_{yz}$, $K_{zx}$, $K_{xy}$  the coefficients of specific
%% -----File: 028.png---Folio 10-------
inductive capacity. Let $1+e$, $1+f$, $1+g$, $a/2$, $b/2$, $c/2$ denote the
pure part of the strain of the medium. The $K$'s will then be
functions of $e$~\&c.\ and we must suppose these functions known, or
at any rate we must assume the knowledge of both the values of
the $K$'s and their differential coefficients at the particular state of
strain in which the medium is when under consideration. The
relations between the above quantities are
\begin{gather*}
4\pi\alpha = K_{xx}X + K_{xy}Y + K_{zx}Z,\quad \text{\&c., \&c.} \\
\begin{aligned}
w &= (X\alpha + Y\beta + Z\gamma)/2 \\
  &= (K_{xx} X^2 + K_{yy} Y^2 + K_{zz} Z^2
   + 2K_{yz} YZ + 2K_{zx} ZX + 2K_{xy} XY)/8\pi.
\end{aligned}
\end{gather*}

It is the second of these expressions for~$w$ which is assumed
below, and the differentiations of course refer only to the~$K$'s.
The equation expressing $P$~\&c.\ in terms of the field is~\Eref{IV}{21} \Secref{40}
below, viz.\
\[
\phi\omega
   = -\tfrac{1}{2} V\mathbf{D}\omega\mathbf{E} - \Pre{\Psi}{\Dop} w\,\omega,
\]
which in our present notation gives the following six equations
\begin{align*}
P &= -\tfrac{1}{2}(-\alpha X + \beta Y + \gamma Z)
         - dw/de,\quad \text{\&c., \&c.}, \\
S &= \tfrac{1}{2} (\beta Z + \gamma Y)
         - dw/da,\quad \text{\&c., \&c.}
\end{align*}

I have shewn in \Secref{41}--\Secref{45} below that these results agree with
particular results obtained by others.


\SectionTitle{Hydrodynamics}

The new work in this subject is given in Section~VI.---``The
Vortex-Atom Theory.'' It is quite unnecessary to translate the
various expressions there used into the Cartesian form. I give
here only the principal equation in its two chief forms, \equationref{VI}{9}
\Secref{89} and \equationref{VI}{11} \Secref{90}, viz.\
\begin{align*}
& P+v - \sigma^2/2 + (4\pi)^{-1} \smalliiint
   (S\sigma\tau \nabla u + u\partial m/\partial t)\, \ds = H, \\
& P+v - \sigma^2/2 + (4\pi)^{-1} \smalliiint
   \{\ds S \nabla u (V\sigma\tau - m\sigma) + ud(m\ds)/dt\} = H.
\end{align*}

In Cartesian notation these are
\begin{align*}
&\int dp/\rho + V + q^2/2 \\
&\qquad - (4\pi)^{-1} \smalliiint
    \bigl\{2[(x'-x)(w\eta - v\zeta) + \cdots + \cdots]/r^3 \\
&\phantom{\qquad - (4\pi)^{-1} \smalliiint}
    + (\partial c/\partial t)/r \bigr\}\,dx'\,dy'\,dz' = H.\\
&\int dp/\rho + V + q^2/2 \\
&\qquad - (4\pi)^{-1} \smalliiint
    \bigl\{(x'-x)[2(w\eta - v\zeta) - cu] + \cdots + \cdots \bigr\}/r^3.
      \,dx'\,dy'\,dz' \\
&\qquad - (4\pi)^{-1} \smalliiint \bigl\{d(cdx'\,dy'\,dz')/dt\bigr\}/r = H.
\end{align*}

%% -----File: 029.png---Folio 11-------

The fluid here considered is one whose motion is continuous
from point to point and which extends to infinity. The volume
integral extends throughout space. The notation is as usual.
It is only necessary to say that~$H$ is a function of the time only,
$r$~is the distance between the points $x', y', z'$ and~$x, y, z$;
\[
c = du/dx + dv/dy + dw/dz;
\]
$d/dt$ is put for differentiation which follows a particle of the fluid,
and $\partial/\partial t$ for that which refers to a fixed point.

The explanation of the unusual length of this essay, which I
feel is called for, is contained in the foregoing description of its
objects. If the objects be justifiable, so must also be the length
which is a necessary outcome of those objects.

%% -----File: 030.png---Folio 12-------


\Chapter{Section II}{Quaternion Theorems}

\Section{1}{Definitions}

\Paragraph{1} As there are two or three symbols and terms which will be
in constant use in the following pages that are new or more general
in their signification than is usual, it is necessary to be perhaps
somewhat tediously minute in a few preliminary definitions and
explanations.

A \emph{function} of a variable in the following essay is to be understood
to mean anything which depends on the variable and which
can be subjected to mathematical operations; the variable itself
being anything capable of being represented by a mathematical
symbol. In Cartesian Geometry the variable is generally a single
scalar. In Quaternions on the other hand a general quaternion
variable is not infrequent, a variable which requires 4 scalars for
its specification, and similarly for the function. In both, however,
either the variable or the function may be a mere symbol of
operation. In the following essay we shall frequently have to
speak of variables and functions which are neither quaternions nor
mere symbols of operation. For instance~$K$ in \Secref{40} below requires
6~scalars to specify it, and it is a function of~$\psi$ which requires
6~scalars and~$\rho$ which requires 3~scalars. When in future the
expression ``any function'' is used it is always to be understood in
the general sense just explained.

We shall frequently have to deal with functions of many independent
vectors, and especially with functions which are linear in
each of the constituent vectors. These functions merely require
to be noticed but not defined.

Hamilton has defined the meaning of the symbolic vector $\nabla$
thus:---
\[
\nabla = i \frac{d}{dx} + j\frac{d}{dy} + k\frac{d}{dz},
\]
%% -----File: 031.png---Folio 13-------
where $i$, $j$, $k$ are unit vectors in the directions of the mutually
perpendicular axes $x$, $y$, $z$. I have found it necessary somewhat to
expand the meaning of this symbol. When a numerical suffix
$1$, $2$, \ldots\ is attached to a~$\nabla$ in any expression it is to indicate that the
differentiations implied in the~$\nabla$ are to refer to and only to other
symbols in the same expression which have the same suffix. After
the implied differentiations have been performed the suffixes are
of course removed. Thus $Q(\alpha, \beta, \gamma, \delta)$ being a quaternion function
of any four vectors $\alpha$, $\beta$, $\gamma$, $\delta$, linear in each
\begin{align*}
Q(\lambda_1 \mu_2 \nabla_1 \nabla_2)
& \equiv
   Q\left( \frac{d\lambda}{dx} \frac{d\mu}{dx} ii \right)
+  Q\left( \frac{d\lambda}{dy} \frac{d\mu}{dx} ji \right)
+  Q\left( \frac{d\lambda}{dz} \frac{d\mu}{dx} ki \right) \\
%
&+ Q\left( \frac{d\lambda}{dx} \frac{d\mu}{dy} ij \right)
+  Q\left( \frac{d\lambda}{dy} \frac{d\mu}{dy} jj \right)
+  Q\left( \frac{d\lambda}{dz} \frac{d\mu}{dy} kj \right) \\
%
&+ Q\left( \frac{d\lambda}{dx} \frac{d\mu}{dz} ik \right)
+  Q\left( \frac{d\lambda}{dy} \frac{d\mu}{dz} jk \right)
+  Q\left( \frac{d\lambda}{dz} \frac{d\mu}{dz} kk \right)
\end{align*}
and again
\[
Q_1(\lambda_1, \mu_2, \nabla_1, \nabla_2) \equiv
Q_3(\lambda_1, \mu_2, \nabla_1 + \nabla_3, \nabla_2).
\]

It is convenient to reserve the symbol~$\Delta$ for a special meaning.
It is to be regarded as a particular form of~$\nabla$, but its differentiations
are to refer to \emph{all} the variables in the term in which it
appears. Thus~$Q$ being as before
\[
Q(\lambda_1, \mu, \Delta, \nabla_1)
= Q_2(\lambda_1, \mu_2, \nabla_1 + \nabla_2, \nabla_1)\footnotemark.
\]
\footnotetext{%
These meanings for $\nabla_1, \nabla_2\ldots\Delta$ I used in a paper on ``Some General Theorems
in Quaternion Integration,'' in the \textit{Mess.\ of Math.}\ Vol.~\textsc{xiv}.\ (1884), p.~26. The
investigations there given are for the most part incorporated below. [Note added,
1892, see preface as to the alteration of $\nabla'$ into $\Delta$.]}

If in a linear expression or function $\nabla_1$ and~$\rho_1$ ($\rho$ being as usual
$\equiv ix +jy + kz$) occur once each they can be interchanged. Similarly
for $\nabla_2$ and~$\rho_2$. So often does this occur that I have thought
it advisable to use a separate symbol~$\zeta_1$ for each of the two $\nabla_1$
and~$\rho_1$, $\zeta_2$ for each of the two $\nabla_2$ and~$\rho_2$ and so for $\zeta_3$,~\&c. If only one
such pair occur there is of course no need for the suffix attached to
$\zeta$. Thus $\zeta$ may be looked upon as a symbolic vector or as a single
term put down instead of three. For $Q(\alpha, \beta)$ being linear in each
of the vectors $\alpha$, $\beta$
\[
\EqnTag{II}{1}
Q(\zeta, \zeta) = Q(\nabla_1, \rho_1)
                = Q(i, i) + Q(j, j) + Q(k, k).
\]

There is one more extension of the meaning of $\nabla$ to be given.
%% -----File: 032.png---Folio 14-------
$u$, $v$, $w$ being the rectangular coordinates of any vector $\sigma$, $\Nabla{\sigma}$ is
defined by the equation
\[
\Nabla{\sigma} = i\frac{d}{du} + j\frac{d}{dv} + k\frac{d}{dw}.
\]
To $\Nabla{\sigma}$ of course are to be attached, when necessary, the suffixes
above explained in connection with~$\nabla$. Moreover just as for
$\nabla_1$, $\rho_1$ we may put $\zeta$, $\zeta$ so also for
$\Nabla{\sigma}_1$, $\sigma_1$ may we put the same.

With these meanings one important result follows at once.
\emph{The $\nabla_1$'s, $\nabla_2$'s, \&c.,\ obey all the laws of ordinary vectors whether
with regard to multiplication or addition}, for the coordinates
$d/dx$, $d/dy$, $d/dz$ of any~$\nabla$ obey with the coordinates of any vector
or any other~$\nabla$ all the laws of common algebra.

Just as $\Nabla{\sigma}$ may be defined as a symbolic vector whose coordinates
are $d/du$, $d/dv$, $d/dw$ so $\phi$ being a linear vector function of
any vector whose coordinates are
\[
(a_1b_1c_1\; a_2b_2c_2\; a_3b_3c_3)\quad(\text{i.e.\ }
\phi i = a_1i + b_1j + c_1k,\quad \text{\&c.}).
\]
$\Pre{\phi}{\Dop}$\footnote{%
I have used an inverted~$D$ to indicate the analogy to Hamilton's inverted~$\Delta$.}
is defined as a symbolic linear vector function whose coordinates
are
\[
(d/da_1,\, d/db_1,\, d/dc_1,\
 d/da_2,\, d/db_2,\, d/dc_2,\
 d/da_3,\, d/db_3,\, d/dc_3),
\]
and to $\Pre{\phi}{\Dop}$ is to be applied exactly the same system of suffixes as
in the case of~$\nabla$. Thus $q$ being any quaternion function of~$\phi$, and~$\omega$
any vector
\begin{align*}
\Pre{\phi}{\Dop}_1\omega\centerdot q_1 =
& -(idq/da_1 + jdq/db_1 + kdq/dc_1) Si\omega  \\
& -(idq/da_2 + jdq/db_2 + kdq/dc_2) Sj\omega  \\
& -(idq/da_3 + jdq/db_3 + kdq/dc_3) Sk\omega.
\end{align*}

The same symbol $\Pre{\phi}{\Dop}$ is used without any inconvenience with a
slightly different meaning. If the independent variable~$\phi$ be a
\emph{self-conjugate} linear vector function it has only six coordinates.
If these are~$PQRSTU$ (i.e.\ $\phi i = Pi + Uj + Tk$,~\&c.)\ $\Pre{\phi}{\Dop}$ is defined
as a self-conjugate linear vector function whose coordinates are
\[
(d/dP,\, d/dQ,\, d/dR,\,
\tfrac{1}{2}d/dS,\, \tfrac{1}{2}d/dT,\, \tfrac{1}{2}d/dU).
\]

We shall frequently have to compare volume integrals with
integrals taken over the bounding surface of the volume, and
again surface integrals with integrals taken round the boundary
of the surface. For this purpose we shall use the following notations
for linear, surface and volume integrals respectively~$\int Q\,d\rho$,
%% -----File: 033.png---Folio 15-------
$\smalliint Q\,d\Sigma$, $\smalliiint Q\,\ds$ where~$Q$ is any function of the position of a point.
Here $d\rho$~is a vector element of the curve, $d\Sigma$ a vector element of
the surface, and~$\ds$ an element of volume. When comparisons
between line and surface integrals are made we take~$d\Sigma$ in such a
direction that~$d\rho$ is in the direction of positive rotation round the
element~$d\Sigma$ close to it. When comparisons between surface and
volume integrals are made $d\Sigma$ is always taken in the direction
\emph{away from} the volume which it bounds.


\Section{2}{Properties of \texorpdfstring{$\zeta$}{zeta}}

\Paragraph{2} The property of~$\zeta$ on which nearly all its usefulness
depends is that if~$\sigma$ be any vector
\[
\sigma = -\zeta S\zeta \sigma,
\]
which is given at once by \equationref{II}{1} of last section.

This gives a useful expression for the conjugate of a linear
vector function of a vector. Let~$\phi$ be the function and~$\omega$, $\tau$ any
two vectors. Then $\phi'$ denoting as usual the conjugate of~$\phi$ we
have
\[
S\omega\phi\tau = S\tau\phi'\omega,
\]
whence putting on the left $\tau =-\zeta S\zeta\tau$ we have
\[
S\tau(-\zeta S\omega\phi\zeta) = S\tau\phi'\omega,
\]
or since $\tau$ is quite arbitrary
\[
\EqnTag{II}{2}
\phi'\omega = -\zeta S\omega\phi\zeta.
\]

From this we at once deduce expressions for the pure part~$\overline{\phi}\omega$
and the rotational part~$V\epsilon\omega$ of~$\phi\omega$ by putting
\[
\EqnTag{II}{3}
\left.
\begin{aligned}
(\phi + \phi')\omega
    &= -\phi\zeta S\omega\zeta - \zeta S\omega\phi\zeta
         = 2\overline{\phi}\omega, \\
%
(\phi - \phi')\omega
    &= -\phi\zeta S\omega\zeta + \zeta S\omega\phi\zeta
         = VV\zeta\phi\zeta.\omega = 2V\epsilon\omega.
\end{aligned}\quad % [** PP: Need a bit of space before brace]
\right\}
\]
And all the other well-known relations between $\phi$ and~$\phi'$ are at
once given e.g.\ $S\zeta\phi\zeta = S\zeta\phi'\zeta$, i.e.\ the ``convergence'' of~$\phi =$ the
``convergence'' of~$\phi'$.

\Paragraph{3} Let $Q (\lambda, \mu)$ be any function of two vectors which is linear
in each. Then if~$\phi\omega$ be any linear vector function of a vector~$\omega$
given by
\[
\EqnTag{II}{4}
\left.
\begin{aligned}
&& &\phi\omega = -\Sigma\beta S\omega\alpha && \\
&\text{we have\qquad\qquad}&
& Q(\zeta, \phi\zeta) = \Sigma Q(-\zeta S\zeta\alpha,\beta)
  = \Sigma Q(\alpha,\beta), &&
\end{aligned}
\right\}
\]
%% -----File: 034.png---Folio 16-------
or more generally
\[
\EqnTag{II}{4$a$}
Q(\zeta, \phi\chi\zeta) = \Sigma Q(\chi'\alpha, \beta).
\]
To prove, it is only necessary to observe that
\begin{flalign*}
&&  \phi\chi\zeta = -\Sigma \beta S\alpha\chi\zeta
    &= -\Sigma \beta S\zeta\chi'\alpha,  &&  \\
&\text{and that}& -\zeta S\zeta\chi'\alpha &= \chi'\alpha.  &&
\end{flalign*}
As a particular case of \hyperref[eqn:II.4]{eq.~(4)} let~$\phi$ have the self-conjugate
value
\Erratum{err16a}% [** PP: Added period after first equation]
\[
\EqnTag{II}{5}
\left.
\begin{aligned}
&& &\phi\omega
    = -\tfrac{1}{2}\Sigma(\beta S\omega\alpha + \alpha S\omega\beta). && \\
&\text{Then}\qquad&
&Q(\zeta, \phi\zeta)
    = \tfrac{1}{2}\Sigma\{Q(\alpha,\beta) + Q(\beta,\alpha)\}, &&
\end{aligned}\right\}
\]
or if $Q(\lambda, \mu)$ is symmetrical in $\lambda$ and~$\mu$
\[
\EqnTag{II}{6}
Q(\zeta, \phi\zeta) = \Sigma Q(\alpha, \beta).
\]

The application we shall frequently make of this is to the case
when for~$\alpha$ we put~$\nabla_1$ and for~$\beta$, $\sigma_1$, where~$\sigma$ is any vector function
of the position of a point. In this case the first expression for~$\phi$
is the strain function and the second expression the pure strain
function resulting from a small displacement~$\sigma$ at every point.
As a simple particular case put $Q(\lambda, \mu) = S\lambda\mu$ so that $Q$ is symmetrical
in $\lambda$ and~$\mu$. Thus~$\phi$ being either of these functions
\[
S\zeta\phi\zeta = S\nabla\sigma.
\]

Another important equation is
\[
\EqnTag{II}{6$a$}
\footnotemark
Q(\zeta, \phi\zeta) = Q(\phi'\zeta, \zeta).
\]
\footnotetext{% [** PP: Keeping ``practice'']
[Note added, 1892. For practice it is convenient to remember this in words:---\emph{A
term in which $\zeta$ and~$\phi\zeta$ occur is unaltered in value by changing them into $\phi'\zeta$
and~$\zeta$ respectively.}]}

This is quite independent of the form of~$\phi$. To prove, observe
that by \equationref{II}{2}
\[
\phi'\zeta = -\zeta_1 S\zeta\phi\zeta_1,
\]
and that $-\zeta S\zeta\phi\zeta_1 = \phi\zeta_1$. Thus we get rid of~$\zeta$ and may now
drop the suffix of~$\zeta_1$ and so get \hyperref[eqn:II.6$a$]{eq.~(6$a$)}. [Notice that by means
of \Eref{II}{6$a$}, \Eref{II}{4$a$} may be deduced from~\Eref{II}{4}; for by~\Eref{II}{6$a$}
\[
Q(\zeta, \phi\chi\zeta) = Q(\chi'\zeta, \phi\zeta)
= \Sigma Q(\chi'\alpha, \beta)\quad \text{by~\Eref{II}{4}}.]
\]


\FNParagraph{3$a$}{3a} A more important result is the expression for $\phi^{-1}\omega$ in
terms of~$\phi$. We assume that
\[
S\phi\lambda\, \phi\mu\, \phi\nu = mS\lambda\mu\nu,
\]
where $\lambda$, $\mu$, $\nu$ are any three vectors and~$m$ is a scalar independent
%% -----File: 035.png---Folio 17-------
of these vectors. Substituting $\zeta_1$, $\zeta_2$, $\zeta_3$ for $\lambda$, $\mu$, $\nu$ and multiplying
by~$S\zeta_1\zeta_2\zeta_3$
\[
\EqnTag{II}{6$b$}
S\zeta_1\,\zeta_2\,\zeta_3\, S\phi\zeta_1\, \phi\zeta_2\, \phi\zeta_3 = 6m,
\]
which gives $m$ in terms of~$\phi$. That $S\zeta_1\,\zeta_2\,\zeta_3\, S\zeta_1\,\zeta_2\,\zeta_3 = 6$ is seen by
getting rid of each pair of $\zeta$'s in succession thus:---
\[
S\zeta_1\, (S\zeta_1\, V\zeta_2\,\zeta_3)\, \zeta_2\,\zeta_3
= -SV\zeta_2\zeta_3 \centerdot \zeta_2\zeta_3
= S(\zeta_2^2\zeta_3 - \zeta_2 S\zeta_2\,\zeta_3)\,\zeta_3
= -2 \zeta_3\zeta_3 = 6.
\]
Next observe that
\[
S\phi\omega\, \phi\zeta_1\, \phi\zeta_2 = mS\omega\,\zeta_1\,\zeta_2.
\]

Multiplying by $V\zeta_1\zeta_2$ and again on the right getting rid of the
$\zeta$s we have
\[
\EqnTag{II}{6$c$}
V\zeta_1\zeta_2\, S\phi\omega\, \phi\zeta_1\, \phi\zeta_2 = -2m\omega,
\]
whence from \equationref{II}{6$b$}
\[
\omega S\zeta_1\zeta_2\zeta_3\, S\phi\zeta_1\, \phi\zeta_2\, \phi\zeta_3
= -3V\zeta_1\zeta_2\, S\phi\omega\, \phi\zeta_1\, \phi\zeta_2,
\]
or changing $\omega$ into $\phi^{-1}\omega$
\[
\EqnTag{II}{6$d$}
\phi^{-1}\omega
= -\frac{3V\zeta_1\zeta_2\, S\omega\, \phi\zeta_1\, \phi\zeta_2}
{S\zeta_1\zeta_2\zeta_3\, S\phi\zeta_1\, \phi\zeta_2\, \phi\zeta_3}.
\]

By \equationref{II}{6$a$} of last section we can also put this in the form
\[
\EqnTag{II}{6$e$}
\phi^{-1}\omega
= -\frac{3V\phi'\zeta_1\, \phi'\zeta_2\, S\omega\zeta_1\zeta_2}
{S\zeta_1\zeta_2\zeta_3\, S\phi'\zeta_1\, \phi'\zeta_2\, \phi'\zeta_3},
\]
so that $\phi^{-1}\omega$ is obtained explicitly in terms of $\phi$ or~$\phi'$.

\Equationref{II}{6$c$} or~\Eref{II}{6$d$} can be put in another useful form which
is more analogous to the ordinary cubic and can be easily deduced
therefrom, or \footnotemark{}less easily from~\Eref{II}{6$d$}, viz.\
\Erratum{err6f}% [** PP: Re-breaking]
\[
\EqnTag{II}{6$f$}
\begin{split}
S\zeta_1\zeta_2\zeta_3
(\phi^3\omega\, S\zeta_1\zeta_2\zeta_3
&- 3\phi^2\omega\, S\zeta_1\zeta_2 \phi\zeta_3 \\
&+ 3\phi\omega\, S\zeta_1\, \phi\zeta_2\, \phi\zeta_3
- \omega S\phi\zeta_1\, \phi\zeta_2\, \phi\zeta_3) = 0.
\end{split}
\]
\footnotetext{[Note added, 1892. The cubic may be obtained in a more useful form from
the equation $\omega S\zeta_1\zeta_2\zeta_3\,
S\phi\zeta_1\, \phi\zeta_2\, \phi\zeta_3
= -3V\zeta_1\zeta_2\, S\phi\omega\, \phi\zeta_1\, \phi\zeta_2$ thus
\begin{align*}
V\zeta_1\zeta_2\, S\phi\omega\, \phi\zeta_1\, \phi\zeta_2
&= \phi\omega S \centerdot \phi\zeta_1 \phi\zeta_2 V\zeta_1\zeta_2
 - \phi\zeta_1 S \centerdot \phi\omega \phi\zeta_2 V\zeta_1\zeta_2
 + \phi\zeta_2 S \centerdot \phi\omega \phi\zeta_1 V\zeta_1\zeta_2 \\
%
&= \phi\omega S \centerdot \phi\zeta_1\, \phi\zeta_2\, V\zeta_1\zeta_2
 - 2\phi\zeta_1\, S \centerdot \phi\omega\, \phi\zeta_2\, V\zeta_1\zeta_2.
\end{align*}
%
\begin{flalign*}
&\text{\indent Again}&
\phi\zeta_1\, S \centerdot \phi\omega\, \phi\zeta_2\, V\zeta_1\zeta_2
&= \phi\zeta_1\, S \centerdot \phi\omega V
\centerdot \phi\zeta_2\, V\zeta_1\zeta_2 && \\ % [** PP: Adding line break]
&& &= \phi\zeta_1\, S\phi\omega (-\zeta_1 S\zeta_2\, \phi\zeta_2
+ \zeta_2\, S\zeta_1\, \phi\zeta_2) && \\
&& &= -\phi(\zeta_1\, S\zeta_1\, \phi\omega) S\zeta_2\, \phi\zeta_2
+ \phi(\zeta_1\, S\zeta_1\, \phi\zeta_2)\, S\zeta_2\, \phi\omega && \\
&& &= \phi^2\omega\, S\zeta\, \phi\zeta - \phi^2\zeta\, S\zeta\, \phi\omega
    = \phi^2\omega\, S\zeta\, \phi\zeta + \phi^3\omega. && \\
%
&\text{\indent Hence}&
&\phi^3\omega - m''\phi^2\omega + m'\phi\omega - m\omega = 0, && \\
&\text{where}&
6m &= S\zeta_1\,\zeta_2\,\zeta_3\, S\phi\zeta_1\, \phi\zeta_2\, \phi\zeta_3 && \\
&& 2m' &= -SV\zeta_1\,\zeta_2\, V\phi\zeta_1\, \phi\zeta_2 && \\
&& m'' &= -S\zeta\, \phi\zeta.] &&
\end{flalign*}} % [** end of footnote]

%% -----File: 036.png---Folio 18-------

As a useful particular case of \equationref{II}{6$d$} we may notice
that by \equationref{II}{4} \Secref{3} if
\Erratum{err18a}% [** PP: Added comma after next equation]
\begin{flalign*}
&&  \phi\omega &= -\sigma_1 S\omega\nabla_1,  && \\
%
\EqnTag{II}{6$g$}
&&  \phi^{-1}\omega
    &= -3V \nabla_1\nabla_2 S\omega\,  \sigma_1\sigma_2 /
          S\nabla_1\nabla_2\nabla_3\, S\sigma_1\sigma_2\sigma_3, && \\
%
\EqnTag{II}{6$h$}
&\text{and}&
    \phi'^{-1}\omega
    &= -3V \sigma_1\sigma_2\, S\omega\,\nabla_1\nabla_2 /
          S\nabla_1\nabla_2\nabla_3\, S\sigma_1\sigma_2\sigma_3. &&
\end{flalign*}

\Paragraph{4} Let $\phi$, $\psi$ be two linear vector functions of a vector.
Then if
\[
S\chi\zeta\, \phi\zeta = S\chi\zeta\, \psi\zeta,
\]
where $\chi$ is a \emph{quite arbitrary} linear vector function
\[
\phi \equiv \psi,
\]
for we may put $\chi\zeta = \tau S\zeta\omega$ where $\tau$ and~$\omega$ are arbitrary vectors, so
that
\begin{flalign*}
&&   S\tau\, \phi\omega &= S\tau\,\psi\omega,  &&  \\
&\text{or}&  \phi\omega &= \psi\omega.  &&
\end{flalign*}
Similarly\footnote{[Note added, 1892. The following slightly more general statement is a practically
much more convenient form of enunciation: if
$S\chi\zeta\, \phi\zeta =  S\chi\zeta\, \psi\zeta$, where~$\chi$ is a
perfectly arbitrary self-conjugate and $\phi$ and~$\psi$ are not necessarily self-conjugate
then $\overline{\phi} = \overline{\psi}$].}
if $\phi$ and~$\psi$ are both self-conjugate and $\chi$~is a quite
arbitrary self-conjugate linear vector function the same relation
holds as can be seen by putting
\[
\chi\zeta = \tau S\zeta\omega + \omega S\zeta\tau.
\]


\Section{5}{Fundamental Property of \texorpdfstring{$\protect\Dop$}{the D operator}}

\Paragraph{5} Just as the fundamental property of $\Nabla{\sigma}$ is that, $Q$~being
any function of~$\sigma$
\[
\delta Q = -Q_1 S\, \delta\sigma\, \Nabla{\sigma}_1,
\]
so we have a similar property of~$\Dop$. $Q$~being any function of~$\phi$ a
linear vector function
\[
\EqnTag{II}{7}
\delta Q = -Q_1 S\,\delta\phi\zeta\, \Pre{\phi}{\Dop}_1 \zeta.
\]
The property is proved in the same way as for~$\nabla$, viz.\ by expanding
$S\, \delta\phi\, \zeta \Pre{\phi}{\Dop}_1 \zeta$ in terms of the coordinates of $\Pre{\phi}{\Dop}$. First let~$\phi$
be not self-conjugate, and let its nine coordinates be
\[
(a_1 b_1 c_1\, a_2 b_2 c_2\, a_3 b_3 c_3).
\]
%% -----File: 037.png---Folio 19-------
Thus
\begin{align*}
   - Q_1\, S\,\delta\phi\,\zeta\,\Pre{\phi}{\Dop}_1 \zeta
&= - Q_1\, S\, \delta\phi i\, \Pre{\phi}{\Dop}_1 i
   - Q_1\, S\, \delta\phi j\, \Pre{\phi}{\Dop}_1 j
   - Q_1\, S\, \delta\phi k\, \Pre{\phi}{\Dop}_1 k, \\
&= \delta a_1\, dQ/da_1 + \delta b_1\, dQ/db_1 + \delta c_1\, dQ/dc_1, \\
&+ \delta a_2\, dQ/da_2 + \delta b_2\, dQ/db_2 + \delta c_2\, dQ/dc_2, \\
&+ \delta a_3\, dQ/da_3 + \delta b_3\, dQ/db_3 + \delta c_3\, dQ/dc_3
 = \delta Q.
\end{align*}
The proposition is exactly similarly proved when~$\phi$ is self-conjugate.


\Section{6}{Theorems in Integration}

\Paragraph{6} Referring back to \Secref{1} above for our notation for linear
surface and volume integrals we will now prove that if $Q$ be \emph{any}
linear function of a
vector\footnote{These two propositions are generalisations of what Tait and Hicks have from
time to time proved. They were first given in the present form by me in the
article already referred to in \Secref{1} above. In that paper the necessary references are
given.}
\begin{align*}
\EqnTag{II}{8}
\smallint Q\, d\rho   &=  \smalliint Q\, (V\,d\Sigma\, \Delta), \\
%
\EqnTag{II}{9}
\smalliint Q\, d\Sigma &= \smalliiint Q\Delta\, \ds.
\end{align*}

To prove the first divide the surface up into a series of
elementary parallelograms by two families of lines---one or more
members of one family coinciding with the given boundary,---apply
the line integral to the boundary of each parallelogram and
sum for the whole. The result will be the linear integral given in
\equationref{II}{8}. Let the sides of one such parallelogram taken in
order in the positive direction be $\alpha$, $\beta+\beta'$, $-\alpha-\alpha'$, $-\beta$; so that
$\alpha'$  and $\beta'$ are infinitely small compared with $\alpha$ and~$\beta$, and we have
the identical relation
\[
0 = \alpha + \beta + \beta' - \alpha -\alpha' - \beta
  = \beta' - \alpha'.
\]

The terms contributed to $\int Q\,d\rho$ by the sides $\alpha$ and $-\alpha - \alpha'$ will
be (neglecting terms of the third and higher orders of small
quantities)
\[
Q\alpha - Q\alpha - Q\alpha' + Q_1\alpha\, S\beta \nabla_1
= -Q\alpha' + Q_1\alpha\, S\beta\nabla_1.
\]
Similarly the terms given by the other two sides will be
\[
Q\beta' - Q_1\beta\, S \alpha \nabla_1
\]
%% -----File: 038.png---Folio 20-------
so that remembering that $\beta' - \alpha' = 0$ and therefore $Q\beta' - Q\alpha' = 0$
we have for the whole boundary of the parallelogram
\[
Q_1 \alpha\, S\beta\nabla_1 - Q_1 \beta\, S\alpha\nabla_1
= Q_1(VV\alpha\beta \centerdot \nabla_1) = QV\, d\Sigma \Delta,
\]
where~$d\Sigma$ is put for~$V\alpha\beta$. Adding for the whole surface we get
\equationref{II}{8}.

\Equationref{II}{9} is proved in an exactly similar way by splitting
the volume up into elementary parallelepipeda by three families
of surfaces one or more members of one of the families coinciding
with the given boundary. If $\alpha$, $\beta$, $\gamma$ be the vector edges of one
such parallelepiped we get a term corresponding to $Q\beta' - Q\alpha'$ viz.\
\[
Q\text{(vector sum of surface of parallelepiped)} \equiv 0,
\]
and we get the sum of three terms corresponding to
\[
Q_1 \alpha\, S\beta\nabla_1 - Q_1 \beta\, S\alpha\nabla_1
\]
above, viz.\
\[
-Q_1(V\beta\gamma)\,  S\alpha\nabla_1
-Q_1(V\gamma\alpha)\, S\beta\nabla_1
-Q_1(V\alpha\beta)\,  S\gamma\nabla_1
= -Q_1 \nabla_1\, S\alpha\beta\gamma,
\]
whence putting $S\alpha\beta\gamma = -\ds$ we get \equationref{II}{9}.

\Paragraph{7} It will be observed that the above theorems have been
proved only for cases where we can put $dQ = -Q_1 S\,d\rho\nabla_1$ i.e.\ when
the space fluxes of~$Q$ are finite. If at any isolated point they are
not finite this point must be shut off from the rest of the space by
a small closed surface or curve as the case may be and this surface
or curve must be reckoned as part of the boundary of the space.
If at a surface (or curve) $Q$ has a discontinuous value so that its
derivatives are there infinite whereas on each side they are
finite, this surface (or curve) must be considered as part of the
boundary and each element of it will occur twice, i.e.\ once for the
part of the space on each side.

In the case of the isolated points, if the surface integral or line
integral round this added boundary vanish, we can of course cease
to consider these points as singular. Suppose $Q$ becomes infinite
at the point $\rho = \alpha$. Draw a small sphere of radius~$a$ and also a
sphere of unit radius with the point~$\alpha$ for centre, and consider the
small sphere to be the added boundary. Let~$d\Sigma'$ be the element
of the unit sphere cut off by the cone which has $\alpha$ for vertex and
the element~$d\Sigma$ of the small sphere for base. Then $d\Sigma = a^2 d\Sigma'$
and we get for the part of the surface integral considered $a^2 \smalliint Q_{d\Sigma}\,d\Sigma'$
where $Q_{d\Sigma}$ is the value of~$Q$ at the element~$d\Sigma$. If then
\[
\text{Lt}_{T(\rho-\alpha) = 0}\ T^2(\rho-\alpha)
Q_{d\Sigma} U(\rho-\alpha) = 0
\]
%% -----File: 039.png---Folio 21-------
the point may be regarded as not singular. If the limiting expression
is finite the added surface integral will be finite. If the
expression is infinite the added surface integral will be generally
but not always infinite. Similarly in the case of an added line
integral if $\text{Lt}_{T(\rho-\alpha)=0} T(\rho-\alpha) QU(\rho-\alpha)$ is zero or finite, the added
line integral will be zero or finite respectively (of course including
in the term finite a possibility of zero value). If this expression
be infinite, the added line integral will generally also be infinite.

This leads to the consideration of potentials which is given
in~\Secref{9}.

\Paragraph{8} Some particular cases of equations \Eref{II}{8} and~\Eref{II}{9} which
(except the last) have been proved by Tait, are very useful.
First put $Q =$ a simple scalar~$P$. Thus
\begin{align*}
\EqnTag{II}{10}
\smallint P\,d\rho    &= \smalliint V\,d\Sigma\, \nabla P, \\
%
\EqnTag{II}{11}
\smalliint P\,d\Sigma &= \smalliiint\nabla P\, \ds.
\end{align*}
If $P$ be the pressure in a fluid $-\smalliint P\,d\Sigma$ is the force resulting from
the pressure on any portion and \equationref{II}{11} shews that $-\nabla P$~is
the force per unit volume due to the same cause. Next put
$Q\omega = S\omega\sigma$ and $V\omega\sigma$. Thus
\begin{align*}
\EqnTag{II}{12}
\smallint S\, d\rho\sigma &= \smalliint S\, d\Sigma\, \nabla\sigma,  \\
%
\EqnTag{II}{13}
\smallint V\, d\rho\sigma &= \smalliint V(V\, d\Sigma\, \nabla\centerdot \sigma)
= \smalliint d\Sigma\, S\nabla\sigma - \smalliint\nabla_1 S\, d\Sigma\sigma_1, \\
%
\EqnTag{II}{14}
\smalliint S\,d\Sigma\sigma &= \smalliiint S\nabla\sigma\, \ds, \\
%
\EqnTag{II}{15}
\smalliint V\,d\Sigma\sigma &= \smalliiint V\nabla\sigma\, \ds.
\end{align*}
Equations \Eref{II}{12} and~\Eref{II}{14} are well-known theorems, and \Eref{II}{13}
and~\Eref{II}{15} will receive applications in the following pages. Green's
Theorem with Thomson's extension of it are, as indeed has been
pointed out by Tait particular cases of these equations.

Equations \Eref{II}{14} and~\Eref{II}{15} applied to an element give the well-known
physical meanings for $S\nabla\sigma$ and~$V\nabla\sigma$\footnote{%
[Note added, 1892. Let me disarm criticism by confessing that what follows
concerning $V\nabla\sigma$ is nonsense.]}.
The first is
obtained by applying~\Eref{II}{14} to any element, and the second (regarding
$\sigma$ as a velocity) is obtained by applying~\Eref{II}{15} to the element
contained by the following six planes each passing infinitely near
to the point considered---(1) two planes containing the instantaneous
axis of rotation, (2) two planes at right angles to this axis,
and (3) two planes at right angles to these four.

%% -----File: 040.png---Folio 22-------

One very frequent application of \equationref{II}{9} may be put in
the following form:---$Q$ being any linear function (varying from
point to point) of $R_1$ and~$\nabla_1$, $R$ being a function of the position of
a point
\[
\EqnTag{II}{16}
\smalliiint Q(R_1, \nabla_1)\, \ds
    = - \smalliiint Q_1(R, \nabla_1)\, \ds + \smalliint Q(R, d\Sigma).
\]


\Section{9}{Potentials}

\Paragraph{9} We proceed at once to the application of these theorems
in integration to Potentials. Although the results about to be
obtained are well-known ones in Cartesian Geometry or are easily
deduced from such results it is well to give this quaternion method
if only for the \emph{collateral} considerations which on account of their
many applications in what follows it is expedient to place in this
preliminary section.

If $R$ is some function of $\rho - \rho'$ where $\rho'$ is the vector coordinate
of some point under consideration and $\rho$ the vector coordinate of
any point in space, we have
\[
\Nabla{\rho} R = - \Nabla{\rho'} R.
\]
Now let $Q(R)$ be any function of~$R$, the coordinates of~$Q$ being
functions of~$\rho$ only. Consider the integral $\smalliiint Q(R)\, \ds$ the variable
of integration being~$\rho$ ($\rho'$~being a constant so far as the integral is
concerned). It does not matter whether the integral is a volume,
surface or linear one but for conciseness let us take it as a volume
integral. Thus we have
\[
\Nabla{\rho'}\smalliiint Q(R)\, \ds =  \smalliiint \Nabla{\rho'} Q(R)\, \ds
                                    = -\smalliiint \Nabla{\rho}_1 Q(R_1)\, \ds.
\]
Now $\Nabla{\rho}$ operating on the whole integral has no meaning so we
may drop the affix to the~$\nabla$ outside and always understand~$\rho'$.
Under the integral sign however we must retain the affix $\rho$ or~$\rho'$
unless a convention be adopted. It is convenient to adopt such a
convention and since~$Q$ will probably contain some $\Nabla{\rho}$ but cannot
possibly contain a~$\Nabla{\rho'}$ we must assume that when~$\nabla$ appears without
an affix under the integral sign the affix~$\rho$ is understood.
With this understanding we see that when~$\nabla$ crosses the integral
sign it must be made to change sign and refer only to the part we
have called~$R$. Thus
\[
\EqnTag{II}{17}
\nabla\smalliiint Q(R)\, \ds = -\smalliiint \nabla_1Q(R_1)\, \ds.
\]

%% -----File: 041.png---Folio 23-------
Generally speaking~$R$ can and will be put as a function
of~$T^{-1} (\rho'-\rho)$ and for this we adopt the single symbol~$u$. Both this
symbol and the convention just explained will be constantly required
in all the applications which follow.

\Paragraph{10} Now let~$Q$ be any function of the position of a point.
Then the potential of the volume distribution of~$Q$, say~$q$, is given
by:---
\[
\EqnTag{II}{18}
q = \smalliiint uQ\, \ds,
\]
the extent of the volume included being supposed given. We
may now prove the following two important propositions
\begin{align*}
\EqnTag{II}{19}
\nabla^2q &= 4\pi Q,  \\
%
\EqnTag{II}{20}
\smalliint S\, d\Sigma\, \nabla\centerdot q &= 4\pi\smalliiint Q\, \ds.
\end{align*}
The latter is a corollary of the former as is seen from \equationref{II}{9}
\Secref{6} above.

\Equationref{II}{19} may be proved thus. If $P$~be any function of
the position of a point which is finite but not necessarily continuous
for all points
\[
\nabla\smalliiint uP\, \ds,
\]
is always finite and if the volume over which the integral extends
is indefinitely diminished, so also is the expression now under
consideration, and this for the point at which is this remnant of
volume. This in itself is an important proposition. The expression,
by \equationref{II}{17}, $=-\smalliiint\nabla uP\, \ds$ and both statements are
obviously true except for the point~$\rho'$. For this point we have
merely to shew that the part of the volume integral just given
contributed by the volume indefinitely near to~$\rho'$ vanishes with
this volume. Divide this near volume up into a series of elementary
cones with~$\rho'$ for vertex. If~$r$ is the (small) height and~$d\omega$ the
solid vertical angle of one of these cones, the part contributed by
this cone is approximately $U (\rho-\rho') P_{\rho'}r\,d\omega/3$ where $P_{\rho'}$ is the
value of~$P$ at the point~$\rho'$. The proposition is now obvious.

Now since
\[
\nabla^2q = \nabla^2\smalliiint uQ\, \ds = \smalliiint\nabla^2 uQ\, \ds,
\]
we see that the only part of the volume integral $\smalliiint uQ\, \ds$ which
need be considered is that given by the volume in the immediate
neighbourhood of~$\rho'$, for at all points except~$\rho'$, $\nabla^2u = 0$. Consider
then our volume and surface integrals only to refer to a small
%% -----File: 042.png---Folio 24-------
sphere with~$\rho'$ for centre and so small that no point is included at
which~$Q$ is discontinuous and therefore $\nabla Q$~infinite. (This last
assumes that~$Q$ is not discontinuous at~$\rho'$. In the case when $Q$ is
discontinuous at~$\rho'$ no definite meaning can be attached to the
expression~$\nabla^2 q$.) We now have
\begin{align*}
\nabla^2 q
&= \nabla^2\smalliiint           uQ\, \ds
= -\nabla  \smalliiint \nabla u\, Q\, \ds \quad \text{[by \Secref{9}]}  \\
&= \nabla  \smalliiint u\, \nabla Q\, \ds - \nabla \smalliint u\, d\Sigma\, Q,
\end{align*}
\equationref{II}{9} \Secref{6} being applied and the centre of the sphere being
not considered as a singular point since the condition of~\Secref{7} is
satisfied, viz.\ that
$\text{Lt}_{T(\rho-\rho')=0}\ T^2(\rho-\rho') uU(\rho-\rho') = 0$. Now putting
$P~\text{ above} = \nabla Q$ we see that the first expression, viz.\
$\nabla\smalliiint u\, \nabla Q\, \ds$
can be neglected and the second gives
\[
\nabla^2 q = \smalliint\nabla u\,d\Sigma\, Q = 4\pi \overline{Q},
\]
where~$\overline{Q}$ is the mean value of~$Q$ over the surface of the sphere and
therefore in the limit~$= Q$. Thus \equationref{II}{19} has been proved.

When~$Q$ has a simple scalar value all the above propositions,
and indeed processes, become well-known ones in the theory of
gravitational potential.

We do not propose to go further into the theory of Potentials
as the work would not have so direct a bearing on what follows as
these few considerations.

%% -----File: 043.png---Folio 25-------


\FNChapter{Section III}{Elastic Solids}{\protect\footnotemark Elastic Solids}

\footnotetext{[Note added, 1892. It would be better to head this section ``Elastic \emph{bodies}''
since except when the strains are assumed small the equations are equally true of
solids and fluids. I may say here that I have proved in the \textit{Proc.~R.~S.~E.} 1890--91,
pp.~106 \textit{et seq.},\ most of the general propositions of this section somewhat
more neatly though the processes are essentially the same as here.]}


\Section{11}{Brief recapitulation of previous work in this branch}

\Paragraph{11} As far as I am aware the only author who has applied
Quaternions to Elasticity is Prof.\ Tait. In the chapter on Kinematics
of his treatise on Quaternions, \S\S~360--371, he has considered
the mathematics of strain with some elaboration and again in the
chapter on Physical Applications, \S\S~487--491, he has done the
same with reference to stress and also its expression in terms of
the displacement at every point of an elastic body.

In the former he has very successfully considered various
useful decompositions of strain into pure and rotational parts and
so far as strain alone is considered, i.e.\ without reference to what
stress brings it about he has left little or nothing to be done. In
the latter he has worked out the expressions for stress by means
of certain vector functions at each point, which express the elastic
properties of the body at that point.

But as far as I can see his method will not easily adapt itself
to the solution of problems which have already been considered
by other methods, or prepare the way for the solution of fresh
problems. To put Quaternions in this position is our present
object. I limit myself to the statical aspect of Elasticity, but I
believe that Quaternions can be as readily, or nearly so, applied to
the Kinetics of the subject.

For the sake of completeness I shall repeat in my own notation
a small part of the work that Tait has given.

%% -----File: 044.png---Folio 26-------

Tait shews (\S~370 of his \textit{Quaternions}) that in any small portion
of a strained medium the strain is homogeneous and (\S~360) that
a homogeneous strain function is a linear vector one. He also
shews (\S~487) that the stress function is a linear vector one and he
obtains expressions (\S\S~487--8) for the force per unit volume due
to the stress, in terms of the space-variation of the stress.


\Section{12}{Strain, Stress-force, Stress-couple}

\Paragraph{12} This last however I give in my own notation. His
expression in \S~370 for the strain function I shall throughout
denote by~$\chi$. Thus
\[
\EqnTag{III}{1}
\chi\omega = \omega - S\omega\,\nabla\centerdot \eta,
\]
where $\eta$ is the displacement that gives rise to the strain.

Let $\chi$ consist of a pure strain~$\psi$ followed by a rotation~$q(\;)q^{-1}$
as explained in Tait's \textit{Quaternions}, \S~365 where he obtains both
$q$ and~$\psi$ in terms of~$\chi$. Thus
\[
\EqnTag{III}{2}
\chi\omega = q\,\psi\omega\, q^{-1}.
\]

When the strain is small $\psi$ takes the convenient form~$\overline{\chi}$
where~$\chi$ stands for the pure part of~$\chi$ so that
\[
\EqnTag{III}{3}
\overline{\chi}\omega
= \omega - \tfrac{1}{2} S\omega\nabla\centerdot \eta
- \tfrac{1}{2} \nabla_1 S\omega\eta_1,
\]
by \equationref{II}{3} \Secref{2} above. Similarly $q(\;)q^{-1}$ becomes $V\theta(\;)$ where
$2\theta = V\nabla\eta$. The truth of these statements is seen by putting in
\equationref{III}{2} for~$q$, $1+\theta/2$ and therefore for~$q^{-1}$, $1-\theta/2$
for~$\psi$, $\overline{\chi}$
and then neglecting all small quantities of an order higher than
the first.

\Paragraph{13} Next let us find the force and couple per unit volume due
to a stress which varies from point to point. Let the stress
function be~$\phi$. Then the force on any part of the body, due to
stress, is
\[
\smalliint \phi\, d\Sigma = \smalliiint \phi\Delta\, \ds,
\]
by \equationref{II}{9} \Secref{6}. Thus \emph{the force per unit volume $= \phi\Delta$}, for
the volume considered in the equation may be taken as the
element~$\ds$.

Again the moment round any arbitrary origin is
\[
\smalliint V\rho\phi\, d\Sigma
= \smalliiint V\rho_1\phi \nabla_1\, \ds
+ \smalliiint V\rho\phi_1 \nabla_1\, \ds,
\]
%% -----File: 045.png---Folio 27-------
by \equationref{II}{9} \Secref{6}. The second term on the right is that due to
the force~$\phi\Delta$ just considered, and the first shews that in addition
to this \emph{there is a couple per unit volume ${}=V\zeta\phi\zeta={}$ twice the
``rotation'' vector} of~$\phi$. Let then~$\overline{\phi}$ be the pure part
and~$V\epsilon(\;)$
the rotatory part of~$\phi$. Thus

\emph{Force} per unit volume
\[
\EqnTag{III}{4}
= \phi\Delta = \overline{\phi}\Delta - V\nabla\epsilon,
\]

\emph{Couple} per unit volume
\[
\EqnTag{III}{5}
= V\zeta \phi\zeta = 2\epsilon.
\]

These results are of course equivalent to those obtained by
Tait, \textit{Quaternions}, \S\S~487--8.

The meanings just given to $\eta$, $\chi$, $\overline{\chi}$, $\psi$, $q$, $\phi$, $\overline{\phi}$ and~$\epsilon$ will be
retained throughout this Section. In all cases of small strain as
we have seen we may use $\psi$ or~$\overline{\chi}$ indifferently and whenever we
wish to indicate that we are considering the physical phenomenon
of pure strain we shall use $\psi$, $\overline{\chi}$ being regarded merely as a
function of~$\chi$. We shall soon introduce a function~$\varpi$ which will
stand towards~$\phi$ somewhat as~$\psi$ towards~$\chi$ and such that when
the strain is small $\varpi = \overline{\phi}$.

It is to be observed that~$\phi\omega$ is the force exerted on a vector
area, which \emph{when strained} is~$\omega$, not the stress on an area which
before strain is~$\omega$. Similarly in equations \Eref{III}{4} and~\Eref{III}{5} the independent
variable of differentiation is~$\rho + \eta$ so that strictly
speaking in~\Eref{III}{4} we should put
$\overline{\phi}_{\rho+\eta}\Delta - V_{\rho+\eta}\nabla\epsilon$. In the case of
small strain these distinctions need not be made.


\Section{14}{Stress in terms of strain}

\Paragraph{14} To express stress in terms of strain we assume any
displacement and consequent strain at every point of the body
and then give to every point a small additional displacement $\delta\eta$
and find in terms of $\psi$ and~$\phi$ the increment $\smalliiint\delta w\, \ds_0$ in the
potential energy of the body, $w\, \ds_0$ being the potential energy of
any element of the body whose volume before strain was $\ds_0$. Thus
\Erratum{err27a}% [** PP: Removed comma after first term]
\begin{align*}
\smalliiint \delta w\, \ds_0
&= \text{(work done by stresses on surface of portion considered)} \\
&- \text{(work done by stresses throughout volume of same portion).}
\end{align*}
%% -----File: 046.png---Folio 28-------
Thus, observing that by \Secref{12} the rotation due to the small
displacement $\delta\eta$ is $V_{\rho+\eta} \nabla\delta\eta/2$, we have
\[
\smalliiint \delta w\, \ds_0
= - \smalliint  S\, \delta\eta\, \phi\, d\Sigma
  + \smalliiint S\, \delta\eta\, \phi_{1\rho+\eta} \nabla_1\, \ds
  + \smalliiint S\epsilon_{\rho+\eta} \nabla\delta\eta\, \ds.
\]
The first of the terms on the right is the work done on the surface
of the portion of the body considered; the second is $-$(work done
by stress-forces $\phi_{\rho+\eta}\Delta$); and the third is $-$(work done by stress-couples
$2\epsilon$). Thus converting the surface integral into a volume
integral by \equationref{II}{9} \Secref{6} above
\begin{align*}
\smalliiint \delta w\, \ds_0
&= -\smalliiint S\, \delta\eta_1\, \phi_{\rho+\eta}         \nabla_1\, \ds
 +  \smalliiint S\, \delta\eta_1\, V\epsilon_{\rho+\eta}    \nabla_1\, \ds \\
&= -\smalliiint S\, \delta\eta_1\, \overline{\phi}_{\rho+\eta} \nabla_1\, \ds.
\end{align*}
Limiting the portion of the body considered to the element $\ds$ we
get
\[
\EqnTag{III}{6}
\delta w = -mS\,\delta\eta_1\, \overline{\phi}_{\rho+\eta} \nabla_1
\]
where $m$ is put for~$\ds/\ds_0$ and therefore may be put by \Secref{3$a$} and~\Secref{3}
above in the various forms
\begin{align*}
\EqnTag{III}{6$a$}
6m &= S\zeta_1\zeta_2\zeta_3\,
      S\chi\zeta_1  \chi\zeta_2  \chi\zeta_3  \\
%
\EqnTag{III}{6$b$}
   &= S\zeta_1\zeta_2\zeta_3\,
      S\chi'\zeta_1 \chi'\zeta_2 \chi'\zeta_3 \\
%
\EqnTag{III}{6$c$}
   &= S\zeta_1\zeta_2\zeta_3\,
       S\psi\zeta_1 \psi\zeta_2  \psi\zeta_3  \\
%
\EqnTag{III}{6$d$}
   &= S\nabla_1\nabla_2\nabla_3\,
      S(\rho+\eta)_1(\rho+\eta)_2(\rho+\eta)_3.
\end{align*}

It is to be observed that since the rotation-vector $\epsilon$~of $\phi$ does not
occur in \equationref{III}{6}, $\epsilon$ and therefore the stress-couple are quite arbitrary
so far as the strain and potential energy are concerned. I do
not know whether this has been pointed out before. Of course
other data in the problem give the stress-couple. In fact it can be
easily shewn that in all cases whether there be equilibrium or not
the external couple per unit volume balances the stress-couple.
[Otherwise the angular acceleration of the element would be
infinite.] Thus if~$\mathbf{M}$ be the external couple per unit volume of
the \emph{unstrained} solid we have always
\[
\EqnTag{III}{7}
\mathbf{M} + 2m\epsilon = 0.
\]
In the particular case of equilibrium $\mathbf{F}$ being the external force
per unit volume of the unstrained solid we have
\[
\EqnTag{III}{8}
\mathbf{F} + m(\overline{\phi}_{\rho+\eta} \Delta - V_{\rho+\eta} \nabla\epsilon) = 0.
\]
The mathematical problem is then the same as if for~$\mathbf{F}/m$ we
substituted $\mathbf{F}/m + V_{\rho+\eta} \nabla(\mathbf{M}/2m)$ and for $\mathbf{M}$,~zero. In the case of
%% -----File: 047.png---Folio 29-------
small strains $m$~may be put $= 1$. In this case then the mathematical
problem is the same as if for~$\mathbf{F}$ we substituted $\mathbf{F} + V \nabla \mathbf{M}/2$
and for $\mathbf{M}$~zero.

Returning to \equationref{III}{6} observe that
\[
\delta \chi \omega
    = - S \chi \omega_{\rho+\eta} \nabla\centerdot \delta \eta
\]
[which is established just as is the equation $\chi\omega = -S\omega \nabla\centerdot (\rho +\eta)$],
so that changing~$\omega$ which is \emph{any} vector into~$\chi^{-1}\omega$
\[
\delta \chi\centerdot \chi^{-1} \omega
    = - S \omega_{\rho+\eta} \nabla \centerdot \delta \eta.
\]
Therefore by \equationref{III}{6} of this section and \equationref{II}{4} \Secref{3} above
we have
\[
\delta w = - mS\delta \chi\, \chi^{-1}\,\zeta \overline{\phi} \zeta.
\]
We must express the differential on the right of this equation in
terms of $\delta \psi$ and~$\delta q$, the latter however disappearing as we should
expect. Now $\chi\omega = q \psi \omega q^{-1}$  [\equationref{III}{2} \Secref{12}] so that remembering
that $\delta\centerdot q^{-1} = - q^{-1}\,\delta q\, q^{-1}$
\begin{flalign*}
&& \delta \chi \omega
&= \delta q\, \psi \omega q^{-1}
   - q \psi \omega q^{-1}\, \delta q\, q^{-1}
   + q\,\delta\psi\, \omega q^{-1}   && \\
&&
&= 2VV\,\delta q\, q^{-1}\centerdot \chi \omega
   + q\, \delta\psi\, \omega q^{-1}, && \\
&\therefore&
\delta w / m
&= -2 S\centerdot V\,\delta q\, q^{-1}\centerdot \zeta \overline{\phi} \zeta
   - Sq\,\delta\psi\, \chi^{-1} \zeta\, q^{-1}\overline{\phi}\zeta, &&
\end{flalign*}
or since $V\,\zeta \overline{\phi} \zeta = 0$,  $\overline{\phi}$ being self-conjugate
\[
\delta w = - m S\, \delta\psi\, \chi^{-1}\, \zeta q^{-1}\overline{\phi}\, \zeta q.
\]
This can be put into a more convenient shape for our present
purpose. First put for~$\chi^{-1}$ its value~$\psi^{-1}q^{-1}(\;)q$ and then apply
\equationref{II}{6$a$} \Secref{3} above, putting for the~$\phi$ of that equation~$q^{-1}(\;)q$
and therefore for the~$\phi'$, $q(\;)q^{-1}$. Thus
\begin{flalign*}
\EqnTag{III}{9}
&&
\delta w &= - m S\, \delta\psi\, \psi^{-1}\, \zeta \varpi \zeta, && \\
%
\EqnTag{III}{10}
&\text{where}&
\varpi \omega &= q^{-1} \overline{\phi} (q\omega q^{-1})q. &&
\end{flalign*}

The physical meaning of this last equation can easily be shewn.
Suppose when there is no rotation that $\overline{\phi}=\varpi'$. Then it is
natural\footnote{%
Observe that we \emph{do not} make this assumption. We really shew that it is
true.}
to assume---in fact it seems almost axiomatic---that the
superimposed rotation~$q(\;)q^{-1}$ should merely so to speak rotate
the stress along with it. Thus if~$\omega$ is some vector area before the
rotation which becomes~$\omega'$ by means of the rotation
\[
q\varpi' \omega\centerdot q^{-1} = \overline{\phi} \omega'.
\]
%% -----File: 048.png---Folio 30-------
But $\omega' = q\omega q^{-1}$, so that
\[
\varpi'\omega = q^{-1}\overline{\phi}(q\omega q^{-1})q = \varpi\omega.
\]
Thus we see that $\varpi$ is as it were~$\overline{\phi}$ with the rotation undone.

Before proceeding further with the calculation let us see what
we have assumed and what \equationref{III}{9} teaches us. The one thing
we have assumed is that the potential energy of the body can be
taken as the sum of the potential energies of its elements, in other
words that no part of the potential energy depends conjointly on
the strains at $P$ and~$Q$ where $P$ and~$Q$ are points separated by a
finite distance. This we must take as an axiom. By it we are
led to the expression for~$\delta w$ in \equationref{III}{9}. This only involves
the variation of the pure strain~$\psi$ \emph{but not the space differential
coefficients of~$\psi$}. This is not an obvious result as far as I can
see but it is I believe always assumed without proof.

We may \emph{now} regard~$w$ as a function of $\psi$~only. Therefore by
\equationref{II}{7} \Secref{5} above
\[
\delta w = - S\, \delta \psi\, \zeta\, \Pre{\psi}{\Dop} w\, \zeta,
\]
therefore by \equationref{III}{9} of this section
\[
S\, \delta\psi\, \zeta\,\Pre{\psi}{\Dop} w\, \zeta
  = mS\, \delta\psi\, \psi^{-1}\,\zeta\varpi\zeta\footnotemark.
\]
\footnotetext{By putting $\delta\psi = \omega S(\;)\omega' + \omega' S(\;)\omega$ in this equation, \equationref{III}{11} can be
deduced but as this method has already been applied in~\Secref{4} I give the one in
the text to shew the variety of Quaternion methods. [Note added,~1892. If we
use the theorem in the foot-note of~\Secref{4}, \equationref{III}{11} follows at once.]}% [** PP: End of footnote, no paragraph]
%
In \equationref{II}{6$a$} \Secref{3} above putting $\phi = \psi^{-1}$ the right-hand member of
this equation becomes $mS\,\delta\psi\,\zeta\,\varpi\psi^{-1}\,\zeta$. Now putting in \equationref{II}{6$a$}
\Secref{3}  $\phi = \varpi\psi^{-1}$ this member becomes $mS\zeta\,\delta\psi\,\psi^{-1}\,\varpi\zeta$ or $mS\,\delta\psi\,\zeta\,\psi^{-1}\varpi\,\zeta$.
Thus we have
\[
2S\,\delta\psi\, \zeta\,\Pre{\psi}{\Dop} w\, \zeta
  = mS\,\delta\psi\, \zeta(\varpi\psi^{-1} + \psi^{-1}\varpi)\zeta.
\]
Now $\varpi\psi^{-1} + \psi^{-1}\varpi$ is self-conjugate. Hence by~\Secref{4} above
\[
\EqnTag{III}{11}
m(\varpi\psi^{-1} + \psi^{-1}\varpi) = 2 \Pre{\psi}{\Dop} w.
\]

This equation can be looked upon as giving~$\varpi$ in terms of the
strain. We can obtain~$\varpi$ however explicitly for $\psi^{-1}\varpi$  is the conjugate
of~$\varpi\psi^{-1}$. Hence from \equationref{III}{11} (because $(\psi^{-1}\varpi + \varpi\psi^{-1})/2$
is the pure part of~$\varpi\psi^{-1}$)
\[
m\, \varpi \psi^{-1}\omega = \Pre{\psi}{\Dop} w\, \omega + V\theta\,\omega
\]
where $\theta$ is a vector to be found. Changing $\omega$ into $\psi\omega$
\[
m\,\varpi\omega = \Pre{\psi}{\Dop} w\, \psi\omega + V \theta\, \psi\omega.
\]
%% -----File: 049.png---Folio 31-------
Now $\varpi$ being self-conjugate $V\zeta\varpi\zeta = 0$. Hence
\begin{align*}
V\zeta \Pre{\psi}{\Dop} w\, \psi\zeta
&= -V\zeta\, V\theta\, \psi\zeta
 = \theta S \zeta\psi\zeta - \psi\zeta\, S\theta\zeta \\
&= \theta S \zeta\psi\zeta + \psi\theta,
\end{align*}
whence
% [** PP: A bit too far left, but leaving alone]
\begin{flalign*}
\EqnTag{III}{12}
\left.\begin{aligned}
&&
\theta
& = (\psi + S\zeta_1\psi\zeta_1)^{-1} V\zeta \Pre{\psi}{\Dop} w\, \psi\zeta \\
%
&\text{or\qquad\qquad}&
\theta
& = (\psi + S\zeta_1\psi\zeta_1)^{-1} V\psi\zeta \Pre{\psi}{\Dop} w\, \zeta
\end{aligned}\;  % [** PP: Need some extra space]
\right\} &&
\end{flalign*}
by \equationref{II}{6$a$} \Secref{3} above. Thus finally
\begin{flalign*}
\EqnTag{III}{13}
\left.\begin{aligned}
&&
m\varpi\omega &= \Pre{\psi}{\Dop} w\, \psi\omega + V\theta\psi\omega && \\
%
&\text{or\qquad\qquad}&
m\varpi\omega &= \Pre{\psi}{\Dop} w\, \psi\omega
    - V\psi\omega(\psi + S\zeta_1\psi\zeta_1)^{-1}
      V\psi\zeta \Pre{\psi}{\Dop} w\, \zeta.
\end{aligned}\!\!\right\} &&
\end{flalign*}

This completely solves the problem of expressing stress in
terms of strain in the most general case.


\Paragraph{15} $\varpi\omega + V\epsilon'\omega$, where, as we saw in last section, $\epsilon'$ is perfectly
arbitrary so far as the strain is concerned, is the force on the
strained area~$\omega$ due to the pure strain~$\psi$. And again
\[
\overline{\phi}\omega + V\epsilon''\omega
\quad\text{or}\quad q(\varpi q^{-1}\ \varpi q)q^{-1} + V\epsilon''\omega
\]
is that due to the strain $q\psi(\;)q^{-1}$ or~$\chi$.

To find the force on an area which before strain was~$\omega_0$ let its
strained value after~$\psi$ has taken place be~$\omega$. Then by equation~(4),
\S~145 of Tait's \textit{Quaternions}, $m\omega_0 = \psi\omega$. Hence
\[
\text{Required force} = \varpi\omega + V\epsilon'\omega
    = \Pre{\psi}{\Dop} w\, \omega_0
      + V\theta\omega_0
      + mV\epsilon'\psi^{-1}\omega_0
\]
by \equationref{III}{13} of last section. If the rotation now take place
this force rotates with it so that the force on the area which was
originally~$\omega_0$ is after the strain $\chi$ or~$q\psi(\;)q^{-1}$
\[
\EqnTag{III}{14}
\tau\omega_0 = q \Pre{\psi}{\Dop} w\,\omega_0 q^{-1}
                 + qV\theta\omega_0\centerdot q^{-1}
                 + V(\epsilon q\psi^{-1}\omega_0\centerdot q^{-1})
\]
where $\epsilon = m\,q\epsilon'q^{-1}$ and
\[
\text{$\therefore\qquad \epsilon = \tfrac{1}{2}$ the stress-couple per un.\ vol.\ of unstrained body}.
\]

This force then is a linear vector function of~$\omega_0$, but in general
even when $\epsilon = 0$ it is not self-conjugate. When both $\epsilon$ and~the
rotation are zero we see that the rotation vector of~$\tau$ is~$\theta$ given by
\equationref{III}{12} of last section.

The stress-force can be shewn as in~\Secref{13} to be~$\tau\Delta$ per unit
volume of the unstrained body. Thus since the corresponding
stress-couple is~$2\epsilon$ the moment exerted by the stresses on any
portion of the body round an arbitrary origin is
\[
\smalliiint V(\rho + \eta)\tau_1\nabla_1\, \ds_0 + 2\smalliiint\epsilon\, \ds_0.
\]
%% -----File: 050.png---Folio 32-------
But this moment may also be put in the form
\begin{flalign*}
&&
&\smalliint V(\rho+\eta)\tau\, d\Sigma_0,     && \\
%
&\text{or\qquad}&
    \smalliiint V\zeta\tau\zeta \, \ds_0
  &+ \smalliiint V(\rho+\eta)\tau_1\nabla_1\, \ds_0
  + \smalliiint V\eta_1\tau\nabla_1\, \ds_0, &&
\end{flalign*}
by \equationref{II}{9} \Secref{6} above. Comparing these results
\begin{flalign*}
\EqnTag{III}{14$a$}
&\text{or}&
\left.
\begin{aligned}
2\epsilon &= V\zeta\tau\zeta + V\eta_1\tau\nabla_1 \\
2\epsilon &= V\rho'_1\tau\nabla_1,
\end{aligned}\; % [** PP: Need some extra space]
\right\} &&
\end{flalign*}
in the notation of next section. This equation may also be
deduced from \equationref{III}{15$l$} below but not so naturally as above.


\Section{16}{The equations of equilibrium}

\Paragraph{16} We require \equationref{III}{9} \Secref{14} above to prove the statement
that no space-variations of $\psi$ or~$q$ are involved in~$w$. It is
also required to shew that the fact that~$q$ also is not involved
in~$w$ is a mathematical sequence of the assumption that the
potential energy of a solid is the sum of the potential energies of
its elements. Assuming these facts however we can arrive at the
equation of stress~\Eref{III}{11} \Secref{14} in a different way from the above.
We shall also obtain quite different expressions for $\varpi$, $\tau$,~\&c.\
and most important of all we shall obtain the equations of
equilibrium by obtaining~$\tau$ explicitly in terms of the displacement
and its space derivatives. From \Secref{12} above we have
\begin{gather*}
\chi\omega = q\psi\omega q^{-1},\qquad
\chi'\omega = \psi(q^{-1}\omega q), \\
%
\EqnTag{III}{15}
\therefore\qquad \chi'\chi = \psi^2 = \Psi,
\end{gather*}
(say) as in Tait's \textit{Quaternions}, \S~365. Thus
$\Psi\omega = \nabla_1 S\omega\, \nabla_2 S\rho'_1\rho'_2$
where~$\rho'$ is put as it will be throughout this section for $\rho + \eta$, the
vector coordinate after displacement of the point~$\rho$. From this as
we have seen in the Introduction we deduce that the coordinates
of~$\Psi$ are the $A$, $B$, $C$, $a$, $b$, $c$ of Thomson and Tait's \textit{Nat.\ Phil.}\
App.~C.

We may as do those authors regard~$w$ as a function of~$\Psi$.
Thus:---
\begin{align*}
\delta w
&= -S\,     \delta\Psi\,   \zeta \Pre{\Psi}{\Dop} w\, \zeta, \\
&= -S\,     \delta\chi'\chi\zeta \Pre{\Psi}{\Dop} w\, \zeta
   -S\chi'\,\delta\chi\,   \zeta \Pre{\Psi}{\Dop} w\, \zeta.
\end{align*}
By \equationref{II}{4$a$} \Secref{3} above, each of the terms in this last
expression
%[** PP: Next two eqns aligned in text, but conceptually *shouldn't* be]
\[
=  -S\,\delta\eta_1\chi \Pre{\Psi}{\Dop} w\, \nabla_1,
\]
\[
\EqnTag{III}{15$a$}
\therefore\qquad \delta w
= -2S\,\delta\eta_1\chi \Pre{\Psi}{\Dop} w\, \nabla_1,
\]
%% -----File: 051.png---Folio 33-------
Comparing this with \equationref{III}{6} \Secref{14} and putting in both
$\delta\eta_1 = \omega'S\omega\rho'$ where $\omega'$, $\omega$ are arbitrary constant vectors, we get
\[
mS\omega'\,\overline{\phi}\omega
= 2S\omega'\chi \Pre{\Psi}{\Dop} w\, \chi'\omega.
\]
Hence, since~$\omega'$ is quite arbitrary
\[
\EqnTag{III}{15$b$}
m\overline{\phi} = 2\chi \Pre{\Psi}{\Dop} w\, \chi'.
\]
From \equationref{III}{10} \Secref{14} above which \emph{defines} $\varpi$ we see that
\[
\EqnTag{III}{15$c$}
m\varpi = 2\psi \Pre{\Psi}{\Dop} w\, \psi,
\]
which we should expect since we have already seen that~$\varpi$ is the
value of~$\overline{\phi}$ when there is no rotation and therefore $\chi = \chi' = \psi$.
Now since
\[
  S\, \delta\psi\, \zeta \Pre{\psi}{\Dop} \zeta
= S\, \delta\Psi\, \zeta \Pre{\Psi}{\Dop} \zeta
= S(\delta\psi\,\psi + \psi\,\delta\psi) \zeta \Pre{\Psi}{\Dop} \zeta,
\]
we deduce by any one of the processes already exemplified that
\[
\Pre{\psi}{\Dop}  = \Pre{\Psi}{\Dop}\, \psi + \psi \Pre{\Psi}{\Dop},
\]
where of course the differentiations of $\Pre{\Psi}{\Dop}$ must not refer to~$\psi$.
We see then from \equationref{III}{15$c$} that
\[
m(\varpi\psi^{-1} + \psi^{-1}\varpi) = 2 \Pre{\psi}{\Dop} w,
\]
which is \equationref{III}{11} \Secref{14}.

Our present purpose however is to find the equations of
equilibrium. Let~$\omega_0$ be the vector area which by the strain~$\chi$
becomes~$\omega$. Thus\footnote{See \Secref{83} below.} as in last section
\[
\EqnTag{III}{15$e$}
m\omega_0 = \chi'\omega.
\]
Further let~$2\epsilon$ be the stress-couple per unit volume of the
unstrained solid so that~$2\epsilon/m$ is the same of the strained solid.
As we know, $\epsilon$~is quite independent of the strain. By \Secref{13}
we see that the force~$\tau\omega_0$ on the area which before strain was~$\omega_0$
is~$\overline{\phi}\omega + V\epsilon\omega/m$. Therefore
\[
\EqnTag{III}{15$f$}
\tau\omega_0
= 2\chi \Pre{\Psi}{\Dop} w\,\omega_0 + V\epsilon\chi'^{-1}\omega_0.
\]
\Erratum{err33a}%[** PP: Added missing ``the'']
We saw in the last section that the force per unit volume of the unstrained
solid is~$\tau\Delta$ and the couple~$2\epsilon$. Hence
\begin{align*}
\EqnTag{III}{15$g$}
\mathbf{F} + \tau\Delta &= 0,  \\
%
\EqnTag{III}{15$h$}
\mathbf{M} +  2\epsilon &= 0,
\end{align*}
%% -----File: 052.png---Folio 34-------
are the equations of equilibrium, where $\mathbf{F}$ and~$\mathbf{M}$ are the external
force and couple per unit volume of the unstrained solid. All that
remains to be done then is to express~$\tau$ in terms of~$\epsilon$ and the
displacement. Putting
\[
\EqnTag{III}{15$i$}
\rho + \eta = \rho',
\]
as already mentioned, we have
\[
\EqnTag{III}{15$j$}
\chi \omega = - \rho'_1 S \omega \nabla_1,
\]
so that by \equationref{II}{6$h$} \Secref{3$a$} above we have
\[
\EqnTag{III}{15$k$}
\chi'^{-1}\omega
= -3V \rho'_1 \rho'_2\, S \omega \nabla_1 \nabla_2
  / S\nabla_1\nabla_2 \nabla_3\, S \rho'_1 \rho'_2 \rho'_3.
\]
Therefore by \equationref{III}{15$f$}
\[
\EqnTag{III}{15$l$}
\tau \omega
= -2\rho'_1 S \nabla_1 \Pre{\Psi}{\Dop} w\, \omega
  -3V\epsilon\, V \rho'_1 \rho'_2\, S \omega \nabla_1 \nabla_2
  / S \nabla_1 \nabla_2 \nabla_3\, S \rho'_1 \rho'_2 \rho'_3.
\]
It is unnecessary to write down what \equationref{III}{15$g$} becomes
when we substitute for~$\tau$, changing~$\epsilon$ into~$-\mathbf{M}/2$ by
\equationref{III}{15$h$}. In the important case however when $\mathbf{M} = 0$, the equation
is quite simple, viz.\
\[
\EqnTag{III}{15$m$}
\mathbf{F} = 2\rho'_1 S \nabla_1 \Pre{\Psi}{\Dop} w\, \Delta.
\]

\textit{Addition to \Secref{16}, Dec.}, 1887 (sent in with the Essay). [The
following considerations occurred just before I was obliged to send
the essay in, so that though I thought them worth giving I had
not time to incorporate them in the text.

It is interesting to consider the case of an isotropic body.
Here~$w$ is a function of the three principal elongations only and
therefore we may in accordance with \Secref{14} and~\Secref{15} suppose it a
function of $a$, $b$, $c$ or in accordance with \Secref{16} of $A$, $B$, $C$ where
\begin{align*}
\EqnTag{III}{$A$}
a &= -S\zeta \psi  \zeta,   \quad&
b &= -S\zeta \psi^2\zeta/2, \quad&
c &= -S\zeta \psi^3\zeta/3. \\
%
\EqnTag{III}{$B$}
A &= -S\zeta \Psi  \zeta,   \quad&
B &= -S\zeta \Psi^2\zeta/2, \quad&
C &= -S\zeta \Psi^3\zeta/3.
\end{align*}

Let us use $x$, $y$, $z$, $X$, $Y$, $Z$ for the differential coefficients of~$w$
with respect to $a$, $b$, $c$, $A$, $B$, $C$ respectively. Thus
\[
dw = x\,da + y\,db +z\,dc
   = -S\,d\psi\, \zeta (x + y\psi + z\psi^2)\zeta
\]
as can be easily proved by means of \equationref{II}{6$a$} \Secref{3} above. But
\begin{align*}
dw &= -S\,d\psi \zeta \Pre{\psi}{\Dop} w\, \zeta. \\
%
\EqnTag{III}{$C$}
\therefore\qquad
\Pre{\psi}{\Dop} w &= x + y\psi + z\psi^2,  \\
%
\intertext{by \Secref{4} above. Similarly we have}
\EqnTag{III}{$D$}
\Pre{\Psi}{\Dop} w &= X + Y\Psi + Z \Psi^2.
\end{align*}
%% -----File: 053.png---Folio 35-------
(Notice in passing that to pass to small strains is quite easy
for $\Pre{\psi}{\Dop} w$ is linear and homogeneous in $\psi-1$ so that $\Pre{\psi}{\Dop} w = x + y \psi$
where~$y$ is a constant and $x + y$  a multiple
of~$a-3$.)\footnote{[Note added, 1892. In the original essay there was a slip here which I have
corrected. It was caused by assuming that~$\psi$ instead of~$\psi-1$ was small for small
strains. In the original I said ``where $y$ is a constant and~$x$ is a multiple of~$a$.'']}

From \equationref{III}{$C$} we see that~$\theta$ in \equationref{III}{12} \Secref{14} is zero
and therefore that from \equationref{III}{13}
\begin{align*}
\EqnTag{III}{$E$}
m\varpi
&= \Pre{\psi}{\Dop} w\, \psi = x\psi + y\psi^2 + z\psi^3. \\
\intertext{Again from \equationref{III}{14} \Secref{15}}
%
\EqnTag{III}{$F$}
\tau\omega
&= xq \omega q^{-1} + y\chi\omega + z\chi\psi\omega
             + V\epsilon\chi'^{-1}\omega. \\
\intertext{Similarly from equations \Eref{III}{15$b$} and~\Eref{III}{15$f$} \Secref{16} we may prove by
\equationref{III}{$D$} that}
%
\EqnTag{III}{$G$}
\tfrac{1}{2}m\overline{\phi}
&= \mathbf{X} \chi\chi'
 + \mathbf{Y}(\chi\chi')^2
 + \mathbf{Z}(\chi\chi')^3,
\end{align*}
\begin{flalign*}
\EqnTag{III}{$H$}
\text{and} &&
\tau = 2X\chi
     + 2Y\chi\chi'\,\chi
     + 2Z\chi\chi'\,\chi\chi'\,\chi + V\epsilon\chi'^{-1}(\;). &&
\end{flalign*}

If we wish to neglect all small quantities above a certain order
the present equations pave the way for suitably treating the
subject. I do not however propose to consider the problem here
as I have not considered it sufficiently to do it justice.]


\TCSection{16$a$}{Variation of temperature}{Variation of Temperature}

\FNParagraph{16$a$}{16a} The~$w$ which appears in the above sections is the same as
the~$w$ which occurs in Tait's \emph{Thermo-dynamics}, \S~209, and therefore
all the above work is true whether the solid experience change of
temperature or not. $w$~will be a function then of the temperature
as well as~$\psi$. To express the complete mathematical problem of
the physical behaviour of a solid we ought of course instead of the
above equations of equilibrium, to have corresponding equations of
motion, viz.\ equations~\Eref{III}{15$h$}, \Eref{III}{15$l$} and $\bigl($instead of~\Eref{III}{15$g$}$\bigr)$
\[
\EqnTag{III}{15$n$}
D \ddot{\rho}' = \mathbf{F} + \tau \Delta,
\]
where~$D$ is the original density of the solid at the point considered.
Further we ought to put down the equations of conduction
of heat and lastly equations \Eref{III}{16$f$} and~\Eref{III}{16$d$} below.

We do not propose to consider the conduction of heat, but
it will be well to shew how the thermo-dynamics of the present
question are treated by Quaternions.

%% -----File: 054.png---Folio 36-------
Let~$t$ be the temperature of the element which was originally
$\ds_0$ and $E\, \ds_0$ its intrinsic energy. Let
\[
\delta H = - S\, \delta\psi\, \zeta M \zeta + N\, \delta t,
\]
where $M, N$ are a linear self-conjugate function and a scalar
respectively, both functions of $\psi$ and~$t$. Here $\delta H\, \ds_0$ is the heat
required to be put into the element to raise its temperature by~$\delta t$
and its pure strain by~$\delta\psi$. Now when $t$~is constant $\delta H$ must be a
perfect differential so that we may put
\[
M = t \Pre{\psi}{\Dop} f,
\]
where $f$ is some function of $\psi$ and~$t$. Thus
\[
\EqnTag{III}{16}
\delta H = -tS\,\delta\psi\, \zeta \Pre{\psi}{\Dop} f\, \zeta
           + N\, \delta t.
\]
Now we have seen in \Secref{14} that the work done on the element
during the increment~$\delta\psi$, divided by~$\ds_0$
\[
= - m S\, \delta\psi\, \psi^{-1}\, \zeta\varpi\zeta.
\]
Thus by the first law of Thermo-dynamics
\begin{align*}
\delta E
&= J\, \delta H - m S\, \delta\psi\, \psi^{-1}\, \zeta\varpi\zeta \\
&= JN\, \delta t - Jt S\, \delta\psi\, \zeta \Pre{\psi}{\Dop} f\, \zeta
  - mS\, \delta\psi\, \psi^{-1}\,  \zeta\varpi\zeta,
\end{align*}
where $J$ is Joule's mechanical equivalent. Thus
\begin{flalign*}
\EqnTag{III}{16$a$}
&&  JN &= dE/dt,              && \\
%
\EqnTag{III}{16$b$}
&\text{and}& \tfrac{1}{2}m(\varpi\psi^{-1} &+ \psi^{-1}\varpi)
   = \Pre{\psi}{\Dop} w,      && \\
%
\EqnTag{III}{16$c$}
&\text{where}& w &= E - Jtf.  &&
\end{flalign*}

To apply the second law we go through exactly the same cycle
as does Tait in his \emph{Thermo-dynamics}, \S~209, viz.\
\[
(\psi,               t)
(\psi + \delta \psi, t)
(\psi + \delta \psi, t + \delta t)
(\psi,               t + \delta t)
(\psi,               t).
\]
We thus get\footnote{[By assuming from the second law that the work done \emph{by} the element in the
cycle, i.e.\ the sum of the works done by it during the first and third steps is~$J \delta t/t$
multiplied by the heat absorbed by the element in the third step. Note added,~1893.]}
\[
J \Pre{\psi}{\Dop} f
= -\tfrac{1}{2} m
   \left(
     \frac{d\varpi}{dt} \psi^{-1} + \psi^{-1} \frac{d\varpi}{dt}
   \right)
= -\frac{d \Pre{\psi}{\Dop} w}{dt}
= -\Pre{\psi}{\Dop}\frac{dw}{dt}
\]
\begin{flalign*}
\EqnTag{III}{16$d$}
&\text{or}& Jf &= - dw/dt &&
\end{flalign*}
the arbitrary function of~$t$ being neglected as not affecting any
physical phenomenon. Substituting for~$w$ from \equationref{III}{16$c$},
\[
\EqnTag{III}{16$e$}
Jt\, df/dt = dE/dt = JN
\]
by \equationref{III}{16$a$}. Thus from \equationref{III}{16}
\[
\EqnTag{III}{16$f$}
\delta H = t\, \delta f,
\]
%% -----File: 055.png---Folio 37-------
so that in elastic solids as in gases we have a convenient function
which is called the ``entropy''. Thus the intrinsic energy~$E\, \ds_{0}$,
the entropy~$f\, \ds_{0}$ and the stress~$\varpi$ have all been determined in
terms of one function~$w$ of $\psi$ and~$t$ which function is therefore in
this general mathematical theory supposed to be known.

If instead of regarding~$w$ (which with the generalised meaning
it now bears may still conveniently be called the ``potential energy''
per unit volume) as the fundamental function of the substance we
regard the intrinsic energy or the entropy as such it will be seen
that one other function of~$\psi$ must also be known. For suppose~$f$
the entropy be regarded as known. Then since $dw/dt = -Jf$
\[
\EqnTag{III}{16$g$}
w = W - J\smallint f\, dt,
\]
where the integral is any \emph{particular} one and~$W$ is a function of
$\psi$~only, supposed known. Again
\begin{flalign*}
&&
E &= w + Jtf && \\
%
\EqnTag{III}{16$h$}
&\text{or}&
E &= W + J(tf - \smallint f\, dt). &&
\end{flalign*}
Thus all the functions are given in terms of $f$ and~$W$. Similarly
if~$E$ be taken as the fundamental function
\begin{align*}
\EqnTag{III}{16$i$}
w &= t(W' - \smallint E\, dt/t^2), \\
%
\EqnTag{III}{16$j$}
Jf &= E/t + \smallint E\, dt/t^2 - W',
\end{align*}
where as before the integral is some particular one and~$W'$ is a
function of $\psi$~only.


\Section{17}{Small strains}

\Paragraph{17} We now make the usual assumption that the strains are
so small that their coordinates can be neglected in comparison
with ordinary quantities such as the coefficients of the linear
vector function $\Pre{\psi}{\Dop} w$. We can deduce this case from the above
more general results.

To the order considered $q = 1$ so that by \equationref{III}{10} \Secref{14} above
\[
\overline{\phi} = \varpi.
\]
We shall use the symbol~$\varpi$ rather than~$\overline{\phi}$ for the same reasons
explained in \Secref{13} above as induce us to use~$\psi$ rather than~$\overline{\chi}$.

Remembering that $\omega_0$ and~$\omega$ are now the same and that~$\psi$ may
be put $= 1$ so that
$V\zeta \Pre{\psi}{\Dop} w\,\psi\zeta = V\zeta \Pre{\psi}{\Dop} w\,\zeta = 0$ and therefore~$\theta$ of
\equationref{III}{12} $= 0$, we have from \equationref{III}{13},
\[
\EqnTag{III}{17}
\varpi\omega = \Pre{\psi}{\Dop} w\,\omega.
\]
%% -----File: 056.png---Folio 38-------

Of course we do not require to go through the somewhat complicated
process of \Secref{14}, \Secref{15} to arrive at this result. In fact in
\equationref{III}{6} \Secref{14}, we may put $m = 1$ and $\Nabla{\rho+\eta} = \nabla$ so that
\[
\delta w = -S\, \delta\eta_1\, \overline{\phi}\nabla_1
         = -S\, \delta\eta_1\, \varpi\nabla_1,
\]
and therefore by \Secref{3} above
\[
\delta w = -S\, \delta\overline{\chi}\, \zeta\varpi\zeta
         = -S\, \delta\psi\, \zeta\varpi\zeta.
\]
But by \Secref{5} above
\[
\delta w = -S\, \delta\psi\, \zeta \Pre{\psi}{\Dop} w\, \zeta,
\]
and therefore by \Secref{4} we get \equationref{III}{17}.

\Paragraph{18} It is convenient here to slightly change the notation.
For $\psi$, $\chi$ we shall now substitute $\psi+1$, $\chi+1$ respectively. This
leads to no confusion as will be seen.

With this notation the strain being small the stress is linear
in~$\psi$ i.e.\ $\Pre{\psi}{\Dop} w$ is linear and therefore~$w$ quadratic. Now for any
such quadratic function
\begin{align*}
\EqnTag{III}{18}
w  &= -S\psi\, \zeta \Pre{\psi}{\Dop} w\,\zeta/2, \\
\intertext{for we have by \Secref{5},}
dw &= -S\, d\psi\, \zeta \Pre{\psi}{\Dop} w\, \zeta.
\end{align*}
Put now $\psi = n\psi'$. Then because $\Pre{\psi}{\Dop} w$ is linear in~$\psi$
\[
\Pre{\psi}{\Dop} w = n\Pre{\psi}{\Dop}' w,
\]
where $\Pre{\psi}{\Dop}' w$ is put for the value of $\Pre{\psi}{\Dop} w$ when the coordinates
of~$\psi'$ are substituted for those of~$\psi$. Thus keeping $\psi'$~constant and
varying~$n$,
\[
dw = -n\,dn\,S\psi'\, \zeta \Pre{\psi}{\Dop}' w\,\zeta,
\]
whence integrating from $n = 0$ to $n = 1$ and changing $\psi'$~into $\psi$
we get \equationref{III}{18}.

From \equationref{III}{18} we see that for small strains we have
\[
\EqnTag{III}{19}
w = -S\psi\, \zeta\varpi\zeta/2.
\]

Now $w$ is quadratic in~$\psi$ and therefore also quadratic in~$\varpi$, so
that regarding~$w$ as a function of~$\varpi$ we have as in \equationref{III}{18}
\[
w = -S\varpi\zeta \Pre{\varpi}{\Dop}  w\,\zeta/2,
\]
so that by \equationref{III}{19} and \Secref{4} above
\[
\EqnTag{III}{20}
\psi = \Pre{\varpi}{\Dop} w.
\]

All these results for small strains are well-known in their Cartesian
form, but it cannot be bias that makes these quaternion
%% -----File: 057.png---Folio 39-------
proofs appear so much more natural and therefore more simple
and beautiful than the ordinary ones.

\Paragraph{19} Let us now consider (as in \Secref{16} is really done) $w$ as a
function of the displacement. Now $w$ is quadratic in~$\psi$, and~$\psi$ is
linear and symmetrical in $\nabla_1$ and~$\eta_1$. In fact from \equationref{III}{3}
\Secref{12} above, remembering that the~$\overline{\chi}$ of that equation is our present
$\psi + 1$, we have
\[
2 \psi \omega = -\eta_1\, S\omega\, \nabla_1 - \nabla_1\, S\omega\, \eta_1.
\]
Therefore we may put
\[
\EqnTag{III}{21}
w = w(\eta_1, \nabla_1, \eta_2, \nabla_2),
\]
where $w(\alpha, \beta, \gamma, \delta)$ is linear in each of its constituents, is symmetrical
in $\alpha$ and~$\beta$, and again in $\gamma$ and~$\delta$, and is also such that
the pair $\alpha$, $\beta$ and the pair $\gamma$, $\delta$ can be interchanged. [This last
statement can be \emph{made} true if not so at first, by substituting for
$w(\alpha, \beta, \gamma, \delta)$,
$w(\alpha, \beta, \gamma, \delta)/2 + w(\gamma, \delta, \alpha, \beta)/2$ as this does not affect
\equationref{III}{21}.] Such a function can be proved to involve
21~independent scalars, which is the number also required to determine
an arbitrary quadratic function of~$\psi$, since~$\psi$ involves six scalars.

Thus we have the two following expressions for $\delta w$, which we
equate
\begin{align*}
-S\, \delta\psi\, \zeta\varpi\zeta
&= w (\delta\eta_1, \nabla_1, \eta_2, \nabla_2)
 + w (\eta_1, \nabla_1, \delta\eta_2, \nabla_2) \\
&= 2w(\delta\eta_1, \nabla_1, \eta_2, \nabla_2),
\end{align*}
\Erratum{err39a}%[** PP: Added closing square brace at end of footnote]
or\footnote{[Note added, 1892. Better thus:---by \Secref{3} above,
\[
-S\, \delta\psi\, \zeta\varpi\zeta
=  2w (\delta\psi\,\zeta, \zeta, \eta_1, \nabla_1)
= -2S\zeta_1\, \delta\psi\, \zeta w (\zeta_1, \zeta, \eta_1, \nabla_1)
\]
therefore by \Secref{4},\hfil
$\varpi\omega = 2\zeta w(\zeta, \omega, \eta_1, \nabla_1)$\hfil \\
for $\zeta w(\zeta, \omega, \eta_1, \nabla_1)$ regarded as a function
of $\omega$ is clearly self conjugate.]} %[** Closing brace]
by \Secref{3} above,
\[
-S\,\delta\eta_1\, \varpi\nabla_1
    = 2w(\delta\eta_1, \nabla_1, \eta_2, \nabla_2).
\]
Now let us put $\delta\eta = \omega'S\omega\rho$ where $\omega'$ and~$\omega$ are arbitrary constant
vectors. We thus get
\[
S\omega'\, \varpi\omega
    = -2w(\omega', \omega, \eta_1, \nabla_1)
    =  2S\omega'\,\zeta w(\zeta, \omega, \eta_1, \nabla_1).
\]
Whence since $\omega'$ is quite arbitrary,
\[
\EqnTag{III}{22}
\varpi\omega = 2\zeta w(\zeta, \omega, \eta_1, \nabla_1).
\]

The statical problem can now be easily expressed. As we saw
in \Secref{14}, equations \Eref{III}{7} and~\Eref{III}{8}, it is simply
\[
\EqnTag{III}{22$a$}
\mathbf{F} + V \nabla \mathbf{M}/2 + \varpi\Delta = 0,
\]
%% -----File: 058.png---Folio 40-------
throughout the mass; and at the surface
\[
\EqnTag{III}{22$b$}
\textbf{F}_S - VU\, d\Sigma\, \textbf{M}/2 - \varpi U\, d\Sigma = 0,
\]
where $\mathbf{F}$, $\mathbf{M}$ are the given external force and couple per unit
volume and $\mathbf{F}_S$ is the given external surface traction per unit
surface. Substituting for~$\varpi$ from \equationref{III}{22}
\[
\EqnTag{III}{23}
\left.
\begin{aligned}
\mathbf{F} + V \nabla \mathbf{M}/2
    + 2\zeta w(\zeta, \Delta, \eta_1, \nabla_1) = 0\phantom{.} \\
\mathbf{F}_S - V U\, d\Sigma\,\mathbf{M}/2
    - 2\zeta w(\zeta, U\,d\Sigma, \eta_1, \nabla_1) = 0.
\end{aligned}
\right\}
\]


\Section{20}{Isotropic Bodies}

\Paragraph{20} The simplest way to treat these bodies is to consider the
(linear) relations between $\varpi$ and~$\psi$.

In the first place notice that~$\psi$ can always be decomposed into
three real elongations (contractions being of course considered as
negative elongations). Thus $i$ being the unit vector in the direction
of such an elongation,
\[
\psi\omega = -\Sigma\, ei\, Si\,\omega.
\]
The elongation $-ei\,Si\,\omega$ will cause a stress symmetrical about the
vector~$i$, i.e.\ a tension~$Ae$ in the direction of~$i$ and a pressure~$Be$ in
all directions at right angles; $A$ and~$B$ being constants (on account
of the linear relation between $\varpi$ and~$\psi$) independent of the direction
of~$i$ (on account of the isotropy of the solid). This stress
may otherwise be described as a tension $(A + B)e$ in the direction
of~$i$ and a hydrostatic pressure~$Be$. Thus
\begin{align*}
\varpi\omega &= -(A+B)\Sigma\, ei\, Si\,\omega - B\omega\,\Sigma e \\
             &=  (A+B)\psi\omega + B\omega\, S\zeta\, \psi\zeta.
\end{align*}

To obtain the values of $A$ and~$B$ in terms of Thomson and
Tait's coefficients $k$ and~$n$ of cubical expansion and rigidity respectively;
first put
\begin{align*}
\psi\omega = e\omega\quad
    &\text{and}\quad \varpi\omega = 3k\, e\omega, \\
%
\intertext{and then put}
\psi\omega = V\lambda\omega\mu\quad
    &\text{and}\quad \varpi\omega = 2nV\lambda\omega\mu,
\end{align*}
$\lambda$ and~$\mu$ being any two vectors perpendicular to each other. We
thus get
\[
A + B = 2n,\qquad B = -(k-2n/3),
\]
whence
\[
\EqnTag{III}{24}
\varpi\omega = 2n\,\psi\omega - (k-2n/3)\omega S\, \zeta\psi\zeta.
\]
%% -----File: 059.png---Folio 41-------
From this we have
\[
\psi\omega = \varpi\omega/2n + \omega S\, \zeta\psi\zeta(k - 2n/3)/2n,
\]
but from the same equation
\begin{gather*}
S\,\zeta\varpi\zeta = 3 k\, S\, \zeta \psi \zeta; \\
\therefore\qquad
\psi \omega = \frac{1}{2n} \varpi \omega
    + \left(\frac{1}{6n} - \frac{1}{9k}\right) \omega S\, \zeta\varpi\zeta.
\EqnTag{III}{25}
\end{gather*}
\Equationref{III}{24} gives stress in terms of strain and~\Eref{III}{25} the converse.

\Paragraph{21} We can now give the various useful forms of~$w$ for
isotropic bodies for from \equationref{III}{19} \Secref{18},
\[
w = - S\psi\, \zeta\varpi\zeta/2.
\]
Therefore from equations \Eref{III}{24} and~\Eref{III}{25} respectively
\begin{align*}
\EqnTag{III}{26}
w &= -n(\psi \zeta)^2
    + \tfrac{1}{2}(k-2n/3) S\, \zeta_1\psi\zeta_1\, S\, \zeta_2\psi\zeta_2, \\
%
\EqnTag{III}{27}
w &= -\frac{1}{4n}(\varpi \zeta)^2
     - \tfrac{1}{2}\left(\frac{1}{6n} - \frac{1}{9k}\right)
     S\, \zeta_1\varpi\zeta_1\, S\, \zeta_2\varpi\zeta_2.
\end{align*}
Therefore again from \Secref{3} above and from \equationref{III}{26},
\[
w = -n S\eta_1 \psi \nabla_1 + \tfrac{1}{2}(k - 2n/3)(S \nabla \eta)^2
\]
or
\[
\EqnTag{III}{28}
\begin{split}
2w(\eta_1, \nabla_1, \eta_2, \nabla_2)
    = n S \nabla_1 \eta_2\, S \nabla_2 \eta_1
    + n S \nabla_1 \nabla_2\, S \eta_1 \eta_2 \\
\shoveright{+ (k-2n/3) S \nabla_1 \eta_1 S \nabla_2 \eta_2.}
\end{split}
\]
Hence from \equationref{III}{22}
\[
\EqnTag{III}{29}
\varpi\omega = - n S \omega \nabla\centerdot \eta
               - n \nabla_1 S \eta_1 \omega
               - (m-n) \omega S \nabla \eta,
\]
where $m$ is put for $k + n/3$. This last could have been deduced at
once from \equationref{III}{24} by substituting for~$\psi \omega$.

Thus the \hyperref[eqn:III.23]{equations~(23)} for the statical problem are
\begin{align*}
\EqnTag{III}{30}
\mathbf{F} + V \nabla \mathbf{M}/2
&= n \nabla^2 \eta + m \nabla S \nabla \eta, \\
%
\EqnTag{III}{31}
-\mathbf{F}_S + V U\, d \Sigma\, \mathbf{M}/2
&= n S U\, d \Sigma\, \nabla \centerdot \eta
 + n \nabla_1 S \eta_1 U\, d \Sigma \\
&\quad + (m-n) U\, d \Sigma\, S \nabla \eta.
\end{align*}

We now proceed to apply these results for small strains in
isotropic bodies to particular cases. These particular cases have
all been worked out by the aid of Cartesian Geometry and they
are given to illustrate the truth of the assertion made in the
Introduction that the consideration of \emph{general problems} is made
simpler by the use of Quaternions instead of the ordinary methods.
%% -----File: 060.png---Folio 42-------


\TCSection{22}{Particular integral of the equation of equilibrium}
  {Particular integral of equation {\upshape(30)}\protect\footnotemark}

\footnotetext{[Note added, 1892. For a neater quaternion treatment of this
problem see \textit{Phil.~Mag.}\ June,~1892, p.~493.]}

\Paragraph{22} Since from \equationref{III}{30} ($\mathbf{F}$~being put for simplicity instead
of $\mathbf{F} + V\nabla \mathbf{M}/2$) we have
\[
n\nabla^2\eta = \mathbf{F} - m\nabla S\nabla\eta
\]
we obtain as a particular case by equations \Eref{II}{18} and~\Eref{II}{19} \Secref{10},
\[
4\pi n\eta = \smalliiint u(\mathbf{F} - m\nabla S \nabla \eta)\, \ds,
\]
where $u$ has the meaning explained in \Secref{9}, and the volume integral
extends over any portion (say the whole) of the body we may
choose to consider. To express $\smalliiint u \nabla S \nabla \eta\, \ds$ as a function of~$\textbf{F}$ put
in this term $u = -\frac{1}{2}U\rho_1\nabla_1$ where~$\rho$ is taken for the $\rho-\rho'$ of \Secref{9},
and apply \equationref{II}{9} \Secref{6}. Thus
\begin{align*}
4\pi n \eta
&= \smalliiint u \mathbf{F}\, \ds
    + \tfrac{1}{2} m \smalliiint U \rho_1 \nabla_1 \nabla S\nabla\eta\, \ds \\
&= \smalliiint u \mathbf{F}\, \ds
    - \tfrac{1}{2} m \smalliiint U \rho \nabla^2 S \nabla \eta \, \ds
    + \text{ a surf.\ int.} \\
&= \smalliiint u \mathbf{F}\, \ds
    - \dfrac{m}{2(m+n)} \smalliiint U \rho S \nabla \mathbf{F}\, \ds
    + \text{ the surf.\ int.}
\end{align*}
for by \equationref{III}{30} $S\nabla \mathbf{F} = (m + n)\nabla^2 S\nabla \eta$. Now (in order to
get rid of any infinite terms due to any discontinuity in~$\mathbf{F}$) apply
\equationref{II}{9} \Secref{6} to the second volume integral. Thus
\[
4\pi n \eta = \smalliiint u \mathbf{F}\, \ds
    + \frac{m}{2(m+n)} \smalliiint S \mathbf{F} \nabla \centerdot U \rho\, \ds
    + \text{ a surf.\ int.}
\]
The surface integral may be neglected as we may thus verify.
Call the volume integral $4\pi n \eta'$. Thus
\begin{gather*}
4\pi n \nabla^2 \eta' = 4 \pi \mathbf{F}
    + \dfrac{m}{2(m+n)} \smalliiint
        S\mathbf{F} \nabla \centerdot \nabla^2 U \rho\, \ds \\
4\pi m \nabla S \nabla \eta' =
    \dfrac{m}{n} \smalliiint S \mathbf{F} \nabla \centerdot \nabla u\, \ds
    + \dfrac{m^2}{2n(m+n)}\smalliiint
        S \mathbf{F} \nabla \centerdot \nabla S \nabla U \rho\, \ds,
\end{gather*}
so that putting $\nabla U \rho = - 2u$ we get
\[
n \nabla^2 \eta' + m \nabla S \nabla \eta' \equiv \mathbf{F},
\]
whence we have as a particular solution of \equationref{III}{30} $\eta = \eta'$ or
\[
\EqnTag{III}{32}
\eta = \dfrac{1}{4 \pi n} \smalliiint u \mathbf{F}\, \ds
     + \dfrac{m}{8\pi n (m+n)} \smalliiint S \mathbf{F} \nabla \centerdot U \rho\, \ds.
\]
%% -----File: 061.png---Folio 43-------

This is generally regarded as a solution of the statical problem
for an infinite isotropic body. In this case some law of convergence
must apply to~$\mathbf{F}$ to make these integrals convergent.
Thomson and Tait (\textit{Nat.\ Phil.}\ \S~730) say that this law is that~$\mathbf{F}r$
converges to zero at infinity. This I think can be disproved by
an example. Put\footnote{%
[Note added, 1892. This is not legitimate since it makes $\mathbf{F} = \infty$ for $r = 0$. The
reasoning is rectified in the \textit{Phil.\ Mag.}\ paper just referred to by putting $\mathbf{F} = 0$ from
$r = 0$ to $r = b$ and $\mathbf{F}r = r^{-a}\lambda$ from $r = b$ to $r = \infty$.]} $\mathbf{F}r = r^{-a}\lambda$ where~$\lambda$ is a constant vector and~$a$ a
positive constant less than unity. \Equationref{III}{32} then gives for the
displacement at the origin due to the part of the integral extending
throughout a sphere whose centre is the origin and radius~$R$
\[
\eta = \dfrac{m+3n}{3n(m+n)} \dfrac{R^{1-a}}{1-a} \lambda.
\]
Putting $R = \infty$, $\eta$ also becomes~$\infty$. The real law of convergence
does not seem to me to be worth seeking as the practical utility
of \equationref{III}{32} is owing to the fact that it is a particular integral.

The present solution of the problem has only to be compared
with the one in Thomson and Tait's \textit{Nat.\ Phil.}\ \S\S~730--1 to see the
immense advantage to be derived from Quaternions.

It is easy to put our result in the form given by them. We
have merely to express $S\mathbf{F}\nabla \centerdot U\rho$ in terms of $\mathbf{F}$ and~$r^2 S\mathbf{F}\nabla \centerdot \nabla u$
where~$r$ is put for the reciprocal of~$u$. Noting that
\[
\nabla u = -u^3\rho,\quad U\rho = u\rho,\quad
S\mathbf{F}\nabla \centerdot \rho = -\mathbf{F},
\]
we have at once
\begin{align*}
S\mathbf{F}\nabla \centerdot U\rho
&= -u  \mathbf{F} -  u^3\rho S\mathbf{F}\rho  \\
S\mathbf{F}\nabla \centerdot \nabla u
&=  u^3\mathbf{F} + 3u^5\rho S\mathbf{F}\rho
\end{align*}
therefore eliminating $\rho S\mathbf{F}\rho$,
\begin{gather*}
S\mathbf{F}\nabla \centerdot U\rho
= -u\mathbf{F}2/3 - r^2 S\mathbf{F}\nabla \centerdot \nabla u/3,  \\
%[** PP: Not pushing to left margin]
\therefore\qquad \eta
= \{ 24\pi n(m+n) \}^{-1} \smalliiint \ds\,
\{ 2(2m+3n)u\mathbf{F} - mr^2 S\mathbf{F}\nabla \centerdot \nabla u \},
\EqnTag{III}{33}
\end{gather*}
which is the required form.

\Paragraph{23} Calling the particular solution~$\eta'$ as before and putting
\[
\eta = \eta' + \eta''
\]
the statical problem is reduced to finding $\eta''$ to satisfy
\[
n\nabla^2 \eta'' + m\nabla S\nabla \eta'' = 0
\]
%% -----File: 062.png---Folio 44-------
and the surface equation either
\begin{flalign*}
&&              \eta' + \eta'' &= \text{ given value},  && \\
&\text{i.e. }&  \eta'' &= \text{ given value,}          && \\
&\text{or }  & \varpi U d \Sigma &=\text{ given surface traction,} &&
\end{flalign*}
i.e.\ by \equationref{III}{29} \Secref{21} above,
\[
nSU\, d\Sigma \nabla\centerdot \eta''
    + n \nabla_1 S \eta_1'' U\, d \Sigma
    + (m-n) U\, d\Sigma S\nabla \eta''
  = \text{ known value.}
\]

This general problem for the spherical shell, the only case
hitherto solved, I do not propose to work out by Quaternions, as
the \emph{methods} adopted are the same as those used by Thomson and
Tait in the same problem. But though each step of the Cartesian
proof would be represented in the Quaternion, the saving in
mental labour which is effected by using the peculiarly happy
notation of Quaternions can only be appreciated by him who has
worked the whole problem in both notations. The only remark
necessary to make is that we may just as easily use vector, surface
or solid, harmonics or indeed quaternion harmonics as ordinary
scalar harmonics.


\Section{24}{Orthogonal coordinates}

\Paragraph{24} It is usual to find what equations~\Eref{III}{22$a$} of \Secref{19} and~\Eref{III}{3}~of
\Secref{12} become when expressed in terms of any orthogonal coordinates.
This can be done much more easily by Quaternions than
Cartesian Geometry. Compare the following investigation with
the corresponding one in Ibbetson's \textit{Math.\ Theory of Elasticity},
Chap.~V.

Let $x$, $y$, $z$ be any orthogonal coordinates, i.e.\ let $x = \text{const.}$,
$y = \text{const.}$, $z = \text{const.}$, represent three families of surfaces cutting
everywhere at right angles. Particular cases are of course the
ordinary Cartesian coordinates, the spherical coordinates $r$, $\theta$, $\phi$
and the cylindrical coordinates $r$, $\phi$, $z$. Let $D_x$, $D_y$, $D_z$ stand for
differentiations \emph{per unit length} perpendicular to the three coordinate
surfaces and let $\lambda$, $\mu$, $\nu$ be the unit vectors in the corresponding
directions. Thus
\[
\nabla = \lambda D_x + \mu D_y + \nu D_z.
\]
Thus, using the same system of suffixes for the $D$'s as was explained
in connection with~$\nabla$ in \Secref{1},
\[
\EqnTag{III}{34}
\phi \Delta = D_{x1}\phi_1\lambda
            + D_{y1}\phi_1\mu
            + D_{z1}\phi_1\nu,
\]
or
\[
\EqnTag{III}{35}
\phi \Delta = D_x(\phi \lambda) + D_y(\phi \mu) + D_z(\phi \nu)
            - \phi(D_x\lambda + D_y\mu + D_z\nu).
\]
%% -----File: 063.png---Folio 45-------

\Paragraph{25} Now to put \equationref{III}{22$a$} \Secref{19} into the present coordinates
all that is required is to express~$\varpi\Delta$ in terms of those coordinates.
Let the coordinates of~$\varpi$ be~$PQRSTU$. Thus from
\equationref{III}{35} we have
\[
\begin{split}
\varpi\Delta
  &= D_x(P\lambda + U\mu + T\nu)
   + D_y(U\lambda + Q\mu + S\nu)
   + D_z(T\lambda + S\mu + R\nu) \\
   &\qquad\qquad - \varpi(D_x\lambda + D_y\mu + D_z\nu).
\end{split}
\]
The first thing then is to find $D_x\lambda$, $D_x\mu$, $D_y\lambda$~\&c.

Let $p_2$, $p_3$ be the principal curvatures normal to $x=\text{const.}$, i.e.\
(by a well-known property of orthogonal surfaces) the curvatures
along the lines of intersection of $x=\text{const.}$, with $z=\text{const.}$, and
$y=\text{const.}$\ $p_2$, $p_3$ will be considered positive when\footnote{%
[Note added, 1892. This is contrary to the usual convention.]}
the positive
value of~$dx$ is on the convex side of the corresponding curvatures.
Similarly for $q_3q_1r_1r_2$. Thus for the coordinates $r$, $\theta$, $\phi$;
\[
p_2 = p_3 = 1/r,\quad
q_3 = \cot\theta/r,\quad
q_1 = r_1 = r_2 = 0.
\]
Again for $r$, $\phi$, $z$; $p_2 = 1/r$ and the rest are each zero.

With these definitions we see geometrically that
\[
\EqnTag{III}{36}
D_x\lambda = -\mu q_1 - \nu r_1, \quad
D_x\mu = \lambda q_1, \quad
D_x\nu = \lambda r_1.
\]
Similarly for $D_y\lambda$, $D_y\mu$, $D_z\lambda$~\&c. Thus
\begin{align*}
\varpi\Delta
  &= \lambda(D_x P + D_y U + D_z T) + \mu(\;) + \nu(\;)  \\
  &+ \lambda(-Qp_2 - Rp_3 + Tr_1 + Uq_1) + \mu(\;) + \nu(\;)  \\
  &+ \lambda\{ (p_2+p_3)P + (q_3+q_1)U + (r_1+r_2)T \}
    + \mu\{\;\} + \nu\{\;\},
\end{align*}
or
\begin{align*}
\varpi\Delta
  &= \lambda\{D_x P + D_y U + D_z T + P(p_2+p_3) - Qp_2 - Rp_3 \notag \\
\EqnTag{III}{37}
  &+ T(2r_1 + r_2) + U(q_3 + 2q_1)\} + \mu\{\;\} + \nu\{\;\}.
\end{align*}

\Paragraph{26} The other chiefly useful thing in transformation of coordinates
in the present subject is the expression for the strain
function~$\psi$ in terms of the coordinates of displacement. Let
$u$, $v$, $w$ be these coordinates. Now by \equationref{III}{3} \Secref{12}, remembering
(\Secref{18}) that $\psi = \overline{\chi} - 1$ we have
\[
-2\psi\omega = S\omega\nabla \centerdot \eta + \nabla_1 S\omega\eta_1,
\]
\begin{flalign*}
&\text{whence} &
-2\psi\lambda
    = \lambda S\lambda D_x\eta
    + \mu     S\lambda D_y\eta
    + \nu     S\lambda D_z\eta - D_x\eta. &&
\end{flalign*}
\begin{flalign*}
&\text{But} &
D_x\eta &= \lambda D_x u
         + \mu     D_x v
         + \nu     D_x w
         + uD_x\lambda
         + vD_x\mu
         + wD_x\nu  && \\
&& &= \lambda(D_x u + vq_1 + wr_1) + \mu(D_x v - uq_1) + \nu(D_x w - ur_1). &&
\end{flalign*}
%% -----File: 064.png---Folio 46-------
Similarly
\begin{align*}
D_y\eta &= \lambda(D_y u-vp_2) + \mu(D_y v+wr_2+up_2) + \nu(D_y w-vr_2) \\
D_z\eta &= \lambda(D_z u-wp_3) + \mu(D_z v-wq_3) + \nu(D_z w+up_3+vq_3) \\
\therefore\qquad
2\psi\lambda &=
    2\lambda(D_x u + v q_1 + wr_1)
   + \mu    (D_y u + D_x v - uq_1 - vp_2) \\
%
\EqnTag{III}{38}
& \phantom{{}=2\lambda(D_x u + v q_1 + wr_1)}
   + \nu    (D_z u + D_x w - ur_1 - wp_3).
\end{align*}
But with Thomson and Tait's notation for pure small strain
\[
2\psi\lambda = 2\lambda e + \mu c + \nu b,
\]
\[
\EqnTag{III}{39}
\left.
\begin{aligned}
\therefore\qquad
\footnotemark
e &= D_xu + vq_1 + wr_1 \\
f &= D_yv + wr_2 + up_2 \\
g &= D_zw + up_3 + vq_3 \\
a &= D_yw + D_zv - wq_3 - vr_2 \\
b &= D_zu + D_xw - ur_1 - wp_3 \\
c &= D_xv + D_yu - vp_2 - uq_1.
\end{aligned}
\right\}
\]
\footnotetext{%
[Note added, 1892. In the \textit{Phil.~Mag.}\ June, 1892, p.~488, there is a mistake in
the equation just preceding equation~31 and there are two mistakes in equation~31.
In the first of these
$I(2D \Pre{\xi}{u} - \Pre{\eta}{\varpi} \Pre{\zeta}{v}
                   - \Pre{\zeta}{\varpi} \Pre{\eta}{w})$ should be
$2I(D \Pre{\xi}{u} - \Pre{\eta}{\varpi} \Pre{\zeta}{v}
                   - \Pre{\zeta}{\varpi} \Pre{\eta}{w})$. In
\equationref{III}{31} all the 2's should be dropped.]}

Thus we have $efgabc$ in terms of the displacement and we
have already in \equationref{III}{24} \Secref{20}, which expresses~$\varpi$ in terms of~$\psi$,
found the values of $PQRSTU$ in terms of $e$,~\&c. Finally the
expression in \equationref{III}{37} for $\varpi\Delta$ gives us the equations of equilibrium
in terms of~$P$, \&c. Thus we have all the materials for
considering any problem with the coordinates we have chosen.

All these results can be at once applied to spherical and cylindrical
coordinates, but as this has nothing to do with our present
purpose---the exemplification of Quaternion methods---we leave
the matter here.

Let us as an example of particular coordinates to which this
section forms a suitable introduction consider St Venant's Torsion
Problem by means of cylindrical coordinates.


\TCSection{27}{Saint-Venant's torsion problem}{Saint-Venant's Torsion Problem}

\Paragraph{27} In this problem we consider the equilibrium of a cylinder
with any given cross-section, subjected to end-couples, but to no
bodily forces and no stress on the curved surface.

%% -----File: 065.png---Folio 47-------

We shall take $r, \phi, z$ as our coordinates, the axis of~$z$ being
parallel to the generating lines of the cylinder. Let $\lambda, \mu, \nu$ be
the unit vectors in the directions of $dr, d\phi, dz$ respectively and let
\[
\eta = u\lambda + v\mu +w \nu
\]
as before.

We shall follow Thomson and Tait's lines of proof---i.e.\ we
shall first find the effect of a simple torsion and then add another
displacement and so try to get rid of stress on the curved surface.

Holding the section $z = 0$ fixed let us give the cylinder a small
torsion of magnitude~$\tau$, i.e.\ let us put
\[
\EqnTag{III}{40}
\eta = \tau z r \mu,
\]
for all points for which $\tau z$ is small.

The practical manipulation of such expressions as this is almost
always facilitated by considering the general value of $Q(\nabla_1, \eta_1)$
where~$Q$ is any function linear in each of its constituents. Thus
in the present case
\[
Q (\nabla_1, \eta_1)
  = \tau \bigl\{ zQ (\lambda, \mu) - zQ(\mu, \lambda) + rQ(\nu,\mu)\bigr\}.
\]
[If~$Q$ is symmetrical in its constituents, e.g.\ in the case of stress
below this reduces to the simple form $Q(\nabla_1, \eta_1) = \tau r Q(\nu, \mu)$.]
From this we at once see that~$\eta$ satisfies the equation of internal
equilibrium
\[
n \nabla^2 \eta + m \nabla S \nabla \eta = 0,
\]
for putting $Q(\alpha, \beta) = \alpha\beta$
\[
\nabla\eta = \tau(2 z \nu - r\lambda )  = \tau \nabla(z^2-r^2/2)
\]
so that both $S\nabla \eta$ and~$\nabla^2\eta = 0$.

Again the value for~$Q$ at once gives us the stress for
\begin{flalign*}
&& \varpi\omega = - n S\omega \nabla
& \centerdot\eta - n \nabla_1 S \omega \eta_1 - (m-n)\omega S \nabla \eta, && \\
%
\EqnTag{III}{41}
&\text{or}& \varpi\omega
&= - n\tau r (\mu S \omega \nu +\nu S\omega \mu), &&
\end{flalign*}
which is a shearing stress $n\tau r$ on the interfaces perpendicular to
$\mu$ and~$\nu$.

Putting $\omega = {}$ the unit normal of the curved surface we have for
the surface traction
\[
\varpi\omega = - n\tau r \nu\, S\omega \mu.
\]

In the figure let the plane of the paper be $z = 0$, $O$ the origin,
$P$~a point on the curved surface and~$OM$ the perpendicular from~$O$
%% -----File: 066.png---Folio 48-------
on the tangent at~$P$. Thus $-rS\omega\mu = OP \cos OPM = PM$, $PM$
being reckoned positive or negative according as it is in the
positive or negative direction of rotation round~$Oz$. Thus we see
that the surface traction is parallel to~$Oz$ and~$= n \tau PM$.
%[Illustration]
\begin{center}
  \includegraphics{./images/cylinder.pdf}
\end{center}

Hence in the case of a circular cylinder a torsion round the
axis satisfies all the conditions of our original problem, but this is
true in no other case.

The surface traction at any point on the plane ends necessary
to produce this strain is $\varpi \nu = n \tau r\mu$ by \equationref{III}{41} so that its
moment round the origin is $n \tau \smalliint r^2\, dA$, where~$dA$ is an element of
area and the integral extends over the whole cross-section.

\Paragraph{28} Let us now assume a further displacement
\[
\EqnTag{III}{42}
\eta = w\nu,
\]
where $w$ is a function of $r, \phi$~only, and let us try to determine~$w$
so that there is still internal equilibrium and so that the stress on
the curved surface due to~$w$ shall neutralise the surface traction
already considered.

In the present case
\[
Q(\nabla_1,\eta_1) = Q(\nabla w, \nu).
\]
Thus $S \nabla \eta = 0$ (since~$\nu$ is perpendicular to~$\nabla w$) and therefore the
equation of internal equilibrium gives
\[
\nabla^2 w = 0.
\]
\begin{flalign*}
&\text{Again}&
\varpi \omega
&= -n S \omega \nabla \centerdot \eta - n \nabla_1 S \omega \eta_1, && \\
%
% [** N.B. No Equations 44 or 45 in this section]
\EqnTag{III}{43}
&\text{or}&
\varpi \omega
&= -n (\nu S \omega \nabla w + \nabla w S \omega \nu), &&
\end{flalign*}
a shear $= nT\nabla w$ on the interfaces perpendicular to $\nabla w$ and~$\nu$.
%% -----File: 067.png---Folio 49-------
Thus putting~$\omega$ for the unit normal to the curved surface, the
% [** PP: Keeping ``neutralize'']
present surface traction will neutralize the former if
\begin{flalign*}
&& S \omega \nabla w
&= - \tau r S \omega \mu = -\tau S \omega \nu (\lambda r) && \\
&&
&= \tfrac{1}{2} \tau S \nu \omega \nabla (r^2), && \\
&\text{i.e.}& dw &/ dn = d (\tau r^2 /2) / ds, &&
\end{flalign*}
where $d/dn$ represents differentiation along the normal outwards
and~$d/ds$ differentiation along the positive direction of the bounding
curve.

We leave the problem here to the theory of complex variables
and Fourier's Theorem.

% [** PP: Changing opening ``='' to ``is equal to'' twice;
% was ``...on the plane end ${} = \varpi \nu...'' and
% ``...total couple ${} = -n\smallint ...$'', which both reads
% awkwardly and breaks badly]
Observe however that the surface traction at any point on the
plane end is equal to\Erratum{err49a}
$\varpi \nu = n \nabla w$ by \equationref{III}{43}, and therefore that the
total couple is equal to
$- n \smalliint r S \mu \nabla w\, dA = n \smalliint (dw/d\phi)\, dA$. This leads to the
usual expression for torsional rigidity.


\Section{29}{Wires}

\Paragraph{29} In the following general treatment of Wires some of the
processes are merely Thomson and Tait's translated into their
shorter Quaternion forms; others are quite different. The two
will be easily distinguished by such as are acquainted with
Thomson and Tait's \emph{Nat.\ Phil.}

The one thing to be specially careful about is the notation and
its exact meaning. This meaning we give at the outset.

The wires we consider are not necessarily naturally straight
but we assume some definite straight condition of the wire as the
``geometrically normal'' condition.

The variable in terms of which we wish to express everything
is~$s$ the distance along the wire from some definite point on it.

Any element of the wire, since it is only \emph{slightly} strained, may
be assumed to have turned as a rigid body from its geometrically
normal position. This rotation is expressed as usual (Tait's
\emph{Quaternions}, \S~354) by the quaternion~$q$; the axis of~$q$ being the
axis of rotation, and the angle of~$q$, half the angle turned through.

$\omega$ is taken so that the rate of this turning per unit length of
the wire is $q \omega q^{-1}$ so that~$\omega$ is the rate of turning per unit of length
when the whole wire is moved as a rigid body so as to bring the
%% -----File: 068.png---Folio 50-------
element under consideration back to its geometrically normal
position. Of course~$\omega$ is a function of~$q$ and its derivative with
reference to~$s$. This function we shall investigate later. The
resolved part of~$q \omega q^{-1}$ parallel to the wire is the vector twist and
the resolved part perpendicular to the wire is in the direction of
the binormal and equal to the curvature. In fact~$\omega$ is the vector
whose coordinates are the $\kappa, \lambda, \tau$ of Thomson and Tait's \emph{Nat.\ Phil.}\
\S~593. When we are given $q$ or~$\omega$ for every point we know the
strain of the wire completely. $\omega_0$~is defined as the naturally
normal value of~$\omega$, i.e.\ the value of~$\omega$ when the wire is unstressed.

As usual we take~$\rho$ as the coordinate vector of any point of the
wire, $\rho$ like the rest of the functions being considered as a function
of~$s$. We shall denote (after Tait, \emph{Quaternions}, Chap.~\textsc{ix}.)\
differentiations with regard to~$s$ by dashes.

We now come to the dynamical symbols. $\mathbf{F}$ and~$\mathbf{M}$ are the
force and couple respectively exerted across any normal section of
the wire on the part of the wire which is on the negative side of
the section by the part on the positive side.

Finally let $\mathbf{X, L}$ be the external force and couple per unit
length exerted upon the wire.

\Paragraph{30} When the wire is strained in any way let us impose a
small additional strain represented by an increment~$\delta \omega$ in~$\omega$ and
an increment~$\delta e$ in the elongation at any point. Then the work
done on the element~$ds$ by the stress-force $= -\delta e S \mathbf{F} \rho'\,ds$ and that
done by the stress-couple $= -Sq \delta \omega q^{-1} \mathbf{M}\, ds$. If (as we assume,
though the assumption is not justified in some useful applications
of the general theory of wires) $\mathbf{F}$ and~$\mathbf{M}$ to be of the same order
of magnitude the former of these expressions can be neglected in
comparison with the latter for~$\delta e$ is a quantity small compared
with~$\delta\omega$. Now the work done on the element by the stress $=$ the
increment of the element's potential energy $= \delta w ds$ where~$w$ is
some function of the strain. Hence
\begin{align*}
\delta w &= -S\, \delta \omega\, q^{-1} \mathbf{M} q. \\
\intertext{Thus $w$ is a function of~$\omega$ only and}
\therefore\qquad \delta w &= -S\, \delta\omega\, \Nabla{\omega} w, \\
%
\intertext{whence we see that}
\EqnTag{III}{46}
q^{-1} \mathbf{M} q &= \Nabla{\omega} w.
\end{align*}

Notice that $\mathbf{M}$ is Thomson and Tait's $\xi$, $\eta$, $\zeta$ and~$q^{-1} \mathbf{M} q$
their~$KLM$ (\emph{Nat.\ Phil}.\ \S\S~594,~614).

%% -----File: 069.png---Folio 51-------

\Paragraph{31} Now since the strain is small, $q^{-1}\mathbf{M}q$ is linear in terms of
the strain and therefore in terms of $\omega$. Hence we see that~$w$
is quadratic in terms of $\omega$. Let us then put
\begin{align*}
w = w_2(\omega - \omega_0, \omega-\omega_0) &+ w_1 (\omega-\omega_0) \\
&+ \text{ a function of temperature only,}
\end{align*}
$w_2$ and~$w_1$ being linear and homogeneous in each of their constituents.
This is the most general quadratic function of~$\omega$.
Now
\[
\Nabla{\omega} w = \zeta w_2(\zeta, \omega - \omega_0)
  + \zeta w_2(\omega - \omega_0, \zeta)
  + \zeta w_1 \zeta.
\]
Putting then $\omega=\omega_0$ and $\Nabla{\omega} w=0$ we get $\zeta w_1\zeta=0$. Operating
on the last by $S\sigma(\;)$ where~$\sigma$ is any vector we see that $w_1=0$.
Thus putting
\[
\Nabla{\omega} w = \phi(\omega - \omega_0),
\]
where $\phi$ has the value given by the last equation and is therefore
self-conjugate we get the two following equations
\[
\EqnTag{III}{47}
w = -S(\omega - \omega_0) \phi(\omega - \omega_0)/2 + w_0,
\]
[as can be seen by a comparison of the last three equations
$w_0$ being the function of the temperature] and
\[
\EqnTag{III}{48}
q^{-1}\mathbf{M}q = \phi(\omega-\omega_0).
\]

When the natural shape of the wire is straight these become
\begin{gather*}
\EqnTag{III}{49}
w = -S \omega\phi\omega/2 + w_0, \\
%
\EqnTag{III}{50}
q^{-1}\mathbf{M}q = \phi\omega,
\end{gather*}
and when further the wire is truly uniform $\phi$ and~$w_0$ are constant
along the wire.

\Paragraph{32} Assuming the truth of these restrictions let us conceive a
rigid body moving about a fixed point which, when placed in a
certain position which we shall call the normal position, has, if
then rotating with any vector angular velocity~$\omega$, a moment
of momentum $= \phi\omega$ where~$\phi$ has the meaning just given. If the
rigid body be made to take a finite rotation $q(\;)q^{-1}$ and then
to move with angular velocity~$q\omega q^{-1}$ its moment of momentum
will be~$q\phi\omega q^{-1}$. Now let a point move along the wire with unit
velocity and let the rigid body so move in unison with it that
when the moving point reaches the point~$s$ the rigid body shall
have made the rotation represented by~$q$ (\Secref{29} above). Thus by
the definition of $\omega$ and~$q$ its angular velocity at any instant
is~$q\omega q^{-1}$ and its moment of momentum therefore $q\phi\omega q^{-1}$ or~$\mathbf{M}$.

%% -----File: 070.png---Folio 52-------

Now consider the equilibrium of the wire when no external
force or couple acts except at its ends. In this case $\mathbf{F}$~is constant
throughout and it is easy to see (what indeed is a particular case
of \equationref{III}{53} below) that
\[
\EqnTag{III}{51}
\mathbf{M}' + V \rho' \mathbf{F} = 0.
\]
Interpreting this equation for our rigid body we get as the law
which governs its motion
\[
d (\text{vect.\ mom.\ of mom.})/dt = -V \rho' \mathbf{F}.
\]
Thus the rigid body will move as if acted upon by a constant
force~$\mathbf{F}$ at the end of the unit vector~$\rho'$ or---since this vector is fixed in
the body---as if acted upon by a constant force acting through a
point fixed in the body. From this kinetic analogue of Kirchhoff's
the mathematical problem of the shape of such a wire as we
are now considering, under the given circumstances, is shewn to
be identical with the general problem of the pendulum of which
the top is a variety.

\Paragraph{33} We will now give the general equations for any wire
under any external actions. The comparison of the Quaternion
treatment of this with the Cartesian as given in Thomson and
Tait's \emph{Nat.\ Phil}.\ \S~614 seems to me to be all in favour of the
former.

The equations of equilibrium of an element~$ds$ are with the
notation explained in \Secref{29} above
\begin{gather*}
d\mathbf{F} + \mathbf{X}ds = 0, \\
Vd\rho \mathbf{F} + d\mathbf{M} + \mathbf{L}\, ds = 0,
\end{gather*}
or dividing by~$ds$
\begin{gather*}
\EqnTag{III}{52}
\mathbf{F}' + \mathbf{X} = 0,   \\
%
\EqnTag{III}{53}
V\rho' \mathbf{F} + \mathbf{M}' + \mathbf{L} = 0.
\end{gather*}
Operating on the last equation by $V\rho'(\;)$ noting that $\rho'^2 = -1$ and
%%%
putting $S\rho' \mathbf{F} = -T$ we get
\[
\EqnTag{III}{54}
\mathbf{F} = \rho' T+ V\rho' (\mathbf{M}' + \mathbf{L}),
\]
whence by equations \Eref{III}{52} and~\Eref{III}{53} respectively
\begin{gather*}
\EqnTag{III}{55}
\mathbf{X} + d\{\rho' T+ V \rho' (\mathbf{M}' + \mathbf{L})\}/ds = 0, \\
%
\EqnTag{III}{56}
S\rho' (\mathbf{M}'+ \mathbf{L}) = 0.
\end{gather*}

%% -----File: 071.png---Folio 53-------

Now by equation~(48) above
% [** PP: No hyperlink; equation repeated here w/manual \tag]
\[
\tag{48}%%* [Sic; it's a copy of 48, not a misnumbered 56.]
q^{-1}\mathbf{M}q = \phi(\omega-\omega_0).
\]
Also by the definition of~$q$
\[
\EqnTag{III}{57}
\rho' = q\lambda q^{-1},
\]
where $\lambda$ is some given constant unit vector. Finally as we are
about to prove
\[
\EqnTag{III}{58}
\omega = 2V q^{-1}q'.
\]

It is usual in the Cartesian treatment to leave the problem in
the form of equations equivalent to the above \Eref{III}{55} to~\Eref{III}{58},
13~scalar equations for the 13~unknown scalars of $\rho$, $T$, $\mathbf{M}$, $q$ and~$\omega$.
We can however as we shall directly in the Quaternion treatment
quite easily reduce the general problem to one vector and one
scalar equation involving the four unknown scalars of~$T$ and~$q$
in terms of which all the other unknowns are explicitly given.

To prove \equationref{III}{58}\footnote{%
This could be deduced from Tait's \textit{Quaternions}, \S~356,
equation~(2). His $\epsilon$
is our~$q\omega q^{-1}$ and his dots our dashes.}
observe that
\[
(q + dq)\sigma(q + dq)^{-1} = q(\sigma + V\omega \sigma\, ds) q^{-1},
\]
where $\sigma$ is any vector. The truth of this is seen by noticing that
$(q + dq)(\;)(q + dq)^{-1}$ is the operator that rotates any vector of
the element $s + ds$ from its geometrically normal position to its
strained position. But we can also get to this final position
by first in the geometrically normal wire making the small strain
$\omega\, ds$ at the given element and then performing the strain of the
wire up to the point~$s$. The first process is represented on the
left of the last equation and the second on the right. Thus
we get
\begin{flalign*}
&&  q^{-1}\,dq\, \sigma &+ \sigma d(q^{-1})\centerdot q = V\omega\sigma\, ds, && \\
&\text{or } \because &
d(q^{-1}) &= -q^{-1}\, dq\, q^{-1},  &&\\
&&  q^{-1}q'\sigma &- \sigma q^{-1}q' = V \omega \sigma, && \\
&\text{whence}& \omega &= 2V\, q^{-1}q'.  &&
\end{flalign*}

Returning to equations \Eref{III}{55} to~\Eref{III}{58} observe that equations
\Eref{III}{57} and~\Eref{III}{58} give $\rho'$ and~$\omega$ as explicit functions of~$q$. Hence
by \equationref{III}{48}
\[
\EqnTag{III}{59}
\mathbf{M} = q \phi (2V\, q^{-1} q' - \omega_0) q^{-1},
\]
%% -----File: 072.png---Folio 54-------
which gives $\mathbf{M}$ also explicitly. Substituting for $\mathbf{M}$ and~$\rho'$ in
equations \Eref{III}{55} and \Eref{III}{56} we have
\begin{gather*}
\EqnTag{III}{60}
\mathbf{X} + \frac{d}{ds}\left\{
    q\lambda q^{-1}\, T + V\, q\lambda q^{-1}
        \Bigl(\mathbf{L} + \frac{d}{ds}\Bigl[
            q\, \phi(2V\, q^{-1}q' - \omega_0)\, q^{-1}\Bigr]
        \Bigr)
    \right\} = 0, \\
%
\EqnTag{III}{61}
S\, q\lambda q^{-1}
        \Bigl(\mathbf{L} + \frac{d}{ds}\Bigl[
            q\, \phi(2V\, q^{-1}q' - \omega_0)\, q^{-1}\Bigr]
        \Bigr) = 0,
\end{gather*}
which are sufficient equations to determine $T$ and~$q$, whereupon
$\mathbf{M}$ is given by \equationref{III}{59}, $\omega$~by \equationref{III}{58} and $\rho'$~and
therefore also~$\rho$ by~\equationref{III}{57}.

Spiral springs can be treated very simply by means of the
above equations, but we have already devoted sufficient space to
this subject.

%% -----File: 073.png---Folio 55-------


\Chapter{Section IV}{Electricity and Magnetism}

\TCSection{34}{%
  \texorpdfstring{{\scshape Electrostatics}---general problem}{Electrostatics}
}{{\scshape Electrostatics}\\[2ex] General Problem}

\Paragraph{34} Merely observing that all the theorems in integration
given in the Preliminary and~\textsc{iv}th chapters of Maxwell's treatise
on \textit{Electricity and Magnetism}, Part~\textsc{i}., are easy particular cases of
equations \Eref{II}{8} and~\Eref{II}{9} \Secref{6} above, we pass on to the one application
of Quaternions that we propose to make in Electrostatics.

This is to find the most general mechanical results arising from
Maxwell's theory of Electrostatics, and to see if they can be
explained by stress in the dielectric. This problem as far as I am
aware has not hitherto in all its generality been attacked though
the most important practical cases have been, as we shall see,
considered by Maxwell, Helmholtz, Korteweg, Lorberg and
Kirchhoff.

It is necessary first of all to indicate as clearly as possible what
I take to be Maxwell's theory of Electricity.

He assumes\footnote{%
Prof.\ J.~J.\ Thomson in his paper on \textit{Electrical Theories}, B.A.\ Reports, 1885,
p.~125, does not credit Maxwell with such a definite and circumscribed theory as
that described in the text, and he is thereby led to find fault with Maxwell's term
``Displacement'' and points out that there is an assumption made with reference
to the connection between the true current and this polarisation (displacement).
He says moreover, ``It is rather difficult to see what is meant in Maxwell's Theory
by the phrase `Quantity of Electricity.'\,'' None of these remarks are called for if
the view I take of Maxwell's theory be correct, and these grounds alone I consider
sufficient for taking that view. The paper of Thomson's here mentioned I shall
frequently have to refer to. [Note added, 1892. In the text I have given much too
rigid a form to Maxwell's theory. What I have called his theory I ought rather
to have called his analogy. Still I think the present foot-note is in the main just.
In my opinion it is no more and no less difficult to see what is meant in Maxwell's
Theory by ``Quantity of Electricity'' than by ``displacement'' since the two are
connected by perfectly definite equations. Of course it is wrong to define ``displacement''
as ``displacement of quantity of electricity'' and then to define ``quantity
of electricity'' in terms of ``displacement'' but Maxwell does not seem to me
even tacitly to do this. Rather he says---the dielectric is polarised; this polarisation
can be represented by a vector~$\mathbf{D}$; electrical quantity can be expressed in
terms of~$\mathbf{D}$; the mathematical connections between electrical quantity and~$\mathbf{D}$
are the same as those between quantity of matter in a space and the displacement
out of that space made by other matter to make room for the given matter; we
will impress this useful analogy firmly on our minds by calling~$\mathbf{D}$ the displacement.
But I have expressed my present views on the meaning of Maxwell's theory much
more fully in \textit{Phil.\ Trans.}\ 1892, p.~685.]} %** end of footnote
all space to be uniformly filled with a certain
substance called Electricity. Whatever electrical actions take place
%% -----File: 074.png---Folio 56-------
depend on the continued or past \textit{motion} of this substance as an
incompressible fluid. If electricity is brought from a distance by
any means and placed in a given space there must be a displacement
of the original electricity outwards from that space and the
quantity of foreign electricity is conveniently measured by the
surface integral of that displacement.

\textit{Dielectrics} are substances in which this displacement tends to
undo itself, so to speak, i.e.\ the original electricity tends to go back
to its primitive position. In \textit{conductors}, on the other hand, there
is no property distinguishing any imported electricity from the
original electricity.

The rate of variation of displacement, whether in dielectrics or
conductors, of course constitutes an electric current as it is conveniently
called.

We have next to consider a vector at each point of space
called the electro-motive force, which depends in some way at
present undefined on the distribution of the displacement in the
dielectrics, the distribution of currents whether in dielectrics or
conductors, and on extra-electrical or semi-electrical action, e.g.\
chemical or mechanical.

If at any point the electro-motive force be multiplied by a
scalar the medium at the point remaining (except electrically)
unchanged, the current in the case of conductors and the displacement
in the case of dielectrics is altered in the same ratio. In
other words the current or the displacement, as the case may be,
is a linear vector function of the electro-motive force, and the
coordinates of the linear vector function\footnote{%
This frequently recurring cumbrous mode of description must be tolerated
unless a single word can be invented for ``a linear vector function of a vector.''
Might I suggest the term ``Hamiltonian?'' Thus we should say that the displacement
is a Hamiltonian of the electro-motive force, the Hamiltonian at any
point being a function of the state of the medium.}
at any point depend
%% -----File: 075.png---Folio 57-------
solely on the state of the medium (whether fluid, solid,~\&c.,\ or
again strained or not) at that point.

To complete the theory we have to explain how the part of the
electro-motive force which is a function of the distribution of displacement
and current depends on this distribution. This explanation
is obtained by making the assumption that the electro-motive
force bears to electricity defined as above exactly the same
energy relation as ordinary force does to matter, i.e.---
\begin{align*}
&\shoveleft{(\text{wk.\ dn.\ on electricity moved during any displacement})} \\
&\qquad\qquad= (\text{total displacement of elect.}) \\
&\qquad\qquad\qquad\times (\text{resolved part of \textsc{e.m.f.}\ in the
direction of displacement}).
\end{align*}

In the \textsc{iv}th part of Maxwell's treatise he gives complete investigations
of the mechanical results flowing from this theory so far
as it refers to currents, but he has not given the general results in
the case of Electrostatics. Nor has he shewn satisfactorily, it
seems to me, that the ordinary laws of Electrostatics flow from his
theory. It is these investigations we now propose to make.

\Paragraph{35} Our notation will be as far as possible the same as
Maxwell's. Thus for the displacement at any point we use~$\mathbf{D}$, and
for the \textsc{e.m.f.}~$\mathbf{E}$. From the connection explained in last section
between $\mathbf{D}$ and~$\mathbf{E}$ we have
\[
\EqnTag{IV}{1}
\mathbf{D} = K\mathbf{E}/4\pi,
\]
where $K$ at any point is some linear vector function depending on
the state of the medium at the point. If the medium change in
any manner not electrical, e.g.\ by means of ordinary strain~$K$ will
in general also suffer change.

Let $w$ be the potential energy per unit volume due to the
electrical configuration. Thus if a small increment~$\delta\mathbf{D}$ be given
to~$\mathbf{D}$ at all points, the increment $\smalliiint\delta w\, \ds$ in $\smalliiint w\, \ds$, the potential
energy of the electrical configuration in any space, = work done on
the electricity in producing the change,
\[
\therefore\qquad \smalliiint\delta w\, \ds
= -\smalliiint S\mathbf{E}\, \delta\mathbf{D}\, \ds,
\]
by the relation stated in \Secref{34} existing between $\mathbf{E}$ and~$\mathbf{D}$. Thus
limiting the space to the element~$\ds$
\[
\EqnTag{IV}{2}
\delta w = -S\mathbf{E}\, \delta\mathbf{D}.
\]
%% -----File: 076.png---Folio 58-------
Now suppose $\mathbf{D} = n\mathbf{D}'$ so that by \equationref{IV}{1} $\mathbf{E} = n\mathbf{E}'$ where~$\mathbf{E}'$,
$\mathbf{D}'$ are corresponding \textsc{e.m.f.} and displacement respectively. Thus
\[
\delta w = -n\, \delta n\, S\mathbf{E}'\mathbf{D}'.
\]
Integrating from $n = 0$ to $n = 1$, and finally changing $\mathbf{D}'$, $\mathbf{E}'$ into
$\mathbf{D}$, $\mathbf{E}$ we get
\[
\EqnTag{IV}{3}
w = -S\mathbf{D}\mathbf{E}/2.
\]
From this we get
\[
\delta w = - S\mathbf{E}\, \delta\mathbf{D}/2
- S\mathbf{D}\, \delta\mathbf{E}/2,
\]
so that by \equationref{IV}{2}
$ S\mathbf{E}\, \delta\mathbf{D}
= S\mathbf{D}\, \delta\mathbf{E}$ or by \equationref{IV}{1}
\[
S\mathbf{E} K\, \delta\mathbf{E} = S\,\delta\mathbf{E}\, K\mathbf{E}.
\]
Hence (because $\mathbf{E}$ and~$\delta\mathbf{E}$ are quite arbitrary) $K$ is self-conjugate
and therefore involves only six instead of nine coordinates\footnote{%
We see from this that $\mathbf{D} = \Nabla{\mathbf{E}} w$ or
$\mathbf{E} = \Nabla{\mathbf{D}} w$ according as $w$ is looked upon as
a function of $\mathbf{E}$ or~$\mathbf{D}$.}.

In electrostatics the line integral of~$\mathbf{E}$ round any closed curve
must be zero, for otherwise making a small conductor coincide
with the curve we shall be able to maintain a current by \Secref{34}, and
so (by the same section) constantly do work on it (i.e.\ as a matter
of fact create heat) without altering the statical configuration.
Hence~$\mathbf{E}$ must have a potential, say~$v$. Thus
\[
\EqnTag{IV}{4}
\mathbf{E} = -\nabla v.
\]
Since in an electrostatic field there is no current in a conductor,
$\mathbf{E} = 0$ throughout any such conductor and therefore $v=\text{const.}$

\Paragraph{36} The charge in any portion of space is defined as the
amount of foreign electricity within that space. Thus the charge
in any space is the surface integral of the displacement outwards.
Thus if there be a charge on the element~$d\Sigma$ of a surface in the
dielectric this charge $= S\, d\Sigma_a\, \mathbf{D}_a + S\, d\Sigma_b\, \mathbf{D}_b$ where $a$, $b$ denote the
two faces of the element (so that $d\Sigma_a = -d\Sigma_b$) and in accordance
with \Secref{1} above $d\Sigma_a$ points away from the region in which the displacement
is~$\mathbf{D}_a$. Thus $\sigma$ being the surface density
\[
\EqnTag{IV}{5}
\sigma = [S\mathbf{D} U\, d\Sigma]_{a+b},
\]
where the notation $[\;]_{a+b}$ is used for $[\;]_a + [\;]_b$. Similarly if there
be finite volume density of foreign electricity, i.e.\ finite volume
density of charge in any space, the charge $= -\smalliint S\mathbf{D}d\Sigma = -\smalliiint S\nabla\mathbf{D}\, \ds$,
so that if~$D$ be the volume density
\[
\EqnTag{IV}{6}
D = -S\nabla\mathbf{D}.
\]
%% -----File: 077.png---Folio 59-------
[The reason for having $+S\,d\Sigma_a\,\mathbf{D}_a$ before and $-S\,d\Sigma\,\mathbf{D}$ here is that
in the former case we were considering a charge \emph{outside} the region
where~$\mathbf{D}_a$ is considered---\emph{between} the regions $a$ and~$b$ in fact---whereas
in the latter case we are considering the charge \emph{inside} the
region where~$\mathbf{D}$ is considered. The same explanation applies to
the sign of~$-S\mathbf{D} U\,d\Sigma$ for the surface of a conductor given below.]

In conductors, as we saw in \Secref{34}, the displacement has virtually
no meaning (except when it is changing and so the phenomenon
of a current takes place) for the foreign electricity and the original
electricity are not to be distinguished. Not so however with the
surface of the dielectric in contact with the conductor. We may
therefore regard the electricity within the body of the conductor
as the original electricity so that the charge is entirely at the
surface. Thus the surface density will be~$-S\mathbf{D} U\,d\Sigma$ where~$d\Sigma$
points away from the conductor and~$\mathbf{D}$ is the displacement in the
dielectric. This may be regarded as a particular case of \equationref{IV}{5}
$\mathbf{D}$~being in accordance with what we have just said considered
as zero in the conductor.

\Paragraph{37} All the volume integrals with which we now have to deal
may be considered either to refer to the whole of space or only to
the dielectrics, as the conductors (except at their surfaces) in all
cases contribute nothing. The boundary of space will be considered
as a surface at infinity and all surfaces where either $\mathbf{D}$ or~$\mathbf{E}$
is discontinuous.

Putting $W = \smalliiint w\, \ds$ we have already found one expression
for~$W$, viz.\
\[
2W = -\smalliiint S\mathbf{D}\mathbf{E}\, \ds.
\]
We now give another. By \equationref{IV}{4}
\begin{align*}
2W &= \smalliiint S\mathbf{D} \nabla v\, \ds  \\
&= \smalliint vS\, d\Sigma\,\mathbf{D} - \smalliiint vS\nabla\mathbf{D}\, \ds,
\end{align*}
by \equationref{II}{9} \Secref{6} above. Thus by equations \Eref{IV}{5} and~\Eref{IV}{6} \Secref{36},
\[
\EqnTag{IV}{7}
2W = \smalliint v\sigma\, ds + \smalliiint vD\, \ds,
\]
where $ds$ is put, as it frequently will be, for an element of surface,
i.e.\ $T\,d\Sigma$. The value of~$W$ which we shall use\footnote{%
This is for the general case following the example of Helmholtz in the particular
case when~$K$ reduces to a single scalar. See \textit{Wiss.\ Abh.}\ vol.~\textsc{i}.\ equation~(2$d$),
p.~805. The \emph{method} adopted in the following investigation is also similar to his.}
is obtained by
combining these two, viz.\
\[
\EqnTag{IV}{8}
W = \smalliint v\sigma\, ds + \smalliiint vD\, \ds
  + \tfrac{1}{2} \smalliiint S\mathbf{D}\mathbf{E}\, \ds.
\]
%% -----File: 078.png---Folio 60-------

So far we have merely been shewing that all the above results
of Maxwell's flow from what in \Secref{34} has been described as his
theory. We now proceed to the actual problem in hand which is
proved from these results however they may be obtained. I may
remark that \emph{some} such investigation as the above seems to me
necessary to make the logic of Maxwell's treatise complete.

\Paragraph{38} Suppose now that~$W$ is the potential energy of some
dynamical system extending throughout space. Let us give to
every point of space a small displacement~$\delta \eta$ vanishing at infinity
and find the consequent increment~$\delta W$ in~$W$. If this can be put in
the form
\[
\EqnTag{IV}{9}
\delta W = - \smalliiint S\, \delta \eta_1\, \phi \nabla_1\, \ds,
\]
we shall have the following expression for~$\mathbf{F}$ the force per unit
volume due to the system
\[
\EqnTag{IV}{10}
\mathbf{F} = \phi \Delta,
\]
and the following expression for~$\mathbf{F}_S$ the force per unit surface at
any surface of discontinuity in~$\phi$
\[
\EqnTag{IV}{11}
\mathbf{F}_s = -[\phi U\, d\Sigma]_{a+b},
\]
the notation being the same as in \equationref{IV}{5} \Secref{36}.

Moreover if~$\phi$ be self-conjugate the forces both throughout the
volume and at surfaces of discontinuity are producible by the
\emph{stress}~$\phi$ as can be seen by \Secref{13} above. [Compare all these statements
with \Secref{14} above.]

For proof, we have by \equationref{II}{9} \Secref{6}
\[
\delta W = -\smalliiint S\, \delta \eta_1\, \phi \nabla_1\, \ds
         = -\smalliint  S\, \delta \eta\, \phi\, d\Sigma
          + \smalliiint S\, \delta \eta\, \phi_1 \nabla_1\, \ds,
\]
where of course the element~$d\Sigma$ is taken \emph{twice}, i.e.\ once for each
face. But
\begin{align*}
\delta W
&= - \text{(work done by the system $\mathbf{F}$, $\mathbf{F}_s$ of forces)} \\
&= \smalliint  S\, \delta \eta\, \mathbf{F}_s ds
 + \smalliiint S\, \delta \eta\, \mathbf{F}\, \ds,
\end{align*}
where the element~$ds$ is taken only \emph{once}. Equating the coefficients
of the arbitrary vector~$\delta\eta$ for each point of space we get the
required equations \Eref{IV}{10} and~\Eref{IV}{11}.

\Paragraph{39} We must then put $\delta W$ where~$W$ is given by \equationref{IV}{8}
in the form given in \equationref{IV}{9}.

We must first define~$\delta$ when applied to a function of the position
of a point. Suppose by means of the small displacement~$\delta\eta$
%% -----File: 079.png---Folio 61-------
any point~$P$ moves to~$P'$. Then~$Q$ being the value at~$P$, before
the displacement~$\delta \eta$, of a function of the position of a point, $Q + \delta Q$
is defined as the value of the function at~$P'$ after the displacement.
Thus even in the neighbourhood of a surface of
discontinuity~$\delta Q$ is a small quantity of the same order as~$\delta\eta$.

Now the charge within any space, that is the quantity of
foreign electricity within that space will not be altered by the
strain.
\[
\EqnTag{IV}{12}
\therefore\qquad \delta(D\, \ds)=0, \qquad \delta(\sigma\, ds) = 0.
\]
To find $\delta\nabla$ we have
\[
S (d\rho + \delta d\rho) (\nabla + \delta \nabla)\centerdot
    = S\, d\rho\, \nabla\centerdot
\]
or, since $\delta d\rho = -S\,d\rho\,\nabla\centerdot \delta \eta$,
\begin{flalign*}
&& S\,d\rho\,   \delta\nabla
&= S\, d\rho\, \nabla_1 S\, \delta\eta_1\,\nabla, && \\
%
\EqnTag{IV}{13}
&\text{whence}& \delta\nabla
&= \nabla_1 S\, \delta\eta_1\, \nabla. &&
\end{flalign*}

The part of~$\delta W$ depending on the first two terms of
\equationref{IV}{8} is by \equationref{IV}{12}
\begin{align*}
&\smalliiint \delta v\, D\, \ds + \smalliint \delta v\, \sigma\, ds && \\
&\quad = - \smalliiint \delta v\, S \nabla \mathbf{D}\, \ds
         + \smalliint  \delta v\, S\, d \Sigma\, \mathbf{D} &
                      &\quad \text{[by equations \Eref{IV}{5} and~\Eref{IV}{6} \Secref{36}]} \\
&\quad = \phantom{-}\smalliiint S \mathbf{D} \nabla\, \delta v\, \ds &
                      &\quad \text{[by \equationref{II}{9} \Secref{6}]}.
\end{align*}
Noticing that $\delta\,\ds = -\ds\, S\nabla\,\delta\eta$ and that $4\pi\delta \mathbf{D} = K \delta \mathbf{E} + \delta K \mathbf{E}$ we
see that the last term in \equationref{IV}{8} contributes
\[
\smalliiint S \mathbf{D}\, \delta\mathbf{E}\, \ds
    - \tfrac{1}{2}\smalliiint S\mathbf{D}\mathbf{E}S\nabla\, \delta\eta\, \ds
    + (8\pi)^{-1} \smalliiint S\mathbf{E}\, \delta K\,\mathbf{E}\, \ds.
\]
Combining the last result with the first term of this we get
\begin{align*}
\smalliiint S\mathbf{D}(\nabla\, \delta v &- \delta(\nabla v))\, \ds
 = -\smalliiint S\mathbf{D}\, \delta \nabla\centerdot v\, \ds \\
&= -\smalliiint S\mathbf{D}\nabla_1S\,\delta\eta_1\,\nabla v\, \ds\\
&= \phantom{-}\smalliiint S\mathbf{D}\nabla_1 S\mathbf{E}\,\delta\eta_1\, \ds.
\end{align*}
Thus we have
\[
\EqnTag{IV}{14}
\delta W = - \smalliiint S\,\delta\eta_1
    (\tfrac{1}{2}\nabla_1 S\mathbf{D}\mathbf{E}
               - \mathbf{E}S\nabla_1 \mathbf{D})\, \ds
    + (8\pi)^{-1}\smalliiint S\mathbf{E}\,\delta K\,\mathbf{E}\, \ds.
\]

\Paragraph{40} Now the increment~$\delta K$ in~$K$ is caused by two things
viz.\ the mere rotation of the body and the change of \emph{shape} of the
body. Let us call these parts $\delta K_r$ and~$\delta K_s$ respectively.

First consider~$\delta K_r$. Suppose the rotation is~$\epsilon$ so that any
vector which was~$\omega$ becomes thereby $\omega + V\epsilon \omega$. Thus the result
%% -----File: 080.png---Folio 62-------
of operating on $\omega + V\epsilon\omega$ by $K + \delta K_r$ is the same as first operating
on~$\omega$ by~$K$ and then rotating. In symbols
\[
(K + \delta K_r) (\omega + V\epsilon\omega) = K\omega + V\epsilon K\omega,
\]
\begin{flalign*}
&\text{whence}&
\delta K_r\, \omega
&= V\epsilon K\omega  - K(V\epsilon\omega). && \\
&\text{Thus}&
S\mathbf{E} \delta K_r\,\mathbf{E}
&= S\mathbf{E}\epsilon K\mathbf{E} - S\mathbf{E}K(V\epsilon\mathbf{E}) && \\
&&
&= 2S\epsilon K\mathbf{E}\mathbf{E} = 8\pi S\epsilon \mathbf{D}\mathbf{E}, &&
\end{flalign*}
whence giving $\epsilon$~its value $V\nabla\delta\eta/2$,
\[
(8\pi)^{-1} S\mathbf{E}\, \delta K_r\,\mathbf{E}
= S\nabla\, \delta\eta\, V\mathbf{D}\mathbf{E}/2.
\]
Substituting in \equationref{IV}{14}
\[
\delta W
= -\smalliiint S\,\delta\eta_1 \{
    V\nabla_1\mathbf{D}\mathbf{E}/2 - \mathbf{E}S\nabla_1\mathbf{D}
    \}\, \ds
    + (8\pi)^{-1}\smalliiint S\mathbf{E}\,\delta K_s\, \mathbf{E}\, \ds,
\]
or
\[
\EqnTag{IV}{15}
\delta W
= \tfrac{1}{2}
  \smalliiint S\,\delta\eta_1\, V\mathbf{D}\, \nabla_1\mathbf{E}\, \ds
+ (8\pi)^{-1}
  \smalliiint S\mathbf{E}\, \delta K_s\, \mathbf{E}\, \ds.
\]

It only remains to consider~$\delta K_s$. $K$ is a function of the pure
strain of the medium and~$\delta K_s$ is the increment in~$K$ due to the
increment in pure strain owing to~$\delta\eta$. Calling this increment of
pure strain~$\delta\psi$ so that by \equationref{III}{3} \Secref{12} above
\[
\delta\psi\omega
= -\delta\eta_1\, S\omega \nabla_1/2
  - \nabla_1 S\omega\, \delta\eta_1/2,
\]
we have
\begin{align*}
\EqnTag{IV}{16}
\delta K_s
&= -S\,\delta\psi\,\zeta \Pre{\psi}{\Dop}_2 \zeta \centerdot K_2, \\
%
\intertext{by \equationref{II}{7} \Secref{5} above. This gives by \equationref{II}{5} \Secref{3}}
\EqnTag{IV}{17}
\delta K_s
&= -S\,\delta\eta_1 \Pre{\psi}{\Dop}_2 \nabla_1\centerdot K_2, \\
%
\EqnTag{IV}{18}
\therefore\qquad
S\mathbf{E}\, \delta K_s\,\mathbf{E}
&= -S\,\delta\eta_1 \Pre{\psi}{\Dop}_2 \nabla_1\, S\mathbf{E} K_2\mathbf{E}.
\end{align*}
Now by equations \Eref{IV}{1} and~\Eref{IV}{3} \Secref{35},
\[
8\pi w = -S\mathbf{E} K\mathbf{E},
\]
so that~$w$ is a function of the independent variables $\mathbf{E}$, $\psi$ (because
$K$ is a function of~$\psi$). Therefore
\Erratum{err62a}% [** PP: In right-hand member, change D to \Dop]
\[
\EqnTag{IV}{19}
(8\pi)^{-1} S\mathbf{E}\, \delta K_s\,\mathbf{E}
= S\delta\eta_1 \Pre{\psi}{\Dop} w\,\nabla_1.
\]
This equation might have been deduced at once thus
\[
(8\pi)^{-1} S\mathbf{E}\, \delta K_s\,\mathbf{E}
= S\,\delta\psi\,\zeta \Pre{\psi}{\Dop} w\,\zeta
= S\,\delta\eta_1 \Pre{\psi}{\Dop} w\,\nabla_1,% <- \Dop
\]
but \equationref{IV}{17} is itself of importance so the above proof is
preferable.

Thus finally from \equationref{IV}{15} we get
\[
\EqnTag{IV}{20}
\delta W
= \smalliiint S\,\delta\eta_1
(V\mathbf{D} \nabla_1\mathbf{E}/2 + \Pre{\psi}{\Dop} w\, \nabla_1)\, \ds.
\]
%% -----File: 081.png---Folio 63-------
We therefore have for~$\phi$ in equations \Eref{IV}{10} and~\Eref{IV}{11} \Secref{38},
\[
\EqnTag{IV}{21}
\phi\omega = -V\mathbf{D}\omega\mathbf{E}/2 - \Pre{\psi}{\Dop} w\, \omega.
\]

This is a self-conjugate function so that as we saw in \Secref{38} it is
a stress which serves to explain forces both throughout the volume
of the dielectric and over any surfaces of discontinuity in~$\mathbf{D}$ or~$\mathbf{E}$\footnotemark.
\footnotetext{As far as I am aware nobody has hitherto attempted to find the electrical
forces much less the stress except in the case when~$\mathbf{D}$ is parallel to $\mathbf{E}$ i.e.\ the
dielectric is electrically isotropic when unstrained. The particular results contained
in \Secref{45} below have been obtained by Korteweg, Lorberg and Kirchhoff as is stated
in Prof.~J.~J.~Thomson's paper (p.~155) referred to in \Secref{34}.}
% [** End of footnote]


\Section{41}{The force in particular cases}

\Paragraph{41} Let us first consider that part $-V\mathbf{D}\Delta\mathbf{E}/2$ of the force
$\bigl($equations~\Eref{IV}{10} \Secref{38} and~\Eref{IV}{21} \Secref{40}$\bigr)$ which does not depend on the
variation of~$K$ with the shape of the body.

Suppose our dielectric is homogeneous and electrically isotropic
so that $K$~is a simple constant scalar. In this case
\[
\EqnTag{IV}{22}
4\pi\mathbf{D} = -K\nabla v,
\]
by equations \Eref{IV}{1} and~\Eref{IV}{4} \Secref{35}. Therefore by equations \Eref{IV}{5}
and~\Eref{IV}{6} \Secref{36},
\begin{align*}
\EqnTag{IV}{23}
4\pi \mathbf{D} &= K\nabla^2v, \\
%
\EqnTag{IV}{24}
4\pi\sigma &= - K[SU\,d\Sigma\,\nabla v]_{a+b}.
\end{align*}
From these we at once get by the theory of potential that
\[
\EqnTag{IV}{25}
Kv = \smalliiint u\mathbf{D}\, \ds + \smalliint u\sigma\, ds.
\]
From this we know by the theory of potential that at the surface
where the charge~$\sigma$ resides $\nabla v$ is discontinuous only with regard
to its normal component and at all other points is continuous.
Thus
\[
\nabla v_a = \nabla v_b + xU\,d\Sigma_a
\]
and by \equationref{IV}{24} $x= 4\pi\sigma/K$ so that
\begin{flalign*}
&&
&(4\pi)^{-1} K(\nabla v_a - \nabla v_b) = \sigma U\,d\Sigma_a, && \\
%
\EqnTag{IV}{26}
&\text{whence}&
&(4\pi)^{-1} KU\,d\Sigma_a(\nabla v_a - \nabla v_b) = -\sigma.  &&
\end{flalign*}
Now the force $\mathbf{F}$ per unit volume is
\begin{flalign*}
&& -V\mathbf{D}\Delta\mathbf{E}/2
   &= -KV\nabla v\Delta\nabla v/8\pi && \\
%
&& &= -K\nabla^2 v\nabla v/4\pi,     && \\
%
\EqnTag{IV}{27}
&\text{or}& \mathbf{F} &= -\mathbf{D}\nabla v, &&
\end{flalign*}
%% -----File: 082.png---Folio 64-------
and the force per unit surface~$\mathbf{F}_s$ is by \equationref{IV}{11} \Secref{38},
\begin{align*}
\tfrac{1}{2}[V\mathbf{D}U\,d\Sigma \mathbf{E}]_{a+b}
&= (8 \pi)^{-1} K [V \nabla v U\, d\Sigma\, \nabla v]_{a+b}  \\
&= (8 \pi)^{-1} K V (\nabla v_a U\, d \Sigma_a\, \nabla v_a
                   - \nabla v_b U\, d \Sigma_a\, \nabla v_b) \\
&= (8 \pi)^{-1} K V (\nabla v_a + \nabla v_b) U\, d \Sigma_a\,
                                               (\nabla v_a - \nabla v_b),
\end{align*}
whence by \equationref{IV}{26}
\[
\EqnTag{IV}{28}
\mathbf{F}_s = -\tfrac{1}{2} \sigma [\nabla v]_{a+b}.
\]

Thus we see that Maxwell's theory as given in \Secref{34} above
reduces to the ordinary theory when $K$~is a single scalar. In fact
two particles containing charges $e\;e'$ apparently repel one another
with a force $ee' /Kr^2$ where $r$~is the distance between them, for by
equations \Eref{IV}{25}, \Eref{IV}{27} and~\Eref{IV}{28} the force in any charged body is that
due to a field of potential~$v$ given by
\[
\EqnTag{IV}{29}
Kv = \Sigma ue.
\]

\Paragraph{42} If the medium when strained remain electrically isotropic
$\Pre{\psi}{\Dop} K$ as well as~$K$ must be a simple scalar. Thus with Thomson
and Tait's notation for strain, which makes the coordinates of
$\psi$, $e$, $f$, $g$, $a/2$, $b/2$, $c/2$ we have
\begin{gather*}
\Pre{\psi}{\Dop} K = dK/de = dK/df = dK/dg  \\
dK/da = dK/db = dK/dc = 0.
\end{gather*}
Therefore $K$ is a function of $e + f + g$ only, i.e.\ of the density ($m$)
of the medium. Thus because
\[
de + df + dg = -dm/m = -d \log m
\]
we get $\Pre{\psi}{\Dop} K = - dK/d \log m = -k$ suppose. Hence
\[
\Pre{\psi}{\Dop} w
= (8 \pi)^{-1} S \mathbf{E} (dK/d \log m) \mathbf{E} = k \mathbf{E}^2/8\pi.
\]
Thus the force $-\Pre{\psi}{\Dop} w\,\Delta$ [equations~\Eref{IV}{21} \Secref{40} and~\Eref{IV}{10} \Secref{38}] resulting
from the change of~$K$ with pure strain is in the case we are
now considering
\[
\EqnTag{IV}{30}
-k \nabla \mathbf{E}^2/8\pi,
\]
and is\footnote{This is the same result as Helmholtz's on the same assumption \textit{Wiss.\ Abh.}\
\textsc{i}.\ p.~798.}
therefore, according as $k$~is positive or negative, in the
direction of or that opposite to that of the most rapid increase of
% [** PP: Added hyphenation to ``electro-motive'' (2 x) for consistency]
the square of the electro-motive\Erratum{err64a} force. Thus even in the case of a
fluid dielectric which has no internal charge but which forms part
of a non-uniform field of (electro-motive) force the surfaces of equal
pressure and therefore the free surface will if originally plane no
longer remain so.


%% -----File: 083.png---Folio 65-------

\TCSection{43}{Nature of the stress}{Nature of the Stress}

\Paragraph{43} We have seen that the stress which serves to explain the
electrostatic forces is that given by equation~(21) \Secref{40}, viz.\
% [** PP: No hyperlink; equation repeated here w/manual \tag]
\[
\tag{21}
\phi \omega = -V \mathbf{D} \omega \mathbf{E}/2 - \Pre{\psi}{\Dop}w\, \omega.
\]

Let us first consider the part $-V \mathbf{D} \omega \mathbf{E}/2$ which does not
depend on the variation of~$K$. Putting~$\omega$ first $= U\mathbf{D}$ and then
$= U \mathbf{E}$ we get
\begin{align*}
\phi U \mathbf{D} &= T\mathbf{D} T\mathbf{E}\centerdot U\mathbf{E}/2,  \\
\phi U \mathbf{E} &= T\mathbf{D} T\mathbf{E}\centerdot U\mathbf{D}/2.
\end{align*}
Therefore putting $\omega$ first $=$ any multiple of $U\mathbf{D} + U \mathbf{E}$ and then
$=$ any multiple of $U \mathbf{D} - U \mathbf{E}$ we get
\begin{align*}
\phi \omega &= \phantom{-}T \mathbf{D} T \mathbf{E} \omega/2,  \\
\phi \omega &= - T \mathbf{D} T \mathbf{E} \omega/2.
\end{align*}
Lastly, since
\[
-V \mathbf{D} \omega \mathbf{E}
=  \omega S \mathbf{D} \mathbf{E}
 - \mathbf{D} S \omega \mathbf{E}
 - \mathbf{E} S \omega \mathbf{D},
\]
we see that if we put $\omega  =$ any multiple of $V \mathbf{D}\mathbf{E}$
\[
\phi \omega = \omega S \mathbf{D} \mathbf{E}/2 = - \omega w.
\]

Thus we see that the stress now considered is a tension along
one of the bisectors of $\mathbf{D}$ and~$\mathbf{E}$ (the bisector of the positive
directions or the negative directions of both) $= T\mathbf{D} T\mathbf{E}/2$, an equal
pressure along the other bisector and a pressure $= w$ perpendicular
to both these directions. When $\mathbf{D}$~is parallel to~$\mathbf{E}$ this at once
reduces to Maxwell's case, viz.\ a tension in the direction of~$\mathbf{E}$ and
a pressure in all directions at right angles each $= w$.

\Paragraph{44} We have now to consider the other part of the stress, viz.
\begin{flalign*}
&&\phi \omega &= - \Pre{\psi}{\Dop} w\, \omega, && \\
%
\EqnTag{IV}{31}
&\text{or}&
  \phi \omega &= \Pre{\psi}{\Dop}_1\omega\, S \mathbf{E}K_1 \mathbf{E}/8\pi. &&
\end{flalign*}
If we assume that $K$ is a function of the density ($m$) of the
medium only we shall have
\[
dK/de = dK/df= dK/dg = -dK/d \log m = -k,
\]
say, and
\[
dK/da = dK/db = dK/dc = 0,
\]
as in \Secref{42}. Here however $k$ is not in general a mere scalar but a
self-conjugate linear vector function. We have then in this case
\[
\phi \omega = -\omega S \mathbf{E} k \mathbf{E}/8\pi,
\]
%% -----File: 084.png---Folio 66-------
which is a hydrostatic pressure or an equal tension in all directions
according as $S \mathbf{E} k \mathbf{E}$ is positive or negative. In this case
the~36 coordinates of $\Pre{\psi}{\Dop}_1\omega\centerdot K_1$ reduce to the 6~of $k$ for each point of
space.

A more general assumption is that $\delta K_s$ (\Secref{40}) depends only on
the elongations in the directions of the principal axes of~$K$.
Taking $i$, $j$, $k$ as unit vectors in these directions we again have
\begin{flalign*}
&& dK/da &= dK/db = dK/dc = 0, && \\
&\text{and thus}&
  \phi i &= iS\mathbf{E} (dK/de) \mathbf{E}/8\pi,  &&
\end{flalign*}
and similarly for $j$ and~$k$, so that the principal axes of the stress
now considered are the principal axes of~$K$.


\FNParagraph{45}{45\protect\footnotemark}
\footnotetext{For references to former proofs of this see foot-note to \Secref{40} above.}
The most natural simple assumption for solid dielectrics
seems to me to be that the medium is electrically isotropic before
strain, and also isotropic with regard to the strain in the sense
that if the strain be, so to speak, merely rotated, $\delta K_s$ will suffer
exactly the same rotation. We may treat this problem exactly as
we did (\Secref{20}) that of stress in terms of strain for an isotropic solid.
Thus splitting up~$\delta\psi$ into its principal elongations, i.e.\ putting
\[
\delta\psi\, \omega = - \Sigma\, \delta ei\, S i \omega,
\]
we shall get, as in \Secref{20},
\begin{align*}
\delta K_s\, \omega
&= -(\alpha - \beta) \Sigma\, \delta ei\, Si \omega
  + \beta \omega \Sigma\, \delta e \\
&= (\alpha - \beta)\, \delta\psi\, \omega
  - \beta \omega S \zeta\, \delta\psi\, \zeta.
\end{align*}
But $\delta K_s \omega = -S\,\delta \psi\, \zeta \Pre{\psi}{\Dop}_1\zeta\centerdot K_1 \omega$ by \equationref{IV}{16} \Secref{40}, so that from
\equationref{IV}{31}
\[
-8 \pi S\,\delta \psi\, \zeta \phi \zeta
  = (\alpha - \beta) S \mathbf{E}\, \delta \psi\, \mathbf{E}
    - \beta \mathbf{E}^2 S \zeta\, \delta\psi\, \zeta,
\]
whence we see by \Secref{4} above that
\[
\EqnTag{IV}{32}
\phi \omega = \bigl\{(\alpha  - \beta) \mathbf{E}S \mathbf{E}\omega
                    + \beta \mathbf{E}^2 \omega\bigr\}/8\pi,
\]
which consists of a pressure in the directions of the lines of force
$= -\alpha \mathbf{E}^2/8\pi$ and another pressure in all directions at right angles
$= -\beta \mathbf{E}^2/8\pi$.

%% -----File: 085.png---Folio 67-------


\TCSection{46}{%
  \texorpdfstring{{\scshape Magnetism}---magnetic potential, force, induction}{Magnetism}
}{{\scshape Magnetism}\\[2ex] Magnetic potential, force, induction}

\Paragraph{46} We now go on to the ordinary theory of magnetism; and
here we shall merely follow Maxwell in his General Theory, so as
to give an opportunity of comparing Quaternion proofs with Cartesian,
as we have already done in Elasticity.

We shall not consider in detail the effect of one small magnet
upon another, as this has already been done by Tait. In connection
with this I am content to remark that I think the treatment
of this problem can be made somewhat simpler than Tait's by
means of \emph{potential}.

Suppose we have a pole~$-m$ at~$O$ and a pole~$+m$ at~$O'$ where
$OO'$ is small. Calling the vector from $O$ to~$O'\,~\overline{OO'}$, let us call the
vector~$m\overline{OO'}~\mu$, so that~$\mu$ is the vector magnetic moment of the
magnet. The potential of~$-m$ at any point~$P$ is~$-mu$, where~$u$ as
usual ${}= PO^{-1}$. Similarly the potential of~$+m$ is~$mu'$, where
$u' = PO'^{-1}$. Therefore the potential of the magnet
\[
{} = m (u'-u) = mS\overline{OO'} \nabla u,
\]
where of course $P$~is the variable point implied by~$\nabla$. Thus the
potential of a small magnet~$\mu$ at any point ${}= S\mu\nabla u$.

Hence the potential of \emph{any} magnet whose magnetic moment
per unit volume at any point is~$\mathbf{I}$ is
\[
\EqnTag{IV}{33}
\Omega = S\nabla\smalliiint u \mathbf{I}\, \ds
       = - \smalliiint S\mathbf{I}\nabla u\, \ds,
\]
according to the convention of \Secref{9} above. By \equationref{II}{9} \Secref{6} this
may be put
\[
\EqnTag{IV}{34}
\Omega =  -\smalliint uS \mathbf{I}\, d \Sigma
        + \smalliiint uS \nabla \mathbf{I}\, \ds,
\]
which shews that we may consider it due to a volume density
$S\nabla\mathbf{I}$ and a surface density $-S\mathbf{I}U\,d\Sigma$\footnote{%
[Note added, 1892. More generally and better $-[S\mathbf{I}U\,d\Sigma]_{a+b}$.]}
of magnetic matter, the surface
density occurring wherever there is discontinuity in~$\mathbf{I}$.

By again considering the poles $m$ and~$-m$ of the small
magnet~$\mu$ we see that its potential energy when placed in a field of magnetic
potential~$\Omega$ is $-S\mu\nabla\Omega$, whence just as we obtained
%% -----File: 086.png---Folio 68-------
\equationref{IV}{33} we now see that the potential energy ($W$) of any magnet in
such a field is
\begin{flalign*}
\EqnTag{IV}{35}
&&
W &= -\smalliiint S \mathbf{I} \nabla \Omega\, \ds, && \\
%
\EqnTag{IV}{36}
&\text{or}&
W &= -\smalliint  \Omega S \mathbf{I}\, d\Sigma
    + \smalliiint \Omega S \nabla \mathbf{I}\, \ds, &&
\end{flalign*}
by \equationref{II}{9} \Secref{6} above, so that the potential energy is just the
same as it would be for the imaginary distribution of magnetic
matter.

\Paragraph{47} The force ($\mathbf{H}$) on a unit magnetic pole at any point
external to the magnet is given by
\begin{align*}
\mathbf{H}
&= -\nabla \Omega = -\nabla S \nabla \smalliiint u \mathbf{I}\, \ds
 = -\nabla^2 \smalliiint u \mathbf{I}\, \ds
  + \nabla V \nabla \smalliiint u \mathbf{I}\, \ds \\
&= -4\pi\mathbf{I} + \nabla\smalliiint V \mathbf{I} \nabla u\, \ds
 = -4\pi\mathbf{I} + \nabla \mathbf{A},
\end{align*}
where
\[
\EqnTag{IV}{37}
\mathbf{A} = \smalliiint V \mathbf{I} \nabla u\, \ds.
\]
Thus we see that for all external points $\mathbf{H} = \nabla \mathbf{A}$, so that~$\mathbf{A}$ is
called the vector magnetic potential. [It is to be observed that
since $\nabla \mathbf{A} = \mathbf{H} + 4 \pi \mathbf{I} =$ a vector, $S \nabla \mathbf{A} = 0$.] $\nabla \mathbf{A}$ is called the
magnetic induction and for it we use the single symbol~$\mathbf{B}$ so that
\begin{flalign*}
\EqnTag{IV}{38}
&& \mathbf{B}
&= \nabla \mathbf{A}.         && \\
%
\EqnTag{IV}{39}
&\text{Thus}& S\nabla\mathbf{B}
&= S \nabla^2 \mathbf{A} = 0, &&
\end{flalign*}
and also by the equation for $\mathbf{H}$ just given
\[
\EqnTag{IV}{40}
\mathbf{B} = \mathbf{H} + 4\pi \mathbf{I}.
\]

This is not the way in which Maxwell defines the magnetic
force and induction, but he shews quite simply (\emph{Elect.\ and Mag.}\
\S\S~398--9) that his definition and the present one are identical.
This can be shewn as easily without analysis at all.

\Paragraph{48} Where $\mathbf{I}$ is discontinuous both $\mathbf{H}$ and~$\mathbf{B}$ are also
discontinuous. From the surface density view we gave in \equationref{IV}{34}
we see that, just as we have the expression, given in \Secref{41} for
$\nabla v_a - \nabla v_b$, so now
\[
\EqnTag{IV}{41}
\mathbf{H}_b - \mathbf{H}_a
    = - 4 \pi U\, d \Sigma_a[S \mathbf{I} U\, d\Sigma]_{a+b},
\]
so that the discontinuity in~$\mathbf{H}$ is entirely normal to the surface of
discontinuity. Further from this equation we have
\begin{flalign*}
&&
& S(\mathbf{H}_b - \mathbf{H}_a)\, d\Sigma_a
 = 4\pi [S \mathbf{I}\, d\Sigma]_{a+b}, && \\
%
&\text{i.e.}&
&\quad [S(\mathbf{H} + 4\pi\mathbf{I})\, d\Sigma]_{a+b} = 0, && \\
%
\EqnTag{IV}{42}
&\text{or}&
&\quad\qquad [S \mathbf{B}\, d\Sigma]_{a+b} = 0 &&
\end{flalign*}
so that the discontinuity in~$\mathbf{B}$ is entirely tangential.
%% -----File: 087.png---Folio 69-------

From this equation we see that for any closed surface whatever
whether it include surfaces of discontinuity in $\mathbf{I}$ or not
\[
\smalliint S\, d\Sigma\,\mathbf{B} = 0.
\]
For adding these surfaces to the boundary of the inclosed space,
in accordance with \Secref{7} above, we see by \equationref{IV}{42} that they
contribute zero to the surface integral; but the total surface integral
is by \equationref{II}{9} \Secref{6} $\smalliiint S\nabla \mathbf{B}\, \ds = 0$ by \equationref{IV}{39}.


\TCSection{49}{Magnetic solenoids and shells}{Magnetic Solenoids and Shells}

\Paragraph{49} A magnet is said to be solenoidal if the imaginary magnetic
matter of \equationref{IV}{34} is entirely on the surface. Thus
for a solenoidal distribution
\[
\EqnTag{IV}{43}
S\nabla\mathbf{I}=0.
\]
In this case the potential is by \equationref{IV}{34}
\[
\EqnTag{IV}{44}
\Omega = -\smalliint uS\mathbf{I}\, d\Sigma.
\]

\Paragraph{50} A simple magnetic shell is defined as a sheet magnetised
everywhere normally to itself and such that, at any point, the
magnetic moment per unit surface is a constant called the strength
of the shell.

Calling the strength~$\phi$ we have for the potential energy at any
point by \equationref{IV}{33}
\[
\EqnTag{IV}{45}
\Omega = -\phi \smalliint S\, d\Sigma\,\nabla u.
\]
Now $-S\,d\Sigma\,\nabla u$ is the solid angle subtended by the element~$d\Sigma$ at
the point considered, so that
\[
\EqnTag{IV}{46}
\Omega = \phi \times \text{(solid angle subtended by shell at point)}.
\]

Thus if $P$ be a point on the positive side of the shell and~$P'$ a
point infinitely near~$P$ but on the negative side
\[
\text{Potential at $P - $ Potential at $P' = 4\pi \phi$},
\]
or what comes to the same thing
\[
-\int_P^{P'} S\mathbf{H}\, d\rho = 4\pi \phi.
\]
This integral may be taken along any path, e.g.\ along a path
which nowhere cuts the shell. The same integral is true if~$\mathbf{H}$ be
the magnetic force due to a whole field of which the shell is only
one of several causes, for the part contributed by the rest of the
%% -----File: 088.png---Folio 70-------
field is zero on account of the infinite proximity of $P$ and~$P'$.
For future use in electro-magnetism observe that this statement
cannot be made if for~$\mathbf{H}$ in the integral be substituted~$\mathbf{B}$.

\Paragraph{51} The condition that any magnet can be divided up into
such shells is at once seen to be that~$\mathbf{I}$ can be put in the form
\[
\EqnTag{IV}{47}
\mathbf{I} = \nabla \phi,
\]
where $\phi$ is some scalar.

In this case the potential is by \equationref{IV}{33}
\begin{flalign*}
&&
\Omega &= -\smalliiint S\mathbf{I}\nabla u\, \ds
        = -\smalliiint S\nabla\phi\nabla u\, \ds,  && \\
\intertext{or by \equationref{II}{9} \Secref{6}}
&&
\Omega &= -\smalliint \phi S\, d\Sigma\, \nabla u
         + \smalliiint\phi\nabla^2 u\, \ds, && \\
%
\EqnTag{IV}{48}
&\text{i.e.}&
\Omega &= -\smalliint \phi S\,d\Sigma\nabla u + 4\pi\phi. &&
\end{flalign*}
Remarking that the solid angle again occurs here it is needless to
interpret the equation further. By \equationref{IV}{37} we have for the
vector potential
\[
\mathbf{A} = \smalliiint V\mathbf{I}\nabla u\, \ds
           = \smalliiint V\nabla\phi\nabla u\, \ds,
\]
or by \equationref{II}{9} \Secref{6}
\[
\EqnTag{IV}{49}
A=\smalliint\phi V\, d\Sigma\, \nabla u.
\]

\Paragraph{52} The potential energy of a magnetic shell of strength~$\phi$
placed in a field of potential~$\Omega$ is of importance. We see by
\equationref{IV}{35} that it is
\[
W = -\phi\smalliint S\, d\Sigma\, \nabla\Omega.
\]
If then the magnets which cause $\Omega$ do not cut the shell anywhere,
so that $-\nabla\Omega = \nabla \mathbf{A}$, we shall have
\begin{flalign*}
\EqnTag{IV}{50}
&&
W &= \phi\smalliint S\, d\Sigma\, \nabla \mathbf{A}
   = \phi\smalliint S\, d\Sigma \mathbf{B},  &&  \\
%
\EqnTag{IV}{51}
&\text{or}&
W &= \phi\int S\, d\rho\, \mathbf{A},        &&
\end{flalign*}
by \equationref{II}{8} \Secref{6}.

Suppose now that $\mathbf{A}$ is caused by another shell of strength~$\phi'$.
Then by \equationref{IV}{49}
\[
\mathbf{A} = \phi'\smalliint V\, d\Sigma'\,\nabla u
           = \phi'\int u\,d\rho',
\]
by \equationref{II}{8} \Secref{6}. Thus finally the potential energy~$M$ of these
two shells is given by
\[
\EqnTag{IV}{52}
M = \phi\phi'\smalliint uS\, d\rho\, d\rho'.
\]

\Paragraph{53} The general theory of induced magnetism when once the
proposition (given in \equationref{IV}{42} \Secref{48}) that $[S\,d\Sigma \mathbf{B}]_{a+b}$ is zero is
%% -----File: 089.png---Folio 71-------
established, is much the same whether treated by Quaternion or
Cartesian notation. We shall therefore not enter into this part of
the subject.


\TCSection{54}{%
  \texorpdfstring{{\scshape Electro-magnetism}---general theory}{Electromagnetism}
}{{\scshape Electro-magnetism}\\[2ex] General theory}\Erratum{err67a}

\Paragraph{54} We now propose to prove the geometrical theorems connected
with Max\-well's general theory of Electro-magnetism by
means of Quaternions.

We assume the dynamical results of Chaps.~V., VI. and~VII.,
and the first six paragraphs of Chap.~VIII. of the fourth part of
his treatise.

These assumptions amount to the following. Connected with
any closed curve in an electro-magnetic field there is a function
\[
\EqnTag{IV}{53}
p = -\int S\mathbf{A}\, d\rho,
\]
where $\mathbf{A}$ is some vector function at every point of the field. The
function~$p$ has the following properties. If any circuit be made
to coincide with the curve the generalised force acting upon the
\emph{electricity} in the circuit is
\[
\EqnTag{IV}{54}
E = -\dot{p}.
\]
Again, if there be a current of electricity~$\gamma$ flowing round this
circuit, the generalised force~$X$, corresponding to any coordinate~$x$
of the position of the circuit due to the field acting upon the \emph{conductor},
is
\[
\EqnTag{IV}{55}
X = \gamma dp/dx.
\]

\Paragraph{55} The first thing to be noticed is that $p$ can be transformed
into a surface integral by \equationref{II}{8} \Secref{6} above.
\begin{flalign*}
\EqnTag{IV}{56}
&\text{Thus}&
&p = -\smalliint S\mathbf{B}\, d\Sigma, && \\
%
\EqnTag{IV}{57}
&\text{where}&
&\quad\mathbf{B} = V\nabla \mathbf{A},       && \\
&\text{so that}&
&\quad S\nabla \mathbf{B} = 0. &&
\end{flalign*}

Next we see by the fundamental connection (\Secref{34} above)
between the \textsc{e.m.f.}\ $\mathbf{E}$ and electricity, that $E$ must equal the line
integral of~$\mathbf{E}$ round the circuit, or
\[
\EqnTag{IV}{58}
E = -\smallint S\mathbf{E}\, d\rho.
\]
%% -----File: 090.png---Folio 72-------

We are now in a position to find $\mathbf{E}$ in terms of $\mathbf{A}$ and~$\mathbf{B}$, i.e.\
of $\mathbf{A}$.

\Paragraph{56} The rate of variation of $p$ is due to two causes, viz.\ the
variation of the field ($\dot{\mathbf{A}}$) and the motion of the circuit ($\dot{\rho}$). In
the time~$\delta t$ then there will be an increment~$\delta\mathbf{A}$ in~$\mathbf{A}$ and an
increment~$\delta d\Sigma$ in~$d\Sigma$ to be considered. Thus
\[
\dot{p} \delta t
= - \smallint  S \delta \mathbf{A}\, d \rho
  - \smalliint S\mathbf{B}\, \delta d\Sigma.
\]
[This amounts to assuming that $\int S \delta \mathbf{A}\, d\rho = \smalliint S \delta \mathbf{B}\, d\Sigma$,  which of
course is true by \equationref{II}{8} \Secref{6}.] Now when the circuit changes
slightly we may suppose the surface over which the new integral
extends to coincide with the original surface and a small strip at
the boundary traced out by the motion ($\dot{\rho} \delta t$) of the boundary.
Thus $\delta d \Sigma$ is zero everywhere except at the boundary and there it
\[
= V (\dot{\rho} \delta t)\, d\rho,
\]
\begin{flalign*}
&\text{so that}&
\dot{p}\, \delta t
&= - \smallint S\, \delta\mathbf{A}\, d\rho
   + \delta t \smallint S \dot{\rho} \mathbf{B}\, d\rho, && \\
%
\intertext{whence dividing by $\delta t$}
\EqnTag{IV}{59}
&&
\dot{p}
&= \smallint S(-\dot{\mathbf{A}} + V\dot{\rho} \mathbf{B})\, d\rho, && \\
%
\intertext{but by equations \Eref{IV}{54} and~\Eref{IV}{58} $\dot{p} = \int S\mathbf{E}\, d\rho$. Thus}
\EqnTag{IV}{60}
&&
\mathbf{E}
&= - \dot{\mathbf{A}} + V\dot{\rho}\mathbf{B} - \nabla\psi,
\end{flalign*}
where $\psi$ is a scalar and $-\nabla\psi$ is put as the most general vector
whose line integral round \emph{any} closed curve is zero.

\Paragraph{57} We now come to the mechanical forces exerted on an
element through which a current~$\mathbf{C}$ per unit volume flows.

We see by \equationref{IV}{55} that the work done by the mechanical
forces on any circuit through which a current of magnitude~$\gamma$ flows
in any small displacement of the circuit equals $\gamma \times$  the increment
in~$p$ caused by the displacement. Give then to each element~$d\rho$ of
the circuit an arbitrary small displacement~$\delta\rho$ and let $\mathbf{F}'$ be the
mechanical force exerted by the field upon the element. Thus as
in last section
\[
- \smallint S \mathbf{F}'\, \delta \rho = \gamma\, \delta p
    = -\gamma \smalliint S \mathbf{B}\, \delta d\Sigma
    = -\gamma \smallint S \,\delta\rho\, d\rho\, \mathbf{B}.
\]
Thus the force $\mathbf{F}'$ on the element~$d\rho$ is $\gamma V\,d\rho\, \mathbf{B}$. But we may
suppose this element to be an element~$\ds$ of volume through which
the current~$\mathbf{C}$ flows. Thus if for $\gamma\, d\rho$ we write $\mathbf{C}\, \ds$, and for $\mathbf{F}'$,
$\mathbf{F}\, \ds$, where $\mathbf{F}$ is the force per unit volume exerted by the field, we
get
\[
\EqnTag{IV}{61}
\mathbf{F} = V \mathbf{C} \mathbf{B}.
\]
%% -----File: 091.png---Folio 73-------

\Paragraph{58} So far we have been able to go by considering the electric
field as a mechanical system, but to go further (as Maxwell points
out) and find how $\mathbf{B}$ or~$\mathbf{A}$ depends on the distribution of current
and displacement in the field we must appeal to experiment. It
has been shewn by experiment that a small circuit produces
exactly the same mechanical effects on magnets as would a small
magnet, at the same point as the circuit, placed with its positive
pole pointing in the direction of the positive normal to the plane
of the circuit when the positive direction round the circuit is
taken as that of the current\footnote{%
See \Secref{1} above for the convention with respect to the relation between the
positive side of a surface and the positive direction round its boundary. Hitherto
we have had no reason for choosing either the right-handed or the left-handed
screw as the type of positive and negative rotation. But to make the statement in
the text correct we must take the former.}. Moreover the magnetic moment
of the magnet which must be placed there is proportional to the
strength of the current~$\times$ the area of the circuit. Further, the
effect of this circuit upon other such small circuits is the same as
the mutual effects of corresponding magnets. We have now only
to consider a finite circuit as split up in the usual way into a
number of elementary circuits to see that a finite circuit will act
upon magnets or upon other circuits exactly like a magnetic shell
of strength proportional to the strength of the current and
boundary coinciding with that of the circuit. The unit current
in the electro-magnetic system is so taken as to make this proportionality
an equality.

The one difference between the circuit and the magnetic shell
is that there is no discontinuity in the magnetic potential in going
round the circuit, so that by \Secref{50} above the line integral of~$\mathbf{H}$
round the circuit will be $4\pi\times$ the strength of the current. In
symbols
\[
\smallint S\mathbf{H}\, d\rho = 4\pi \smalliint S\mathbf{C}\, d\Sigma
\]
for any curve, so that by \equationref{II}{8} \Secref{6}
\begin{flalign*}
\EqnTag{IV}{62}
&&  4\pi\mathbf{C} = V\nabla\mathbf{H}, && \\
%
\EqnTag{IV}{63}
&\text{whence}&  S\nabla\mathbf{C} = 0, &&
\end{flalign*}
which of course is a direct result of our original assumption that
electricity moves like an incompressible fluid. Maxwell tacitly
assumes this by making the assumption that only one coordinate
is required to express the motion of electricity in a circuit.

%% -----File: 092.png---Folio 74-------

\Paragraph{59} We are now in a position to identify the $\mathbf{B}$ we are now
using with the magnetic induction for which we have already used
the same symbol.

We see by \equationref{IV}{50} \Secref{52} that the mechanical force on the
shell corresponding to any coordinate~$x$ is
\[
-\phi d \smalliint S\, d\Sigma\, \mathbf{B}'/dx,
\]
where $\mathbf{B}'$ is the magnetic induction; and by equations \Eref{IV}{53}
and~\Eref{IV}{55} that the force on the corresponding electric circuit is
\[
- \phi d \smalliint S\, d\Sigma\, \mathbf{B}/dx,
\]
therefore $\mathbf{B} = \mathbf{B}'$ wherever there is no magnetism. And where
there is magnetism $\mathbf{B}$ is not $= \mathbf{H}$ for $S \nabla \mathbf{B} = 0$, as we have seen.
Thus $\mathbf{B} = \mathbf{B}' $ at all points. In other words the two vectors are
identical and we are justified in using the same symbol for the
two.

This practically ends the general theory of electro-magnetism.
We content ourselves with one more application of Quaternions
in this subject. We give it because it exhibits in a striking
manner the advantages of Quaternion methods.


\TCSection{60}{Electro-magnetic stress}{Electro-magnetic phenomena explained by Stress}

\Paragraph{60} In \Secref{46} \equationref{IV}{35} we have seen that the potential
energy of a magnetic element $ = \mathbf{I}\, \ds$ in a magnetic field is $S\mathbf{IH}\, \ds$
when $\mathbf{H}$ has a potential. Maxwell assumes that the same
expression is true whether $\mathbf{H}$ have a potential or not. Assuming
this point\footnote{%
I do not defend the legitimacy of this assumption. It seems to me bold to
assume that a magnet possesses any such thing as potential energy in a field which
has no potential. If we assume $\mathbf{H}$ \emph{and its derivatives} to be continuous throughout
our typical element~$\ds$ of volume containing a great number of molecules (both
material and magnetic) the force on a magnetic molecule~$\mu$ consisting of two poles
is $-S\mu\nabla \centerdot \mathbf{H}$ and the force per unit volume $-S \mathbf{I} \nabla \centerdot \mathbf{H}$, which is only identical with the
expression $-\nabla_1 S \mathbf{I} \mathbf{H}_1$ obtained below when $\mathbf{H}$ has a potential. With this expression
a stress cannot be found that produces the force. If, however, $\mathbf{H}$ and its derivatives
be not assumed continuous in this manner the force on the magnet $\mu$ is quite indeterminate
whether the magnetic pole or the molecular current view of magnetism
be taken, unless it be specified in what way the poles and currents are distributed in
the element of volume.} with him we can find the force and couple acting on
the medium and a stress which will produce that force and
couple. The force and couple due to the magnetism of an element
%% -----File: 093.png---Folio 75-------
is obtained by giving the element an arbitrary translation and
rotation and assuming that the work done by this force and
couple = the decrement in the potential energy of the element.
Thus the force per unit volume is $-\nabla_1 S\mathbf{IH}_1$ for the decrement in
the potential energy due to a small translation $\delta\rho$ is $S\,\delta\rho\,\nabla_1 S\mathbf{IH}_1$.
Similarly the couple $\mathbf{M}$ is given by
\[
\EqnTag{IV}{64}
\mathbf{M} = V\mathbf{IH}
\]
for the decrement $-S\mathbf{M}\,\delta\omega$ in the potential energy due to a small
rotation~$\delta\omega$ is $-SV\,\delta\omega\, \mathbf{I}\centerdot \mathbf{H} = -S\,\delta\omega\, V\mathbf{I}\mathbf{H}$. The total force $\mathbf{F}$ per
unit volume is the sum of that just given and that given by
\equationref{IV}{61}, so that
\[
\mathbf{F} = V\mathbf{CB} - \nabla_1 S\mathbf{IH}_1.
\]
Therefore by \equationref{IV}{62}
\begin{align*}
4\pi\mathbf{F}
&= VV\nabla\mathbf{H}\centerdot\mathbf{B} - 4\pi\nabla_1 S\mathbf{IH}_1 \\
&= -\mathbf{H}_1 S\nabla_1\mathbf{B} + \nabla_1 S\mathbf{BH}_1
                                     - 4\pi\nabla_1 S\mathbf{IH}_1 \\
&= -\mathbf{H}_1 S\nabla_1\mathbf{B} + \nabla_1 S\mathbf{HH}_1.
\end{align*}
Now $S\nabla\mathbf{B} = 0$ so that $\mathbf{H}_1 S\nabla_1\mathbf{B} = \mathbf{H}S\Delta\mathbf{B}$, and again
\begin{flalign*}
&&
\nabla_1 S\mathbf{H}\mathbf{H}_1
&= \nabla(\mathbf{H}^2)/2,               && \\
%
\EqnTag{IV}{65}
&\text{so that}&
\mathbf{F} &= \phi\Delta,                && \\
%
\EqnTag{IV}{66}
&\text{where}&
8\pi\phi\omega &= -2\mathbf{H} S\omega\mathbf{B}
                  + \omega\mathbf{H}^2.   && \\
\intertext{From this we get}
&&
8\pi V\zeta\phi\zeta
&= 2V\mathbf{B}\mathbf{H} = 8\pi V\mathbf{I}\mathbf{H}, && \\
%
\EqnTag{IV}{67}
&\text{so that}&
\mathbf{M}
&= V\zeta\phi\zeta.                        &&
\end{flalign*}

From these two results $\bigl($equations \Eref{IV}{65} and~\Eref{IV}{67}$\bigr)$ we see by
\Secref{13} above that the stress~$\phi$ will produce all the mechanical effects
of the field.

This stress, as can be seen by giving~$\omega$ the required values in
\equationref{IV}{66}, is one of pressure $-\mathbf{H}^2/8\pi$ in all directions at right
angles to~$\mathbf{B}$ and of tension $-S\mathbf{H}(2\mathbf{B}-\mathbf{H})/8\pi$ in the direction
of~$\mathbf{H}$. When there is no magnetism $\mathbf{H} = \mathbf{B}$ so that this pressure and
tension become equal and their directions at right angles to and
along~$\mathbf{B}$ respectively. In fact we then have
\[
\EqnTag{IV}{68}
8\pi\phi\omega = -\mathbf{B}\omega\mathbf{B}.
\]
%% -----File: 094.png---Folio 76-------



\Chapter{Section V}{Hydrodynamics}

\Tocline{61}{Preliminary}

\Paragraph{61} In the applications I am about to make in this I have
practically nothing new to shew except the utility of Quaternion
methods in the general theory of Hydrodynamics in all its parts.

I therefore take a treatise (Greenhill's article in the \emph{Encyc.\
Brit.}\ on this subject) and work out the general theory on the lines
of the treatise. This is more necessary than at first sight it would
seem, for I believe mathematicians who have studied Quaternions
are under the impression that the method does not lend itself
conveniently to the establishment and treatment of such equations
as the Lagrangian and those of Cauchy's integrals. With our
meaning of~$\nabla$, however, the Quaternion treatment of these equations
is as much simpler than the Cartesian as in the case of the
Eulerian equations.

Down to \equationref{V}{13} below the subject has already been
handled by Prof.\ Hicks\footnote{%
[Note added, 1892. Prof.~Tait's name ought to be added to Prof.~Hicks's.]}
in his \emph{Quaternion treatment of Strains
and Fluid motion} (\emph{Quart.\ Journ.\ Math}.~\textsc{xiv}.\ [1877] p.~271). I do
not hesitate to go over the ground again as my methods are
different from his.


\Section{62}{Notation}

\Paragraph{62} For the vector velocity at any point we shall use~$\sigma$, for
the density~$D$, for the force per unit mass~$\mathbf{F}$, for its potential, when
it has one,~$v$, and for the pressure~$p$. For time-flux which follows
a particle we shall use $d/dt$ or the Newtonian dot, and for that
which refers to a fixed point of space $\partial/\partial t$\footnotemark.
\footnotetext{[Note added, 1892. I am aware that this is contrary to the usual English
custom, but that custom---of interchanging the meanings of $d/dt$ and~$\partial/\partial t$ as given
in the text---seems to me out of harmony with the meaning attached to~$\partial$ in other
branches of Mathematics. At any rate I have respectable fellow-sinners, e.g.\
Kirchhoff in his \textit{Mechanik}, zweite Vorlesung, et seq.]}

%% -----File: 095.png---Folio 77-------

Thus
\[
\EqnTag{V}{1}
d/dt = \partial/\partial t - S\sigma\nabla.
\]
This equation is given on p.~446 of Greenhill's article already
mentioned. In future we shall refer to this article simply as
``Greenhill's article.''


\TCSection{63}{Euler's equations}{Euler's Equations}

\Paragraph{63} To find the equation of continuity, with Greenhill, we
merely express symbolically that the rate of increase of the mass of
the fluid in any space equals the rate at which it is flowing
through the boundary. Thus $M$~being the mass in any space,
\[
\EqnTag{V}{2}
\partial M/\partial t = \smalliint DS\sigma\, d\Sigma.
\]
This is Greenhill's equation~(1) p.~445. By it and \equationref{II}{9}
\Secref{6} above we have
\[
\partial M/\partial t = \smalliiint S\nabla(D\sigma)\, \ds,
\]
whence reducing the volume to the element~$\ds$,
\[
\EqnTag{V}{3}
\partial D/\partial t = S\nabla(D\sigma).
\]
This is Greenhill's equation~(2). Thus by \equationref{V}{1} of last
section,
\[
\EqnTag{V}{4}
\dot{D}/D = S\nabla\sigma.
\]

\Paragraph{64} To obtain Euler's equations of motion we express that
the vector sum of the impressed forces for any volume equals the
vector sum of the bodily forces plus the vector sum of the pressures
on the surface. Thus
\begin{align*}
\smalliiint  D\dot{\sigma}\, \ds
&= \smalliiint  D\mathbf{F}\, \ds - \smalliint p\, d\Sigma  \\
&= \smalliiint (D\mathbf{F} - \nabla p)\, \ds,
\end{align*}
by \equationref{II}{9} \Secref{6} above. Applying this to the element~$\ds$ and
dividing by~$D\, \ds$ we have
\[
\EqnTag{V}{5}
\dot{\sigma} = \mathbf{F} - \nabla p/D,
\]
or by \equationref{V}{1} \Secref{62}
\[
\EqnTag{V}{6}
\partial\sigma/\partial t - S\sigma\nabla \centerdot \sigma
= \mathbf{F} - \nabla p/D.
\]
This is Greenhill's equations~(4)~(5)~(6).

%% -----File: 096.png---Folio 78-------

If $F(\rho, t) = 0$ ($F$ a scalar) be the equation of a surface always
containing the same particles $\dot{F}= 0$, or by \equationref{V}{1}
\[
\EqnTag{V}{7}
\partial F/ \partial t - S \sigma \nabla F = 0.
\]
This is Greenhill's equation~(7).

\Paragraph{65} Let us now put
\[
\EqnTag{V}{8}
\smallint dp/D = P.
\]
This of course assumes that $D$ is a function of $p$~only, which is not
always the case, for instance in a gas where diffusion of heat is
taking place. If $\mathbf{F}$ have a potential~$v$, $\mathbf{F} = - \nabla v$. Thus
\equationref{V}{5} becomes
\[
\EqnTag{V}{9}
\dot{\sigma} = -\nabla(v + P),
\]
and \equationref{V}{6}
\[
\EqnTag{V}{10}
\partial\sigma/\partial t - S\sigma\nabla\centerdot\sigma = -\nabla(v+P).
\]
Now
\[
S\sigma\nabla_1\centerdot \sigma_1
    = V \sigma V\nabla_1 \sigma_1 + \nabla_1 S \sigma \sigma_1
    = V \sigma V\nabla   \sigma   + \nabla(\sigma^2)/2.
\]
Thus \equationref{V}{10} becomes
\[
\EqnTag{V}{11}
\partial\sigma/\partial t + 2V\epsilon\sigma + \nabla R = 0,
\]
where the scalar~$R$ is put for $P + v - \sigma^2 / 2$ and the vector~$\epsilon$ for
$V\nabla\sigma/2$. This is Greenhill's equations~(8)~(9)~(10).

If $\epsilon = 0$, i.e.\ $V\nabla\sigma = 0$, we have  $\sigma = \nabla\phi$, whence our equation
becomes
\begin{flalign*}
&&               \nabla(\partial\phi/\partial t + R) &= 0,  && \\
%
\EqnTag{V}{12}
&\text{so that}&        \partial\phi/\partial t + R  &= H,  &&
\end{flalign*}
where $H$ is a function of $t$ only. In next section we shall obtain
in the case of an infinite fluid a generalisation of this which I
believe has not hitherto been obtained. Here we have made the
assumption that if $V \nabla \sigma= 0$ at one epoch it will be so always.
This we shall prove later.

\Paragraph{66} Greenhill next considers the case of steady motion. In
this case $\partial\sigma/\partial t = 0$, so that \equationref{V}{11} becomes
\[
\EqnTag{V}{13}
2V \epsilon \sigma + \nabla R = 0,
\]
and therefore the surface $R = \text{const.}$\ contains both vortices and
stream-lines and the relation $dR/dn = 2q\omega\,\sin\theta$ given on p.~446 of
Greenhill's article is the natural interpretation of our equation.

So far we have been going over much the same ground as
%% -----File: 097.png---Folio 79-------
Hicks, but now we enter upon applications of Quaternions that I
think have not been made before.

\Paragraph{67} Greenhill next considers rotating axes and finds the form
of the equations of motion when referred to these. Let $\sigma$ be the
velocity referred to them; so that if the axes have at any time
made the rotation~$q(\;)q^{-1}$ the real velocity will be~$q \sigma q^{-1}$. Thus\footnote{%
This is quoted as a known result because it occurs generally in the subject of
\emph{Rigid Dynamics}. No Quaternion proof, however as far as I am aware, has been
given. We therefore give one here. What is meant by rotating axes may be thus
explained.---Instead of choosing as our coordinates the vectors $\alpha'$, $\beta' \dotsc$  which occur
in any problem, we take others $\alpha$, $\beta\dotsc$ such that $\alpha' = q \alpha q^{-1}$, $\beta' = q \beta q^{-1} \dotsc$ $q(\;)q^{-1}$
may be called the integral rotation of the axes. Thus if we say that the vector
angular velocity of these axes themselves is~$\omega$ we mean that the real angular
velocity is~$q \omega q^{-1}$, so that, as can be seen by putting in \equationref{V}{$a$} below $\alpha = \text{const.}$,
or as in \equationref{IV}{58} \Secref{33} above, $\omega = 2V  q^{-1} \dot{q}$. (This maybe established also by
Tait's \textit{Quaternions}, \S~356, equation~(2), from which $q\omega q^{-1} = 2V \dot{q} q^{-1}$ or $\omega = 2V  q^{-1} \dot{q}$).
Again, when we say that the rate of increase of~$\alpha$ in space is~$\tau$, we mean that
$\dot{\alpha}' = q\tau q^{-1}$ or $\tau = q^{-1}\centerdot d(q\alpha q^{-1})/dt\centerdot q$, or
\begin{flalign*}
\EqnTag{V}{$a$}
&&
\tau &= \dot{\alpha} + 2VVq^{-1} \dot{q}\centerdot\alpha, && \\
%
\EqnTag{V}{$b$}
&\text{or}&
\tau &= \dot{\alpha} + V \omega \alpha.                  &&
\end{flalign*}
This could have been proved with fewer symbols and more explanation, but the
above seems to me the most characteristic Quaternion proof. We might have
started with not quite so general an explanation of reference to rotating axes and
so refrained from introducing the \emph{integral} rotation, and therefore also~$q$.},
% [** End of footnote]
as always with rotating axes, if~$\alpha$ be any vector function of a
particle the rate of increase of~$\alpha$ in space is
\[
\dot{\alpha}  + V \omega \alpha,
\]
where $\omega$ is the angular velocity referred to the same system of
rotating axes. Thus we see that the acceleration of a particle
$=\dot{\sigma} + V \omega \sigma$. The velocity $\sigma = \dot{\rho}' + V \omega \rho'$, where $\rho'$ is the vector
coordinate referred to our present axes (i.e.\ in the notation of the
footnote $\rho' = q\rho q^{-1}$). Thus our equation of motion is now
\[
\dot{\sigma} + V \omega \sigma = \mathbf{F} - \Nabla{\rho'} p/D,
\]
but now
\[
d/dt = \partial/\partial t - S{\dot{\rho}'} \Nabla{\rho'}\centerdot
     = \partial/\partial t - S(\sigma - \omega\rho')\Nabla{\rho'},
\]
whence changing $\rho'$~into $\rho$ we have
\[
\EqnTag{V}{14}
\partial\sigma/\partial t
    - S(\sigma - \omega\rho) \nabla\centerdot \sigma + V\omega\sigma
   = \mathbf{F} - \nabla p/D,
\]
which is the equation Greenhill gives on p.~446.

%% -----File: 098.png---Folio 80-------


\TCSection{68}{The Lagrangian equations}
          {The Lagrangian Equations\protect\footnotemark}

\footnotetext{%
[Note added, 1892. At the time of writing the essay I did not notice that
these equations are a particular case of the general equation for an elastic body
already established (see equations~\Eref{III}{15$m$} \Secref{16} and~\Eref{III}{15$n$} \Secref{16$a$})
\[
\left.
D \ddot{\rho}'
    = \mathbf{F} - 2\rho'_1 S\nabla_1 \Pre{\Psi}{\Dop} w\, \Delta.\right]
\]}
% [** End of footnote]

\Paragraph{68} We now consider the history of a single particle and for
this we require different notation.

We consider the vector coordinate ($\rho$) of a particle as a function
of some other vector~$\alpha$ (say the initial value of~$\rho$) and of~$t$.

We first require the connection between $\Nabla{\alpha}$ and $\Nabla{\rho}$. We shall
drop the affix~$\alpha$ and retain~$\rho$, so that now \emph{not} $Q (\nabla_1, \rho_1)$ but $Q (\nabla_1, \alpha_1)$
\[
= Q(\zeta,\zeta).
\]
\begin{flalign*}
&\text{\indent Now}&
S\, d\alpha\, \nabla &= S\, d\rho\, \Nabla{\rho},        && \\
%
&\text{but}&
               d\rho &= -\rho_1 S\, d\alpha\, \nabla_1,  && \\
%
&& \therefore\qquad
S\, d\alpha\, \nabla &= -S\, d\alpha\, \nabla_1 S \rho_1 \Nabla{\rho},  &&
\end{flalign*}
or since $d\alpha$ is perfectly arbitrary
\[
\EqnTag{V}{15}
\nabla = - \nabla_1 S \rho_1 \Nabla{\rho}.
\]
Thus operating upon \equationref{V}{9} \Secref{65} by $\nabla_1S\rho_1(\;)$ and remembering
that in that equation~$\nabla$ must be changed to $\Nabla{\rho}$ we get
\[
\EqnTag{V}{16}
\nabla_1 S \rho_1 \dot{\sigma} = \nabla (v + P),
\]
and this is our new equation of motion (Greenhill's~(1)~(2)~(3)
p.~448).

In \equationref{III}{1} \Secref{12} above let us put for $\rho + \eta$, $\rho$; and for $\rho$, $\alpha$.
\begin{flalign*}
\EqnTag{V}{17}
&\text{Thus}& \chi\omega &= -S\omega \nabla\centerdot \rho.  &&
\end{flalign*}
Hence by \equationref{III}{6$d$} \Secref{14} above,
\[
\text{strained vol.\ of el./original vol.}
= S \nabla_1 \nabla_2 \nabla_3 S \rho_1 \rho_2 \rho_3 / 6,
\]
whence we see that
\[
\EqnTag{V}{18}
D S \nabla_1 \nabla_2 \nabla_3\, S \rho_1 \rho_2 \rho_3 = 6 D_0,
\]
where $D_0$ is a constant which when~$\alpha$ is taken as the initial value
of~$\rho$ is the original density. This is the equation of continuity.
%% -----File: 099.png---Folio 81-------


\Section{69}{Cauchy's integrals of these equations}

\Paragraph{69} To obtain these integrals we require to express $\Nabla{\rho}$ in
terms of~$\nabla$. Now \equationref{V}{15} of last section expresses~$\nabla$ as a
linear function of~$\Nabla{\rho}$. The converse we have already seen how to
get. In fact from \equationref{II}{6$h$} \Secref{3$a$} above
\[
\EqnTag{V}{19}
2J \Nabla{\rho} = -V\rho_1\rho_2\, S\nabla_1\nabla_2\nabla,
\]
where, with Greenhill, for brevity $J$ is put for $S\nabla_1\nabla_2\nabla_3\, S\rho_1\rho_2\rho_3/6$.
The only function we wish to apply this to is $V \Nabla{\rho}\sigma$. We have
\begin{align*}
2JV \Nabla{\rho}\sigma
&= V\sigma_3\, V\rho_1\rho_2\, S\nabla_1\nabla_2\nabla_3 \\
&= (\rho_2 S\rho_1\sigma_3 - \rho_1 S\rho_2\sigma_3)\,S\nabla_1\nabla_2\nabla_3,
\end{align*}
or interchanging the suffixes 1 and~2 in the last term
\[
\EqnTag{V}{20}
JV \Nabla{\rho}\sigma
= \rho_2 S\rho_1\sigma_3\, S\nabla_1\nabla_2\nabla_3,
\]
which gives the spin at any instant in terms of our present independent
variables $\alpha$ and~$t$.

\Paragraph{70} In order to obtain Cauchy's integral of \equationref{V}{16}
operate on it by $V\nabla(\;)$. Thus
\[
0 = V(\nabla_1 + \nabla_2)\nabla_1 S\rho_1 \dot{\sigma}_2
  = S\rho_1\dot{\sigma}_2\centerdot V\nabla_2\nabla_1.
\]
Now
\[
d(S\rho_1\sigma_2 \centerdot V\nabla_2\nabla_1)/dt
  = S\sigma_1\sigma_2 \centerdot V\nabla_2\nabla_1
    + S\rho_1\dot{\sigma}_2 \centerdot V\nabla_2\nabla_1.
\]
The first term of this last expression is zero since the sign is
changed by interchanging the suffixes. From the last expression
then
\begin{align*}
S\rho_1\sigma_2 \centerdot V\nabla_2\nabla_1
&= \text{constant}  \\
&= \text{initial value of } S\zeta\sigma_2 V\nabla_2\zeta
 = -V\nabla\sigma_0,
\end{align*}
where $\sigma_0$ is the initial value of~$\sigma$. Changing the suffixes and substituting
in \equationref{V}{20} we have
\[
JV \Nabla{\rho}\sigma = -\rho_1 S\nabla\sigma_0\nabla_1,
\]
or giving $J$ its value~$D_0/D$ from \equationref{V}{18} and putting $V\Nabla{\rho}\sigma = \epsilon$,
$V\nabla\sigma_0 = \epsilon_0$, we have
\[
\EqnTag{V}{21}
\epsilon/D = -S(\epsilon_0/D_0)\nabla \centerdot \rho.
\]
This is Greenhill's equations~(4)~(5)~(6) p.~448.

The physical interpretation of this equation is quite easy.
Consider a small vector~$d\alpha$ drawn in the fluid initially. At the
%% -----File: 100.png---Folio 82-------
time~$t$ this will have become $d\rho = -S\,d\alpha\,\nabla\centerdot \rho$. Thus we see that if
a small vector~$c\epsilon_0/D_0$ be drawn in the fluid initially it will at the
time~$t$ be~$c\epsilon/D$, from which we infer that an element of a vortex
filament will always remain an element of a vortex filament;
or, a vortex filament or tube always remains a vortex filament
or tube. Again we see that $T \epsilon /D$ at any time varies directly
with the elongation in the direction of the vortex filament so
that $T\epsilon$ varies as that elongation~$\times$ the density, i.e.\ inversely as
the cross-section of a small vortex tube at the point. This is not
% [** PP: Keeping ``show'']
the easiest way of arriving at these results, but it is well to show
in passing how easy of interpretation are our results.

We see from \equationref{V}{21} that if $\epsilon_0 = 0$, $\epsilon = 0$. In other words,
if the motion have a velocity potential at one instant it will have
one always.


\Section{71}{Flow, circulation, vortex-motion}

\Paragraph{71} We are about to consider vortex-motion from another
point of view, viz.\ that of circulation.

In \Secref{12} we saw that a strain due to a small displacement~$\eta$
could be decomposed into a pure strain followed by a rotation, the
vector rotation being~$V\nabla\eta/2$. If now for~$\eta$ we put the small
vector~$\sigma\,\delta t$ we see the propriety of calling $V\nabla\sigma/2$ the \emph{spin}. This
therefore is taken as a \emph{definition} of spin. Greenhill does not take
this (usual) course but uses the property we shall immediately
prove concerning circulation and spin to lead to his definition.

It is not necessary here to define flow and circulation. Putting
$\sigma = \nabla \phi$ we see that for irrotational motion the flow $= -\int S \sigma d \rho = \int d \phi$
from one point to another is the increment in the velocity potential.
Thus for mutually reconcilable paths it is always the same.

Taking the circulation round a closed curve in the general case,
the curve not inclosing any singular region of the fluid, we may
transform the line integral into a surface integral by \equationref{II}{8}
\Secref{6} above. Thus
\[
\EqnTag{V}{22}
-\smallint S \sigma\, d\rho = - \smalliint S\, d\Sigma\, \nabla \sigma,
\]
so that the circulation round the curve equals twice the surface
integral of the spin. Hence Greenhill's definition of the spin.
%% -----File: 101.png---Folio 83-------

\Paragraph{72} From \equationref{V}{9} \Secref{65} we see that
\begin{flalign*}
&&
-\dfrac{d}{dt}\left(\int_A^B S\sigma\, d\rho\right)
&= -\int_A^B S\sigma\, d\sigma - \int_A^B S\dot{\sigma}\, d\rho  && \\
&&
&= -\tfrac{1}{2}\int_A^B d(\sigma^2) - \int_A^B d(v+P),  && \\
%
\EqnTag{V}{23}
&\text{or}&
-\dfrac{d}{dt}\left(\int_A^B S\sigma\, d\rho\right)
% [** PP: Hard-coded bracket size; contents aren't tall]
&= -\Bigl[v + P + \sigma^2/2 \Bigr]_A^B. &&
\end{flalign*}
Therefore for a closed curve
\[
d(\smallint S\sigma\, d\rho)/dt = 0,
\]
so that the circulation round the curve, and therefore the corresponding
surface integral of the spin remains always constant.
Taking the curve round a small vortex tube we once more arrive
at the propositions enunciated in \Secref{70} about vortices.

Thus at any point of a vortex tube the strength which is
defined as the product of the spin into the cross-section is constant
throughout all time. Also it is the same for all points of the tube,
for by \equationref{II}{9} \Secref{6} we have
\[
  \smalliint  S\,d\Sigma\,\nabla\sigma
= \smalliiint S\nabla^2 \sigma\, \ds = 0
\]
for any portion of the tube. But the only part contributing to
the surface integral is the ends of the part of the tube considered,
so that the strength at these two ends is the same, and is therefore
constant for the whole tube.

\Paragraph{73} These propositions are proved in the paper by Hicks
already referred to. There is yet a third method of proof which,
like Hicks's, is derived directly from the \equationref{V}{9} of motion.

We have from \equationref{V}{1} \Secref{62}
\[
\label{eqn:V.24} % [** PP: Footnoted tag, handle manually]
\tag*{(24)\protect\footnotemark}
\frac{d}{dt}\nabla - \nabla\frac{d}{dt}
= -S\sigma\nabla\centerdot \nabla + \nabla S\sigma\nabla\centerdot
= \nabla_1 S\sigma_1\nabla\centerdot\,;
\]
therefore\qquad
$dV\nabla\sigma/dt - V\nabla\dot{\sigma}
= V(\nabla_1 S\sigma_1\nabla_2\centerdot \sigma_2)$.

\footnotetext{%
[Note added, 1892. When first giving this in the \textit{Mess.\ of Math.}\ 1884 and
when again putting it in the present essay, I was unaware of Prof.\ Tait's paper in
\textit{Proc.\ R.~S.~E.}\ 1869--70, p.~143, where in the present notation he has for an incompressible
fluid,
(1) $d/dt = \partial/\partial t - S\sigma\nabla$ [given in Cartesian form],
(2) $\dot{\sigma} = -\nabla v - \nabla p/D$,
(3) $S\nabla\sigma=0$,
(4) $\nabla\dot{\sigma} - d(\nabla\sigma)/dt
    = -S\nabla\sigma\nabla\centerdot\sigma$,
(5) $V\nabla\dot{\sigma}=0$,
(6) $d\nabla\sigma/dt = -S\nabla\sigma\nabla\centerdot\sigma$.
Perhaps I have not interpreted Prof.~Tait's notation which is but briefly described
correctly, but (4)~should apparently be $V\nabla\dot{\sigma} - d(\nabla\sigma)/dt = S\nabla\sigma\nabla\centerdot \sigma$, and so agree
with (5) and~(6). It will be seen that the whole of \S\S~512--13 of Tait's \textit{Quaternions},
3rd ed., is contained in this essay. I cannot at present recall whether this is
owing wholly or in part to my being indebted to other old papers of Prof.~Tait, or
whether in writing his 3rd ed.\ he has arrived independently at the same treatment,
but am inclined to the latter belief.]} %[** End of footnote]

%% -----File: 102.png---Folio 84-------

Now by \equationref{V}{9} \Secref{65}, $V \nabla \dot{\sigma} = 0$, so that
\begin{gather*}
\EqnTag{V}{25}
d V \nabla \sigma/dt = V \nabla_1 \sigma_2 S \nabla_2 \sigma_1, \\
\therefore\qquad d(\rho + x V \nabla \sigma)/dt
    = \sigma + \dot{x} V \nabla \sigma
             + x V \nabla_1 \sigma_2\, S \nabla_2 \sigma_1.
\end{gather*}

Now
$\sigma_1 S\, \nabla_2\sigma_2 \nabla_1
  = V \sigma_2 \nabla_1 S\, \nabla_2\sigma_1
  + V \nabla_1 \nabla_2 S\, \sigma_1\sigma_2
  + V \nabla_2 \sigma_2 S\, \nabla_1\sigma_1$,
but $V \nabla_1\nabla_2 S\, \sigma_1\sigma_2 = 0$, for an interchange of suffixes leads to a change
of sign. Therefore
\[
  V \nabla_1 \sigma_2 S \nabla_2 \sigma_1
= V \nabla   \sigma   S \nabla   \sigma
  - S \nabla \sigma \nabla\centerdot \sigma.
\]

Put now $x = c/D$ where $c$ is some small constant. Thus
\begin{flalign*}
&&
\dot{x}/c = -\dot{D}/D^2
&= -S\nabla\sigma /D,  && \\
%
\EqnTag{V}{26}
&\text{and we get}&
d(\rho + cV\nabla \sigma\centerdot /D)\,dt
&= \sigma - cS\,\nabla\sigma\, \nabla\centerdot \sigma\centerdot/D. &&
\end{flalign*}

This shews that $\rho' \equiv \rho + cV\nabla\sigma/D$ is the variable vector of a
particle, for the equation asserts that the time-flux of~$\rho' =$ the
velocity at~$\rho'$. Thus again we get the laws of vortices given in
\Secref{70}.


\Section{74}{Irrotational Motion}

\Paragraph{74} We use this heading merely to connect what follows with
what Greenhill has under the same on p.~449. It is not very
appropriate here.

For irrotational motion we have seen that we may put
\[
\EqnTag{V}{27}
\sigma = \nabla \phi.
\]

If the fluid be incompressible we have further
\[
\EqnTag{V}{28}
0 = S \nabla \sigma = \nabla^2 \phi.
\]

Let $T$ be the kinetic energy of this liquid. Thus
\[
-2T = D \smalliiint (\nabla \phi)^2\, \ds
    = D \smalliint  \phi S\, d\Sigma\, \nabla \phi
    - D \smalliiint \phi \nabla^2 \phi\, \ds
\]
by \equationref{II}{9} \Secref{6} above. Thus by \equationref{V}{28}
\[
\EqnTag{V}{29}
T = -\tfrac{1}{2} D \smalliint \phi S\, d\Sigma\, \nabla\phi.
\]

Hence we see that if we have a vector~$\sigma$ which satisfies the
equations $V \nabla \sigma = 0$ and~$S \nabla \sigma = 0$ throughout any singly-connected
space, or more generally satisfies the second of these equations
%% -----File: 103.png---Folio 85-------
throughout \emph{any} space, and also has zero circulation round any
closed curve in that space\footnote{%
To insure that~$\phi$ is single valued and so, that the part of the surface integral
of \equationref{V}{29} due to ``barriers'' is zero.};
and further satisfies the equation
$S\, d\Sigma\, \sigma = 0$  at the boundary; then the function
\[
\smalliiint \sigma^2\, \ds = 0,
\]
or $ \sigma = 0$ at all points of the space, for each element of the integral
being essentially negative must be zero.

\Paragraph{75} The following important theorem now follows: If there
are given; at every point of any region the convergence $S \nabla \sigma$ and
the spin $V \nabla \sigma$, at every point of the boundary the normal velocity
(and therefore $S\, d\Sigma\, \sigma$), and the circulation round every cavity which
increases the cyclomatic number of the region; then the motion
(as given by~$\sigma$ at every point) if possible, is unique. For let~$\sigma$ be
one possible velocity, and if possible $\sigma + \tau$ another. Thus~$\tau$
satisfies all the conditions that~$\sigma$ does at the end of last section,
and therefore $\tau = 0$ at every point or~$\sigma$ is unique.


\Section{76}{Motion of a solid through a liquid}

\Paragraph{76} We assume that there is no circulation of the liquid for
any cycle; in other words, that if all the solids be brought to rest,
so will be also the liquid.

We shall take axes of reference fixed in the (one) moving solid.
In the foot-note to \Secref{67} is explained how the \emph{rotation} of these axes
is taken account of. The effect of the \emph{translation} of the origin
will be easily found.

Let $\sigma$, $\omega$ be the linear and angular velocities respectively of
the moving solid. Let us put
\[
\EqnTag{V}{30}
\phi = - S \sigma \psi - S \omega \chi,
\]
where $\phi$ is the required velocity potential of the liquid, and $\psi$
and~$\chi$ are two vector functions of the position of a point independent
of $\sigma$ and~$\omega$. Let us see whether $\psi$ and~$\chi$ \emph{can} be found so as to
satisfy these conditions. $\sigma$ and~$\omega$ are of course assumed as quite
arbitrary.

% [** PP: No hyperlink; equation repeated here]
The conditions are first the equation~(28) of continuity
\[
\tag{28}
\nabla^2 \phi = 0,
\]
%% -----File: 104.png---Folio 86-------
which gives
\[
\EqnTag{V}{31}
\nabla^2 \psi = 0, \nabla^2\chi = 0,
\]
and second, the equality of the normal velocity $-SU\, d\Sigma\, \nabla\phi$ of the
liquid at any point of the boundary with that of the boundary at
the same point. Thus for a fixed boundary
\[
   S \sigma (S\, d\Sigma\, \nabla \centerdot \psi)
 + S \omega (S\, d\Sigma\, \nabla \centerdot \chi) = 0,
\]
whence on account of the arbitrariness of $\sigma$ and~$\omega$
\[
\EqnTag{V}{32}
   S\, d\Sigma\, \nabla\centerdot \psi
 = S\, d\Sigma\, \nabla\centerdot \chi = 0.
\]

For a moving boundary again we have
\begin{flalign*}
&&
S\, d\Sigma\, \nabla\phi
&= S\, d\Sigma\, (\sigma + V \omega \rho),  && \\
&\text{or}&
S \sigma (d \Sigma + S\, d\Sigma\, \nabla\centerdot \psi)
&+ S \omega (V \rho\, d\Sigma + S\, d\Sigma\, \nabla\centerdot \chi)
= 0,                                        && \\
%
\intertext{which gives}
\EqnTag{V}{33}
&&
d\Sigma
&+ S\, d\Sigma\, \nabla\centerdot\psi = 0,  && \\
%
\EqnTag{V}{34}
&&
V\rho\, d\Sigma
&+ S\, d\Sigma\, \nabla\centerdot\chi = 0.  &&
\end{flalign*}

Now it is well known that $\psi$ and~$\chi$ can be determined to satisfy
all these conditions. In fact $x$ being a coordinate of either $\psi$ or~$\chi$
these conditions amount to:--- $\nabla^2 x = 0$ throughout the space, and
$S\, d\Sigma\, \nabla x = {}$ given value at the boundary.

\Paragraph{77} We do not propose to find $\psi$ and~$\chi$ in any particular
case. We leave p.~454 of Greenhill's article and go on to p.~455,
i.e.\ we proceed to find the equations of motion of the solid. For
this purpose we require the kinetic energy of the system. Calling
this~$T$ we shall have
\[
\EqnTag{V}{35}
T = -S \sigma\, \Sigma\sigma/2
   - S \omega\, \Phi\sigma - S \omega\, \Omega\omega/2,
\]
where $\Sigma$, $\Omega$ and~$\Phi$ are linear vector functions and the first two are
self-conjugate. This is the most general\footnote{%
For let $A(\sigma, \sigma) + B(\omega, \sigma) + C(\omega, \omega)$ be this general function $A$, $B$, and~$C$ being
scalar functions each linear in each of its constituents, and let
\[
\Sigma\sigma = \zeta \{A(\zeta,\sigma) + A(\sigma,\zeta)\}, \quad
  \Phi\sigma = \zeta   B(\zeta,\sigma), \quad
\Omega\omega = \zeta \{C(\zeta,\omega) + C(\omega,\zeta)\}.
\]} % [** End of footnote]
quadratic function of
$\omega$ and~$\sigma$. It involves 21~independent constants, six in~$\Sigma$, six in~$\Omega$
and nine in~$\Phi$. When $\psi$ and~$\chi$ are known $\Sigma$, $\Phi$ and~$\Omega$ are all
known. We will obtain the expressions for them, and for simplicity
will assume that the origin is at the centre of gravity of the
moving solid. Let~$M$ be the mass of this last and~$\mu\omega$ (where, as is
well-known, $\mu$~is a self-conjugate linear vector function) the
moment of momentum. Thus by \equationref{V}{29}
\[
2T = -D \smalliint \phi S\, d\Sigma\, \nabla\phi
    - M \sigma^2 - S \omega \mu \omega.
\]
%% -----File: 105.png---Folio 87-------
Putting $S\,d\Sigma\, \nabla\phi = S\,d\Sigma (\sigma + V\omega\rho)$ at the moving boundary and
zero at the fixed,
\[
2T = D \smalliint (S \sigma \psi + S \omega \chi)
     (S\sigma\, d\Sigma + S\omega\rho\, d\Sigma)
   - M \sigma^2 - S\omega\,\mu\omega
\]
where the surface integral must be taken only over the moving
boundary. Thus noting that $\Sigma$ and~$\Omega$ are self-conjugate
\begin{align*}
\EqnTag{V}{36}
\Sigma \lambda
&= M\lambda - \tfrac{1}{2} D \smalliint
      (\psi S\lambda\, d\Sigma +  d\Sigma\, S\lambda\psi), \\
%
\EqnTag{V}{37}
\Omega\lambda
&= \mu\lambda - \tfrac{1}{2} D \smalliint
      (\chi S\lambda\rho\, d\Sigma + V \rho\, d\Sigma\, S\lambda\chi), \\
%
\EqnTag{V}{38}
-S \lambda \Phi \lambda'
&= \tfrac{1}{2} D \smalliint
      (S\lambda\chi\, S\lambda'\, d\Sigma
     + S\lambda\rho\, d\Sigma\, S\lambda'\psi).
\end{align*}

These surface integrals can be simplified, for in each of these
equations the first surface integral equals the second. In
\equationref{V}{36} in order to prove this we have merely to put for $d\Sigma$, $-S\,d\Sigma\,\nabla \centerdot \psi$
$\bigl($\equationref{V}{33}$\bigr)$ when we shall find by \equationref{II}{9} \Secref{6} above
(since we may now suppose the integrations to extend over the
\emph{whole} boundary), that
\[
\smalliint \psi S \lambda\, d\Sigma
= - \smalliiint \psi_1 S \nabla_1 \nabla_2 S \lambda \psi_2\, \ds
=   \smalliint d\Sigma\, S \lambda \psi.
\]

Similarly for \equationref{V}{37}. Again in \equationref{V}{38},
\[
\smalliint S\lambda\chi\, S\lambda'\, d\Sigma
= - \smalliiint S \nabla_1 \nabla_2\, S\lambda\chi_1\, S\lambda'\psi_2\, \ds
=   \smalliint  S \lambda\rho\, d\Sigma\, S\lambda'\psi.
\]

Thus finally for $\Sigma$, $\Omega$ and~$\Phi$
\begin{align*}
\EqnTag{V}{39}
\Sigma\lambda
&= M \lambda - D \smalliint \psi S \lambda\, d\Sigma
 = M \lambda - D \smalliint d\Sigma\, S\lambda\psi,  \\
%
\EqnTag{V}{40}
\Omega\lambda
&= \mu\lambda - D \smalliint \chi S \lambda \rho\, d\Sigma
 = \mu\lambda - D \smalliint V \rho\, d\Sigma\, S\lambda\chi, \\
%
\EqnTag{V}{41}
-S\lambda \Phi\lambda'
&=  D \smalliint S \lambda\chi\, S \lambda'\, d \Sigma
 =  D \smalliint S \lambda\rho\, d\Sigma\, S \lambda' \psi.
\end{align*}

This last may be put in the following four forms
\begin{align*}
\EqnTag{V}{42}
\Phi' \lambda
 &= -D \smalliint d\Sigma\, S\lambda\chi
  = -D \smalliint \psi S \lambda\rho\, d\Sigma, \\
%
\EqnTag{V}{43}
\Phi \lambda
 &= -D \smalliint \chi S \lambda\, d\Sigma
  = -D \smalliint V \rho\, d \Sigma\, S \lambda \psi.
\end{align*}

Thus assuming that $\psi$ and~$\chi$ are determined we have found $T$
as a quadratic function of $\sigma$ and~$\omega$.

\Paragraph{78} If $\mathbf{P}, \mathbf{G}$ be the linear and angular impulses of the system
respectively our equations of motion are
\begin{align*}
\mathbf{\dot{P}} + V \omega \mathbf{P} &= \mathbf{F},  \\
\mathbf{\dot{G}} + V \omega \mathbf{G} &= \mathbf{M},
\end{align*}
by the footnote to \Secref{67} above. Here $\mathbf{F}$ and~$\mathbf{M}$ are the external
force and couple applied to the body. Now at the instant under
consideration $\mathbf{P} = \Nabla{\sigma} T$, $\mathbf{G} = \Nabla{\omega} T$. But if the origin had been at
the point~$-\rho$, $\mathbf{P}$ would still be $\Nabla{\sigma} T$ whereas $\mathbf{G}$ would be $\Nabla{\omega} T + V \rho \mathbf{P}$.
%% -----File: 106.png---Folio 88-------
Thus differentiation with regard to~$t$ does not affect the form of~$\mathbf{P}$
but does that of~$\mathbf{G}$. In fact using the last stated values of $\mathbf{P}$
and~$\mathbf{G}$ along with the last two equations and eventually putting
$\dot{\rho} = \sigma$, $\rho=0$ we get
\begin{align*}
\EqnTag{V}{44}
&d \Nabla{\sigma} T/dt + V\omega \Nabla{\sigma} T = \mathbf{F},  \\
%
\EqnTag{V}{45}
&d \Nabla{\omega} T/dt + V\omega \Nabla{\omega} T
                       + V\sigma \Nabla{\sigma} T = \mathbf{M}.
\end{align*}

Now from \equationref{V}{35} we see that
\begin{align*}
\EqnTag{V}{46}
\Nabla{\sigma} T &= \Sigma \sigma +  \Phi' \omega, \\
%
\EqnTag{V}{47}
\Nabla{\omega} T &=   \Phi \sigma + \Omega \omega,
\end{align*}
whence from equations \Eref{V}{44} and~\Eref{V}{45}
\begin{align*}
\EqnTag{V}{48}
&\Sigma \dot{\sigma} + \Phi' \dot{\omega}
    + V\omega\, \Sigma\sigma
    + V\omega\, \Phi'\omega = \mathbf{F},  \\
%
\EqnTag{V}{49}
&\Phi \dot{\sigma} + \Omega \dot{\omega}
    +  V\sigma\, \Sigma\sigma
    + (V\sigma\, \Phi' \omega + V\omega\, \Phi\sigma)
    +  V\omega\, \Omega\omega = \mathbf{M}.
\end{align*}

Thus we see that with Quaternion notation even the general
equations of motion are not too complicated to write down
conveniently.

We now leave Greenhill's article and proceed to certain
theorems not contained therein. The Cartesian treatment of
these subjects will be found in Lamb's treatise on \textit{The Motion of
Fluids}, Chapters \textsc{vi.} and~\textsc{ix.}, which are headed \textit{Vortex Motion} and
\textit{Viscosity} respectively.


\Section{79}{The velocity in terms of the convergences and spins}

\Paragraph{79} We have seen in \Secref{75} that if the spin and convergence
are given for each point of a bounded fluid and the normal
velocity at each point of the boundary, there is if any, but one
possible motion. We shall see directly that this unique value
always exists.

At present we observe that we cannot find a motion giving an
assigned spin and convergence for each point and \emph{any} assigned
velocity for each point of the boundary. If however with such
data a motion be possible we can find the velocity at any point
explicitly. By \equationref{II}{19} \Secref{10} above we have
\[
4\pi\sigma = -\nabla \smalliiint \nabla u \sigma\, \ds,
\]
but by \equationref{II}{9} \Secref{6}
\begin{gather*}
\smalliiint \nabla u \sigma\, \ds
  = \smalliint u\, d\Sigma\, \sigma - \smalliiint u \nabla\sigma\, \ds \\
%
\EqnTag{V}{50}
\therefore\qquad
4\pi\sigma
  = \smalliint  [V] \nabla u\, d\Sigma\, \sigma
  - \smalliiint [V] \nabla u\, \nabla\sigma\, \ds,
\end{gather*}
%% -----File: 107.png---Folio 89-------
where the square brackets indicate that we may or may not
retain the $V$ at our convenience. This equation may be put
\[
\EqnTag{V}{51}
4\pi\sigma
= \smalliint  [V](\nabla u\, d\Sigma\,\sigma - u\,d\Sigma\,\nabla\sigma)
+ \smalliiint u\nabla^2\sigma \, \ds.
\]
Both of these equations solve the question now proposed.

If the fluid be considered infinite the surface integral of
\equationref{V}{50} vanishes if~$\sigma$ converges to zero at infinity and also
that of \equationref{V}{51} if in addition the spin and convergence
converge to a quantity infinitely small compared with the reciprocal
of the distance of the surface.

\Paragraph{80} We may now consider the case when $\nabla\sigma$ (spin and convergence)
is given at all points and $S\,d\Sigma\,\sigma$ (normal velocity) at the
boundary. It must be observed that the given value of the spin
must be distributed in a solenoidal manner for
\[
S\nabla V\, \nabla\sigma = 0.
\]

Whenever the quaternion
\[
\EqnTag{V}{52}
4\pi q = -\smalliiint \nabla u\, \nabla\sigma\, \ds
       =  \nabla \smalliiint u\, \nabla\sigma\, \ds,
\]
is a vector, we see that all the conditions except the boundary
ones are satisfied by putting $\sigma = q$. Let us then see when
$q$~reduces to a vector. By \equationref{II}{9} \Secref{6} above
\[
4\pi Sq = -\smalliint  uS\, d\Sigma\, \nabla\sigma
         + \smalliiint uS\, \nabla^2\sigma\, \ds.
\]
Thus $q$ is a vector if this surface integral vanish. The surface
integral vanishes if all the vortices form closed curves within the
space. If they do not we must extend the space and make them
form closed curves outside the original space. Extending the
volume integral accordingly, we may put
\[
\EqnTag{V}{53}
4\pi\sigma' = -\smalliiint \nabla u\, \nabla\sigma\, \ds,
\]
and now $\nabla\sigma' = \nabla\sigma$. Suppose then that
\[
\EqnTag{V}{54}
\sigma = \sigma' +\sigma''.
\]
We thus get $\nabla\sigma'' = 0$ and may therefore put
\[
\EqnTag{V}{56}
\sigma'' = \nabla\phi,
\]
where $\phi$ satisfies the equations
\begin{flalign*}
\EqnTag{V}{57}
&&
&\nabla^2\phi = 0,  && \\
%
\EqnTag{V}{58}
&\text{and}&
S\, d\Sigma\, \nabla\phi = S\, &d\Sigma(\sigma - \sigma')
                         = \text{known quantity.} &&
\end{flalign*}

Now it is well known that $\phi$ can be determined so as to satisfy
%% -----File: 108.png---Folio 90-------
these two equations. Therefore the problem under discussion
always admits of solution.

The above is equivalent to Lamb's \S\S~128--131. His \S~130 is
the natural interpretation of the equation
\[
4\pi\sigma = -\smalliiint V \nabla u\, \nabla\sigma\, \ds.
\]
His \S~132 is seen at once from \equationref{V}{50} above. For in the
case he considers we, in accordance with \Secref{7} above, take each side
of the surface of discontinuity as a part of the boundary. Now
$[S\, d\Sigma\, \sigma]_{a+b} = 0$ so that for this part of the boundary we can leave
out the part $S\, d\Sigma\, \sigma$ and we get
\[
\EqnTag{V}{59}
4\pi\sigma = \smalliint  V \nabla u\, V\,  d\Sigma\, \sigma
           - \smalliiint V \nabla u\, \nabla\sigma\, \ds,
\]
so that if we regard $[V\, d\Sigma\, \sigma]_{a+b}$ as $-2\times$ an element of a vortex,
we get the same law for these vortex \emph{sheets} as for the vortices in
the rest of the fluid.

\Paragraph{81} The velocity potential due to a single vortex filament of
strength~$d\theta$ is at once obtained by putting in \equationref{V}{50} for
$V \nabla \sigma\, \ds$, $2\, d\theta\, d\rho$. Thus calling $\sigma'$ the part of the velocity due to
the filament
\[
2\pi\sigma' = -d\theta \smallint  V \nabla u\, d\rho
            = -d\theta \smalliint u_1 V \nabla_1 V\, d\Sigma\, \nabla_1
\]
by \equationref{II}{8} \Secref{6} above. Now
\[
V \nabla_1 V\, d\Sigma\, \nabla_1
  = -d\Sigma\, \nabla_1^2 + \nabla_1 S\, d\Sigma\, \nabla_1.
\]
Hence, since $\nabla^2 u = 0$ for all points not on the filament
\[
\EqnTag{V}{60}
2\pi\sigma' = \nabla (d\theta \smalliint S\, d\Sigma\, \nabla u).
\]
Thus the velocity potential $= (2\pi)^{-l} \times$ the strength~$\times$ the solid
angle subtended by the filament at the point considered.


\SectionTitle{Kinetic Energy}

\Paragraph{82} Let us put
\begin{flalign*}
\EqnTag{V}{61}
&&
4\pi q &= \smalliiint u \nabla \sigma\, \ds,  && \\
%
\EqnTag{V}{62}
&\text{so that}&
\sigma &= \nabla q.                           &&
\end{flalign*}

We may now find expressions for $T$, the kinetic energy, in one
or two interesting forms. We have
\begin{flalign*}
&&
2T &= - \smalliiint D \sigma^2\, \ds,  && \\
&\text{whence}&
2T  = - \smalliiint D S\sigma\, \nabla q\, \ds
   &= - \smalliint  D S\sigma\, d\Sigma\, q
      + \smalliiint D_1 S\sigma_1\, \nabla_1 q\, \ds.  &&
\end{flalign*}

%% -----File: 109.png---Folio 91-------

We assume that the fluid extends to infinity and that the
vortices and convergences are all at a finite distance, so that at
infinity $\sigma$ is of order~$1/R^2$ and $q$ of order~$1/R$. Thus the surface
integral vanishes and we have
\[
\EqnTag{V}{63}
2T = \smalliiint D_1 S \sigma_1\, \nabla_1 q\, \ds,
\]
or, putting in the value of~$q$ from \equationref{V}{61},
\[
\EqnTag{V}{64}
T = (8 \pi)^{-1} \smalliiint\smalliiint
    u D_1 S\sigma_1\, \nabla_1 \nabla_2 \sigma_2\, \ds_1\, \ds_2.
\]
These are Lamb's equations (28) and~(29), p.~160, generalised.

Similarly his equation~30 generalised is
\begin{flalign*}
\EqnTag{V}{65}
&&
2T
&= \smalliiint D_1 S \rho\sigma_1\, \nabla_1 \sigma_1\, \ds, && \\
&\text{for}&
\smalliiint D_1 S \rho \sigma_1\, \nabla_1\sigma_1\, \ds
&= \smalliint  D S \rho \sigma\, d\Sigma\, \sigma
 - \smalliiint D S \zeta\sigma\zeta\sigma\, \ds,  &&
\end{flalign*}
by \equationref{II}{9} \Secref{6} above. But
\[
\zeta\sigma\zeta = 2\zeta S\sigma\zeta - \zeta^2\sigma = \sigma,
\]
and the surface integral vanishes as before.


\Section{83}{Viscosity}

\Paragraph{83} To consider viscosity the assumption is made that the
shearing stress which causes it is $\mu \times{}$ the rate of shear of the
moving fluid. $\mu$~is assumed independent of the velocity and
experiment seems to shew that it is also independent of the
density. This last we assume though the variation with density
can be easily treated.

Consider a general strain $\chi\omega$. Since
\begin{gather*}
mS\lambda\mu\nu = S \chi\lambda\, \chi\mu\, \chi\nu
                = S \lambda \chi' V\, \chi\mu\, \chi\nu\footnotemark, \\
V \chi\mu\, \chi\nu = m \chi'^{-1} V \mu \nu.
\end{gather*}
\footnotetext{%
Kelland and Tait's \textit{Quaternions}, chap.~x.\ equation~($n$). We have already used
a particular case of this in \Secref{15}, above.}%
% [** PP: End of footnote, no paragraph]
Putting $\mu$ and~$\nu$ for any two vectors perpendicular to~$\omega$ we see
that the normal~$U \omega$ to any interface becomes~$U \chi'^{-1} \omega$ by the strain.
Hence the interface~$\omega$ experiences a shear (strain) which equals
the resolved part of $\chi U \omega$ perpendicular to the vector $\chi'^{-1} \omega$ equals
resolved part of $(\chi - \chi'^{-1})U \omega$ perpendicular to~$\chi'^{-1} \omega$. Now when
$\chi$ is the strain function due to a small displacement~$\sigma\,dt$, by \Secref{12}
above
\begin{flalign*}
\EqnTag{V}{66}
&&
\chi\omega &= \omega - S \omega\, \nabla \centerdot \sigma\, dt,  && \\
%% -----File: 110.png---Folio 92-------
\EqnTag{V}{67}
&\text{whence}&
\chi'^{-1} \omega &= \omega + \nabla_1 S \omega \sigma_1\, dt, &&
\end{flalign*}
and we see that the shear is the resolved part of
\[
-(SU\omega\nabla\centerdot \sigma + \nabla_1 SU\omega\sigma_1)\, dt
\]
perpendicular to $\omega$, i.e.\ parallel to the interface~$\omega$. From this we
see by our assumption concerning viscosity that any element of
the fluid is subject to a stress~$\phi$ given by
\[
\phi \omega
  = -R\omega - \mu(S\omega\, \nabla\centerdot \sigma
                 + \nabla_1 S \omega \sigma_1)
\]
$R$ being a scalar. Now we define the pressure~$p$ by putting
\begin{gather*}
3p = S \zeta \phi \zeta = 3R  + 2 \mu S\, \nabla\sigma \\
%
\EqnTag{V}{68}
\therefore\qquad
\phi\omega = -p\omega + \tfrac{2}{3} \mu \omega S\, \nabla\sigma
    - \mu(S\omega\nabla\centerdot \sigma
        + \nabla_1 S\omega\sigma_1).
\end{gather*}
The equation of motion is
\[
D \dot{\sigma} = D\mathbf{F} + \phi\Delta,
\]
or
\[
\EqnTag{V}{69}
D(\partial\sigma/\partial t - S\sigma\, \nabla\centerdot \sigma)
  = D \dot{\sigma}
  = D \mathbf{F} - \nabla p - \mu\nabla S\, \nabla\sigma/3
    - \mu\nabla^2 \sigma.
\]
If $\mu$ be not as stated independent of~$D$ this equation must be
made to contain certain space fluxes of~$\mu$ which are quite easy to
insert.

\Paragraph{84} In \S~179 of Lamb's treatise he considers the dissipation
function due to the viscosity of a fluid in a way that seems to me
misleading if not wrong. It appears as if he should add to the
first expression of that section
\[
u\,dp_{xx}/dx + v\,dp_{xy}/dx + w\,dp_{xz}/dx.
\]
He refers to Stokes, \textit{Camb.\ Trans.}\ vol.~\textsc{ix.}\ p.~58. Let us give the
quaternion treatment of Stokes's method.

If $T$ is the kinetic energy of any portion of the fluid
\begin{flalign*}
&&
2T &= -\smalliiint D\sigma^2\, \ds.  && \\
%
&\text{Thus}&
\because\quad
&d(D\, \ds)/dt = 0,                  && \\
%
&&
\dot{T}
&= -\smalliiint DS\sigma \dot{\sigma}\, \ds,    && \\
%
&\text{but}&
D\dot{\sigma}
&= D\mathbf{F} + \phi_1 \nabla_1,    && \\
%
&\text{so that}&
\dot{T}
&= -\smalliiint DS \sigma \mathbf{F}\, \ds
   -\smalliiint S \sigma \phi_1 \nabla_1\, \ds  && \\
%
\EqnTag{V}{70}
&\text{or}&
\dot{T} = -\smalliiint DS \sigma \mathbf{F}\, \ds
&- \smalliint S \sigma \phi\, d\Sigma
+ \smalliiint S \sigma_1 \phi \nabla_1\, \ds.  &&
\end{flalign*}

If now $\phi$ be expressed in terms of $p$ and~$\mu$ we have $\dot{T}$~depending
on $\mathbf{F}$, $p$ and~$\mu$. The part of~$\dot{T}$ depending on $\mathbf{F}$ and~$p$ represents
energy stored up as potential energy of position and potential
%% -----File: 111.png---Folio 93-------
energy of strain respectively, but the part depending on~$\mu$ represents
a loss of energy to the system we are considering the energy
being converted into heat.

Thus putting $\phi = p + \varpi$ we have by \equationref{V}{68}
\[
\EqnTag{V}{71}
\left.
\begin{aligned}
\varpi \omega
&= \tfrac{2}{3} \mu\omega S \nabla\sigma
      - \mu(S\omega\nabla\centerdot \sigma + \nabla_1 S\omega\sigma_1) \\
&= \tfrac{2}{3} \mu\omega S \zeta\psi\zeta + 2\mu\psi\omega,
\end{aligned}\right\}
\]
where $\psi$ is given by \equationref{V}{75} below.

Thus we see that the rate of loss of energy is
\[
\EqnTag{V}{72}
\smalliint  S \sigma   \varpi\, d\Sigma
- \smalliiint S \sigma_1 \varpi\nabla_1\, \ds.
\]
The surface integral is the work done by viscosity against the
moving fluid at the boundary and the volume integral is considered
due to the work done against the straining of the fluid.
Thus we put
\[
\EqnTag{V}{73}
F = -S\sigma_1 \varpi \nabla_1,
\]
and call $F$ the ``dissipation function.''

By \equationref{II}{6} \Secref{3} above
\[
\EqnTag{V}{74}
F = - S \psi \zeta \varpi \zeta,
\]
where $\psi$ is the rate of pure strain of the fluid, i.e.\
\[
\EqnTag{V}{75}
2\psi\omega
  = -S\omega\,\nabla\centerdot \sigma - \nabla_1 S \omega \sigma_1.
\]

Again by \equationref{III}{18} \Secref{18} above because $F$ is quadratic in~$\psi$
\[
\EqnTag{V}{76}
F = -S\psi\zeta \Pre{\psi}{\Dop} F\, \zeta/2,
\]
so that from \equationref{V}{74} we have, by \Secref{4} above,
\[
\EqnTag{V}{77}
\varpi = \Pre{\psi}{\Dop} F/2.
\]

Substituting for $\varpi$ from \equationref{V}{71} in \equationref{V}{74}
\[
\EqnTag{V}{78}
F = - \tfrac{2}{3} \mu(S \zeta \psi \zeta )^2
    - 2\mu\psi \zeta\psi\zeta,
\]
which gives $F$ in terms of $\psi$.

%% -----File: 112.png---Folio 94-------


\Chapter{Section VI}{The Vortex-Atom Theory}

\Tocline{85}{Preliminary}

\Paragraph{85} If Quaternions can give valuable hints or indicate a
promising method of dealing with the highly interesting mathematical
theory of Vortex-Atoms, I think this alone ought to be
sufficient defence of its claims to be within the range of practical
methods of investigation.

In what follows I think I may be said to have indicated a
hopeful path to follow in order to test to some extent the soundness
of this theory.


\Section{86}{Statement of Sir Wm.~Thomson's and Prof.~Hicks's theories}

\Paragraph{86} Sir Wm.~Thomson's theory is so well-known that it is not
necessary to state it in detail. Matter is some differentiation of
space which can vary its position carrying with it so to speak
certain phenomena, some of which admit of definite quantitative
measurement. Perhaps the most important of these phenomena
is what is called mass. Now, says in effect Sir Wm. Thomson, if
we suppose all space filled with an incompressible perfect fluid the
vortices in it are just such differentiations of space. They also
carry about with them definite quantitative phenomena. Can we
prove that these hypothetical vortices would act upon one another
as do the atoms of matter, the laws of whose action are contained
in the various Sciences, e.g.\ Physics, Chemistry and Physiology?
The problem in its first stages at any rate is a mathematical one,
but during the many years it has been before the mathematical
world very little progress has been made with it.

Hicks's extension of this theory is perhaps not so well known,
but it seems to me quite as interesting and more likely to tally
with the known phenomena of matter. He enunciates his theory
%% -----File: 113.png---Folio 95-------
in the \emph{Proceedings of the Cambridge Philosophical Society}, vol.~\textsc{iii}.
p.~276. It differs from Thomson's simply in assuming that the
fluid does not quite fill space---that there are in it bubbles\footnote{%
By ``bubbles'' of course I do not mean spaces occupied by another kind of
fluid of smaller density but actual vacua. Thus a bubble may start into existence
where none previously existed, or again a bubble may completely disappear.}.
These bubbles will find their way to where the pressure is least,
i.e.\ speaking generally to the centre of some at least of the vortices.
Thus we have another source of differentiation of space
and general considerations seem to point to \emph{these} differentiations
being the true atoms, though of course ``atom'' is no longer a
descriptive term.


\Section{87}{General considerations concerning these theories}

\Paragraph{87} In Maxwell's article \emph{Atom} in the \emph{Encyc.\ Brit.}\ p.~45, he
says---``One of the first if not the very first desideratum in a
complete theory of matter is to explain first mass and second
gravitation\ldots\ldots\ In Thomson's theory the mass of bodies requires
explanation. We have to explain the inertia of what is only a
mode of motion and inertia is a property of matter, not of modes
of motion. It is true that a vortex ring at any given instant has
a definite momentum and a definite energy, but to shew that
bodies built up of vortex rings would have such momentum and
energy as we know them to have is in the present state of the
theory a very difficult task.

``It may seem hard to say of an infant theory that it is bound
to explain gravitation.''

Now as Hicks tells us what induced him to give his theory
was the promise it gave of explaining gravitation. But I believe
nowhere does he point out the still more important result that
probably on his theory we can explain inertia.

The statement of the principal property of inertia put scientifically
is that the motion of the centre of gravity of any two
bodies approximates more and more nearly to uniform velocity in
a straight line, the more nearly they are isolated from external
influence. But this property is probably true of Hicks's bubbles.
The centre of gravity of any portion of the fluid containing certain
bubbles $(1)$, $(2) \dotsc (n)$ will if approximately isolated from all the
%% -----File: 114.png---Folio 96-------
rest of the fluid move approximately in a straight line, but this
amounts to saying that the centre of volume of the $n$~bubbles will
also move uniformly in a straight line \emph{if} the centre of the whole
volume (bubbles \emph{and} fluid) considered, move similarly. These
conditions are not evidently true, but I think they are probably
so when we consider groups of large numbers of bubbles.

Whether this theory explains gravitation is one of the principal
questions to be considered in the following first trial.


\Section{88}{Description of the method here adopted}

\Paragraph{88} Finding it practically impossible to consider the real
problem of a number of bubbles in the liquid, I consider the fluid
continuous and not incompressible. Let us assume that
\[
\EqnTag{VI}{1}
D = \tanh (p/c),
\]
where $c$ is some very small constant which in the limit $= 0$. For
ordinary values of~$p$, $D \text{ very nearly} = 1$. It is only when
$p$~becomes comparable with~$c$ that~$D$ varies. When $p=0$, $D=0$.
Also
\[
\EqnTag{VI}{2}
P = \smallint(dp/D) = c \log \sinh (p/c).
\]
For ordinary values of $p$ then, $P = p$ but when $p$ becomes comparable
with~$c$, $P$~varies and when $p = 0$, $P = -\infty$. If we assume
then for the greater part of the fluid that $p$~is large compared
with the small quantity~$c$ we see that the fluid will have almost
exactly the same properties as a liquid containing bubbles. We
may now apply the equations of motion of \Secref{64}, \Secref{65}.

Now we know the velocity at any point of an infinite fluid in
terms of the spins and convergences at all points. From this we
may deduce the acceleration in terms of the spins, convergences
and time-fluxes. It so happens that by the equations in \Secref{73} we
can get rid of the time-fluxes of the spins. This greatly simplifies
the further discussion of the problem. The equation that we thus
deduce forms the starting point of our investigation, for the
phenomena of gravitation, electro-magnetism, stress,~\&c.,\ are exhibited
and measured by the \emph{acceleration} in bodies due to their relative
positions.

The equation we obtain at once gives a generalisation of the
integral, \equationref{V}{12} \Secref{65} above.

At the end of this section I give an equation which is rather
%% -----File: 115.png---Folio 97-------
complicated but which promises to enable us to deal more
rigorously with our problem than I profess to have done below.

We proceed to the investigation just indicated.


\Section{89}{Acceleration in terms of the convergences, their time-fluxes, and the spins}

\Paragraph{89} Let us put for the convergence and twice the spin $m$
and~$\tau$ respectively, so that
\begin{align*}
\EqnTag{VI}{3}
S \nabla \sigma &= m,     \\
%
\EqnTag{VI}{4}
V \nabla \sigma &= \tau.
\end{align*}

We have seen that in \equationref{V}{50} \Secref{79} we may neglect the
surface integral. Thus
\begin{flalign*}
\EqnTag{VI}{5}
&&
4\pi\sigma
&= \nabla \smalliiint u (m + \tau)\, \ds,  && \\
&&
\therefore\qquad 4\pi\partial\sigma/\partial t
&= \nabla \smalliiint u \partial(m+\tau)/\partial t\centerdot\, \ds. && \\
&\text{\indent By \Secref{73}}&
\dot{\tau}
&= \tau S \nabla \sigma - S\tau\nabla\centerdot \sigma, &&
\end{flalign*}
but by \equationref{V}{1} \Secref{62}
\begin{align*}
\partial\tau/\partial t
&= \dot{\tau} + S \sigma \nabla \centerdot \tau,  \\
\therefore\qquad \partial\tau/\partial t
&= S \sigma \nabla\centerdot \tau + \tau S\nabla\sigma
 - S \tau\nabla\centerdot \sigma,                 \\
%
\intertext{or because $S \nabla \tau = 0$}
\EqnTag{VI}{6}
\partial\tau/\partial t
&= \tau S \Delta\sigma - \sigma S\Delta\tau = V \nabla V \sigma \tau \\
%
\EqnTag{VI}{7}
\therefore\qquad 4\pi\partial\sigma/\partial t
&= \nabla \smalliiint u(V\nabla V\sigma\tau
                      + \partial m/\partial t)\, \ds.
\end{align*}

This equation is not in a convenient form for our purpose, for
$\tau$ and~$m$ may be and most probably will be discontinuous, so that
it is advisable to get rid of their derivatives. Moreover it is
advisable to allow the time-flux of~$m$ to appear only under the
form $d(m\, \ds)/dt$ because in case of pulsations the time integral of
this expression will be zero, whereas we can predicate no such
thing of $(\partial m/\partial t)\, \ds$, or indeed of $\dot{m}\, \ds$. Let us first then consider
the first term in \equationref{VI}{7}, viz.\ $\nabla \smalliiint u V \nabla V \sigma \tau\, \ds$. Using
\equationref{II}{9} \Secref{6}, and neglecting the surface integral at infinity, this
becomes
\begin{align*}
&-\nabla \smalliiint V \nabla u V \sigma \tau\, \ds
= \nabla V \nabla \smalliiint u V \sigma \tau\, \ds\qquad \tag*{[\Secref{9}]} \\
%
&= \nabla^2 \smalliiint u V \sigma \tau\, \ds
 - \nabla S \nabla \smalliiint u V \sigma \tau\, \ds  \\
%
&= 4\pi V \sigma\tau + \nabla \smalliiint S\sigma\tau\, \nabla u\, \ds,
\end{align*}
by \equationref{II}{19} \Secref{10} above.

%% -----File: 116.png---Folio 98-------

Remembering now (\Secref{65} above), that
\[
\dot{\sigma} = \partial\sigma/\partial t - V\sigma\tau - \nabla(\sigma^2)/2,
\]
we see by \equationref{VI}{7} that
\[
\EqnTag{VI}{8}
4\pi\dot{\sigma} = -2\pi\nabla\centerdot \sigma^2
    + \nabla \smalliiint S\sigma\tau\, \nabla u\, \ds
    + \nabla \smalliiint u (\partial m/\partial t)\, \ds,
\]
whence by \equationref{V}{9} \Secref{65} above
\[
\EqnTag{VI}{9}
P + v - \sigma^2/2
+ (4 \pi)^{-1} \smalliiint
    (S\sigma\tau\, \nabla u + u\,\partial m/\partial t)\, \ds = H,
\]
where $H$ is a function of the time only. This is a generalisation
of \equationref{V}{12} \Secref{65}, for assuming that $\tau = 0$ we know by \Secref{79}
that $\partial\phi/\partial t = (4 \pi)^{-1}\! \smalliiint u(\partial m/\partial t)\, \ds$.

\Paragraph{90} This integral equation may be put into several different
forms by means of \equationref{II}{9} \Secref{6} above. The form that is
useful to us is the one in which instead of $\partial m/\partial t$ we have $d(m\, \ds)/dt$
as we have already seen. Now $d(\ds)/dt = -m\, \ds$. Therefore
\begin{align*}
d(m\, \ds)/dt
&= (\dot{m} - m^2) \, \ds \\
&= (\partial m/\partial t - S\sigma\,\nabla m - m^2)\, \ds.
\end{align*}

Substituting from this for $\partial m/\partial t$, and then transforming by
\equationref{II}{9} \Secref{6} so as to get rid of the space variations of~$m$ involved
in $S\sigma\,\nabla m$ we get
\begin{align*}
   \smalliiint u\, (\partial m/\partial t)\, \ds
&= \smalliiint u\,d(m\, \ds)/dt
 + \smalliiint (um^2 - mS\sigma\, \nabla u - u m^2)\, \ds \\
&= \smalliiint u\, d(m\, \ds)/dt - \smalliiint m S\sigma\, \nabla u\, \ds.
\end{align*}

Thus from equations \Eref{VI}{8} and~\Eref{VI}{9}
\[
\EqnTag{VI}{10}
4\pi\dot{\sigma} = -2\pi\nabla \centerdot \sigma^2
    + \nabla \smalliiint S \nabla u(V \sigma\tau - m\sigma)\, \ds
    + \nabla \smalliiint u\,d(m\, \ds)/dt,
\]
and
\[
\EqnTag{VI}{11}
P + v - \sigma^2/2
    + (4\pi)^{-1} \smalliiint \{\ds\,
        S \nabla u(V\sigma\tau - m\sigma) + u\,d(m\, \ds)/dt\} = H.
\]


\TCSection{91}{Sir Wm.~Thomson's theory}{Sir Wm.~Thomson's Theory}

\Paragraph{91} We are now in a position to examine the two theories.
We first take Thomson's, which is considerably the simpler, and
which therefore serves as an introduction to the other. We have
then $m = 0$. Thus equations \Eref{VI}{10} and~\Eref{VI}{11} become
\begin{align*}
\EqnTag{VI}{12}
&4\pi\dot{\sigma}
    = -2\pi\nabla\centerdot \sigma^2
      + \nabla \smalliiint S\sigma\tau\, \nabla u\, \ds, \\
%
\EqnTag{VI}{13}
&p/D - \sigma^2/2
      + (4\pi)^{-1} \smalliiint S\sigma\tau\, \nabla u\, \ds = H,
\end{align*}
for in the present theories we may put $v = 0$.

We shall consider the two terms on the right of \equationref{VI}{12}
separately. The second term gives then an apparent force per
unit mass due to a potential
\[
\EqnTag{VI}{14}
-(4\pi)^{-1} \smalliiint S\sigma\tau\, \nabla u\, \ds.
\]

%% -----File: 117.png---Folio 99-------

Comparing this with \equationref{IV}{33} \Secref{46} above we see that this
potential is the same as that of a magnetic system given by
\[
\EqnTag{VI}{15}
4 \pi \mathbf{I} = V \sigma \tau.
\]

Now this magnetism is zero where $\tau$ is zero. In other words
it is only present where the vortex-atoms are, and therefore it
cannot be so distributed as to give an apparent force of gravitation,
for taking the view of magnetic matter expressed in \equationref{IV}{34}
\Secref{46} we see that there must, in every complete vortex-atom, be
a sum of magnetic matter exactly $= 0$. Nor again is it likely to
explain the phenomena of permanent magnets, because, assuming
that for any given small space including many vortex-atoms
$\smalliiint V\sigma\tau\, \ds$ is not zero, the apparent force produced will affect all
other parts of space independently of whether this same integral
for them is not or is zero. But to explain the phenomena of
permanent magnets we must assume that the effect takes place
only on portions of space where there is positive magnetic matter.
This term then gives us no phenomena analogous to physical
phenomena. As a matter of fact it probably has no visible effect
on large groups of vortices, for there is no reason to suppose that
the vector~$V\sigma\tau$ is distributed otherwise than at random.

\Paragraph{92} Let us now consider the other term in \equationref{VI}{12}, and
neglecting the term already considered, put
\[
\EqnTag{VI}{16}
\dot{\sigma} = -\nabla\centerdot \sigma^2/2.
\]

The phenomena resulting from this are the same as would
follow from a stress in a medium, the stress being an equal tension
in all directions $= -\sigma^2/2$. Now comparing \equationref{IV}{62} \Secref{58} with
\equationref{VI}{4} \Secref{89}, we see that $\sigma$ depends on $\tau/4\pi$ in exactly the
same way as does~$\mathbf{H}$ on~$\mathbf{C}$. The question then arises---is the stress
we are now considering equivalent in its mechanical effects upon
the vortex-atoms to the stress given by \equationref{IV}{68} \Secref{60} above,
which explains the mechanical effect of one current on another?
We saw in \Secref{60} that this stress is a tension $-{\mathbf{H}}^2/8\pi$ along the
vector~$\mathbf{H}$ and an equal pressure in all directions at right angles.
The effects then would be the same only if $\sigma$~be at right angles to
the surface of our atom. But this is obviously not in general the
case. From this analogy we can see however what approximately
will happen to our atom. For instead of $\sigma$~being at right angles
to the surface it is in all probability very nearly tangential.
Assuming that it is actually tangential we see that at the surface
%% -----File: 118.png---Folio 100-------
of the atom we have a tension exactly corresponding to the
pressure which in the electric analogue will be exerted on this
surface. In other words, the atoms will act on each other very
approximately in what may be called a converse way to the small
circuits in the electric analogue; i.e.\ where, in the electric analogue
there is an attraction, in the hydrodynamic case there will be an
apparent repulsion, and vice~vers�.

Now each vortex-atom forms a small circuit and therefore acts
in the converse way to a small magnet. In other words, each atom
acts upon each other atom as if it were charged with \emph{attracting}
magnetic matter. Thus we see that if we could suppose certain
extra atomic vortices to exist and to be disposed throughout space
in what at present must be considered quite an artificial manner
with reference to the \emph{atomic} vortices we could rear up a fabric
which would explain gravitation. This conception however is of
very little use for our present purpose.

From these considerations I think we have every reason to
believe that Thomson's theory in its native simplicity does not
promise to lead us to the physical phenomena of matter. We
pass on therefore to Hicks's.


\TCSection{93}{Prof.~Hicks's theory}{Prof.~Hicks's Theory}

\Paragraph{93} Hicks in his theory, as I understand him, assumes that
the bubbles always remain associated with the same particles of
the fluid. This of course is probably not the case. By reason of
the variation of the pressure with the time it is probable that
evanescent bubbles start into existence and disappear at various
parts of the fluid. This requires some few preliminary remarks.

The particles of the fluid with which bubbles are permanently
(i.e.\ throughout the greater part of each small but not infinitely
small interval of time) associated are those where the intensity of
spin is greatest. If the intensity of spin is quite various at
different points we shall thus have vortices where there is
generally no bubble, extra material vortices in fact. We must
suppose these distributed quite at random till the more exact
mathematical treatment of our problem leads us to suppose otherwise.
Now evanescent bubbles will occur rather in these vortices
than in parts of the fluid where is no vortex at all (if we may
%% -----File: 119.png---Folio 101-------
suppose such parts to exist), and of course they will occur more
readily in stronger than in weaker vortices. At the present stage
of the theory then we may suppose evanescent bubbles to occur in
all parts of the fluid. As a first approximation to the consideration
of the effect of these bubbles we may assume a part~$m'$ of~$m$
to be continuously distributed through space. Putting
\[
\EqnTag{VI}{17}
m = m' + M,
\]
we must suppose $M$ to be present only at the material bubbles
where it is probably discontinuous, whereas $m'$~is continuous
throughout both the material vortices and the rest of the fluid.

Now when on account of variation of pressure $m'$ and~$M$ are
affected---is it probable that in the neighbourhood of a permanent
bubble $m'$ and~$M$ are of the same or opposite signs? To answer
this question observe that what we call~$m'$ continuously distributed
is really a series of discontinuous values of~$m$ scattered at random
through space so that~$m'$ is probably very small. A decrement of
pressure will cause an increment of evanescent bubbles, i.e.\ a
decrement in~$m'$. An increment of the permanent bubbles will
also take place the magnitude of which by what has been said
concerning~$m'$ will not be by any means accounted for by the
decrement in~$m'$. There will therefore also be a decrement in~$M$.
Similarly for an increment in the pressure. $M$ and~$m'$ may
therefore be assumed to be of the same sign.

After noticing that $m'$~is continuous and therefore that there is
no objection to introducing its space variations we are furnished
with all the materials necessary for discussing our problem. The
equation we shall use is~\Eref{VI}{10} of \Secref{90} above. We divide its discussion
into two parts as follows.


% [** PP: To prevent an overfull hbox, \Paragraph does not indent 94.]
\Section{94}{Consideration of all the terms except \texorpdfstring{$-\nabla\centerdot(\sigma^2)/2$}{the last}}\Erratum{err101a}% [** PP: Added \centerdot]

\Paragraph{94} The reasons that we have already seen in \Secref{91} for
neglecting $\nabla \smalliiint S\sigma\tau\, \nabla u\, \ds$ still hold good so we put this aside.
This is not the case with $-\nabla \smalliiint m S\sigma\, \nabla u\, \ds$  but we can neglect
$\nabla \smalliiint u\, d(m\, \ds)/dt$ for the average value of $d(m\, \ds)/dt$ for any particle
is zero. The only term to consider then is $-\nabla\smalliiint mS\sigma\, \nabla u\, \ds$.
Putting as in \equationref{VI}{17} $m = m' + M$ we have
\[
-\nabla \smalliiint M  S\sigma\, \nabla u\, \ds
-\nabla \smalliiint m' S\sigma\, \nabla u\, \ds.
\]

%% -----File: 120.png---Folio 102-------

The first term of this can be neglected for the same reasons as
for neglecting that containing~$V\sigma\tau$. Applying \equationref{II}{9} \Secref{6} to
the last term and neglecting the surface integral as usual we get
\[
\EqnTag{VI}{18}
  \nabla \smalliiint u S \nabla(m' \sigma)\, \ds
= \nabla \smalliiint u \{m'(m' +M) + S\sigma\, \nabla m'\}\, \ds.
\]
The last term can probably be neglected though we cannot give
such good reasons as for the other terms we have neglected. At
any rate in places not near permanent bubbles $S\sigma\nabla m'$ is as likely
to be positive as negative and vice~vers�, so that such portions of
space will on the whole produce no effect on the permanent
bubbles. If $S\sigma\,\nabla m'$ contributes anything for parts of space in the
neighbourhood of permanent bubbles we must be content at
present with the assumption either that the contribution is in
general positive or that if it be negative it is not sufficient to
cancel the effect of the positive term~$m'M$. Remembering that
$m'$~is small compared with~$M$ we are left with the positive
term~$m' M$. This as can be easily seen from the form of \equationref{VI}{18}
leads to an apparent law of gravitation for our permanent
bubbles.

\Paragraph{95} The gravitational mass which we must on this supposition
assign to each permanent bubble varies as the average value
of~$m' M$ for that bubble\footnote{%
There is one important difference to be noticed between this and Hicks's
explanation of gravitation. His depends on the \emph{synchronous} pulsations of distant
vortices. I do not wish to imply that I do not believe in the existence of such
synchronous pulsations, but by the above we see that gravitation can probably
be explained independently of them. As a matter of fact such synchronous
pulsations probably actually occur on account of the variation of~$H$ with the time.}.
Now we saw in \Secref{87} that the probable
measure of mass of a permanent bubble was proportional to its
average size. Do these two results agree? I cannot say, but even
if they do not these considerations would still explain the motions
of the solar system, but if the sun and Jupiter (say) were to
collide their subsequent motion would not be that due to the
collision of two bodies the ratio of whose masses is that which is
accepted as the ratio of the sun's and Jupiter's. As a matter
of fact however I should imagine that the average value of~$m'M$
for a permanent bubble is proportional to its average volume and
this simply as a consequence of the reasoning in \Secref{87} above.

A conclusion at any rate to be drawn from the above is that
there is a \emph{presumption} in favour of Hicks's theory explaining
gravitation.

%% -----File: 121.png---Folio 103-------


\Section{96}{Consideration of the \texorpdfstring{term $-\nabla\centerdot(\sigma^2)/2$}{final term}}

\Paragraph{96} In considering this term we adopt the method of \Secref{92}
and consider an electric analogue. The analogue is an electro-magnetic
field for which in the notation of \Secref{46} to~\Secref{60} above at
every point,\Erratum{err103a} % [** PP: Changed period to comma]
\[
\EqnTag{VI}{20}
\mathbf{H} = \sigma.
\]
For this field we have at once
\[
\EqnTag{VI}{21}
4\pi \mathbf{C} = V\nabla \mathbf{H} = \tau.
\]
The distribution of the magnetism in the field is somewhat arbitrary,
but in the notation of equations \Eref{IV}{61} and~\Eref{V}{62} \Secref{82} it will be
found that everything is satisfied by putting
\[
\EqnTag{VI}{22}
4\pi \mathbf{I} = -\nabla S q.
\]
This gives as it should $S\nabla (\mathbf{H} + 4\pi \mathbf{I}) = 0$, which is in fact the only
equation it is necessary to satisfy. We have further
\begin{flalign*}
&&
\mathbf{B} = \mathbf{H} + 4\pi\mathbf{I}
&= \sigma - \nabla Sq = \nabla Vq,       && \\
%
\EqnTag{VI}{23}
&\text{so that}& \mathbf{A} &= Vq.       &&
\end{flalign*}

Thus all the important vectors in the analogue are determined.
It remains to compare the mechanical effects of the analogue with
the term $-\nabla\centerdot \sigma^2/2$.

{\stretchyspace
$\mathbf{I}$, it must be observed, is not confined to the bubbles, but is
distributed throughout space.}


\Paragraph{97} If we now assume that bubbles have not always existed
in the positions which we call permanent, there cannot at the
surface of the bubbles be any circulation round them. This makes
the velocity at the surface almost normal to it, so that the stress
given in \equationref{IV}{66} \Secref{60} reduces to a tension $-S\mathbf{H} (2\mathbf{B}-\mathbf{H})/8\pi$
over the surface, i.e.\ we have a pressure
\[
-\sigma^2/8\pi + S\sigma\,\nabla V q/4\pi.
\]
Now on account of the absence of circulation $\nabla V q$ is very small
and may therefore be neglected. Thus we get a \emph{pressure} $-\sigma^2/8\pi$,
and are thus led once more to a ``converse'' of the analogue.
This at once\footnote{%
[Note added, 1892. Because the density of attracting magnetic matter of the
analogue $= -S\nabla \mathbf{H}/4\pi = -m/4\pi$.]}
leads to another reason for the law of gravitation if
%% -----File: 122.png---Folio 104-------
the pulsations are synchronous. This we have already seen to be
probable.

The present consideration of the subject is merely to point to
a \emph{method} of investigating the theory of vortex-atoms. I therefore
leave the subject here, not attempting to force the phenomena we
have been considering to tally with the known phenomena of
electricity and magnetism. Nevertheless I may say that the
prospect of discussing these things by means of the present subject
can scarcely be considered as distant after what has gone before.

To sum up, this first application of the method leads to a presumption
in favour of Hicks's theory leading to an explanation of
both the important properties of matter---inertia and the law of
gravitation---and there is also reason from it to hope that the
phenomena of electro-magnetism are not unlikely to receive an
explanation. Thomson's theory on the other hand would seem to
fail in the first two at any rate of those endeavours.


\Paragraph{98} We close the essay with the fulfilment of the promise
made towards the end of \Secref{88}. In that section it will be remembered
we considered a hypothetical fluid for which $D = \tanh (p/c)$,
and made this do duty for a liquid containing bubbles. Strictly
speaking our liquid is \emph{bounded} at the bubbles and therefore as a
bounded liquid should it be treated. For such a liquid we require
an equation corresponding to \equationref{VI}{10} \Secref{90}, and if possible
\equationref{VI}{11} also. This last I have been unable to obtain, and I
am not sure that to solve the problem explicitly is possible.

Our problem only deals with an incompressible fluid, but as
the removal of this restriction does not greatly complicate the
work we will consider the general case of a bounded compressible
fluid. We have
\begin{align*}
4\pi\sigma
&=  \nabla^2\smalliiint u\sigma\, \ds
 = -\nabla  \smalliiint \nabla u\sigma\, \ds \\
&=  \nabla  \smalliiint u(\tau + m)\, \ds
  - \nabla  \smalliint  u\, d\Sigma\,\sigma,
\end{align*}
whence
\[
4\pi\partial\sigma/\partial t
= \nabla \smalliiint d \{u(\tau + m)\, \ds\}/dt
 -\nabla \smalliint  d (u\, d\Sigma\, \sigma)/dt.
\]
The justification of using $\partial/\partial t$ on the left and~$d/dt$ under the integral
sign will appear if the increment $(\partial\sigma/\partial t)\, dt$ in~$\sigma$ at a given
point in the time~$dt$ be considered. It will be observed that the
meaning here to be attached to~$\dot{u}$ will be $-S\sigma\, \nabla u$, as in the
differentiation~$\partial/\partial t$ with regard to the time the origin of~$u$ is
assumed to be fixed.

%% -----File: 123.png---Folio 105-------

I shall now merely indicate the method of procedure. By the
method exhibited in \Secref{89} we can prove that
\begin{align*}
d(u\tau\, \ds)/dt
&= -V\nabla u\,V\sigma\tau\, \ds - u\sigma S\tau\Delta\, \ds \\
\intertext{and that}
d(um\, \ds)/dt
&= -mS\sigma\, \nabla u\, \ds + u\,d(m\, \ds)/dt.
\end{align*}
From this we can deduce that
\begin{align*}
4\pi\,\partial\sigma/\partial t
&= \nabla V\, \nabla\smalliiint uV\sigma\tau\, \ds
 + \nabla\smalliiint \bigl\{-mS\sigma\, \nabla u\, \ds + u\,d(m\, \ds)/dt\bigr\} \\
&- \nabla\smalliint u\sigma S\tau\, d\Sigma
 - \nabla\smalliint u\, d(d\Sigma\,\sigma)/dt
 + \nabla\smalliint d\Sigma\, \sigma S\sigma\, \nabla u,
\end{align*}
and from this again we get
\[
\EqnTag{VI}{24}
\dot{\sigma} = \nabla w + \nabla w' + \nabla v,
\]
where $w$ and~$w'$ are scalars and~$v$ a vector given by
\[
\EqnTag{VI}{25}
4\pi w = -2\pi\sigma^2
    + \smalliiint\{\ds\,S\, \nabla u(V\sigma\tau - m\sigma)
                   + u\,d(m\, \ds)/dt\},
\]
so that $w$ is in fact the $H-v-P$ of \equationref{VI}{11} \Secref{90}
\begin{align*}
\EqnTag{VI}{26}
4\pi w'
&= \smalliint(S\, d\Sigma\, \sigma S\sigma\, \nabla u
   - u\, dS\, d\Sigma\, \sigma/dt),  \\
%
\EqnTag{VI}{27}
4\pi v
&= \smalliint(-u\sigma S\tau\, d\Sigma
              + V\,d\Sigma\,\sigma S\sigma\, \nabla u
              - u\, dV\, d\Sigma\, \sigma/dt).
\end{align*}
This last equation may be put in what for our purposes is the
more convenient form
\[
\EqnTag{VI}{28}
4\pi v = -\smalliint\{u\sigma S\tau\, d\Sigma + d(uV\, d\Sigma\, \sigma)/dt\}.
\]
Again it may be put in a form free from~$d/dt$; for $V\,d\Sigma\,\dot{\sigma} = 0$
because the surface is a free surface and $d(d\Sigma)/dt = \nabla_1 S\,d\Sigma\,\sigma_1$\footnote{%
[Note added, 1892. This should be $d(d\Sigma)/dt = -m\,d\Sigma + \nabla_1 S\,d\Sigma\,\sigma_1$, and therefore
\equationref{VI}{29} should be
\[
4\pi v = \smalliint\{V\, d\Sigma\, \sigma S\sigma\, \nabla u
                   - u(\sigma S\tau\, d\Sigma
                   - mV\, d\Sigma\, \sigma
                   + V\nabla_1\sigma S\, d\Sigma\sigma_1)\}.
\]
This does not affect our present problem because $m=0$ in our case.]}, as
can easily be proved by considerations similar to those in \Secref{83}.
Thus
\[
\EqnTag{VI}{29}
4\pi v = \smalliint(- u\sigma S\tau\, d\Sigma
                    + V\,d \Sigma\, \sigma S\sigma\, \nabla u
                    - uV\nabla_1\sigma S\, d\Sigma\, \sigma_1).
\]

In the problem we have to discuss $m = 0$, so that $w$ gives only
terms which we discussed in considering Sir Wm.~Thomson's
theory. Observing that if in~$w$ we change~$m\, \ds$ into $-S\,d\Sigma\,\sigma$ we
get for the part of~$w$ containing~$m$, $w'$; we see that $w'$ only gives
terms that we have virtually discussed under Hicks's theory. We
have however entirely neglected~$v$. Are we justified in this? In
the first place we have seen that if the bubbles have not always
been associated with those parts of the fluid with which they now
are there is round every bubble absolutely no circulation. This
%% -----File: 124.png---Folio 106-------
% [** PP: Keeping ``shows'']
shows that for any one bubble $\smalliint V\, d\Sigma\, \sigma = 0$\footnote{%
One way out of many of proving this is as follows. Divide the bubble up into
a number of infinitely near sections by planes perpendicular to the unit vector~$\alpha$.
For any one section $\int S\,d\rho\,\sigma=0$. Consider~$d\Sigma$ to be the element of the surface cut
off by the following four planes, (1)~the plane of section considered, (2)~the consecutive
plane of section, (3)~the two planes perpendicular to~$d\rho$, and through the
extremities of the element~$d\rho$. Thus if $x$~be the distance between the two sections,
$d\rho=x^{-1}V\alpha\, d\Sigma$, whence
\[
\smalliint S\alpha\, d\Sigma\,\sigma=0,
\]
for the surface of the bubble between the two sections. But adding, we may suppose
this integral to extend over the whole bubble. Thus $S\alpha\smalliint V\,d\Sigma\,\sigma=0$ for the
whole bubble; therefore $\alpha$ being a quite arbitrary unit vector we have for the whole
bubble $\smalliint V\, d\Sigma\, \sigma=0$.},
%*end of footnote text
and therefore we are
justified in neglecting the last term in \equationref{VI}{28}. We are
probably also justified in neglecting the first term, for probably $\tau$
is very nearly tangential to the surface and therefore $S\tau\, d\Sigma = 0$.

[Note added, 1892. The whole of this last section is in rather
a nebulous stage, and since writing it I have not had sufficient
leisure to return to the matter. I hesitated whether to include it
in the present issue. But since, notwithstanding the absence of
any reliable results, it serves very well to illustrate how investigations
are conducted by Quaternions, I have thought it worth publishing.

Should anybody feel inclined to attempt to apply the method
or an analogous one it is well to note that in the \textit{Phil.\ Mag.}\ June,
1892, p.~490, I have given the more general result sought in this
last section. As the result is not there proved I give one proof
which seems instructive. It exhibits the great variety of suitable
quaternion methods of dealing with physical questions. It furnishes
incidentally a fourth quaternion proof of the properties of
vortices. It also illustrates how special quaternion methods
developed for use in one branch of Physics at once prove themselves
useful in other branches.

Adopting the notation and terminology of \textit{Phil.\ Trans.}\ 1892,
p.~686, \S\S~5--7, let $\sigma$ and~$\tau$ be taken as an intensity and flux respectively,
$\tau$~still being ${} = V\, \nabla\sigma$ and therefore $\tau' = V\, \nabla'\sigma'$. Thus
$\sigma'$ and~$\tau'$ are the actual velocity and double spin respectively and
the equation of motion is
\[
\dot{\sigma}' = -\nabla'(v+P).
\]
Putting $\sigma' = \chi'^{-1}\sigma$ and operating on the equation by~$\chi'$
\[
\dot{\sigma} +\chi' d\left(\chi'^{-1}\right)/dt\centerdot \sigma
    = -\nabla(v+P).
\]
%% -----File: 125.png---Folio 107-------
Now
\begin{gather*}
\chi'd(\chi'^{-1})/dt\centerdot \sigma
= -\dot{\chi}'\chi'^{-1}\sigma
= -\dot{\chi}'\sigma'
= \nabla_1 S\sigma_1'\sigma'
= \nabla\centerdot \sigma'^2/2. \\
%
\EqnTag{VI}{$A$}
\therefore\qquad
\dot{\sigma} + \nabla\centerdot \sigma'^2/2
= -\nabla(v+P),
\end{gather*}
which can easily be deduced from or utilised to prove Lord
Kelvin's theorem concerning ``flow,'' \equationref{V}{23} \Secref{72} above.

As $d/dt$ is commutative with~$\nabla$, $\dot{\tau}=0$ or $\tau$ is an absolute constant
for each element of matter. This being interpreted at once
gives the well-known properties of vortices in their usual form.

If in the equation $4\pi(v+P)=S\centerdot \nabla^2\smalliiint u(v+P)\, \ds$ we carry
one~$\nabla$ across the integral sign, get rid of its differentiations which
affect~$u$ by \equationref{II}{9} \Secref{6} above, and then do the same with the
other~$\nabla$ we get
\[
4\pi(v+P)
  = \smalliint\{(v+P)S\, d\Sigma\, \nabla u - uS\,d\Sigma\,\nabla(v+P)\}
   +\smalliiint u\nabla^2(v+P)\, \ds.
\]
At surfaces of discontinuity in $\sigma$, $v$ and~$P$ will both be continuous,
so that instead of $\smalliint(v+P)S\, d\Sigma\,\nabla u$ we may write $\smalliint_b(v+P)S\,d\Sigma\,\nabla u$.
In the last equation substitute throughout for $\nabla(v + P)$ from
\equationref{VI}{$A$}. Thus
\Erratum{err107a}% [** PP: Added \centerdot below]
\begin{gather*}
\begin{aligned}
4\pi(v+P) = \smalliint_b (v+P)S\, d\Sigma\, \nabla u
&+ \smalliint  uS\, d\Sigma\, \dot{\sigma}
  -\smalliiint uS\, \nabla\dot{\sigma}\, \ds  \\
&+ \left(\smalliint  uS\, d\Sigma\, \nabla\centerdot \sigma'^2/2
        -\smalliiint u\nabla^2\sigma'^2\, \ds/2\right)
\end{aligned} \\
{} = \smalliint_b (v+P)S\, d\Sigma\, \nabla u
    +\smalliint   uS\, d\Sigma\, \dot{\sigma}
    -\smalliiint  uS\nabla\dot{\sigma}\, \ds
    +\smalliiint  S\, \nabla u\, \nabla\centerdot(\sigma'^2)\, \ds/2,% [** <- \centerdot]
\end{gather*}
which is equation~(36) of the \textit{Phil.\ Mag.}\ paper.

If the standard position and present position of matter coincide
it is quite easy to prove that
\begin{align*}
dS\, d\Sigma'\, \sigma'/dt
&= S\, d\Sigma(\dot{\sigma} + V\nabla_1\sigma\sigma_1) \\
d(S\nabla'\sigma'\, \ds')/dt
&= S\nabla(\dot{\sigma} + V\nabla_1\sigma\sigma_1)\, \ds.
\end{align*}
Substituting for $S\,d\Sigma\,\dot{\sigma}$, $S\nabla\dot{\sigma}$ from these in the last equation we
get equation~(32) of the \textit{Phil.\ Mag.}\ paper; but this equation can
also be proved directly. It should be noticed that equations
(34) and~(35) of that paper have been wrongly written down from
equation~(32). In each read ${} + \nabla\centerdot(\sigma^2/2)$ for ${} - \nabla\centerdot(\sigma^2/2)$.]\Erratum{err107b}% [** PP: Added two \centerdots]

\cleardoublepage

\enlargethispage{50pt}

\SectionTitle{Transcriber's Notes}

\pdfbookmark[0]{Transcriber's Notes}{Transcriber's Notes}

\fancyhead{}
\fancyhead[C]{{\scshape\small transcriber's notes.}}
\ifthenelse{\boolean{ForPrinting}}
 {\fancyhead[RO,LE]{\thepage}}
 {\fancyhead[R]{\thepage}}

Minor spelling inconsistencies (\textit{foot-note}/\textit{footnote},
\textit{neutralise}/\textit{neutralize}, and
\textit{shew}/\textit{show}) have been retained.

Changes listed below are marked with double asterisks in the \LaTeX\
file.
\smallskip

%% Changes of spelling:
%  caligraphy -> calligraphy (p. ix)
%  paralleled -> parallelled (p. vii)

The punctuation on displayed equations has been placed at the end of
the equation instead of after the tag at the right margin. Aside from
this, equation punctuation has been retained unless it conflicts
obtrusively with grammar.
\bigskip

Page numbers refer to the folio numbers in the scanned images.

\begin{itemize}

\item p.~7: \hyperref[err7a]{Set long verbal equation as a display.} \\
  (``$w\times\text{the element of volume}\ldots$'')

\item p.~33: \hyperref[err33a]{Just after equation~($15f$), add missing
word.} (``We saw in \emph{the} last section\dots'')

\item p.~49: \hyperref[err49a]{Replace opening equals sign by ``is
  equal to'' just above \S\,29.} (Two times.)

\item p.~62: \hyperref[err62a]{Replace ``$\mathrm{D}$'' by~``$\Dop$''
  in right-hand member of display following equation~(19).}

\item p.~101: \hyperref[err101a]{Add dot in title of \S\,94
  (``$-\nabla\centerdot(\sigma^2)/2$'').}

\item p.~107: Add missing dot after~$\nabla$ three times: in last
  member of \hyperref[err107a]{second-to-last display}, and twice in
  \hyperref[err107b]{last line.}

\end{itemize}

%% List of minor changes:
\iffalse
\item p.~6: \hyperref[err6a]{Re-break displayed equation at
  second equals sign.} \\
  (``$A=a(f+g)+bc/2=\cdots$'')

\item p.~16: \hyperref[err16a]{Add period to first line of
  equation~(5).}

\item p.~17: \hyperref[err6f]{Re-break equation~(6$f$).}

\item p.~18: \hyperref[err18a]{Add comma to equation just before
  equation~($6g$).}

\item p.~27: \hyperref[err27a]{Remove comma in middle of display
  in first paragraph of \S\,14.} \\
  $\bigl($``$(\ldots\text{on surface of portion considered}) - (\text{work done}\ldots)$''$\bigr)$

\item p.~39: \hyperref[err39a]{Add missing closing square bracket at
  end of footnote.}

\item p.~64: \hyperref[err64a]{Hyphenate ``electro-motive'' twice in
  the final paragraph of \S\,42.}

\item p.~67: \hyperref[err67a]{Reformat title of \S\,54 for
  consistency with \S\S\,34 and~46.}

\item p.~103: \hyperref[err103a]{Change period to comma just before
  equation~(20).}
\fi

\cleardoublepage

\phantomsection
\pdfbookmark[0]{Project Gutenberg License}{Project Gutenberg License}
\fancyhead[C]{{\scshape\small licensing.}}

\begin{PGtext}
End of the Project Gutenberg EBook of Utility of Quaternions in Physics, by 
Alexander McAulay

*** END OF THIS PROJECT GUTENBERG EBOOK UTILITY OF QUATERNIONS IN PHYSICS ***

***** This file should be named 26262-pdf.pdf or 26262-pdf.zip *****
This and all associated files of various formats will be found in:
        http://www.gutenberg.org/2/6/2/6/26262/

Produced by Joshua Hutchinson, Andrew D. Hwang, Carolyn
Bottomley and the Online Distributed Proofreading Team at
http://www.pgdp.net (This ebook was produced from images
from the Cornell University Library: Historical Mathematics
Monographs collection.)


Updated editions will replace the previous one--the old editions
will be renamed.

Creating the works from public domain print editions means that no
one owns a United States copyright in these works, so the Foundation
(and you!) can copy and distribute it in the United States without
permission and without paying copyright royalties.  Special rules,
set forth in the General Terms of Use part of this license, apply to
copying and distributing Project Gutenberg-tm electronic works to
protect the PROJECT GUTENBERG-tm concept and trademark.  Project
Gutenberg is a registered trademark, and may not be used if you
charge for the eBooks, unless you receive specific permission.  If you
do not charge anything for copies of this eBook, complying with the
rules is very easy.  You may use this eBook for nearly any purpose
such as creation of derivative works, reports, performances and
research.  They may be modified and printed and given away--you may do
practically ANYTHING with public domain eBooks.  Redistribution is
subject to the trademark license, especially commercial
redistribution.



*** START: FULL LICENSE ***

THE FULL PROJECT GUTENBERG LICENSE
PLEASE READ THIS BEFORE YOU DISTRIBUTE OR USE THIS WORK

To protect the Project Gutenberg-tm mission of promoting the free
distribution of electronic works, by using or distributing this work
(or any other work associated in any way with the phrase "Project
Gutenberg"), you agree to comply with all the terms of the Full Project
Gutenberg-tm License (available with this file or online at
http://gutenberg.org/license).


Section 1.  General Terms of Use and Redistributing Project Gutenberg-tm
electronic works

1.A.  By reading or using any part of this Project Gutenberg-tm
electronic work, you indicate that you have read, understand, agree to
and accept all the terms of this license and intellectual property
(trademark/copyright) agreement.  If you do not agree to abide by all
the terms of this agreement, you must cease using and return or destroy
all copies of Project Gutenberg-tm electronic works in your possession.
If you paid a fee for obtaining a copy of or access to a Project
Gutenberg-tm electronic work and you do not agree to be bound by the
terms of this agreement, you may obtain a refund from the person or
entity to whom you paid the fee as set forth in paragraph 1.E.8.

1.B.  "Project Gutenberg" is a registered trademark.  It may only be
used on or associated in any way with an electronic work by people who
agree to be bound by the terms of this agreement.  There are a few
things that you can do with most Project Gutenberg-tm electronic works
even without complying with the full terms of this agreement.  See
paragraph 1.C below.  There are a lot of things you can do with Project
Gutenberg-tm electronic works if you follow the terms of this agreement
and help preserve free future access to Project Gutenberg-tm electronic
works.  See paragraph 1.E below.

1.C.  The Project Gutenberg Literary Archive Foundation ("the Foundation"
or PGLAF), owns a compilation copyright in the collection of Project
Gutenberg-tm electronic works.  Nearly all the individual works in the
collection are in the public domain in the United States.  If an
individual work is in the public domain in the United States and you are
located in the United States, we do not claim a right to prevent you from
copying, distributing, performing, displaying or creating derivative
works based on the work as long as all references to Project Gutenberg
are removed.  Of course, we hope that you will support the Project
Gutenberg-tm mission of promoting free access to electronic works by
freely sharing Project Gutenberg-tm works in compliance with the terms of
this agreement for keeping the Project Gutenberg-tm name associated with
the work.  You can easily comply with the terms of this agreement by
keeping this work in the same format with its attached full Project
Gutenberg-tm License when you share it without charge with others.

1.D.  The copyright laws of the place where you are located also govern
what you can do with this work.  Copyright laws in most countries are in
a constant state of change.  If you are outside the United States, check
the laws of your country in addition to the terms of this agreement
before downloading, copying, displaying, performing, distributing or
creating derivative works based on this work or any other Project
Gutenberg-tm work.  The Foundation makes no representations concerning
the copyright status of any work in any country outside the United
States.

1.E.  Unless you have removed all references to Project Gutenberg:

1.E.1.  The following sentence, with active links to, or other immediate
access to, the full Project Gutenberg-tm License must appear prominently
whenever any copy of a Project Gutenberg-tm work (any work on which the
phrase "Project Gutenberg" appears, or with which the phrase "Project
Gutenberg" is associated) is accessed, displayed, performed, viewed,
copied or distributed:

This eBook is for the use of anyone anywhere at no cost and with
almost no restrictions whatsoever.  You may copy it, give it away or
re-use it under the terms of the Project Gutenberg License included
with this eBook or online at www.gutenberg.org

1.E.2.  If an individual Project Gutenberg-tm electronic work is derived
from the public domain (does not contain a notice indicating that it is
posted with permission of the copyright holder), the work can be copied
and distributed to anyone in the United States without paying any fees
or charges.  If you are redistributing or providing access to a work
with the phrase "Project Gutenberg" associated with or appearing on the
work, you must comply either with the requirements of paragraphs 1.E.1
through 1.E.7 or obtain permission for the use of the work and the
Project Gutenberg-tm trademark as set forth in paragraphs 1.E.8 or
1.E.9.

1.E.3.  If an individual Project Gutenberg-tm electronic work is posted
with the permission of the copyright holder, your use and distribution
must comply with both paragraphs 1.E.1 through 1.E.7 and any additional
terms imposed by the copyright holder.  Additional terms will be linked
to the Project Gutenberg-tm License for all works posted with the
permission of the copyright holder found at the beginning of this work.

1.E.4.  Do not unlink or detach or remove the full Project Gutenberg-tm
License terms from this work, or any files containing a part of this
work or any other work associated with Project Gutenberg-tm.

1.E.5.  Do not copy, display, perform, distribute or redistribute this
electronic work, or any part of this electronic work, without
prominently displaying the sentence set forth in paragraph 1.E.1 with
active links or immediate access to the full terms of the Project
Gutenberg-tm License.

1.E.6.  You may convert to and distribute this work in any binary,
compressed, marked up, nonproprietary or proprietary form, including any
word processing or hypertext form.  However, if you provide access to or
distribute copies of a Project Gutenberg-tm work in a format other than
"Plain Vanilla ASCII" or other format used in the official version
posted on the official Project Gutenberg-tm web site (www.gutenberg.org),
you must, at no additional cost, fee or expense to the user, provide a
copy, a means of exporting a copy, or a means of obtaining a copy upon
request, of the work in its original "Plain Vanilla ASCII" or other
form.  Any alternate format must include the full Project Gutenberg-tm
License as specified in paragraph 1.E.1.

1.E.7.  Do not charge a fee for access to, viewing, displaying,
performing, copying or distributing any Project Gutenberg-tm works
unless you comply with paragraph 1.E.8 or 1.E.9.

1.E.8.  You may charge a reasonable fee for copies of or providing
access to or distributing Project Gutenberg-tm electronic works provided
that

- You pay a royalty fee of 20% of the gross profits you derive from
     the use of Project Gutenberg-tm works calculated using the method
     you already use to calculate your applicable taxes.  The fee is
     owed to the owner of the Project Gutenberg-tm trademark, but he
     has agreed to donate royalties under this paragraph to the
     Project Gutenberg Literary Archive Foundation.  Royalty payments
     must be paid within 60 days following each date on which you
     prepare (or are legally required to prepare) your periodic tax
     returns.  Royalty payments should be clearly marked as such and
     sent to the Project Gutenberg Literary Archive Foundation at the
     address specified in Section 4, "Information about donations to
     the Project Gutenberg Literary Archive Foundation."

- You provide a full refund of any money paid by a user who notifies
     you in writing (or by e-mail) within 30 days of receipt that s/he
     does not agree to the terms of the full Project Gutenberg-tm
     License.  You must require such a user to return or
     destroy all copies of the works possessed in a physical medium
     and discontinue all use of and all access to other copies of
     Project Gutenberg-tm works.

- You provide, in accordance with paragraph 1.F.3, a full refund of any
     money paid for a work or a replacement copy, if a defect in the
     electronic work is discovered and reported to you within 90 days
     of receipt of the work.

- You comply with all other terms of this agreement for free
     distribution of Project Gutenberg-tm works.

1.E.9.  If you wish to charge a fee or distribute a Project Gutenberg-tm
electronic work or group of works on different terms than are set
forth in this agreement, you must obtain permission in writing from
both the Project Gutenberg Literary Archive Foundation and Michael
Hart, the owner of the Project Gutenberg-tm trademark.  Contact the
Foundation as set forth in Section 3 below.

1.F.

1.F.1.  Project Gutenberg volunteers and employees expend considerable
effort to identify, do copyright research on, transcribe and proofread
public domain works in creating the Project Gutenberg-tm
collection.  Despite these efforts, Project Gutenberg-tm electronic
works, and the medium on which they may be stored, may contain
"Defects," such as, but not limited to, incomplete, inaccurate or
corrupt data, transcription errors, a copyright or other intellectual
property infringement, a defective or damaged disk or other medium, a
computer virus, or computer codes that damage or cannot be read by
your equipment.

1.F.2.  LIMITED WARRANTY, DISCLAIMER OF DAMAGES - Except for the "Right
of Replacement or Refund" described in paragraph 1.F.3, the Project
Gutenberg Literary Archive Foundation, the owner of the Project
Gutenberg-tm trademark, and any other party distributing a Project
Gutenberg-tm electronic work under this agreement, disclaim all
liability to you for damages, costs and expenses, including legal
fees.  YOU AGREE THAT YOU HAVE NO REMEDIES FOR NEGLIGENCE, STRICT
LIABILITY, BREACH OF WARRANTY OR BREACH OF CONTRACT EXCEPT THOSE
PROVIDED IN PARAGRAPH F3.  YOU AGREE THAT THE FOUNDATION, THE
TRADEMARK OWNER, AND ANY DISTRIBUTOR UNDER THIS AGREEMENT WILL NOT BE
LIABLE TO YOU FOR ACTUAL, DIRECT, INDIRECT, CONSEQUENTIAL, PUNITIVE OR
INCIDENTAL DAMAGES EVEN IF YOU GIVE NOTICE OF THE POSSIBILITY OF SUCH
DAMAGE.

1.F.3.  LIMITED RIGHT OF REPLACEMENT OR REFUND - If you discover a
defect in this electronic work within 90 days of receiving it, you can
receive a refund of the money (if any) you paid for it by sending a
written explanation to the person you received the work from.  If you
received the work on a physical medium, you must return the medium with
your written explanation.  The person or entity that provided you with
the defective work may elect to provide a replacement copy in lieu of a
refund.  If you received the work electronically, the person or entity
providing it to you may choose to give you a second opportunity to
receive the work electronically in lieu of a refund.  If the second copy
is also defective, you may demand a refund in writing without further
opportunities to fix the problem.

1.F.4.  Except for the limited right of replacement or refund set forth
in paragraph 1.F.3, this work is provided to you 'AS-IS' WITH NO OTHER
WARRANTIES OF ANY KIND, EXPRESS OR IMPLIED, INCLUDING BUT NOT LIMITED TO
WARRANTIES OF MERCHANTIBILITY OR FITNESS FOR ANY PURPOSE.

1.F.5.  Some states do not allow disclaimers of certain implied
warranties or the exclusion or limitation of certain types of damages.
If any disclaimer or limitation set forth in this agreement violates the
law of the state applicable to this agreement, the agreement shall be
interpreted to make the maximum disclaimer or limitation permitted by
the applicable state law.  The invalidity or unenforceability of any
provision of this agreement shall not void the remaining provisions.

1.F.6.  INDEMNITY - You agree to indemnify and hold the Foundation, the
trademark owner, any agent or employee of the Foundation, anyone
providing copies of Project Gutenberg-tm electronic works in accordance
with this agreement, and any volunteers associated with the production,
promotion and distribution of Project Gutenberg-tm electronic works,
harmless from all liability, costs and expenses, including legal fees,
that arise directly or indirectly from any of the following which you do
or cause to occur: (a) distribution of this or any Project Gutenberg-tm
work, (b) alteration, modification, or additions or deletions to any
Project Gutenberg-tm work, and (c) any Defect you cause.


Section  2.  Information about the Mission of Project Gutenberg-tm

Project Gutenberg-tm is synonymous with the free distribution of
electronic works in formats readable by the widest variety of computers
including obsolete, old, middle-aged and new computers.  It exists
because of the efforts of hundreds of volunteers and donations from
people in all walks of life.

Volunteers and financial support to provide volunteers with the
assistance they need, is critical to reaching Project Gutenberg-tm's
goals and ensuring that the Project Gutenberg-tm collection will
remain freely available for generations to come.  In 2001, the Project
Gutenberg Literary Archive Foundation was created to provide a secure
and permanent future for Project Gutenberg-tm and future generations.
To learn more about the Project Gutenberg Literary Archive Foundation
and how your efforts and donations can help, see Sections 3 and 4
and the Foundation web page at http://www.pglaf.org.


Section 3.  Information about the Project Gutenberg Literary Archive
Foundation

The Project Gutenberg Literary Archive Foundation is a non profit
501(c)(3) educational corporation organized under the laws of the
state of Mississippi and granted tax exempt status by the Internal
Revenue Service.  The Foundation's EIN or federal tax identification
number is 64-6221541.  Its 501(c)(3) letter is posted at
http://pglaf.org/fundraising.  Contributions to the Project Gutenberg
Literary Archive Foundation are tax deductible to the full extent
permitted by U.S. federal laws and your state's laws.

The Foundation's principal office is located at 4557 Melan Dr. S.
Fairbanks, AK, 99712., but its volunteers and employees are scattered
throughout numerous locations.  Its business office is located at
809 North 1500 West, Salt Lake City, UT 84116, (801) 596-1887, email
business@pglaf.org.  Email contact links and up to date contact
information can be found at the Foundation's web site and official
page at http://pglaf.org

For additional contact information:
     Dr. Gregory B. Newby
     Chief Executive and Director
     gbnewby@pglaf.org


Section 4.  Information about Donations to the Project Gutenberg
Literary Archive Foundation

Project Gutenberg-tm depends upon and cannot survive without wide
spread public support and donations to carry out its mission of
increasing the number of public domain and licensed works that can be
freely distributed in machine readable form accessible by the widest
array of equipment including outdated equipment.  Many small donations
($1 to $5,000) are particularly important to maintaining tax exempt
status with the IRS.

The Foundation is committed to complying with the laws regulating
charities and charitable donations in all 50 states of the United
States.  Compliance requirements are not uniform and it takes a
considerable effort, much paperwork and many fees to meet and keep up
with these requirements.  We do not solicit donations in locations
where we have not received written confirmation of compliance.  To
SEND DONATIONS or determine the status of compliance for any
particular state visit http://pglaf.org

While we cannot and do not solicit contributions from states where we
have not met the solicitation requirements, we know of no prohibition
against accepting unsolicited donations from donors in such states who
approach us with offers to donate.

International donations are gratefully accepted, but we cannot make
any statements concerning tax treatment of donations received from
outside the United States.  U.S. laws alone swamp our small staff.

Please check the Project Gutenberg Web pages for current donation
methods and addresses.  Donations are accepted in a number of other
ways including checks, online payments and credit card donations.
To donate, please visit: http://pglaf.org/donate


Section 5.  General Information About Project Gutenberg-tm electronic
works.

Professor Michael S. Hart is the originator of the Project Gutenberg-tm
concept of a library of electronic works that could be freely shared
with anyone.  For thirty years, he produced and distributed Project
Gutenberg-tm eBooks with only a loose network of volunteer support.


Project Gutenberg-tm eBooks are often created from several printed
editions, all of which are confirmed as Public Domain in the U.S.
unless a copyright notice is included.  Thus, we do not necessarily
keep eBooks in compliance with any particular paper edition.


Most people start at our Web site which has the main PG search facility:

     http://www.gutenberg.org

This Web site includes information about Project Gutenberg-tm,
including how to make donations to the Project Gutenberg Literary
Archive Foundation, how to help produce our new eBooks, and how to
subscribe to our email newsletter to hear about new eBooks.
\end{PGtext}

% %%%%%%%%%%%%%%%%%%%%%%%%%%%%%%%%%%%%%%%%%%%%%%%%%%%%%%%%%%%%%%%%%%%%%%% %
%                                                                         %
% End of the Project Gutenberg EBook of Utility of Quaternions in Physics, by 
% Alexander McAulay                                                       %
%                                                                         %
% *** END OF THIS PROJECT GUTENBERG EBOOK UTILITY OF QUATERNIONS IN PHYSICS ***
%                                                                         %
% ***** This file should be named 26262-t.tex or 26262-t.zip *****        %
% This and all associated files of various formats will be found in:      %
%         http://www.gutenberg.org/2/6/2/6/26262/                         %
%                                                                         %
% %%%%%%%%%%%%%%%%%%%%%%%%%%%%%%%%%%%%%%%%%%%%%%%%%%%%%%%%%%%%%%%%%%%%%%% %

\end{document}

### lprep configuration
@ControlwordReplace = (
  ['\\Preface',"\n\nPreface.\n"],
  ['\\McAulay','McAULAY'],
  ['\\TransNote',"Transcriber's Note"],
  ['\\TransNoteText',"\nThis book may be easily recompiled for printing. Please see the\nsource file preamble for instructions.\n"]
  );
@ControlwordArguments = (
  ['\\HeadChapter',1, 1, "\n\n","\n",1,1,'',"\n\n",1,0,'',''],
  ['\\FNChapter',1, 1, "\n\n","\n",1,0,'','',1,1,'',"\n\n"],
  ['\\Chapter',1, 1, "\n\n","\n",1,1,'',"\n\n"],
  ['\\Tocline',1,0,'','',1,0,'',''],
  ['\\TCSection',1,0,'','',1,0,'',"\n"],
  ['\\Section',1,0,'',"\n"],
  ['\\Paragraph',1,1,'','.'],
  ['\\FNParagraph',1,0,'','',1,1,'','.'],
  ['\\Eref',1,0,'','',1,1,'(',')'],
  ['\\equationref',1,0,'','',1,1,'equation~(',')'],
  ['\\Equationref',1,0,'','',1,1,'Equation~(',')'],
  ['\\Secref',1,1,'\\S~',''],
  ['\\Erratum',1,0,'',''],
  ['\\hyperref',0,0,'','']
  );

###
This is pdfeTeX, Version 3.141592-1.30.5-2.2 (Web2C 7.5.5) (format=pdflatex 2008.5.6)  11 AUG 2008 02:47
entering extended mode
**26262-t.tex
(./26262-t.tex
LaTeX2e <2003/12/01>
Babel <v3.8g> and hyphenation patterns for english, usenglishmax, dumylang, noh
yphenation, greek, monogreek, ancientgreek, ibycus, loaded.
(/usr/share/texmf-texlive/tex/latex/base/book.cls
Document Class: book 2004/02/16 v1.4f Standard LaTeX document class
(/usr/share/texmf-texlive/tex/latex/base/bk12.clo
File: bk12.clo 2004/02/16 v1.4f Standard LaTeX file (size option)
)
\c@part=\count79
\c@chapter=\count80
\c@section=\count81
\c@subsection=\count82
\c@subsubsection=\count83
\c@paragraph=\count84
\c@subparagraph=\count85
\c@figure=\count86
\c@table=\count87
\abovecaptionskip=\skip41
\belowcaptionskip=\skip42
\bibindent=\dimen102
)

LaTeX Warning: You have requested, on input line 149, version
               `2005/09/16' of document class book,
               but only version
               `2004/02/16 v1.4f Standard LaTeX document class'
               is available.

(/usr/share/texmf-texlive/tex/latex/amslatex/amsmath.sty
Package: amsmath 2000/07/18 v2.13 AMS math features
\@mathmargin=\skip43
For additional information on amsmath, use the `?' option.
(/usr/share/texmf-texlive/tex/latex/amslatex/amstext.sty
Package: amstext 2000/06/29 v2.01
(/usr/share/texmf-texlive/tex/latex/amslatex/amsgen.sty
File: amsgen.sty 1999/11/30 v2.0
\@emptytoks=\toks14
\ex@=\dimen103
)) (/usr/share/texmf-texlive/tex/latex/amslatex/amsbsy.sty
Package: amsbsy 1999/11/29 v1.2d
\pmbraise@=\dimen104
) (/usr/share/texmf-texlive/tex/latex/amslatex/amsopn.sty
Package: amsopn 1999/12/14 v2.01 operator names
)
\inf@bad=\count88
LaTeX Info: Redefining \frac on input line 211.
\uproot@=\count89
\leftroot@=\count90
LaTeX Info: Redefining \overline on input line 307.
\classnum@=\count91
\DOTSCASE@=\count92
LaTeX Info: Redefining \ldots on input line 379.
LaTeX Info: Redefining \dots on input line 382.
LaTeX Info: Redefining \cdots on input line 467.
\Mathstrutbox@=\box26
\strutbox@=\box27
\big@size=\dimen105
LaTeX Font Info:    Redeclaring font encoding OML on input line 567.
LaTeX Font Info:    Redeclaring font encoding OMS on input line 568.
\macc@depth=\count93
\c@MaxMatrixCols=\count94
\dotsspace@=\muskip10
\c@parentequation=\count95
\dspbrk@lvl=\count96
\tag@help=\toks15
\row@=\count97
\column@=\count98
\maxfields@=\count99
\andhelp@=\toks16
\eqnshift@=\dimen106
\alignsep@=\dimen107
\tagshift@=\dimen108
\tagwidth@=\dimen109
\totwidth@=\dimen110
\lineht@=\dimen111
\@envbody=\toks17
\multlinegap=\skip44
\multlinetaggap=\skip45
\mathdisplay@stack=\toks18
LaTeX Info: Redefining \[ on input line 2666.
LaTeX Info: Redefining \] on input line 2667.
) (/usr/share/texmf-texlive/tex/latex/amsfonts/amssymb.sty
Package: amssymb 2002/01/22 v2.2d
(/usr/share/texmf-texlive/tex/latex/amsfonts/amsfonts.sty
Package: amsfonts 2001/10/25 v2.2f
\symAMSa=\mathgroup4
\symAMSb=\mathgroup5
LaTeX Font Info:    Overwriting math alphabet `\mathfrak' in version `bold'
(Font)                  U/euf/m/n --> U/euf/b/n on input line 132.
)) (/usr/share/texmf-texlive/tex/latex/base/ifthen.sty
Package: ifthen 2001/05/26 v1.1c Standard LaTeX ifthen package (DPC)
) (/usr/share/texmf-texlive/tex/latex/base/inputenc.sty
Package: inputenc 2004/02/05 v1.0d Input encoding file
(/usr/share/texmf-texlive/tex/latex/base/latin1.def
File: latin1.def 2004/02/05 v1.0d Input encoding file
))

LaTeX Warning: You have requested, on input line 155, version
               `2006/05/05' of package inputenc,
               but only version
               `2004/02/05 v1.0d Input encoding file'
               is available.

(/usr/share/texmf-texlive/tex/latex/graphics/graphicx.sty
Package: graphicx 1999/02/16 v1.0f Enhanced LaTeX Graphics (DPC,SPQR)
(/usr/share/texmf-texlive/tex/latex/graphics/keyval.sty
Package: keyval 1999/03/16 v1.13 key=value parser (DPC)
\KV@toks@=\toks19
) (/usr/share/texmf-texlive/tex/latex/graphics/graphics.sty
Package: graphics 2001/07/07 v1.0n Standard LaTeX Graphics (DPC,SPQR)
(/usr/share/texmf-texlive/tex/latex/graphics/trig.sty
Package: trig 1999/03/16 v1.09 sin cos tan (DPC)
) (/usr/share/texmf-texlive/tex/latex/config/graphics.cfg
File: graphics.cfg 2001/08/31 v1.1 graphics configuration of teTeX/TeXLive
)
Package graphics Info: Driver file: pdftex.def on input line 80.
(/usr/share/texmf-texlive/tex/latex/graphics/pdftex.def
File: pdftex.def 2005/06/20 v0.03m graphics/color for pdftex
\Gread@gobject=\count100
))
\Gin@req@height=\dimen112
\Gin@req@width=\dimen113
) (/usr/share/texmf-texlive/tex/latex/base/alltt.sty
Package: alltt 1997/06/16 v2.0g defines alltt environment
) (/usr/share/texmf-texlive/tex/latex/tools/indentfirst.sty
Package: indentfirst 1995/11/23 v1.03 Indent first paragraph (DPC)
) (/usr/share/texmf-texlive/tex/latex/footmisc/footmisc.sty
Package: footmisc 2005/03/17 v5.3d a miscellany of footnote facilities
\FN@temptoken=\toks20
\footnotemargin=\dimen114
\c@pp@next@reset=\count101
Package footmisc Info: Declaring symbol style bringhurst on input line 817.
Package footmisc Info: Declaring symbol style chicago on input line 818.
Package footmisc Info: Declaring symbol style wiley on input line 819.
Package footmisc Info: Declaring symbol style lamport-robust on input line 823.

Package footmisc Info: Declaring symbol style lamport* on input line 831.
Package footmisc Info: Declaring symbol style lamport*-robust on input line 840
.
) (/usr/share/texmf-texlive/tex/latex/ltxmisc/perpage.sty
Package: perpage 2002/12/20 v1.0 Reset counters per page
)

LaTeX Warning: You have requested, on input line 169, version
               `2006/07/15' of package perpage,
               but only version
               `2002/12/20 v1.0 Reset counters per page'
               is available.

(/usr/share/texmf-texlive/tex/latex/txfonts/txfonts.sty
Package: txfonts 2005/01/03 v3.2
LaTeX Font Info:    Redeclaring symbol font `operators' on input line 20.
LaTeX Font Info:    Overwriting symbol font `operators' in version `normal'
(Font)                  OT1/cmr/m/n --> OT1/txr/m/n on input line 20.
LaTeX Font Info:    Overwriting symbol font `operators' in version `bold'
(Font)                  OT1/cmr/bx/n --> OT1/txr/m/n on input line 20.
LaTeX Font Info:    Overwriting symbol font `operators' in version `bold'
(Font)                  OT1/txr/m/n --> OT1/txr/bx/n on input line 21.
\symitalic=\mathgroup6
LaTeX Font Info:    Overwriting symbol font `italic' in version `bold'
(Font)                  OT1/txr/m/it --> OT1/txr/bx/it on input line 25.
LaTeX Font Info:    Redeclaring math alphabet \mathrm on input line 27.
LaTeX Font Info:    Redeclaring math alphabet \mathbf on input line 28.
LaTeX Font Info:    Overwriting math alphabet `\mathbf' in version `normal'
(Font)                  OT1/cmr/bx/n --> OT1/txr/bx/n on input line 28.
LaTeX Font Info:    Overwriting math alphabet `\mathbf' in version `bold'
(Font)                  OT1/cmr/bx/n --> OT1/txr/bx/n on input line 28.
LaTeX Font Info:    Redeclaring math alphabet \mathit on input line 29.
LaTeX Font Info:    Overwriting math alphabet `\mathit' in version `normal'
(Font)                  OT1/cmr/m/it --> OT1/txr/m/it on input line 29.
LaTeX Font Info:    Overwriting math alphabet `\mathit' in version `bold'
(Font)                  OT1/cmr/bx/it --> OT1/txr/m/it on input line 29.
LaTeX Font Info:    Overwriting math alphabet `\mathit' in version `bold'
(Font)                  OT1/txr/m/it --> OT1/txr/bx/it on input line 30.
LaTeX Font Info:    Redeclaring math alphabet \mathsf on input line 39.
LaTeX Font Info:    Overwriting math alphabet `\mathsf' in version `normal'
(Font)                  OT1/cmss/m/n --> OT1/txss/m/n on input line 39.
LaTeX Font Info:    Overwriting math alphabet `\mathsf' in version `bold'
(Font)                  OT1/cmss/bx/n --> OT1/txss/m/n on input line 39.
LaTeX Font Info:    Overwriting math alphabet `\mathsf' in version `bold'
(Font)                  OT1/txss/m/n --> OT1/txss/b/n on input line 40.
LaTeX Font Info:    Redeclaring math alphabet \mathtt on input line 49.
LaTeX Font Info:    Overwriting math alphabet `\mathtt' in version `normal'
(Font)                  OT1/cmtt/m/n --> OT1/txtt/m/n on input line 49.
LaTeX Font Info:    Overwriting math alphabet `\mathtt' in version `bold'
(Font)                  OT1/cmtt/m/n --> OT1/txtt/m/n on input line 49.
LaTeX Font Info:    Overwriting math alphabet `\mathtt' in version `bold'
(Font)                  OT1/txtt/m/n --> OT1/txtt/b/n on input line 50.
LaTeX Font Info:    Redeclaring symbol font `letters' on input line 57.
LaTeX Font Info:    Overwriting symbol font `letters' in version `normal'
(Font)                  OML/cmm/m/it --> OML/txmi/m/it on input line 57.
LaTeX Font Info:    Overwriting symbol font `letters' in version `bold'
(Font)                  OML/cmm/b/it --> OML/txmi/m/it on input line 57.
LaTeX Font Info:    Overwriting symbol font `letters' in version `bold'
(Font)                  OML/txmi/m/it --> OML/txmi/bx/it on input line 58.
\symlettersA=\mathgroup7
LaTeX Font Info:    Overwriting symbol font `lettersA' in version `bold'
(Font)                  U/txmia/m/it --> U/txmia/bx/it on input line 66.
LaTeX Font Info:    Redeclaring math alphabet \mathfrak on input line 69.
LaTeX Font Info:    Redeclaring symbol font `symbols' on input line 76.
LaTeX Font Info:    Overwriting symbol font `symbols' in version `normal'
(Font)                  OMS/cmsy/m/n --> OMS/txsy/m/n on input line 76.
LaTeX Font Info:    Overwriting symbol font `symbols' in version `bold'
(Font)                  OMS/cmsy/b/n --> OMS/txsy/m/n on input line 76.
LaTeX Font Info:    Overwriting symbol font `symbols' in version `bold'
(Font)                  OMS/txsy/m/n --> OMS/txsy/bx/n on input line 77.
LaTeX Font Info:    Redeclaring symbol font `AMSa' on input line 92.
LaTeX Font Info:    Overwriting symbol font `AMSa' in version `normal'
(Font)                  U/msa/m/n --> U/txsya/m/n on input line 92.
LaTeX Font Info:    Overwriting symbol font `AMSa' in version `bold'
(Font)                  U/msa/m/n --> U/txsya/m/n on input line 92.
LaTeX Font Info:    Overwriting symbol font `AMSa' in version `bold'
(Font)                  U/txsya/m/n --> U/txsya/bx/n on input line 93.
LaTeX Font Info:    Redeclaring symbol font `AMSb' on input line 101.
LaTeX Font Info:    Overwriting symbol font `AMSb' in version `normal'
(Font)                  U/msb/m/n --> U/txsyb/m/n on input line 101.
LaTeX Font Info:    Overwriting symbol font `AMSb' in version `bold'
(Font)                  U/msb/m/n --> U/txsyb/m/n on input line 101.
LaTeX Font Info:    Overwriting symbol font `AMSb' in version `bold'
(Font)                  U/txsyb/m/n --> U/txsyb/bx/n on input line 102.
LaTeX Font Info:    Redeclaring math alphabet \mathbb on input line 105.
\symsymbolsC=\mathgroup8
LaTeX Font Info:    Overwriting symbol font `symbolsC' in version `bold'
(Font)                  U/txsyc/m/n --> U/txsyc/bx/n on input line 112.
LaTeX Font Info:    Redeclaring symbol font `largesymbols' on input line 119.
LaTeX Font Info:    Overwriting symbol font `largesymbols' in version `normal'
(Font)                  OMX/cmex/m/n --> OMX/txex/m/n on input line 119.
LaTeX Font Info:    Overwriting symbol font `largesymbols' in version `bold'
(Font)                  OMX/cmex/m/n --> OMX/txex/m/n on input line 119.
LaTeX Font Info:    Overwriting symbol font `largesymbols' in version `bold'
(Font)                  OMX/txex/m/n --> OMX/txex/bx/n on input line 120.
\symlargesymbolsA=\mathgroup9
LaTeX Font Info:    Overwriting symbol font `largesymbolsA' in version `bold'
(Font)                  U/txexa/m/n --> U/txexa/bx/n on input line 128.
LaTeX Info: Redefining \not on input line 1042.
) (/usr/share/texmf-texlive/tex/latex/fancyhdr/fancyhdr.sty
\fancy@headwidth=\skip46
\f@ncyO@elh=\skip47
\f@ncyO@erh=\skip48
\f@ncyO@olh=\skip49
\f@ncyO@orh=\skip50
\f@ncyO@elf=\skip51
\f@ncyO@erf=\skip52
\f@ncyO@olf=\skip53
\f@ncyO@orf=\skip54
) (/usr/share/texmf-texlive/tex/latex/fancyhdr/extramarks.sty
\@temptokenb=\toks21
) (/usr/share/texmf-texlive/tex/latex/geometry/geometry.sty
Package: geometry 2002/07/08 v3.2 Page Geometry
\Gm@cnth=\count102
\Gm@cntv=\count103
\c@Gm@tempcnt=\count104
\Gm@bindingoffset=\dimen115
\Gm@wd@mp=\dimen116
\Gm@odd@mp=\dimen117
\Gm@even@mp=\dimen118
\Gm@dimlist=\toks22
(/usr/share/texmf-texlive/tex/latex/geometry/geometry.cfg)) (/usr/share/texmf-t
exlive/tex/latex/hyperref/hyperref.sty
Package: hyperref 2003/11/30 v6.74m Hypertext links for LaTeX
\@linkdim=\dimen119
\Hy@linkcounter=\count105
\Hy@pagecounter=\count106
(/usr/share/texmf-texlive/tex/latex/hyperref/pd1enc.def
File: pd1enc.def 2003/11/30 v6.74m Hyperref: PDFDocEncoding definition (HO)
) (/usr/share/texmf-texlive/tex/latex/config/hyperref.cfg
File: hyperref.cfg 2002/06/06 v1.2 hyperref configuration of TeXLive
)
Package hyperref Info: Option `hyperfootnotes' set `false' on input line 1830.
Package hyperref Info: Option `bookmarks' set `true' on input line 1830.
Package hyperref Info: Option `linktocpage' set `true' on input line 1830.
Package hyperref Info: Option `pdfdisplaydoctitle' set `true' on input line 183
0.
Package hyperref Info: Option `pdfpagelabels' set `true' on input line 1830.
Package hyperref Info: Option `bookmarksopen' set `true' on input line 1830.
Package hyperref Info: Option `colorlinks' set `true' on input line 1830.
Package hyperref Info: Hyper figures OFF on input line 1880.
Package hyperref Info: Link nesting OFF on input line 1885.
Package hyperref Info: Hyper index ON on input line 1888.
Package hyperref Info: Plain pages ON on input line 1893.
Package hyperref Info: Backreferencing OFF on input line 1900.
Implicit mode ON; LaTeX internals redefined
Package hyperref Info: Bookmarks ON on input line 2004.
(/usr/share/texmf-texlive/tex/latex/ltxmisc/url.sty
\Urlmuskip=\muskip11
Package: url 2005/06/27  ver 3.2  Verb mode for urls, etc.
)
LaTeX Info: Redefining \url on input line 2143.
\Fld@menulength=\count107
\Field@Width=\dimen120
\Fld@charsize=\dimen121
\Choice@toks=\toks23
\Field@toks=\toks24
Package hyperref Info: Hyper figures OFF on input line 2618.
Package hyperref Info: Link nesting OFF on input line 2623.
Package hyperref Info: Hyper index ON on input line 2626.
Package hyperref Info: backreferencing OFF on input line 2633.
Package hyperref Info: Link coloring ON on input line 2636.
\Hy@abspage=\count108
\c@Item=\count109
)
*hyperref using driver hpdftex*
(/usr/share/texmf-texlive/tex/latex/hyperref/hpdftex.def
File: hpdftex.def 2003/11/30 v6.74m Hyperref driver for pdfTeX
(/usr/share/texmf-texlive/tex/latex/psnfss/pifont.sty
Package: pifont 2004/09/15 PSNFSS-v9.2 Pi font support (SPQR) 
LaTeX Font Info:    Try loading font information for U+pzd on input line 63.
(/usr/share/texmf-texlive/tex/latex/psnfss/upzd.fd
File: upzd.fd 2001/06/04 font definitions for U/pzd.
)
LaTeX Font Info:    Try loading font information for U+psy on input line 64.
(/usr/share/texmf-texlive/tex/latex/psnfss/upsy.fd
File: upsy.fd 2001/06/04 font definitions for U/psy.
))
\Fld@listcount=\count110
\@outlinefile=\write3
)

LaTeX Warning: You have requested, on input line 257, version
               `2007/02/07' of package hyperref,
               but only version
               `2003/11/30 v6.74m Hypertext links for LaTeX'
               is available.

\c@pcabs@footnote=\count111
\Tocskip=\skip55
(./26262-t.aux)
\openout1 = `26262-t.aux'.

LaTeX Font Info:    Checking defaults for OML/txmi/m/it on input line 462.
LaTeX Font Info:    Try loading font information for OML+txmi on input line 462
.
(/usr/share/texmf-texlive/tex/latex/txfonts/omltxmi.fd
File: omltxmi.fd 2000/12/15 v3.1
)
LaTeX Font Info:    ... okay on input line 462.
LaTeX Font Info:    Checking defaults for T1/cmr/m/n on input line 462.
LaTeX Font Info:    ... okay on input line 462.
LaTeX Font Info:    Checking defaults for OT1/cmr/m/n on input line 462.
LaTeX Font Info:    ... okay on input line 462.
LaTeX Font Info:    Checking defaults for OMS/txsy/m/n on input line 462.
LaTeX Font Info:    Try loading font information for OMS+txsy on input line 462
.
(/usr/share/texmf-texlive/tex/latex/txfonts/omstxsy.fd
File: omstxsy.fd 2000/12/15 v3.1
)
LaTeX Font Info:    ... okay on input line 462.
LaTeX Font Info:    Checking defaults for OMX/txex/m/n on input line 462.
LaTeX Font Info:    Try loading font information for OMX+txex on input line 462
.
(/usr/share/texmf-texlive/tex/latex/txfonts/omxtxex.fd
File: omxtxex.fd 2000/12/15 v3.1
)
LaTeX Font Info:    ... okay on input line 462.
LaTeX Font Info:    Checking defaults for U/txexa/m/n on input line 462.
LaTeX Font Info:    Try loading font information for U+txexa on input line 462.

(/usr/share/texmf-texlive/tex/latex/txfonts/utxexa.fd
File: utxexa.fd 2000/12/15 v3.1
)
LaTeX Font Info:    ... okay on input line 462.
LaTeX Font Info:    Checking defaults for PD1/pdf/m/n on input line 462.
LaTeX Font Info:    ... okay on input line 462.
LaTeX Font Info:    Try loading font information for OT1+txr on input line 462.

(/usr/share/texmf-texlive/tex/latex/txfonts/ot1txr.fd
File: ot1txr.fd 2000/12/15 v3.1
) (/usr/share/texmf-texlive/tex/context/base/supp-pdf.tex (/usr/share/texmf-tex
live/tex/context/base/supp-mis.tex
loading : Context Support Macros / Miscellaneous (2004.10.26)
\protectiondepth=\count112
\scratchcounter=\count113
\scratchtoks=\toks25
\scratchdimen=\dimen122
\scratchskip=\skip56
\scratchmuskip=\muskip12
\scratchbox=\box28
\scratchread=\read1
\scratchwrite=\write4
\zeropoint=\dimen123
\onepoint=\dimen124
\onebasepoint=\dimen125
\minusone=\count114
\thousandpoint=\dimen126
\onerealpoint=\dimen127
\emptytoks=\toks26
\nextbox=\box29
\nextdepth=\dimen128
\everyline=\toks27
\!!counta=\count115
\!!countb=\count116
\recursecounter=\count117
)
loading : Context Support Macros / PDF (2004.03.26)
\nofMPsegments=\count118
\nofMParguments=\count119
\MPscratchCnt=\count120
\MPscratchDim=\dimen129
\MPnumerator=\count121
\everyMPtoPDFconversion=\toks28
)
-------------------- Geometry parameters
paper: letterpaper
landscape: --
twocolumn: --
twoside: true
asymmetric: --
h-parts: 90.91446pt, 0.63\paperwidth , 136.3717pt
v-parts: 95.39737pt, 0.7\paperheight , 143.09605pt
hmarginratio: 2:3
vmarginratio: 2:3
lines: --
heightrounded: --
bindingoffset: 0.0pt
truedimen: --
includehead: --
includefoot: --
includemp: --
driver: pdftex
-------------------- Page layout dimensions and switches
\paperwidth  614.295pt
\paperheight 794.96999pt
\textwidth  387.00883pt
\textheight 556.47656pt
\oddsidemargin  18.64447pt
\evensidemargin 64.10172pt
\topmargin  -8.74641pt
\headheight 12.0pt
\headsep    19.8738pt
\footskip   30.0pt
\marginparwidth 98.0pt
\marginparsep   7.0pt
\columnsep  10.0pt
\skip\footins  10.8pt plus 4.0pt minus 2.0pt
\hoffset 0.0pt
\voffset 0.0pt
\mag 1000
\@twosidetrue \@mparswitchtrue 
(1in=72.27pt, 1cm=28.45pt)
-----------------------
(/usr/share/texmf-texlive/tex/latex/graphics/color.sty
Package: color 1999/02/16 v1.0i Standard LaTeX Color (DPC)
LaTeX Info: Redefining \color on input line 71.
(/usr/share/texmf-texlive/tex/latex/config/color.cfg
File: color.cfg 2001/08/31 v1.1 color configuration of teTeX/TeXLive
)
Package color Info: Driver file: pdftex.def on input line 125.
)
Package hyperref Info: Link coloring ON on input line 462.
(/usr/share/texmf-texlive/tex/latex/hyperref/nameref.sty
Package: nameref 2003/12/03 v2.21 Cross-referencing by name of section
\c@section@level=\count122
)
LaTeX Info: Redefining \ref on input line 462.
LaTeX Info: Redefining \pageref on input line 462.
(./26262-t.out) (./26262-t.out)
\openout3 = `26262-t.out'.

LaTeX Font Info:    Try loading font information for OT1+txtt on input line 471
.
(/usr/share/texmf-texlive/tex/latex/txfonts/ot1txtt.fd
File: ot1txtt.fd 2000/12/15 v3.1
)
LaTeX Font Info:    Try loading font information for U+txsya on input line 492.

(/usr/share/texmf-texlive/tex/latex/txfonts/utxsya.fd
File: utxsya.fd 2000/12/15 v3.1
)
LaTeX Font Info:    Try loading font information for U+txsyb on input line 492.

(/usr/share/texmf-texlive/tex/latex/txfonts/utxsyb.fd
File: utxsyb.fd 2000/12/15 v3.1
)
LaTeX Font Info:    Try loading font information for U+txmia on input line 492.

(/usr/share/texmf-texlive/tex/latex/txfonts/utxmia.fd
File: utxmia.fd 2000/12/15 v3.1
)
LaTeX Font Info:    Try loading font information for U+txsyc on input line 492.

(/usr/share/texmf-texlive/tex/latex/txfonts/utxsyc.fd
File: utxsyc.fd 2000/12/15 v3.1
) [1

{/var/lib/texmf/fonts/map/pdftex/updmap/pdftex.map}] [2

]
! pdfTeX warning (ext4): destination with the same identifier (name{page.1}) ha
s been already used, duplicate ignored
<to be read again> 
                   \penalty 
l.573 \clearpage
                 [1] <./images/pubmark.pdf, id=244, 469.755pt x 134.5025pt>
File: ./images/pubmark.pdf Graphic file (type pdf)
<use ./images/pubmark.pdf>
! pdfTeX warning (ext4): destination with the same identifier (name{page.2}) ha
s been already used, duplicate ignored
<to be read again> 
                   \penalty 
l.586 \clearpage
                 [2

 <./images/pubmark.pdf>] [3

] [4

]
! pdfTeX warning (ext4): destination with the same identifier (name{page.1}) ha
s been already used, duplicate ignored
<to be read again> 
                   \penalty 
l.738 
       [1


]
Underfull \vbox (badness 1874) has occurred while \output is active []

! pdfTeX warning (ext4): destination with the same identifier (name{page.2}) ha
s been already used, duplicate ignored
<to be read again> 
                   \penalty 
l.808 
       [2]
Underfull \vbox (badness 1874) has occurred while \output is active []

! pdfTeX warning (ext4): destination with the same identifier (name{page.3}) ha
s been already used, duplicate ignored
<to be read again> 
                   \penalty 
l.861 
       [3]
! pdfTeX warning (ext4): destination with the same identifier (name{page.4}) ha
s been already used, duplicate ignored
<to be read again> 
                   \penalty 
l.907 
       [4]
LaTeX Font Info:    Try loading font information for OMS+txr on input line 909.

(/usr/share/texmf-texlive/tex/latex/txfonts/omstxr.fd
File: omstxr.fd 2000/12/15 v3.1
)
LaTeX Font Info:    Font shape `OMS/txr/m/n' in size <12> not available
(Font)              Font shape `OMS/txsy/m/n' tried instead on input line 909.

Underfull \vbox (badness 6332) has occurred while \output is active []

[5] [6] (./26262-t.toc [7


])
\tf@toc=\write5
\openout5 = `26262-t.toc'.

[8]
LaTeX Font Info:    Font shape `OMS/txr/m/n' in size <10> not available
(Font)              Font shape `OMS/txsy/m/n' tried instead on input line 1151.

! pdfTeX warning (ext4): destination with the same identifier (name{page.1}) ha
s been already used, duplicate ignored
<to be read again> 
                   \penalty 
l.1157 B
        ut, what has produced this notion, that the subject of [1



]
Underfull \vbox (badness 1874) has occurred while \output is active []

! pdfTeX warning (ext4): destination with the same identifier (name{page.2}) ha
s been already used, duplicate ignored
<to be read again> 
                   \penalty 
l.1214 
        [2]
! pdfTeX warning (ext4): destination with the same identifier (name{page.3}) ha
s been already used, duplicate ignored
<to be read again> 
                   \penalty 
l.1295 
        [3]
Underfull \vbox (badness 6332) has occurred while \output is active []

! pdfTeX warning (ext4): destination with the same identifier (name{page.4}) ha
s been already used, duplicate ignored
<to be read again> 
                   \penalty 
l.1339 
        [4]
! pdfTeX warning (ext4): destination with the same identifier (name{page.5}) ha
s been already used, duplicate ignored
<to be read again> 
                   \penalty 
l.1365 \[
          [5]
! pdfTeX warning (ext4): destination with the same identifier (name{page.6}) ha
s been already used, duplicate ignored
<to be read again> 
                   \penalty 
l.1435 T
        o express the part of the stress ($P$~\&c.)\ which depends on [6]
! pdfTeX warning (ext4): destination with the same identifier (name{page.7}) ha
s been already used, duplicate ignored
<to be read again> 
                   \penalty 
l.1498 T
        o put down the equations of motion let $X_x$, $Y_x$, $Z_x$ be the [7]
! pdfTeX warning (ext4): destination with the same identifier (name{page.8}) ha
s been already used, duplicate ignored
<to be read again> 
                   \penalty 
l.1557 \SectionTitle{Electricity}
                                  [8] [9] [10] [11

] [12] [13] [14] [15] [16] [17] [18] [19] [20] [21] [22] [23] [24

] [25] [26] [27] [28] [29] [30] [31] [32]
LaTeX Font Info:    Font shape `OMS/txr/m/it' in size <12> not available
(Font)              Font shape `OMS/txsy/m/n' tried instead on input line 3057.

[33] [34] [35] [36] [37] [38] [39] [40] [41] [42] [43] [44] [45] [46] [47] <./i
mages/cylinder.pdf, id=1007, 130.4875pt x 88.33pt>
File: ./images/cylinder.pdf Graphic file (type pdf)
<use ./images/cylinder.pdf> [48 <./images/cylinder.pdf>] [49] [50] [51] [52] [5
3] [54] [55

]
Underfull \vbox (badness 10000) has occurred while \output is active []

[56] [57] [58] [59] [60] [61] [62]
Overfull \hbox (2.23517pt too wide) in paragraph at lines 4862--4867
\OT1/txr/m/n/12 From this we know by the the-ory of po-ten-tial that at the sur
-face where the charge $\OML/txmi/m/it/12 ^^[$
 []

[63] [64] [65] [66] [67] [68] [69] [70] [71] [72] [73] [74] [75] [76] [77

] [78] [79] [80] [81] [82] [83] [84] [85] [86] [87] [88] [89] [90] [91] [92] [9
3] [94] [95] [96

] [97] [98] [99] [100] [101] [102] [103] [104] [105] [106] [107] [108] [109] [1
10

] [111] [112

]
Underfull \vbox (badness 10000) has occurred while \output is active []

[113]
Underfull \vbox (badness 10000) has occurred while \output is active []

[114]
Underfull \vbox (badness 10000) has occurred while \output is active []

[115]
Underfull \vbox (badness 10000) has occurred while \output is active []

[116]
Underfull \vbox (badness 10000) has occurred while \output is active []

[117]
Underfull \vbox (badness 10000) has occurred while \output is active []

[118]
Underfull \vbox (badness 10000) has occurred while \output is active []

[119] [120] (./26262-t.aux)

 *File List*
    book.cls    2004/02/16 v1.4f Standard LaTeX document class
    bk12.clo    2004/02/16 v1.4f Standard LaTeX file (size option)
 amsmath.sty    2000/07/18 v2.13 AMS math features
 amstext.sty    2000/06/29 v2.01
  amsgen.sty    1999/11/30 v2.0
  amsbsy.sty    1999/11/29 v1.2d
  amsopn.sty    1999/12/14 v2.01 operator names
 amssymb.sty    2002/01/22 v2.2d
amsfonts.sty    2001/10/25 v2.2f
  ifthen.sty    2001/05/26 v1.1c Standard LaTeX ifthen package (DPC)
inputenc.sty    2004/02/05 v1.0d Input encoding file
  latin1.def    2004/02/05 v1.0d Input encoding file
graphicx.sty    1999/02/16 v1.0f Enhanced LaTeX Graphics (DPC,SPQR)
  keyval.sty    1999/03/16 v1.13 key=value parser (DPC)
graphics.sty    2001/07/07 v1.0n Standard LaTeX Graphics (DPC,SPQR)
    trig.sty    1999/03/16 v1.09 sin cos tan (DPC)
graphics.cfg    2001/08/31 v1.1 graphics configuration of teTeX/TeXLive
  pdftex.def    2005/06/20 v0.03m graphics/color for pdftex
   alltt.sty    1997/06/16 v2.0g defines alltt environment
indentfirst.sty    1995/11/23 v1.03 Indent first paragraph (DPC)
footmisc.sty    2005/03/17 v5.3d a miscellany of footnote facilities
 perpage.sty    2002/12/20 v1.0 Reset counters per page
 txfonts.sty    2005/01/03 v3.2
fancyhdr.sty    
extramarks.sty    
geometry.sty    2002/07/08 v3.2 Page Geometry
geometry.cfg
hyperref.sty    2003/11/30 v6.74m Hypertext links for LaTeX
  pd1enc.def    2003/11/30 v6.74m Hyperref: PDFDocEncoding definition (HO)
hyperref.cfg    2002/06/06 v1.2 hyperref configuration of TeXLive
     url.sty    2005/06/27  ver 3.2  Verb mode for urls, etc.
 hpdftex.def    2003/11/30 v6.74m Hyperref driver for pdfTeX
  pifont.sty    2004/09/15 PSNFSS-v9.2 Pi font support (SPQR) 
    upzd.fd    2001/06/04 font definitions for U/pzd.
    upsy.fd    2001/06/04 font definitions for U/psy.
 omltxmi.fd    2000/12/15 v3.1
 omstxsy.fd    2000/12/15 v3.1
 omxtxex.fd    2000/12/15 v3.1
  utxexa.fd    2000/12/15 v3.1
  ot1txr.fd    2000/12/15 v3.1
supp-pdf.tex
   color.sty    1999/02/16 v1.0i Standard LaTeX Color (DPC)
   color.cfg    2001/08/31 v1.1 color configuration of teTeX/TeXLive
 nameref.sty    2003/12/03 v2.21 Cross-referencing by name of section
 26262-t.out
 26262-t.out
 ot1txtt.fd    2000/12/15 v3.1
  utxsya.fd    2000/12/15 v3.1
  utxsyb.fd    2000/12/15 v3.1
  utxmia.fd    2000/12/15 v3.1
  utxsyc.fd    2000/12/15 v3.1
./images/pubmark.pdf
  omstxr.fd    2000/12/15 v3.1
./images/cylinder.pdf
 ***********

 ) 
Here is how much of TeX's memory you used:
 6129 strings out of 95148
 76537 string characters out of 1184452
 134140 words of memory out of 1000000
 8370 multiletter control sequences out of 10000+50000
 69014 words of font info for 159 fonts, out of 500000 for 2000
 73 hyphenation exceptions out of 8191
 28i,19n,43p,260b,477s stack positions out of 1500i,500n,5000p,200000b,5000s
PDF statistics:
 2041 PDF objects out of 300000
 758 named destinations out of 131072
 459 words of extra memory for PDF output out of 10000
</usr/share/texmf-texlive/fonts/type1/public/txfonts/rtxi.pfb></usr/share/tex
mf-texlive/fonts/type1/public/txfonts/txex.pfb></usr/share/texmf-texlive/fonts/
type1/public/txfonts/txsya.pfb></usr/share/texmf-texlive/fonts/type1/public/txf
onts/txmia.pfb></usr/share/texmf-texlive/fonts/type1/public/txfonts/txexa.pfb><
/usr/share/texmf-texlive/fonts/type1/public/txfonts/txsy.pfb></usr/share/texmf-
texlive/fonts/type1/public/txfonts/rtxmi.pfb></usr/share/texmf-texlive/fonts/ty
pe1/public/txfonts/rtxr.pfb>{/usr/share/texmf-texlive/fonts/enc/dvips/base/8r.e
nc}</usr/share/texmf-texlive/fonts/type1/urw/times/utmri8a.pfb></usr/share/texm
f-texlive/fonts/type1/urw/times/utmr8a.pfb></usr/share/texmf-texlive/fonts/type
1/public/txfonts/rtxsc.pfb></usr/share/texmf-texlive/fonts/type1/urw/times/utmb
8a.pfb></usr/share/texmf-texlive/fonts/type1/public/txfonts/txtt.pfb>
Output written on 26262-t.pdf (134 pages, 732335 bytes).
